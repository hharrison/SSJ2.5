\defclass {WatsonUDist}

Extends the class \class{ContinuousDistribution} for the
\emph{Watson U}  distribution (see \cite{tDUR73a,tSTE70a,tSTE86b}).
Given a sample of $n$ independent uniforms $u_i$ over $[0,1]$,
the \emph{Watson} statistic $U_n^2$ is defined by
\begin{latexonly}%
\begin {eqnarray*}
    W_n^2 &=& \frac{1}{12n} +
            \sum_{j=1}^n \left\{u_{(j)} - \frac{(j- 1/2)}{n}\right\}^2, \\
    U_n^2 &=& W_n^2  - n\left (\bar u_n - 1/2\right)^2.
                                                   \eqlabel {eq:WatsonU}
\end {eqnarray*}
\end{latexonly}%
\begin{htmlonly}
\eq
    W_n^2 = {1/ 12n} +
            \sum_{j=1}^n \left\{u_{(j)} - {(j- 0.5)/ n}\right\}^2, \\
    U_n^2 = W_n^2  - n\left (\bar u_n - 1/2\right)^2.
\endeq
\end{htmlonly}%
  where the $u_{(j)}$ are the $u_i$ sorted in increasing order, and
  $\bar u_n$ is the average of the observations $u_{i}$.
 The distribution function (the cumulative probabilities)
 is defined as $F_n(x) = P[U_n^2 \le x]$.

\bigskip\hrule

%%%%%%%%%%%%%%%%%%%%%%%%%%%%%%%%%%%%%%%%
\begin{code}
\begin{hide}
/*
 * Class:        WatsonUDist
 * Description:  Watson U  distribution 
 * Environment:  Java
 * Software:     SSJ 
 * Copyright (C) 2001  Pierre L'Ecuyer and Université de Montréal
 * Organization: DIRO, Université de Montréal
 * @author       Richard Simard
 * @since

 * SSJ is free software: you can redistribute it and/or modify it under
 * the terms of the GNU General Public License (GPL) as published by the
 * Free Software Foundation, either version 3 of the License, or
 * any later version.

 * SSJ is distributed in the hope that it will be useful,
 * but WITHOUT ANY WARRANTY; without even the implied warranty of
 * MERCHANTABILITY or FITNESS FOR A PARTICULAR PURPOSE.  See the
 * GNU General Public License for more details.

 * A copy of the GNU General Public License is available at
   <a href="http://www.gnu.org/licenses">GPL licence site</a>.
 */
\end{hide}
package umontreal.iro.lecuyer.probdist;
\begin{hide}
import umontreal.iro.lecuyer.util.*;
import umontreal.iro.lecuyer.functions.MathFunction;
\end{hide} 

public class WatsonUDist extends ContinuousDistribution\begin{hide} {
   private static final double XSEPARE = 0.15;
   private static final double PI = Math.PI;
   private static final int JMAX = 10;
   protected int n;

   private static class Function implements MathFunction {
      protected int n;
      protected double u;

      public Function (int n, double u) {
         this.n = n;
         this.u = u;
      }

      public double evaluate (double x) {
         return u - cdf(n,x);
      }
   }

   private static double cdfn (int n, double x) {
      // The 1/n correction for the cdf, for x < XSEPARE
      double terme;
      double v = Math.exp (-0.125/x);
      double somme = 0;
      int j = 0;

      do {
         double a = (2*j + 1)*(2*j + 1);
         terme = Math.pow (v, (double)(2*j + 1)*(2*j + 1));
         double der = terme*(a - 4.0*x)/(8.0*x*x);
         somme += (5.0*x - 1.0/12.0) * der / 12.0;
         der = terme* (a*a - 24.0*a*x + 48.0*x*x)/ (64.0*x*x*x*x);
         somme += x*x*der/6.0;
         ++j;
      } while (!(terme <= Math.abs(somme) * Num.DBL_EPSILON || j > JMAX));
      if (j > JMAX)
         System.err.println (x + ": watsonU:  somme 1/n has not converged");

      v = -2.0*somme/(n*Math.sqrt (2.0*PI*x));
      return v;
   }\end{hide}
\end{code}
%%%%%%%%%%%%%%%%%%%%%%%%%%%%%%%%%%%%%%%%%
\subsubsection* {Constructor}

\begin{code}

   public WatsonUDist (int n)\begin{hide} {
      setN (n);
   }\end{hide}
\end{code}
\begin{tabb}
   Constructs a {\em Watson U\/} distribution for a sample of size $n$.

\end{tabb}

%%%%%%%%%%%%%%%%%%%%%%%%%%%%%%%%%%%
\subsubsection* {Methods}

\begin{code}\begin{hide}

   public double density (double x) {
      return density (n, x);
   }

   public double cdf (double x) {
      return cdf (n, x);
   }

   public double barF (double x) {
      return barF (n, x);
   }

   public double inverseF (double u) {
      return inverseF (n, u);
   }

   public double getMean() {
      return WatsonUDist.getMean (n);
   }

   public double getVariance() {
      return WatsonUDist.getVariance (n);
   }

   public double getStandardDeviation() {
      return WatsonUDist.getStandardDeviation (n);
   }\end{hide}

   public static double density (int n, double x)\begin{hide} {
      if (n < 2)
         throw new IllegalArgumentException ("n < 2");

      if (x <= 1.0/(12.0*n) || x >= n/12.0 || x >= XBIG)
         return 0.0;

      final double EPS = 1.0 / 100.0;
      return (cdf(n, x + EPS) - cdf(n, x - EPS)) / (2.0 * EPS);

/*
// This is the asymptotic density for n -> infinity
      int j;
      double signe;
      double v;
      double terme;
      double somme;

      if (x > XSEPARE) {
         // this series converges rapidly for x > 0.15
         v = Math.exp (-(x*2.0*PI*PI));
         signe = 1.0;
         somme = 0.0;
         j = 1;
         do {
            terme = j*j*Math.pow (v, (double)j*j);
            somme += signe*terme;
            signe = -signe;
            ++j;
         } while (!(terme < Num.DBL_EPSILON || j > JMAX));
         if (j > JMAX)
            System.err.println ("watsonU:  sum1 has not converged");
         return 4.0*PI*PI*somme;
      }

      // this series converges rapidly for x <= 0.15
      v = Math.exp (-0.125/x);
      somme = v;
      double somme2 = v;
      j = 2;
      do {
         terme = Math.pow (v, (double)(2*j - 1)*(2*j - 1));
         somme += terme;
         somme2 += terme * (2*j - 1)*(2*j - 1);
         ++j;
      } while (!(terme <= somme2 * Num.DBL_EPSILON || j > JMAX));
      if (j > JMAX)
         System.err.println ("watsonU:  sum2 has not converged");

      final double RACINE = Math.sqrt (2.0*PI*x);
      return -somme/(x*RACINE) + 2*somme2 * 0.125/ ((x*x) * RACINE);
*/
   }\end{hide}
\end{code}
\begin{tabb} Computes the density of the {\em Watson U\/} distribution
  with parameter $n$.
\end{tabb}
\begin{code}

   public static double cdf (int n, double x)\begin{hide} {
      if (n < 2)
         throw new IllegalArgumentException ("n < 2");

      if (x <= 1.0/(12.0*n))
         return 0.0;
      if (x > 3.95 || x >= n/12.0)
         return 1.0;

      if (2 == n) {
         if (x <= 1.0/24.0)
            return 0.0;
         if (x >= 1.0/6.0)
            return 1.0;
         return 2.0*Math.sqrt(2.0*x - 1.0/12.0);
      }

      if (x > XSEPARE)
         return 1.0 - barF (n, x);

      // this series converges rapidly for x <= 0.15
      double terme;
      double v = Math.exp (-0.125/x);
      double somme = v;
      int j = 2;

      do {
         terme = Math.pow (v, (double)(2*j - 1)*(2*j - 1));
         somme += terme;
         ++j;
      } while (!(terme <= somme * Num.DBL_EPSILON || j > JMAX));
      if (j > JMAX)
         System.err.println (x + ": watsonU:  sum2 has not converged");

      v = 2.0*somme/Math.sqrt (2.0*PI*x);
      v += cdfn(n, x);
      if (v >= 1.0)
         return 1.0;
      if (v <= 0.0)
         return 0.0;
       return v;
   }\end{hide}
\end{code}
\begin{tabb}
  Computes the Watson $U$ distribution function, i.e. returns
  $P[U_n^2 \le x]$, where $U_n^2$ is the Watson statistic  
  defined in (\ref{eq:WatsonU}). We use the asymptotic distribution for
$n \to\infty$, plus a correction in $O(1/n)$, as given in \cite{tCSO96a}.
%  It is given by
%   \begin{equation}
%    P(U_\infty^2 \le x)  \ = \ 1 + 2 \sum_{j=1}^\infty (-1)^j e^{-2 j^2 \pi^2 x}
%                                                      \eqlabel{eq:DistWU1}
%   \end{equation}
%  This sum converges extremely fast except for small $x$, where alternating
%  successive terms give rise to numerical instability.
%  But with the Poisson summation formula \cite{mLAN73a},
%  the sum can be transformed to
%  \begin{equation}
%     P(U_\infty^2 \le x)  \ = \  \sqrt{\frac2{\pi x}}\; \sum_{j=0}^\infty
%        e^{- (2 j+1)^2/8 x}                           \eqlabel{eq:DistWU2}
%  \end{equation}
%  which can be used for small $x$.
%  The absolute difference between the returned value and $P[U_n^2 \le x]$ 
%  is estimated to be less than 0.01 for $n \ge 8$, for the asymptotic formula.

% With the 1/n correction, there should be 4 decimals digits of precision. 
 \end{tabb}
\begin{code}

   public static double barF (int n, double x)\begin{hide} {
      if (n < 2)
         throw new IllegalArgumentException ("n < 2");

      if (x <= 1.0/(12.0*n))
         return 1.0;
      if (x >= XBIG || x >= n/12.0)
         return 0.0;

      if (2 == n)
         return 1.0 - 2.0*Math.sqrt(2.0*x - 1.0/12.0);

      if (x > XSEPARE) {
         // this series converges rapidly for x > 0.15
         double terme, ter;
         double v = Math.exp (-2.0*PI*PI*x);
         double signe = 1.0;
         double somme = 0.0;
         double son = 0.0;
         int j = 1;

         do {
            terme = Math.pow (v, (double)j*j);
            somme += signe*terme;
            double h = 2*j*PI*x;
            ter = (5.0*x - h*h - 1.0/12.0)*j*j;
            son += signe*terme*ter;
            signe = -signe;
            ++j;
         } while (!(terme < Num.DBL_EPSILON || j > JMAX));
         if (j > JMAX)
            System.err.println (x + ": watsonU:  sum1 has not converged");
         v = 2.0*somme + PI*PI*son/(3.0*n);
         if (v <= 0.0) 
            return 0.0;
         if (v >= 1.0)
            return 1.0;
         return v;
      }

      return (1.0 - cdf(n, x));
   }\end{hide}
\end{code}
\begin{tabb}
  Computes the complementary distribution function  $\bar F_n(x)$,
  where $F_n$ is the {\em Watson\/} $U$ distribution with parameter $n$.
\end{tabb}
\begin{code}

   public static double inverseF (int n, double u)\begin{hide} {
      if (n < 2)
         throw new IllegalArgumentException ("n < 2");
      if (u < 0.0 || u > 1.0)
         throw new IllegalArgumentException ("u must be in [0,1]");
      if (u >= 1.0)
         return n/12.0;
      if (u <= 0.0)
         return 1.0/(12.0*n);

      if (2 == n)
         return 1.0/24.0 + u*u/8.0;

      Function f = new Function (n,u);
      return RootFinder.brentDekker (0.0, 2.0, f, 1e-7);
   }\end{hide}
\end{code}
\begin{tabb}
  Computes $x = F_n^{-1}(u)$, where $F_n$ is the 
  {\em Watson\/} $U$ distribution with parameter $n$.
\end{tabb}
\begin{code}

   public static double getMean (int n)\begin{hide} {
      return 1.0/12.0;
   }\end{hide}
\end{code}
\begin{tabb} Returns the mean of the {\em Watson\/} $U$  distribution with
   parameter $n$.
\end{tabb}
\begin{htmlonly}
   \return{Returns the mean}
\end{htmlonly}
\begin{code}

   public static double getVariance (int n)\begin{hide} {
      return (n - 1)/(360.0*n);
   }\end{hide}
\end{code}
\begin{tabb} Returns the variance of the {\em Watson\/} $U$ distribution with
   parameter $n$.
\end{tabb}
\begin{htmlonly}
   \return{the variance}
\end{htmlonly}
\begin{code}

   public static double getStandardDeviation (int n)\begin{hide} {
      return Math.sqrt (WatsonUDist.getVariance (n));
   }\end{hide}
\end{code}
\begin{tabb} Returns the standard deviation of the {\em Watson\/} $U$
  distribution with parameter $n$.
\end{tabb}
\begin{htmlonly}
   \return{the standard deviation}
\end{htmlonly}
\begin{code}

   public int getN()\begin{hide} {
      return n;
   }\end{hide}
\end{code}
 \begin{tabb} Returns the parameter $n$ of this object.
 \end{tabb}
\begin{code}

   public void setN (int n)\begin{hide} {
      if (n < 2)
         throw new IllegalArgumentException ("n < 2");
      this.n = n;
      supportA = 1.0/(12.0*n);
      supportB = n/12.0;
   }\end{hide}
\end{code}
 \begin{tabb} Sets the parameter $n$ of this object.
 \end{tabb}
 \begin{code}

   public double[] getParams ()\begin{hide} {
      double[] retour = {n};
      return retour;
   }\end{hide}
\end{code}
\begin{tabb}
   Return an array containing the parameter $n$ of this object.
\end{tabb}
\begin{hide}\begin{code}

   public String toString ()\begin{hide} {
      return getClass().getSimpleName() + " : n = " + n;
   }\end{hide}
\end{code}
\begin{tabb}
   Returns a \texttt{String} containing information about the current distribution.
\end{tabb}\end{hide}
\begin{code}\begin{hide}
}\end{hide}
\end{code}
