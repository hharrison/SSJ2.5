\defclass {GammaDist}

Extends the class \class{ContinuousDistribution} for
the {\em gamma\/} distribution \cite[page 337]{tJOH95a} with
shape parameter $\alpha > 0$ and scale parameter
$\lambda > 0$.
The density is
\begin{htmlonly}
\eq
   f(x) = \lambda^{\alpha} x^{\alpha - 1}e^{-\lambda x}/\Gamma(\alpha),
   \qquad \mbox {for } x > 0,
\endeq
\end{htmlonly}
\begin{latexonly}
\eq
  f(x) = \frac{\lambda^\alpha x^{\alpha - 1}e^{-\lambda x}}
        {\Gamma(\alpha)},
   \qquad \mbox {for  } x > 0,\eqlabel{eq:fgamma}
\endeq
\end{latexonly}
where $\Gamma$ is the gamma function, defined by
\eq
    \Gamma (\alpha) = \int_0^\infty x^{\alpha-1} e^{-x} dx.
                                                       \eqlabel{eq:Gamma}
\endeq
In particular, $\Gamma(n) = (n-1)!$ when $n$ is a positive integer.


\bigskip\hrule

%%%%%%%%%%%%%%%%%%%%%%%%%%%%%%%%%%%%%%%%
\begin{code}
\begin{hide}
/*
 * Class:        GammaDist
 * Description:  gamma distribution
 * Environment:  Java
 * Software:     SSJ 
 * Copyright (C) 2001  Pierre L'Ecuyer and Université de Montréal
 * Organization: DIRO, Université de Montréal
 * @author       
 * @since

 * SSJ is free software: you can redistribute it and/or modify it under
 * the terms of the GNU General Public License (GPL) as published by the
 * Free Software Foundation, either version 3 of the License, or
 * any later version.

 * SSJ is distributed in the hope that it will be useful,
 * but WITHOUT ANY WARRANTY; without even the implied warranty of
 * MERCHANTABILITY or FITNESS FOR A PARTICULAR PURPOSE.  See the
 * GNU General Public License for more details.

 * A copy of the GNU General Public License is available at
   <a href="http://www.gnu.org/licenses">GPL licence site</a>.
 */
\end{hide}
package umontreal.iro.lecuyer.probdist;
\begin{hide}
import umontreal.iro.lecuyer.util.Num;
import umontreal.iro.lecuyer.functions.MathFunction;
import umontreal.iro.lecuyer.util.RootFinder;
\end{hide}

public class GammaDist extends ContinuousDistribution\begin{hide} {
   private double alpha;
   private double lambda;
   private double logFactor;      // Log (lambda^alpha / Gamma (alpha))
   private static final double ALIM = 1.0E5;

   private static class Function implements MathFunction {
      // For MLE
      private int n;
      private double empiricalMean;
      private double sumLn;

      public Function (int n, double empiricalMean, double sumLn) {
         this.n = n;
         this.empiricalMean = empiricalMean;
         this.sumLn = sumLn;
      }

      public double evaluate (double x) {
         if (x <= 0.0) return 1.0e200;
         return (n * Math.log (empiricalMean / x) + n * Num.digamma (x) - sumLn);
      }
   }


   private static class myFunc implements MathFunction {
      // For inverseF
      protected int d;
      protected double alp, u;

      public myFunc (double alp, int d, double u) {
         this.alp = alp;
         this.d = d;
         this.u = u;
      }

      public double evaluate (double x) {
         return u - GammaDist.cdf(alp, d, x);
      }
   }


\end{hide}\end{code}
%%%%%%%%%%%%%%%%%%%%%%%%%%%%%%%%%%%%%%%%%
\subsubsection* {Constructors}

\begin{code}

   public GammaDist (double alpha)\begin{hide} {
      setParams (alpha, 1.0, decPrec);
   }\end{hide}
\end{code}
  \begin{tabb} Constructs a \texttt{GammaDist} object with parameters
  $\alpha$ = \texttt{alpha} and $\lambda=1$.
  \end{tabb}
\begin{code}

   public GammaDist (double alpha, double lambda)\begin{hide} {
      setParams (alpha, lambda, decPrec);
   }\end{hide}
\end{code}
  \begin{tabb} Constructs a \texttt{GammaDist} object with parameters
  $\alpha$ = \texttt{alpha}  and $\lambda$ = \texttt{lambda}.
  \end{tabb}
\begin{code}

   public GammaDist (double alpha, double lambda, int d)\begin{hide} {
      setParams (alpha, lambda, d);
   }\end{hide}
\end{code}
\begin{tabb}
   Constructs a \texttt{GammaDist} object with parameters $\alpha$ =
   \texttt{alpha} and  $\lambda$ = \texttt{lambda}, and approximations of
   roughly \texttt{d} decimal digits of precision when computing functions.
\end{tabb}

%%%%%%%%%%%%%%%%%%%%%%%%%%%%%%%%%%%
\subsubsection* {Methods}
\begin{hide}
\begin{code}

   private static double mybelog (double x)
   /*
    * This is the function  1 + (1 - x*x + 2*x*log(x)) / ((1 - x)*(1 - x))
    */
   {
      if (x < 1.0e-30)
         return 0.0;
      if (x > 1.0e30)
         return 2.0*(Math.log(x) - 1.0) / x;
      if (x == 1.0)
         return 1.0;

      double t = 1.0 - x;
      if (x < 0.9 || x > 1.1) {
         double w = (t + x*Math.log(x)) / (t*t);
         return 2.0 * w;
      }

      // For x near 1, use a series expansion to avoid loss of precision.
      double term;
      final double EPS = 1.0e-12;
      double tpow = 1.0;
      double sum = 0.5;
      int j = 3;
      do {
         tpow *= t;
         term = tpow / (j * (j - 1));
         sum += term;
         j++;
      } while (Math.abs (term / sum) > EPS);
      return 2.0*sum;
   }


   public double density (double x) {
      if (x <= 0)
         return 0.0;
      double z = logFactor + (alpha - 1.0) * Math.log(x) - lambda * x;
      if (z > -XBIGM)
         return Math.exp (z);
      else
         return 0.0;
   }

   public double cdf (double x) {
      return cdf (alpha, lambda, decPrec, x);
   }

   public double barF (double x) {
      return barF (alpha, lambda, decPrec, x);
   }

   public double inverseF (double u) {
      return inverseF (alpha, decPrec, u)/lambda;
   }

   public double getMean() {
      return GammaDist.getMean (alpha, lambda);
   }

   public double getVariance() {
      return GammaDist.getVariance (alpha, lambda);
   }

   public double getStandardDeviation() {
      return GammaDist.getStandardDeviation (alpha, lambda);
   }
\end{code}
\end{hide}\begin{code}

   public static double density (double alpha, double lambda, double x)\begin{hide} {
      if (alpha <= 0)
         throw new IllegalArgumentException ("alpha <= 0");
      if (lambda <= 0)
         throw new IllegalArgumentException ("lambda <= 0");

      if (x <= 0)
         return 0.0;
      double z = alpha * Math.log (lambda*x) - lambda*x - Num.lnGamma (alpha);
      if (z > -XBIGM)
         return Math.exp (z) / x;
      else
         return 0.0;
   }\end{hide}
\end{code}
\begin{tabb} Computes the density function (\ref{eq:fgamma}) at $x$.
\end{tabb}
\begin{code}

   public static double cdf (double alpha, double lambda, int d, double x)\begin{hide} {
      return cdf (alpha, d, lambda*x);
   }\end{hide}
\end{code}
\begin{tabb} Returns an approximation of the gamma distribution
function with  parameters $\alpha$ = \texttt{alpha} and
$\lambda$ = \texttt{lambda}\html{.}\latex{, whose density is given
by (\ref{eq:fgamma}).
%  One has $\Gamma (\alpha) = (\alpha-1)!$ when $\alpha$ is an integer.
 The approximation is an improved version of the algorithm in \cite{tBAT70a}.}
 The function tries to return $d$ decimals digits of precision.
%, but there is no guarantee.
 For $\alpha$ not too large (e.g., $\alpha \le 1000$),
 $d$ gives a good idea of the precision attained.
\end{tabb}
\begin{code}

   public static double cdf (double alpha, int d, double x)\begin{hide} {
      if (alpha <= 0.0)
        throw new IllegalArgumentException ("alpha <= 0");
      if (d <= 0)
        throw new IllegalArgumentException ("d <= 0");
      if (x <= 0.0)
         return 0.0;
      if (1.0 == alpha)
         return ExponentialDist.cdf (1.0, x);

      if (alpha > 10.0) {
         if (x > alpha * 10.0)
            return 1.0;
      } else {
         if (x > XBIG)
            return 1.0;
      }

      if (alpha >= ALIM) {
         double d2 = x + 1.0/3.0 - alpha - 0.02/alpha;
         double S = alpha - 1.0/2.0;
         double w = mybelog(S/x);
         double y = d2 * Math.sqrt(w/x);
         return NormalDist.cdf01 (y);
      }

      if (x <= 1.0 || x < alpha) {
         double factor, z, rn, term;
         factor = Math.exp (alpha*Math.log (x) - x - Num.lnGamma (alpha));
         final double EPS = EPSARRAY[d];
         z = 1.0;
         term = 1.0;
         rn = alpha;
         do {
            rn += 1.0;
            term *= x/rn;
            z += term;
         } while (term >= EPS * z);
         return z*factor/alpha;

      } else
         return 1.0 - barF (alpha, d, x);
   }\end{hide}
\end{code}
  \begin{tabb}
  Equivalent to \texttt{cdf (alpha, 1.0, d, x)}.
  \end{tabb}
\begin{code}

   public static double barF (double alpha, double lambda, int d, double x)\begin{hide} {
      return barF (alpha, d, lambda*x);
   }\end{hide}
\end{code}
\begin{tabb} Computes the complementary distribution function.
\end{tabb}
\begin{code}

   public static double barF (double alpha, int d, double x)\begin{hide} {
      if (alpha <= 0.0)
        throw new IllegalArgumentException ("alpha <= 0");
      if (d <= 0)
        throw new IllegalArgumentException ("d <= 0");
      if (x <= 0.0)
         return 1.0;
      if (1.0 == alpha)
         return ExponentialDist.barF (1.0, x);

      if (alpha >= 70.0) {
         if (x >= alpha * XBIG)
            return 0.0;
      } else {
         if (x >= XBIGM)
            return 0.0;
      }

      if (alpha >= ALIM) {
         double d2 = x + 1.0/3.0 - alpha - 0.02/alpha;
         double S = alpha - 1.0/2.0;
         double w = mybelog(S/x);
         double y = d2 * Math.sqrt(w/x);
         return NormalDist.barF01 (y);
      }

      if (x <= 1.0 || x < alpha)
         return 1.0 - cdf (alpha, d, x);

      double[] V = new double[6];
      final double EPS = EPSARRAY[d];
      final double RENORM = 1.0E100;
      double R, dif;
      int i;
      double factor = Math.exp (alpha*Math.log (x) - x - Num.lnGamma (alpha));

      double A = 1.0 - alpha;
      double B = A + x + 1.0;
      double term = 0.0;
      V[0] = 1.0;
      V[1] = x;
      V[2] = x + 1.0;
      V[3] = x * B;
      double res = V[2]/V[3];

      do {
         A += 1.0;
         B += 2.0;
         term += 1.0;
         V[4] = B * V[2] - A * term * V[0];
         V[5] = B * V[3] - A * term * V[1];
         if (V[5] != 0.0) {
            R = V[4]/V[5];
            dif = Math.abs (res - R);
            if (dif <= EPS*R)
               return factor*res;
            res = R;
         }
         for (i = 0; i < 4; i++)
            V[i] = V[i + 2];
         if (Math.abs (V[4]) >= RENORM) {
            for (i = 0; i < 4; i++)
               V[i] /= RENORM;
         }
      } while (true);
   }\end{hide}
\end{code}
  \begin{tabb}
  Same as \method{barF}{double,double,int,double}~\texttt{(alpha, 1.0, d, x)}.
 \end{tabb}
\begin{code}

   public static double inverseF (double alpha, double lambda, int d,
                                  double u)\begin{hide} {
      return inverseF (alpha, d, u)/lambda;
   }\end{hide}
\end{code}
\begin{tabb}  Computes the inverse distribution function.
\end{tabb}
\begin{code}

   public static double inverseF (double alpha, int d, double u)\begin{hide} {
      if (alpha <= 0.0)
         throw new IllegalArgumentException ("alpha <= 0");
      if (u > 1.0 || u < 0.0)
         throw new IllegalArgumentException ("u not in [0,1]");
      if (u <= 0.0)
         return 0;
      if (u >= 1.0)
         return Double.POSITIVE_INFINITY;
      if (d <= 0)
         throw new IllegalArgumentException ("d <= 0");
      if (d > 15)
         d = 15;
      final double EPS = Math.pow (10.0, -d);

      double sigma = GammaDist.getStandardDeviation (alpha, 1.0);
      double x = NormalDist.inverseF (alpha, sigma, u);
      double v = GammaDist.cdf (alpha, d, x);
      double xmax;
      if (alpha < 1.0)
         xmax = 100.0;
      else
         xmax = alpha + 40.0 * sigma;
      myFunc f = new myFunc (alpha, d, u);

     if (u <= 1.0e-8 || alpha <= 1.5) {
         if (v < u)
            return RootFinder.bisection (x, xmax, f, EPS);
         else
            return RootFinder.bisection (0, x, f, EPS);
      } else {
          if (v < u)
            return RootFinder.brentDekker (x, xmax, f, EPS);
         else
            return RootFinder.brentDekker (0, x, f, EPS);
      }
   }\end{hide}
\end{code}
\begin{tabb} Same as
\method{inverseF}{double,double,int,double}~\texttt{(alpha, 1, d, u)}.
\end{tabb}
\begin{code}

   public static double[] getMLE (double[] x, int n)\begin{hide} {
      double parameters[];
      double sum = 0.0;
      double sumLn = 0.0;
      double empiricalMean;
      double alphaMME;
      double a;
      final double LN_EPS = Num.LN_DBL_MIN - Num.LN2;

      parameters = new double[2];
      if (n <= 0)
         throw new IllegalArgumentException ("n <= 0");
      for (int i = 0; i < n; i++)
      {
         sum += x[i];
         if (x[i] <= 0.0)
            sumLn += LN_EPS;
         else
            sumLn += Math.log (x[i]);
      }
      empiricalMean = sum / (double) n;

      sum = 0.0;
      for (int i = 0; i < n; i++) {
         sum += (x[i] - empiricalMean) * (x[i] - empiricalMean);
      }

      alphaMME = (empiricalMean * empiricalMean * (double) n) / sum;
      if ((a = alphaMME - 10.0) <= 0) {
         a = 1.0e-5;
      }

      Function f = new Function (n, empiricalMean, sumLn);
      parameters[0] = RootFinder.brentDekker (a, alphaMME + 10.0, f, 1e-7);
      parameters[1] = parameters[0] / empiricalMean;

      return parameters;
   }\end{hide}
\end{code}
\begin{tabb}
  Estimates the parameters $(\alpha,\lambda)$ of the gamma distribution
   using the maximum likelihood method, from the $n$ observations
   $x[i]$, $i = 0, 1,\ldots, n-1$. The estimates are returned in a two-element
    array, in regular order: [$\alpha$, $\lambda$].
   \begin{detailed}
   The maximum likelihood estimators are the values $(\hat\alpha , \hat\lambda)$
   that satisfy the equations:
   \begin{eqnarray*}
      \frac{1}{n} \sum_{i=1}^{n} \ln(x_i) - \ln(\bar{x}_n) & = &
       \psi(\hat{\alpha}) - \ln(\hat{\alpha})\\
      \hat{\lambda} \bar{x}_n & = & \hat{\alpha}
   \end{eqnarray*}
   where $\bar x_n$ is the average of $x[0],\dots,x[n-1]$, and
   $\psi$ is the logarithmic derivative of the Gamma function
   $\psi(x) = \Gamma'(x) / \Gamma(x)$ (\cite[page 361]{tJOH95a}).
   \end{detailed}
\end{tabb}
\begin{htmlonly}
   \param{x}{the list of observations to use to evaluate parameters}
   \param{n}{the number of observations to use to evaluate parameters}
   \return{returns the parameters [$\hat{\alpha}$, $\hat{\lambda}$]}
\end{htmlonly}
\begin{code}

   public static GammaDist getInstanceFromMLE (double[] x, int n)\begin{hide} {
      double parameters[] = getMLE (x, n);
      return new GammaDist (parameters[0], parameters[1]);
   }\end{hide}
\end{code}
\begin{tabb}
   Creates a new instance of a gamma distribution with parameters $\alpha$ and $\lambda$
   estimated using the maximum likelihood method based on the $n$ observations
   $x[i]$, $i = 0, 1, \ldots, n-1$.
\end{tabb}
\begin{htmlonly}
   \param{x}{the list of observations to use to evaluate parameters}
   \param{n}{the number of observations to use to evaluate parameters}
\end{htmlonly}
\begin{code}

   public static double getMean (double alpha, double lambda)\begin{hide} {
      if (alpha <= 0.0)
         throw new IllegalArgumentException ("alpha <= 0");
      if (lambda <= 0.0)
         throw new IllegalArgumentException ("lambda <= 0");

      return (alpha / lambda);
   }\end{hide}
\end{code}
\begin{tabb}  Computes and returns the mean $E[X] = \alpha/\lambda$
   of the gamma distribution with parameters $\alpha$ and $\lambda$.
\end{tabb}
\begin{htmlonly}
   \return{the mean of the gamma distribution $E[X] = \alpha / \lambda$}
\end{htmlonly}
\begin{code}

   public static double getVariance (double alpha, double lambda)\begin{hide} {
      if (alpha <= 0.0)
         throw new IllegalArgumentException ("alpha <= 0");
      if (lambda <= 0.0)
         throw new IllegalArgumentException ("lambda <= 0");

      return (alpha / (lambda * lambda));
   }\end{hide}
\end{code}
\begin{tabb}  Computes and returns the variance $\mbox{Var}[X] = \alpha/\lambda^2$
   of the gamma distribution with parameters $\alpha$ and $\lambda$.
\end{tabb}
\begin{htmlonly}
   \return{the variance of the gamma distribution $\mbox{Var}[X] = \alpha / \lambda^2$}
\end{htmlonly}
\begin{code}

   public static double getStandardDeviation (double alpha, double lambda)\begin{hide} {
      if (alpha <= 0.0)
         throw new IllegalArgumentException ("alpha <= 0");
      if (lambda <= 0.0)
         throw new IllegalArgumentException ("lambda <= 0");

      return (Math.sqrt(alpha) / lambda);
   }\end{hide}
\end{code}
\begin{tabb}  Computes and returns the standard deviation of the gamma
   distribution with parameters $\alpha$ and $\lambda$.
\end{tabb}
\begin{htmlonly}
   \return{the standard deviation of the gamma distribution}
\end{htmlonly}
\begin{code}

   public double getAlpha()\begin{hide} {
      return alpha;
   }\end{hide}
\end{code}
  \begin{tabb} Return the parameter $\alpha$ for this object.
  \end{tabb}
\begin{code}

   public double getLambda()\begin{hide} {
      return lambda;
   }\end{hide}
\end{code}
\begin{tabb} Return the parameter $\lambda$ for this object.
\end{tabb}
\begin{code}

   public void setParams (double alpha, double lambda, int d)\begin{hide} {
      if (alpha <= 0)
         throw new IllegalArgumentException ("alpha <= 0");
      if (lambda <= 0)
         throw new IllegalArgumentException ("lambda <= 0");

      this.alpha   = alpha;
      this.lambda  = lambda;
      this.decPrec = d;
      logFactor    = alpha * Math.log(lambda) - Num.lnGamma (alpha);
      supportA = 0.0;
    } \end{hide}
\end{code}
\begin{tabb}
\end{tabb}
\begin{code}

   public double[] getParams ()\begin{hide} {
      double[] retour = {alpha, lambda};
      return retour;
   }\end{hide}
\end{code}
\begin{tabb}
   Return a table containing the parameters of the current distribution.
   This table is put in regular order: [$\alpha$, $\lambda$].
\end{tabb}
\begin{hide}\begin{code}

   public String toString ()\begin{hide} {
      return getClass().getSimpleName() + " : alpha = " + alpha + ", lambda = " + lambda;
   }\end{hide}
\end{code}
\begin{tabb}
   Returns a \texttt{String} containing information about the current distribution.
\end{tabb}\end{hide}
\begin{code}\begin{hide}
}\end{hide}
\end{code}
