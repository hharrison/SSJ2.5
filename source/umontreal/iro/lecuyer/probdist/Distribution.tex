\defclass {Distribution}

This interface should be implemented by all classes supporting 
discrete and continuous distributions. % over the \emph{real numbers}.
It specifies the signature of methods that compute
%%the density $f(x)$, 
the distribution function $F(x)$,
%=P[X\le x]$, 
the complementary distribution function $\bar F(x)$,
and the inverse distribution function $ F^{-1} (u)$.
%for a random variable $X$ having a univariate continuous distribution.
%
%For distributions over integers, use the class 
%\externalclass{umontreal.iro.lecuyer.probdist}{DiscreteDistributionInt}.
It also specifies the signature of methods that returns the mean,
the variance and the standard deviation.

\bigskip\hrule

\begin{code}
\begin{hide}
/*
 * Class:        Distribution
 * Description:  interface for all discrete and continuous distributions
 * Environment:  Java
 * Software:     SSJ 
 * Copyright (C) 2001  Pierre L'Ecuyer and Université de Montréal
 * Organization: DIRO, Université de Montréal
 * @author       
 * @since

 * SSJ is free software: you can redistribute it and/or modify it under
 * the terms of the GNU General Public License (GPL) as published by the
 * Free Software Foundation, either version 3 of the License, or
 * any later version.

 * SSJ is distributed in the hope that it will be useful,
 * but WITHOUT ANY WARRANTY; without even the implied warranty of
 * MERCHANTABILITY or FITNESS FOR A PARTICULAR PURPOSE.  See the
 * GNU General Public License for more details.

 * A copy of the GNU General Public License is available at
   <a href="http://www.gnu.org/licenses">GPL licence site</a>.
 */
\end{hide}
package umontreal.iro.lecuyer.probdist;


public interface Distribution\begin{hide} {\end{hide}

   public double cdf (double x);
\end{code}
\begin{tabb} Returns the distribution function $F(x)$.
%% \eq
%%   F(x) = P[X\le x] = \int_{-\infty}^x f(s)ds.
%% \endeq
\end{tabb}
\begin{htmlonly}
   \param{x}{value at which the distribution function is evaluated}
   \return{distribution function evaluated at \texttt{x}}
\end{htmlonly}
\begin{code}

   public double barF (double x);
\end{code}
\begin{tabb} Returns $\bar F(x) = 1 - F(x)$.
\end{tabb}
\begin{htmlonly}
 \param{x}{value at which the complementary distribution function is evaluated}
 \return{complementary distribution function evaluated at \texttt{x}}
\end{htmlonly}
\begin{code}

   public double inverseF (double u);
\end{code}
\begin{tabb}  Returns the inverse distribution function
   $F^{-1}(u)$, defined in (\ref{eq:inverseF}).
%%  = \inf\{x\in\RR : F(x)\ge u\}$,  
%%    where $0\le u\le 1$.
%   The default implementation uses binary search to find the inverse of a
%   generic continuous distribution function $F$, evaluated at $u$.
%   The returned value has approximately \texttt{decPrec} decimal digits of
%   precision.
%%
%   If \texttt{detail} is true, the method will print detailed information
%   about the inversion process.
% \pierre{Va imprimer o\`u?  Probablement \`a enlever.}
%   Restriction: $0 \le u \le 1$.
\end{tabb}
\begin{htmlonly}
   \param{u}{value in the interval $(0,1)$ for which the inverse 
     distribution function is evaluated}
   \return{the inverse distribution function evaluated at \texttt{u}}
%  \exception{IllegalArgumentException}{if $u$ is  not in the interval $(0,1)$}
%  \exception{ArithmeticException}{if the inverse cannot be computed,
%    for example if it would give infinity in a theoritical context}
\end{htmlonly}
\begin{code}

   public double getMean();
\end{code}
\begin{tabb}   Returns the mean of the distribution function.
\end{tabb}
\begin{code}

   public double getVariance();
\end{code}
\begin{tabb}   Returns the variance of the distribution function.
\end{tabb}
\begin{code}

   public double getStandardDeviation();
\end{code}
\begin{tabb}   Returns the standard deviation of the distribution function.
\end{tabb}
\begin{code}

   public double[] getParams();
\end{code}
\begin{tabb}  Returns the parameters of the distribution function in the same
 order as in the constructors.
\end{tabb}
\begin{code}\begin{hide}
}\end{hide}
\end{code}
