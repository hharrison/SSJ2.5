\defclass {CauchyDist}

Extends the class \class{ContinuousDistribution} for
the {\em Cauchy\/} distribution \cite[page 299]{tJOH95a}
with location parameter $\alpha$
and scale parameter $\beta > 0$.
The density function is given by
\begin{htmlonly}
\eq
  f (x) = \beta/(\pi[(x - \alpha)^2 + \beta^2])\mbox{ for }-\infty < x < \infty.
\endeq
The distribution function is
\eq
  F (x) = 1/2 + \arctan ((x - \alpha)/\beta)/\pi,
 \qquad \qquad  \mbox{for } -\infty < x < \infty,
\endeq
\end{htmlonly}
\begin{latexonly}
\eq
    f (x) = \frac{\beta}{\pi[(x-\alpha)^2 + \beta^2]},
            \qquad \qquad  \mbox{for } -\infty < x < \infty . \eqlabel{eq:fcuachy}
\endeq
The distribution function is
\eq
  F (x) = \frac12 + \frac{\arctan ((x - \alpha)/\beta)}{\pi},
 \qquad \qquad  \mbox{for } -\infty < x < \infty,
\endeq
\end{latexonly}
and its inverse is
\eq
F^{-1} (u) = \alpha + \beta\tan (\pi(u - 1/2)).
\qquad \mbox{for } 0 < u < 1.
\endeq

\bigskip\hrule\bigskip

%%%%%%%%%%%%%%%%%%%%%%%%%%%%%%%%%%%%%%%%
\begin{code}\begin{hide}
/*
 * Class:        CauchyDist
 * Description:  Cauchy distribution
 * Environment:  Java
 * Software:     SSJ 
 * Copyright (C) 2001  Pierre L'Ecuyer and Université de Montréal
 * Organization: DIRO, Université de Montréal
 * @author       Richard Simard
 * @since        March 2009

 * SSJ is free software: you can redistribute it and/or modify it under
 * the terms of the GNU General Public License (GPL) as published by the
 * Free Software Foundation, either version 3 of the License, or
 * any later version.

 * SSJ is distributed in the hope that it will be useful,
 * but WITHOUT ANY WARRANTY; without even the implied warranty of
 * MERCHANTABILITY or FITNESS FOR A PARTICULAR PURPOSE.  See the
 * GNU General Public License for more details.

 * A copy of the GNU General Public License is available at
   <a href="http://www.gnu.org/licenses">GPL licence site</a>.
 */
\end{hide}
package umontreal.iro.lecuyer.probdist;
\begin{hide}
import umontreal.iro.lecuyer.util.Misc;
import optimization.*;
\end{hide}

public class CauchyDist extends ContinuousDistribution\begin{hide} {
   private double alpha;
   private double beta;

   private static class Optim implements Uncmin_methods
   {
      private int n;
      private double[] xi;

      public Optim (double[] x, int n)
      {
         this.n = n;
         this.xi = new double[n];
         System.arraycopy (x, 0, this.xi, 0, n);
      }

      public double f_to_minimize (double[] p)
      {
         double sum = 0.0;

         if (p[2] <= 0.0)               // barrier at 0
            return 1.0e200;

         for (int i = 0; i < n; i++)
            sum -= Math.log (density (p[1], p[2], xi[i]));

         return sum;
      }

      public void gradient (double[] x, double[] g)
      {
      }

      public void hessian (double[] x, double[][] h)
      {
      }
   }
\end{hide}
\end{code}
%%%%%%%%%%%%%%%%%%%%%%%%%%%%%%%%%%%%%%%%%
\subsubsection* {Constructors}

\begin{code}

   public CauchyDist()\begin{hide} {
      setParams (0.0, 1.0);
   }\end{hide}
\end{code}
  \begin{tabb} Constructs a \texttt{CauchyDist} object
   with parameters $\alpha=0$ and $\beta=1$.
  \end{tabb}
\begin{code}

   public CauchyDist (double alpha, double beta)\begin{hide} {
      setParams (alpha, beta);
   }\end{hide}
\end{code}
 \begin{tabb} Constructs a \texttt{CauchyDist} object with parameters
   $\alpha=$ \texttt{alpha} and $\beta=$ \texttt{beta}.
 \end{tabb}

%%%%%%%%%%%%%%%%%%%%%%%%%%%%%%%%%%%
\subsubsection* {Methods}
\begin{code}\begin{hide}

   public double density (double x) {
      return density (alpha, beta, x);
   }

   public double cdf (double x) {
      return cdf (alpha, beta, x);
   }

   public double barF (double x) {
      return barF (alpha, beta, x);
   }

   public double inverseF (double u){
      return inverseF (alpha, beta, u);
   }

   public double getMean() {
      return CauchyDist.getMean (alpha, beta);
   }

   public double getVariance() {
      return CauchyDist.getVariance (alpha, beta);
   }

   public double getStandardDeviation() {
      return CauchyDist.getStandardDeviation (alpha, beta);
   }\end{hide}

   public static double density (double alpha, double beta, double x)\begin{hide} {
      if (beta <= 0.0)
         throw new IllegalArgumentException ("beta <= 0");
      double t = (x - alpha)/beta;
      return 1.0/(beta * Math.PI*(1 + t*t));
   }\end{hide}
\end{code}
\begin{tabb} Computes the density function.
\end{tabb}
\begin{code}

   public static double cdf (double alpha, double beta, double x)\begin{hide} {
      // The integral was computed analytically using Mathematica
      if (beta <= 0.0)
         throw new IllegalArgumentException ("beta <= 0");
      double z = (x - alpha)/beta;
      if (z < -0.5)
         return Math.atan(-1.0/z)/Math.PI;
      return Math.atan(z)/Math.PI + 0.5;
   }\end{hide}
\end{code}
 \begin{tabb}
  Computes the  distribution function.
 \end{tabb}
\begin{code}

   public static double barF (double alpha, double beta, double x)\begin{hide} {
      if (beta <= 0.0)
         throw new IllegalArgumentException ("beta <= 0");
      double z = (x - alpha)/beta;
      if (z > 0.5)
         return Math.atan(1./z)/Math.PI;
      return 0.5 - Math.atan(z)/Math.PI;
   }\end{hide}
\end{code}
  \begin{tabb}
  Computes the complementary distribution.
 \end{tabb}
\begin{code}

   public static double inverseF (double alpha, double beta, double u)\begin{hide} {
      if (beta <= 0.0)
         throw new IllegalArgumentException ("beta <= 0");
     if (u < 0.0 || u > 1.0)
        throw new IllegalArgumentException ("u must be in [0,1]");
     if (u <= 0.0)
        return Double.NEGATIVE_INFINITY;
     if (u >= 1.0)
        return Double.POSITIVE_INFINITY;
     if (u < 0.5)
        return alpha - 1.0/Math.tan (Math.PI*u) * beta;
     return alpha + Math.tan (Math.PI*(u - 0.5)) * beta;
   }\end{hide}
\end{code}
  \begin{tabb}
  Computes the inverse of the distribution.
 \end{tabb}
\begin{code}

   public static double[] getMLE (double[] x, int n)\begin{hide} {
      double sum = 0.0;

      if (n <= 0)
         throw new IllegalArgumentException ("n <= 0");

      Optim system = new Optim (x, n);

      double[] parameters = new double[2];
      double[] xpls = new double[3];
      double[] param = new double[3];
      double[] fpls = new double[3];
      double[] gpls = new double[3];
      int[] itrcmd = new int[2];
      double[][] a = new double[3][3];
      double[] udiag = new double[3];

      param[1] = EmpiricalDist.getMedian (x, n);

      int m = Math.round ((float) n / 4.0f);
      double q3 = Misc.quickSelect (x, n, 3 * m);
      double q1 = Misc.quickSelect (x, n, m);
      param[2] = (q3 - q1) / 2.0;

      Uncmin_f77.optif0_f77 (2, param, system, xpls, fpls, gpls, itrcmd, a, udiag);

      for (int i = 0; i < 2; i++)
         parameters[i] = xpls[i+1];

      return parameters;
   }\end{hide}
\end{code}
\begin{tabb}
  Estimates the parameters $(\alpha,\beta)$ of the Cauchy distribution
  using the maximum likelihood method, from the $n$ observations
   $x[i]$, $i = 0, 1,\ldots, n-1$. The estimates are returned in a two-element
    array, in regular order: [$\alpha$, $\beta$].
   \begin{detailed}
   The estimates of the parameters are given by maximizing numerically the
   log-likelihood function, using the Uncmin package \cite{iSCHa,iVERa}.
   \end{detailed}
\end{tabb}
\begin{htmlonly}
   \param{x}{the list of observations to use to evaluate parameters}
   \param{n}{the number of observations to use to evaluate parameters}
   \return{returns the parameters [$\hat{\alpha}$, $\hat{\beta}$]}
\end{htmlonly}
\begin{code}

   public static CauchyDist getInstanceFromMLE (double[] x, int n)\begin{hide} {
      double parameters[] = getMLE (x, n);
      return new CauchyDist (parameters[0], parameters[1]);
   }\end{hide}
\end{code}
\begin{tabb}
   Creates a new instance of a Cauchy distribution with parameters $\alpha$ and $\beta$
   estimated using the maximum likelihood method based on the $n$ observations
   $x[i]$, $i = 0, 1, \ldots, n-1$.
\end{tabb}
\begin{htmlonly}
   \param{x}{the list of observations to use to evaluate parameters}
   \param{n}{the number of observations to use to evaluate parameters}
\end{htmlonly}
\begin{code}

   public static double getMean (double alpha, double beta)\begin{hide} {
      if (beta <= 0.0)
         throw new IllegalArgumentException ("beta <= 0");

      throw new UnsupportedOperationException("Undefined mean");
   }\end{hide}
\end{code}
\begin{tabb} Throws an exception since the mean does not exist.
\end{tabb}
\begin{htmlonly}
   %\return{the mean of the Cauchy distribution.}%
   \exception{UnsupportedOperationException}{the mean of the Cauchy distribution is undefined.}
\end{htmlonly}
\begin{code}

   public static double getVariance (double alpha, double beta)\begin{hide} {
      if (beta <= 0.0)
         throw new IllegalArgumentException ("beta <= 0");

      return Double.POSITIVE_INFINITY;
   }\end{hide}
\end{code}
\begin{tabb}  Returns $\infty$ since the variance does not exist.
\end{tabb}
\begin{htmlonly}
   \return{$\infty$.}
\end{htmlonly}
\begin{code}

   public static double getStandardDeviation (double alpha, double beta)\begin{hide} {
      return Double.POSITIVE_INFINITY;
   }\end{hide}
\end{code}
\begin{tabb} Returns $\infty$ since the standard deviation does not exist.
\end{tabb}
\begin{htmlonly}
   \return{$\infty$}
\end{htmlonly}
\begin{code}

   public double getAlpha()\begin{hide} {
      return alpha;
   }\end{hide}
\end{code}
  \begin{tabb}
  Returns the value of $\alpha$ for this object.
 \end{tabb}
\begin{code}

   public double getBeta()\begin{hide} {
      return beta;
   }
\end{hide}
\end{code}
  \begin{tabb}
  Returns the value of $\beta$ for this object.
 \end{tabb}
\begin{code}

   public void setParams (double alpha, double beta)\begin{hide} {
      if (beta <= 0.0)
         throw new IllegalArgumentException ("beta <= 0");
      this.alpha = alpha;
      this.beta = beta;
   }\end{hide}
\end{code}
  \begin{tabb}
  Sets the value of the parameters $\alpha$ and $\beta$ for this object.
 \end{tabb}
\begin{code}

   public double[] getParams ()\begin{hide} {
      double[] retour = {alpha, beta};
      return retour;
   }\end{hide}
\end{code}
\begin{tabb}
   Return a table containing parameters of the current distribution.
   This table is put in regular order: [$\alpha$, $\beta$].
\end{tabb}
\begin{hide}\begin{code}

   public String toString ()\begin{hide} {
      return getClass().getSimpleName() + " : alpha = " + alpha + ", beta = " + beta;
   }\end{hide}
\end{code}
\begin{tabb}
   Returns a \texttt{String} containing information about the current distribution.
\end{tabb}\end{hide}
\begin{code}\begin{hide}
}\end{hide}
\end{code}
