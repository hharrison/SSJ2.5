\defclass {ChiDist}

Extends the class \class{ContinuousDistribution} for the {\em chi\/}
  distribution \cite[page 417]{tJOH95a} with shape parameter
 $\latex{\nu}\html{v} > 0$,  where the number of degrees of freedom
 $\latex{\nu}\html{v}$ is a positive integer.
The density function is given by
\begin{htmlonly}
\eq
f (x) = e^{-x^2/2} x^{v - 1} / (2^{(v/2) - 1}\Gamma (v/2))\mbox{ for }x > 0,
\endeq
\end{htmlonly}
\begin{latexonly}
\eq f (x) = \frac {e^{-x^2 /2} x^{\nu-1}}{2^{(\nu /2)-1}\Gamma (\nu /2)},
              \qquad\mbox { for } x > 0,
              \eqlabel{eq:Fchi}
\endeq
\end{latexonly}
where $\Gamma (x)$ is the gamma function defined in
\latex{(\ref{eq:Gamma})}\html{\class{GammaDist}}.
The distribution function is
\eq F (x) = \latex{\frac{1}{\Gamma(\nu/2)}}\html{1/\Gamma(v/2)}
   \int_0^{x^2/2} t^{\latex{\nu}\html{v}/2-1}e^{-t}\ dt.
\endeq
% F(x) obtained by integration with Mathematica.
% Formula for the incomplete gamma taken
% from http://mathworld.wolfram.com/IncompleteGammaFunction.html
% Found to be equivalent to GammaDist
It is equivalent to the gamma distribution function with parameters
$\alpha=\latex{\nu}\html{v}/2$ and $\lambda=1$, evaluated at $x^2/2$.

\bigskip\hrule

%%%%%%%%%%%%%%%%%%%%%%%%%%%%%%%%%%%%%%%%
\begin{code}
\begin{hide}
/*
 * Class:        ChiDist
 * Description:  chi distribution
 * Environment:  Java
 * Software:     SSJ 
 * Copyright (C) 2001  Pierre L'Ecuyer and Université de Montréal
 * Organization: DIRO, Université de Montréal
 * @author       
 * @since

 * SSJ is free software: you can redistribute it and/or modify it under
 * the terms of the GNU General Public License (GPL) as published by the
 * Free Software Foundation, either version 3 of the License, or
 * any later version.

 * SSJ is distributed in the hope that it will be useful,
 * but WITHOUT ANY WARRANTY; without even the implied warranty of
 * MERCHANTABILITY or FITNESS FOR A PARTICULAR PURPOSE.  See the
 * GNU General Public License for more details.

 * A copy of the GNU General Public License is available at
   <a href="http://www.gnu.org/licenses">GPL licence site</a>.
 */
\end{hide}
package umontreal.iro.lecuyer.probdist;\begin{hide}
import umontreal.iro.lecuyer.util.Num;
import umontreal.iro.lecuyer.functions.MathFunction;
\end{hide}

public class ChiDist extends ContinuousDistribution\begin{hide} {
   private int nu;
   private double C1;

   private static class Function implements MathFunction {
      protected int n;
      protected double sum;

      public Function (double s, int n)
      {
         this.n = n;
         this.sum = s;
      }

      public double evaluate (double k)
      {
         if (k < 1.0) return 1.0e200;
         return (sum + n * (Num.lnGamma (k / 2.0) - 0.5*(Num.LN2) - Num.lnGamma ((k + 1.0) / 2.0)));
      }
   }
\end{hide}
\end{code}
%%%%%%%%%%%%%%%%%%%%%%%%%%%%%%%%%%%%%%%%%
\subsubsection* {Constructor}

\begin{code}

   public ChiDist (int nu)\begin{hide} {
      setNu (nu);
   }\end{hide}
\end{code}
  \begin{tabb} Constructs a \texttt{ChiDist} object.
  \end{tabb}

%%%%%%%%%%%%%%%%%%%%%%%%%%%%%%%%%%%
\subsubsection* {Methods}

\begin{code}\begin{hide}

   public double density (double x) {
       if (x <= 0.0)
         return 0.0;
      return Math.exp ((nu - 1)*Math.log (x) - x*x/2.0 - C1);
   }

   public double cdf (double x) {
      return cdf (nu, x);
   }

   public double barF (double x) {
      return barF (nu, x);
   }

   public double inverseF (double u) {
      return inverseF (nu, u);
   }

   public double getMean() {
      return ChiDist.getMean (nu);
   }

   public double getVariance() {
      return ChiDist.getVariance (nu);
   }

   public double getStandardDeviation() {
      return ChiDist.getStandardDeviation (nu);
   }\end{hide}

   public static double density (int nu, double x)\begin{hide} {
      if (nu <= 0)
         throw new IllegalArgumentException ("nu <= 0");
      if (x <= 0.0)
         return 0.0;
      return Math.exp ((nu - 1)*Math.log (x) - x*x/2.0
                         - (nu/2.0 - 1.0)*Num.LN2 - Num.lnGamma (nu/2.0));
   }\end{hide}
\end{code}
\begin{tabb} Computes the density function.
\end{tabb}
\begin{code}

   public static double cdf (int nu, double x)\begin{hide} {
      if (x <= 0.0)
         return 0.0;
      return GammaDist.cdf (nu/2.0, 15, x*x/2.0);
   }\end{hide}
\end{code}
 \begin{tabb}
  Computes the  distribution function by using the
  gamma distribution function.
 \end{tabb}
\begin{code}

   public static double barF (int nu, double x)\begin{hide} {
      if (x <= 0.0)
         return 1.0;
      return GammaDist.barF (nu/2.0, 15, x*x/2.0);
   }\end{hide}
\end{code}
  \begin{tabb}
  Computes the complementary distribution.
 \end{tabb}
\begin{code}

   public static double inverseF (int nu, double u)\begin{hide} {
       double res =  GammaDist.inverseF (nu/2.0, 15, u);
       return Math.sqrt (2*res);
   }\end{hide}
\end{code}
\begin{tabb}  Returns the inverse distribution function computed
  using the gamma inversion.
\end{tabb}
\begin{code}

   public static double[] getMLE (double[] x, int n)\begin{hide} {
      double[] parameters = new double[1];

      double mean = 0.0;
      for (int i = 0; i < n; i++)
         mean += x[i];
      mean /= (double) n;

      double var = 0.0;
      for (int i = 0; i < n; i++)
         var += ((x[i] - mean) * (x[i] - mean));
      var /= (double) n;

      double k = Math.round (var + mean * mean) - 5.0;
      if (k < 1.0)
         k = 1.0;

      double sum = 0.0;
      for (int i = 0; i < n; i++) {
         if (x[i] > 0.0)
            sum += Math.log (x[i]);
         else
            sum -= 709.0;
      }

      Function f = new Function (sum, n);
      while (f.evaluate(k) > 0.0)
         k++;
      parameters[0] = k;

      return parameters;
   }\end{hide}
\end{code}
\begin{tabb}
   Estimates the parameter $\nu$ of the chi distribution
   using the maximum likelihood method, from the $n$ observations
   $x[i]$, $i = 0, 1, \ldots, n-1$. The estimate is returned in element 0
   of the returned array.
\end{tabb}
\begin{htmlonly}
   \param{x}{the list of observations to use to evaluate parameters}
   \param{n}{the number of observations to use to evaluate parameters}
   \return{returns the parameter [$\hat{\nu}$]}
\end{htmlonly}
\begin{code}

   public static ChiDist getInstanceFromMLE (double[] x, int n)\begin{hide} {
      double parameters[] = getMLE (x, n);
      return new ChiDist ((int) parameters[0]);
   }\end{hide}
\end{code}
\begin{tabb}
   Creates a new instance of a chi distribution with parameter $\nu$ estimated using
   the maximum likelihood method based on the $n$ observations $x[i]$,
   $i = 0, 1, \ldots, n-1$.
\end{tabb}
\begin{htmlonly}
   \param{x}{the list of observations to use to evaluate parameters}
   \param{n}{the number of observations to use to evaluate parameters}
\end{htmlonly}
\begin{code}

   public static double getMean (int nu)\begin{hide} {
      if (nu <= 0)
         throw new IllegalArgumentException ("nu <= 0");
      return  Num.RAC2 * Num.gammaRatioHalf(nu / 2.0);
   }\end{hide}
\end{code}
\begin{tabb}  Computes and returns the mean
\begin{latexonly}
   $$E[X] =  \frac{\sqrt{2}\,\Gamma( \frac{\nu + 1}{2} )}{\Gamma(\frac{\nu}{2})}$$
\end{latexonly}
   of the chi distribution with parameter $\nu$.
\end{tabb}
\begin{htmlonly}
   \return{the mean of the chi distribution
    $E[X] = \sqrt{2}\Gamma((\nu + 1) / 2) / \Gamma(\nu / 2)$}
\end{htmlonly}
\begin{code}

   public static double getVariance (int nu)\begin{hide} {
      if (nu <= 0)
         throw new IllegalArgumentException ("nu <= 0");
      double mean = ChiDist.getMean(nu);
      return (nu - (mean * mean));
   }\end{hide}
\end{code}
\begin{tabb}  Computes and returns the variance
\begin{latexonly}
  $$\mbox{Var}[X] = \frac{2\,\Gamma(\frac{\nu}{2}) \Gamma(1 + \frac{\nu}{2}) -
                    \Gamma^2(\frac{\nu + 1}{2})}{\Gamma (\frac{\nu}{2})}$$
\end{latexonly}
   of the chi distribution with parameter $\nu$.
\end{tabb}
\begin{htmlonly}
   \return{the variance of the chi distribution
    $\mbox{Var}[X] = 2 [ \Gamma(\nu / 2) \Gamma(1 + \nu / 2) - \Gamma^2(1/2 (\nu + 1)) ]
     / \Gamma (\nu / 2)$}
\end{htmlonly}
\begin{code}

   public static double getStandardDeviation (int nu)\begin{hide} {
      return Math.sqrt (ChiDist.getVariance (nu));
   }\end{hide}
\end{code}
\begin{tabb}  Computes and returns the standard deviation of the chi distribution
   with parameter $\nu$.
\end{tabb}
\begin{htmlonly}
   \return{the standard deviation of the chi distribution}
\end{htmlonly}
\begin{code}

   public int getNu()\begin{hide} {
      return nu;
   }
\end{hide}
\end{code}
  \begin{tabb}
  Returns the value of $\nu$ for this object.
 \end{tabb}
\begin{code}

   public void setNu (int nu)\begin{hide} {
      if (nu <= 0)
         throw new IllegalArgumentException ("nu <= 0");
      this.nu = nu;
      supportA = 0.0;
      C1 = (nu/2.0 - 1.0)*Num.LN2 + Num.lnGamma (nu/2.0);
   }\end{hide}
\end{code}
  \begin{tabb}
  Sets the value of $\nu$ for this object.
 \end{tabb}
\begin{code}

   public double[] getParams ()\begin{hide} {
      double[] retour = {nu};
      return retour;
   }\end{hide}
\end{code}
\begin{tabb}
   Return a table containing parameters of the current distribution.
\end{tabb}
\begin{hide}\begin{code}

   public String toString ()\begin{hide} {
      return getClass().getSimpleName() + " : nu = " + nu;
   }\end{hide}
\end{code}
\begin{tabb}
   Returns a \texttt{String} containing information about the current distribution.
\end{tabb}\end{hide}
\begin{code}\begin{hide}
}\end{hide}
\end{code}

