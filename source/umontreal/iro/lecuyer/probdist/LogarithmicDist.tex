\defclass {LogarithmicDist}

Extends the class \class{DiscreteDistributionInt} for
the {\em logarithmic\/} distribution. It has shape parameter
$\theta$, where $0 < \theta <1$.
Its mass function is
\eq
    p(x) = \latex{\frac{-\theta^x} {x\log(1- \theta)}}
           \html{-\theta^x/(x\log(1- \theta)}
             \qquad \mbox{for } x = 1,2,3,\dots  \eqlabel{eq:flogar}
\endeq
Its distribution function is
\begin{htmlonly}
\eq
   F(x) = {-1/\log(1- \theta)}\sum_{i=1}^x {\theta^i}/{i}, &
      \mbox { for } x > 0.
\endeq
\end{htmlonly}
\begin{latexonly}
\eq
   F(x) =
     \frac{-1}{\log (1 - \theta)}\sum_{i=1}^x \frac{\theta^i}{i}, \qquad
      \mbox { for } x = 1, 2, 3, \ldots
\endeq
and  is   0 for $ x\le 0$.
\end{latexonly}

\bigskip\hrule

%%%%%%%%%%%%%%%%%%%%%%%%%%%%%%%%%%%%%%%%
\begin{code}
\begin{hide}
/*
 * Class:        LogarithmicDist
 * Description:  logarithmic distribution
 * Environment:  Java
 * Software:     SSJ 
 * Copyright (C) 2001  Pierre L'Ecuyer and Université de Montréal
 * Organization: DIRO, Université de Montréal
 * @author       
 * @since

 * SSJ is free software: you can redistribute it and/or modify it under
 * the terms of the GNU General Public License (GPL) as published by the
 * Free Software Foundation, either version 3 of the License, or
 * any later version.

 * SSJ is distributed in the hope that it will be useful,
 * but WITHOUT ANY WARRANTY; without even the implied warranty of
 * MERCHANTABILITY or FITNESS FOR A PARTICULAR PURPOSE.  See the
 * GNU General Public License for more details.

 * A copy of the GNU General Public License is available at
   <a href="http://www.gnu.org/licenses">GPL licence site</a>.
 */
\end{hide}
package umontreal.iro.lecuyer.probdist;
\begin{hide}
import umontreal.iro.lecuyer.util.Num;
import umontreal.iro.lecuyer.util.RootFinder;
import umontreal.iro.lecuyer.functions.MathFunction;
\end{hide}

public class LogarithmicDist extends DiscreteDistributionInt\begin{hide} {

   private double theta;
   private double t;

   private static class Function implements MathFunction {
      protected double mean;

      public Function (double mean) {
         this.mean = mean;
      }

      public double evaluate (double x) {
         if (x <= 0.0 || x >= 1.0) return 1.0e200;
         return (x + mean * (1.0 - x) * Math.log1p (-x));
      }
   }
\end{hide}\end{code}

%%%%%%%%%%%%%%%%%%%%%%%%%%%%%%%%%%%%%%%%%
\subsubsection* {Constructor}
\begin{code}

   public LogarithmicDist (double theta)\begin{hide} {
      setTheta (theta);
   }\end{hide}
\end{code}
  \begin{tabb} Constructs a logarithmic distribution with parameter $\theta = $
  \texttt{theta}.
 \end{tabb}

%%%%%%%%%%%%%%%%%%%%%%%%%%%%%%%%%%%
\subsubsection* {Methods}
\begin{code}\begin{hide}

   public double prob (int x) {
      if (x < 1)
         return 0;
      return t*Math.pow (theta, x)/x;
   }

   public double cdf (int x) {
      if (x < 1)
         return 0;
      double res = prob (1);
      double term = res;
      for (int i = 2; i <= x; i++) {
         term *= theta;
         res += term/i;
      }
      return res;
   }

   public double barF (int x) {
      if (x <= 1)
         return 1.0;
      double res = prob (x);
      double term = res;
      int i = x + 1;
      while (term > EPSILON) {
         term *= theta*(i-1)/i;
         res += term;
      }
      return res;
   }

   public int inverseFInt (double u) {
      return inverseF (theta, u);
   }

   public double getMean() {
      return LogarithmicDist.getMean (theta);
   }

   public double getVariance() {
      return LogarithmicDist.getVariance (theta);
   }

   public double getStandardDeviation() {
      return LogarithmicDist.getStandardDeviation (theta);
   }\end{hide}

   public static double prob (double theta, int x)\begin{hide} {
      if (theta <= 0 || theta >= 1)
         throw new IllegalArgumentException ("theta not in range (0,1)");
      if (x < 1)
         return 0;
      return -1.0/Math.log1p(-theta) * Math.pow (theta, x)/x;
   }\end{hide}
\end{code}
 \begin{tabb}
   Computes the logarithmic probability $p(x)$%
\latex{ given in (\ref{eq:flogar}) }.
 \end{tabb}
\begin{code}

   public static double cdf (double theta, int x)\begin{hide} {
      if (theta <= 0 || theta >= 1)
         throw new IllegalArgumentException ("theta not in range (0,1)");
      if (x < 1)
         return 0;
      double res = prob (theta, 1);
      double term = res;
      for (int i = 2; i <= x; i++) {
         term *= theta;
         res += term/i;
      }
      return res;
   }\end{hide}
\end{code}
  \begin{tabb} Computes the distribution function $F(x)$.
  \end{tabb}
\begin{code}

   public static double barF (double theta, int x)\begin{hide} {
      if (theta <= 0 || theta >= 1)
         throw new IllegalArgumentException ("theta not in range (0,1)");
      if (x <= 1)
         return 1.0;
      double res = prob (theta, x);
      double term = res;
      int i = x + 1;
      while (term > EPSILON) {
         term *= theta*(i-1)/i;
         res += term;
      }
      return res;
   }\end{hide}
\end{code}
  \begin{tabb} Computes the complementary distribution function.
\emph{WARNING:} The complementary distribution function is defined as 
$\bar F(x) = P[X \ge x]$.
 \end{tabb}
\begin{code}\begin{hide}

   public static int inverseF (double theta, double u) {
      throw new UnsupportedOperationException();
   }\end{hide}

   public static double[] getMLE (int[] x, int n)\begin{hide} {
      if (n <= 0)
         throw new IllegalArgumentException ("n <= 0");

      double parameters[];
      parameters = new double[1];
      double sum = 0.0;
      for (int i = 0; i < n; i++) {
         sum += x[i];
      }

      double mean = (double) sum / (double) n;

      Function f = new Function (mean);
      parameters[0] = RootFinder.brentDekker (1e-15, 1.0-1e-15, f, 1e-7);

      return parameters;
   }\end{hide}
\end{code}
\begin{tabb}
   Estimates the parameter $\theta$ of the logarithmic distribution
   using the maximum likelihood method, from the $n$ observations 
   $x[i]$, $i = 0, 1, \ldots, n-1$. The estimate is returned in element 0
   of the returned array. 
   \begin{detailed}
   The maximum likelihood estimator $\hat{\theta}$ satisfies the equation
   (see \cite[page 122]{mEVA00a})
   \begin{eqnarray*}
      \bar{x}_n = \frac{-\hat{\theta}}{(1 - \hat{\theta}) \ln(1 - \hat{\theta})}
   \end{eqnarray*}
   where  $\bar{x}_n$ is the average of $x[0], \ldots, x[n-1]$.
   \end{detailed}
\end{tabb}
\begin{htmlonly}
   \param{x}{the list of observations used to evaluate parameters}
   \param{n}{the number of observations used to evaluate parameters}
   \return{returns the parameter [$\hat{\theta}$]}
\end{htmlonly}
\begin{code}

   public static LogarithmicDist getInstanceFromMLE (int[] x, int n)\begin{hide} {
      double parameters[] = getMLE (x, n);
      return new LogarithmicDist (parameters[0]);
   }\end{hide}
\end{code}
\begin{tabb}
   Creates a new instance of a logarithmic distribution with parameter
   $\theta$ estimated using the maximum likelihood method based on the $n$
   observations $x[i]$, $i = 0, 1, \ldots, n-1$.
\end{tabb}
\begin{htmlonly}
   \param{x}{the list of observations to use to evaluate parameters}
   \param{n}{the number of observations to use to evaluate parameters}
\end{htmlonly}
\begin{code}

   public static double getMean (double theta)\begin{hide} {
      if (theta <= 0.0 || theta >= 1.0)
         throw new IllegalArgumentException ("theta not in range (0,1)");

      return ((-1 / Math.log1p(-theta)) * (theta / (1 - theta)));
   }\end{hide}
\end{code}
\begin{tabb}  Computes and returns the mean
\begin{latexonly}
   $$E[X] = \frac{-\theta}{(1 - \theta)\ln(1 - \theta)}$$
\end{latexonly}
   of the logarithmic distribution with parameter $\theta = $ \texttt{theta}.
\end{tabb}
\begin{htmlonly}
   \return{the mean of the logarithmic distribution
    $E[X] = -\theta / ((1 - \theta)  ln(1 - \theta))$}
\end{htmlonly}
\begin{code}

   public static double getVariance (double theta)\begin{hide} {
      if (theta <= 0.0 || theta >= 1.0)
         throw new IllegalArgumentException ("theta not in range (0,1)");

      double v = Math.log1p(-theta);
      return ((-theta * (theta + v)) / ((1 - theta) * (1 - theta) * v * v));
   }\end{hide}
\end{code}
\begin{tabb}  Computes and returns the variance
\begin{latexonly}
  $$\mbox{Var}[X] = \frac{-\theta (\theta + \ln(1 - \theta))}{[(1 - \theta)
    \ln(1 - \theta)]^2}$$
\end{latexonly}
   of the logarithmic distribution with parameter $\theta =$ \texttt{theta}.
\end{tabb}
\begin{htmlonly}
   \return{the variance of the logarithmic distribution
    $\mbox{Var}[X] = -\theta (\theta + ln(1 - \theta)) / ((1 - \theta)^2
(ln(1 - \theta))^2)$}
\end{htmlonly}
\begin{code}

   public static double getStandardDeviation (double theta)\begin{hide} {
      return Math.sqrt (LogarithmicDist.getVariance (theta));
   }\end{hide}
\end{code}
\begin{tabb}  Computes and returns the standard deviation of the
   logarithmic distribution with parameter $\theta = $ \texttt{theta}.
\end{tabb}
\begin{htmlonly}
   \return{the standard deviation of the logarithmic distribution}
\end{htmlonly}
\begin{code}

   public double getTheta()\begin{hide} {
      return theta;
   }
\end{hide}
\end{code}
\begin{tabb}
   Returns the $\theta$ associated with this object.
\end{tabb}
%% \pierre{Add \texttt{setTheta} method here, and similar
%% \texttt{setParams} methods
%%   for all classes that follow.  In most cases, the code in the constructor
%%   should be moved to \texttt{setParams} and the constructor should call it. }
\begin{code}

   public void setTheta (double theta)\begin{hide} {
      if (theta <= 0 || theta >= 1)
         throw new IllegalArgumentException ("theta not in range (0,1)");
      this.theta = theta;
      t = -1.0/Math.log1p (-theta);
      supportA = 1;
   }\end{hide}
\end{code}
\begin{tabb}
   Sets the $\theta$ associated with this object.
\end{tabb}
\begin{code}

   public double[] getParams ()\begin{hide} {
      double[] retour = {theta};
      return retour;
   }\end{hide}
\end{code}
\begin{tabb}
   Return a table containing the parameters of the current distribution.
\end{tabb}
\begin{hide}\begin{code}

   public String toString ()\begin{hide} {
      return getClass().getSimpleName() + " : theta = " + theta;
   }\end{hide}
\end{code}
\begin{tabb}
   Returns a \texttt{String} containing information about the current distribution.
\end{tabb}\end{hide}
\begin{code}\begin{hide}
}\end{hide}
\end{code}
