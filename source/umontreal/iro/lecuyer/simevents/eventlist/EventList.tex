\defclass {EventList}

An interface for implementations of event lists.
Different implementations are provided in SSJ:
doubly-linked list, splay tree, Henricksen's method, etc.
The {\em events\/} in the event list are objects of the class
\externalclass{umontreal.iro.lecuyer.simevents}{Event}.
The method \clsexternalmethod{umontreal.iro.lecuyer.simevents}{Sim}{init}{EventList}
permits one to select
the actual implementation used in a simulation \cite{sKIN85a}.

To allow the user to print the event list, the 
\externalmethod{java.lang}{Object}{toString}{} method
from the \class{Object} class should be reimplemented in all {\tt EventList}
implementations.  It will return a string in the following format:
``{\tt Contents of the event list }\emph{event list class}{\tt :}'' for the first line and
each subsequent line has format 
``\emph{scheduled event time}{\tt , }\emph{event priority}~{\tt :}~\emph{event string}''.
The \emph{event string} is obtained by calling the
{\tt toString} method of the event objects.
The string should not end with the end-of-line character.

The following example is the event list of the bank example,
printed at 10h30.  See {\tt examples.pdf} for more information.

\begin{verbatim}
Contents of the event list SplayTree:
   10.51,        1 : BankEv$Arrival@cfb549
   10.54,        1 : BankEv$Departure@8a7efd
      11,        1 : BankEv$3@86d4c1
      14,        1 : BankEv$4@f9f9d8
      15,        1 : BankEv$5@820dda
\end{verbatim}
%$  Without that, Emacs believes we are in math mode.

% The event time (obtained by calling the
% \clsexternalmethod{}{Event}{time}{}) must not be modified
% by implementations of this interface because the \class{Event} class
% already takes care of that.

\bigskip\hrule

%%%%%%%%%%%%%%%%%%%%%%%%%%%%%%%%%%%%%%%%%%%%%%%%%%%%%%%%%%%%%%%%%%
\begin{code}
\begin{hide}
/*
 * Class:        EventList
 * Description:  interface for implementations of event lists
 * Environment:  Java
 * Software:     SSJ 
 * Copyright (C) 2001  Pierre L'Ecuyer and Université de Montréal
 * Organization: DIRO, Université de Montréal
 * @author       
 * @since

 * SSJ is free software: you can redistribute it and/or modify it under
 * the terms of the GNU General Public License (GPL) as published by the
 * Free Software Foundation, either version 3 of the License, or
 * any later version.

 * SSJ is distributed in the hope that it will be useful,
 * but WITHOUT ANY WARRANTY; without even the implied warranty of
 * MERCHANTABILITY or FITNESS FOR A PARTICULAR PURPOSE.  See the
 * GNU General Public License for more details.

 * A copy of the GNU General Public License is available at
   <a href="http://www.gnu.org/licenses">GPL licence site</a>.
 */
\end{hide}
package umontreal.iro.lecuyer.simevents.eventlist; \begin{hide}

import java.util.ListIterator; 
import umontreal.iro.lecuyer.simevents.Event;
\end{hide}

public interface EventList extends Iterable<Event>\begin{hide} {
\end{hide}
 
   public boolean isEmpty();
\end{code}
 \begin{tabb}  Returns {\tt true} if and only if the event list is empty
   (no event is scheduled).
 \end{tabb}
\begin{htmlonly}
   \return{{\tt true} if the event list is empty}
\end{htmlonly}
\begin{code}

   public void clear();
\end{code}
 \begin{tabb}  Empties the event list, i.e., cancels all events.
 \end{tabb}
\begin{code}

   public void add (Event ev);
\end{code}
 \begin{tabb}  Adds a new event in the event list, according to
  the time of {\tt ev}.
  If the event list contains events scheduled to happen at the same time as
  {\tt ev}, {\tt ev} must be added after all these events.
 \end{tabb}
\begin{htmlonly}
   \param{ev}{event to be added}
\end{htmlonly}
\begin{code}

   public void addFirst (Event ev);
\end{code}
 \begin{tabb}  Adds a new event at the beginning of the event list.  The given
   event {\tt ev} will occur at the current simulation time.
 \end{tabb}
\begin{htmlonly}
   \param{ev}{event to be added}
\end{htmlonly}
\begin{code}

   public void addBefore (Event ev, Event other);
\end{code}
 \begin{tabb}  Same as \method{add}{}, but adds the new event {\tt ev}
  immediately before the event {\tt other} in the list.
 \end{tabb}
\begin{htmlonly}
   \param{ev}{event to be added}
   \param{other}{reference event before which {\tt ev} will be added}
\end{htmlonly}
\begin{code}

   public void addAfter (Event ev, Event other);
\end{code}
 \begin{tabb}  Same as \method{add}{}, but adds the new event {\tt ev}
  immediately after the event {\tt other} in the list.
 \end{tabb}
\begin{htmlonly}
   \param{ev}{event to be added}
   \param{other}{reference event after which {\tt ev} will be added}
\end{htmlonly}
\begin{code}

   public Event getFirst();
\end{code}
 \begin{tabb} Returns the first event in the event list.
 If the event list is empty, returns {\tt null}.
 \end{tabb}
\begin{htmlonly}
   \return{the first event in the event list, or {\tt null} if the list is empty}
\end{htmlonly}
\begin{code}

   public Event getFirstOfClass (String cl);
\end{code}
 \begin{tabb} Returns the first event of the class {\tt cl} (a subclass of
  {\tt Event}) in the event list.  If no such event is found, returns
  {\tt null}.
 \end{tabb}
\begin{htmlonly}
   \return{the first event of class {\tt cl}, or {\tt null} if no such event exists in the list}
\end{htmlonly}
\begin{code}

   public <E extends Event> E getFirstOfClass (Class<E> cl);
\end{code}
 \begin{tabb} Returns the first event of the class {\tt E} (a subclass of
  {\tt Event}) in the event list. If no such event is found, returns
  {\tt null}.
 \end{tabb}
\begin{htmlonly}
   \return{the first event of class {\tt cl}, or {\tt null} if no such event exists in the list}
\end{htmlonly}
\begin{code}

   public ListIterator<Event> listIterator();
\end{code}
\begin{tabb}
   Returns a list iterator over the elements of the class {\tt Event} in this list.
\end{tabb}
\begin{htmlonly}
   \return{a list iterator over the elements of the class {\tt Event} in this list}
\end{htmlonly}
\begin{code}

   public boolean remove (Event ev);
\end{code}
 \begin{tabb}  Removes the event {\tt ev} from the event list (cancels this event).
 Returns {\tt true} if and only if the event removal has succeeded.
 \end{tabb}
\begin{htmlonly}
   \param{ev}{event to be removed}
   \return{{\tt true} if the event was successfully removed from the list}
\end{htmlonly}
\begin{code}

   public Event removeFirst();
\end{code}
 \begin{tabb}  Removes the first event from the event list (to cancel or
    execute this event).  Returns the removed event.  If the list is empty,
    then {\tt null} is returned.
 \end{tabb}
\begin{htmlonly}
   \return{the first event removed from the list, or {\tt null} if the list is empty}
\end{htmlonly}
\begin{code}\begin{hide}
}\end{hide}
\end{code}
