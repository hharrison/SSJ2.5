\defclass {RandomVariateGen}

This is the base class for all random variate generators over the real line.
It specifies the signature of the \method{nextDouble}{} method, which is
normally called to generate a real-valued random variate whose distribution
has been previously selected.
A random variate generator object can be created simply by invoking the
constructor of this class with previously created
\externalclass{umontreal.iro.lecuyer.rng}{RandomStream} and
\externalclass{umontreal.iro.lecuyer.probdist}{Distribution}
objects, or by invoking the constructor of a subclass.
By default, all random variates will be generated via
inversion by calling the
\externalmethod{umontreal.iro.lecuyer.probdist}{Distribution}{inverseF}{}
method for the distribution, even though this can be inefficient in some cases.
For some of the distributions, there are subclasses with special and
more efficient methods to generate the random variates.

%% The generation algorithm can be changed for a given distribution object
%% by calling an appropriate method on it.
%% Each distribution also has a static method for generating random variates,
%% which also uses inversion by default, unless specified otherwise.
%% The generating algorithm can be changed by invoking an appropriate static
%% method.  To avoid side effects, it is recommended to set the static algorithm
%% once and keep it for the duration of the program.
%% \pierre{Another good reason for using subclasses instead!}
%% If one wants to generate random variates with more than one algorithms,
%% one should create separate objects.

For generating many random variates, creating an object and calling
the non-static method is more efficient when the generating algorithm
involves a significant setup.
When no work is done at setup time, the static methods are usually
slightly faster.


\bigskip\hrule

\begin{code}
\begin{hide}
/*
 * Class:        RandomVariateGen
 * Description:  base class for all random variate generators over the reals 
 * Environment:  Java
 * Software:     SSJ 
 * Copyright (C) 2001  Pierre L'Ecuyer and Université de Montréal
 * Organization: DIRO, Université de Montréal
 * @author       
 * @since

 * SSJ is free software: you can redistribute it and/or modify it under
 * the terms of the GNU General Public License (GPL) as published by the
 * Free Software Foundation, either version 3 of the License, or
 * any later version.

 * SSJ is distributed in the hope that it will be useful,
 * but WITHOUT ANY WARRANTY; without even the implied warranty of
 * MERCHANTABILITY or FITNESS FOR A PARTICULAR PURPOSE.  See the
 * GNU General Public License for more details.

 * A copy of the GNU General Public License is available at
   <a href="http://www.gnu.org/licenses">GPL licence site</a>.
 */
\end{hide}
package umontreal.iro.lecuyer.randvar;\begin{hide}
import umontreal.iro.lecuyer.rng.RandomStream;
import umontreal.iro.lecuyer.probdist.Distribution;\end{hide}

public class RandomVariateGen \begin{hide} {

   protected RandomStream stream;
   // the stream used for generating random variates

   protected Distribution dist;
   // the distribution used by this generator

\end{hide}
\end{code}

%%%%%%%%%%%%%%%%%%%%%%%%%%%%%%%%%%%%%%%%%%%%%
\subsubsection* {Constructor}
\begin{code}\begin{hide}
 // This constructor is needed for subclasses with no associated distribution.
   protected RandomVariateGen() {}\end{hide}

   public RandomVariateGen (RandomStream s, Distribution dist) \begin{hide} {
      this.stream = s;
      this.dist   = dist;
   }\end{hide}
\end{code}
  \begin{tabb}  Creates a new random variate generator from the
    distribution \texttt{dist}, using stream \texttt{s}.
 \end{tabb}
\begin{htmlonly}
   \param{s}{random stream used for generating uniforms}
   \param{dist}{continuous distribution object of the generated values}
\end{htmlonly}

%%%%%%%%%%%%%%%%%%%%%%%%%%%%%%%%%%%%%%%%%%%%%%
\subsubsection* {Methods}
\begin{code}

   public double nextDouble() \begin{hide} {
      return dist.inverseF (stream.nextDouble());
   }\end{hide}
\end{code}
  \begin{tabb} Generates a random number from the continuous distribution
    contained in this object.
    By default, this method uses inversion by calling the
    \externalmethod{umontreal.iro.lecuyer.probdist}{ContinuousDistribution}{inverseF}{}
    method of the distribution object.
    Alternative generating methods are provided in subclasses.
 \end{tabb}
\begin{htmlonly}
   \return{the generated value}
\end{htmlonly}
\begin{code}

   public void nextArrayOfDouble (double[] v, int start, int n) \begin{hide} {
      if (n <= 0)
         throw new IllegalArgumentException ("n must be positive.");
      for (int i = 0; i < n; i++)
         v[start + i] = nextDouble();
   }\end{hide}
\end{code}
\begin{tabb}
   Generates \texttt{n} random numbers from the continuous distribution
   contained in this object.  These numbers are stored in the array \texttt{v},
   starting from index \texttt{start}.
   By default, this method calls \method{nextDouble()}{} \texttt{n}
   times, but one can override it in subclasses for better efficiency.
 \end{tabb}
\begin{htmlonly}
   \param{v}{array in which the variates will be stored}
   \param{start}{starting index, in \texttt{v}, of the new variates}
   \param{n}{number of variates to generate}
\end{htmlonly}
\begin{code}

   public RandomStream getStream() \begin{hide} { return stream; }
\end{hide}
\end{code}
  \begin{tabb}  Returns the \class{RandomStream} used by this generator.
 \end{tabb}
\begin{htmlonly}
   \return{the stream associated to this object}
\end{htmlonly}
\begin{code}

   public void setStream (RandomStream stream)\begin{hide} {
      this.stream = stream;
   }\end{hide}
\end{code}
\begin{tabb}
   Sets the \class{RandomStream} used by this generator to \texttt{stream}.
\end{tabb}
\begin{code}

   public Distribution getDistribution() \begin{hide} {
      return dist;
   }\end{hide}
\end{code}
\begin{tabb}
  Returns the \class{Distribution} used by this generator.
\end{tabb}
\begin{htmlonly}
   \return{the distribution associated to that object}
\end{htmlonly}
\begin{code}

   public String toString ()\begin{hide} {
      if (dist != null)
         return getClass().getSimpleName() + " with  " + dist.toString();
      else
         return getClass().getSimpleName() ;
   }\end{hide}
\end{code}
\begin{tabb}
   Returns a \texttt{String} containing information about the current generator.
\end{tabb}

\begin{code}\begin{hide}
}\end{hide}
\end{code}
