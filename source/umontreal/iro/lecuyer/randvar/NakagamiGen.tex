\defclass {NakagamiGen}

This class implements random variate generators for the {\em Nakagami\/} 
distribution. See the definition in 
\externalclass{umontreal.iro.lecuyer.probdist}{NakagamiDist} of package
\texttt{probdist}.

\bigskip\hrule

\begin{code}
\begin{hide}
/*
 * Class:        NakagamiGen
 * Description:  random variate generators for the Nakagami distribution.
 * Environment:  Java
 * Software:     SSJ 
 * Copyright (C) 2001  Pierre L'Ecuyer and Université de Montréal
 * Organization: DIRO, Université de Montréal
 * @author       
 * @since

 * SSJ is free software: you can redistribute it and/or modify it under
 * the terms of the GNU General Public License (GPL) as published by the
 * Free Software Foundation, either version 3 of the License, or
 * any later version.

 * SSJ is distributed in the hope that it will be useful,
 * but WITHOUT ANY WARRANTY; without even the implied warranty of
 * MERCHANTABILITY or FITNESS FOR A PARTICULAR PURPOSE.  See the
 * GNU General Public License for more details.

 * A copy of the GNU General Public License is available at
   <a href="http://www.gnu.org/licenses">GPL licence site</a>.
 */
\end{hide}
package umontreal.iro.lecuyer.randvar;\begin{hide}
import umontreal.iro.lecuyer.rng.*;
import umontreal.iro.lecuyer.probdist.*;
\end{hide}

public class NakagamiGen extends RandomVariateGen \begin{hide} {
   // Distribution parameters
   protected double a;
   protected double lambda;
   protected double c;
\end{hide}
\end{code}

%%%%%%%%%%%%%%%%%%%%%%%%%%%%%%%%%%%%%%%%%%%%%%%%
\subsubsection* {Constructors}
\begin{code}

   public NakagamiGen (RandomStream s, double a, double lambda, double c) \begin{hide} {
      super (s, new NakagamiDist (a, lambda, c));
      setParams (a, lambda, c);
   }\end{hide}
\end{code}
  \begin{tabb}  Creates a new Nakagami generator with parameters $a=$ \texttt{a}, 
 $\lambda =$ \texttt{lambda} and $c =$ \texttt{c}, using stream \texttt{s}.
\end{tabb}
\begin{code}

   public NakagamiGen (RandomStream s, NakagamiDist dist) \begin{hide} {
      super (s, dist);
      if (dist != null)
         setParams (dist.getA(), dist.getLambda(), dist.getC());
   }\end{hide}
\end{code}
\begin{tabb} Creates a new generator for the distribution \texttt{dist},
     using stream \texttt{s}.
\end{tabb}


%%%%%%%%%%%%%%%%%%%%%%%%%%%%%%%%%%%%%%%%%%%%%%%%5
\subsubsection* {Methods}
\begin{code}

   public static double nextDouble (RandomStream s, double a, double lambda,
                                    double c)\begin{hide} {
      return NakagamiDist.inverseF (a, lambda, c, s.nextDouble());
   }\end{hide}
\end{code}
\begin{tabb} Generates a variate from the {\em Nakagami\/} distribution with
 parameters $a=$ \texttt{a}, 
 $\lambda =$ \texttt{lambda} and $c =$ \texttt{c}, using stream \texttt{s}.
\end{tabb}
\begin{htmlonly}
   \param{s}{the random stream}
   \param{a}{the location parameter}
   \param{lambda}{the scale parameter}
   \param{c}{the shape parameter}
   \return{Generates a variate from the {\em Nakagami\/} distribution}
\end{htmlonly}
\begin{code}

   public double getA()\begin{hide} {
      return a;
   }\end{hide}
\end{code}
  \begin{tabb} Returns the location parameter $a$ of this object.
  \end{tabb}
\begin{htmlonly}
   \return{the location parameter mu}
\end{htmlonly}
\begin{code}

   public double getLambda()\begin{hide} {
      return lambda;
   }\end{hide}
\end{code}
  \begin{tabb} Returns the scale parameter $\lambda$ of this object.
  \end{tabb}
\begin{htmlonly}
   \return{the scale parameter mu}
\end{htmlonly}
\begin{code}

   public double getC()\begin{hide} {
      return c;
   }\end{hide}
\end{code}
  \begin{tabb} Returns the shape parameter $c$ of this object.
  \end{tabb}
\begin{htmlonly}
   \return{the shape parameter mu}
\end{htmlonly}
\begin{code}\begin{hide}

   protected void setParams (double a, double lambda, double c) {
      if (lambda <= 0.0)
         throw new IllegalArgumentException ("lambda <= 0");
      if (c <= 0.0)
         throw new IllegalArgumentException ("c <= 0");
      this.a = a;
      this.lambda = lambda;
      this.c = c;
   }
}\end{hide}
\end{code}
