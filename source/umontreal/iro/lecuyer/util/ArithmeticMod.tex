\defmodule {ArithmeticMod}

This class provides facilities to compute multiplications of scalars, of
vectors and of matrices modulo m. All algorithms are present in three 
different versions. These allow operations on \texttt{double}, \texttt{int} and
\texttt{long}. The \texttt{int} and \texttt{long} versions work exactly like the 
\texttt{double} ones.

%%%%%%%%%%%%%%%%%%%%%%%%%%%%%%%%%%%%%%%%%%%
\bigskip\hrule

\begin{code}\begin{hide}
/*
 * Class:        ArithmeticMod
 * Description:  multiplications of scalars, vectors and matrices modulo m
 * Environment:  Java
 * Software:     SSJ 
 * Copyright (C) 2001  Pierre L'Ecuyer and Université de Montréal
 * Organization: DIRO, Université de Montréal
 * @author       
 * @since

 * SSJ is free software: you can redistribute it and/or modify it under
 * the terms of the GNU General Public License (GPL) as published by the
 * Free Software Foundation, either version 3 of the License, or
 * any later version.

 * SSJ is distributed in the hope that it will be useful,
 * but WITHOUT ANY WARRANTY; without even the implied warranty of
 * MERCHANTABILITY or FITNESS FOR A PARTICULAR PURPOSE.  See the
 * GNU General Public License for more details.

 * A copy of the GNU General Public License is available at
   <a href="http://www.gnu.org/licenses">GPL licence site</a>.
 */
\end{hide}
package umontreal.iro.lecuyer.util;


public class ArithmeticMod \begin{hide} {

   //private constants
   private static final double two17    =  131072.0;
   private static final double two53    =  9007199254740992.0;

   //prevent the creation of the object
   private ArithmeticMod() {};

 \end{hide}
\end{code}

%%%%%%%%%%%%%%%%%%%%%%%%%%%%%%%%%%%%
\subsubsection* {Methods using \texttt{double}}

\begin{code}

   public static double multModM (double a, double s, double c, double m) \begin{hide} {
      int a1;
      double v = a * s + c;
      if (v >= two53 || v <= -two53 ) {
         a1 = (int)(a / two17);
         a -= a1 * two17;
         v  = a1 * s;
         a1 = (int)(v / m);
         v -= a1 * m;
         v  = v * two17 + a * s + c;
      }
      a1 = (int)(v / m);
      if ((v -= a1 * m) < 0.0)
         return v += m;
      else
         return v;
   } \end{hide}
\end{code}
\begin{tabb} Computes $(a \times s + c) \bmod m$. Where \texttt{m} must 
 be smaller than $2^{35}$. Works also if \texttt{s} or \texttt{c} are negative.
 The result is always positive (and thus always between 0 and \texttt{m} - 1).
\end{tabb}
\begin{htmlonly}
  \param{a}{the first factor of the multiplication}
  \param{s}{the second factor of the multiplication}
  \param{c}{the second term of the addition}
  \param{m}{the modulus}
  \return{the result of the multiplication and the addition modulo \texttt{m}}
\end{htmlonly}
\begin{code}

   public static void matVecModM (double A[][], double s[], double v[],
                                  double m) \begin{hide} {
      int i;
      double x[] = new double[v.length];
      for (i = 0; i < v.length;  ++i) {
         x[i] = 0.0;
         for(int j = 0; j < s.length; j++)
            x[i] = multModM (A[i][j], s[j], x[i], m);
      }
      for (i = 0; i < v.length;  ++i)
         v[i] = x[i];
   } \end{hide}
\end{code}
\begin{tabb} Computes the result of $\mathtt{A} \times \mathtt{\mathbf{s}}
  \bmod m$ and puts the
  result in \texttt{v}. Where \texttt{s} and \texttt{v} are both column vectors. This 
  method works even if \texttt{s} = \texttt{v}.
\end{tabb}
\begin{htmlonly}
  \param{A}{the multiplication matrix}
  \param{s}{the multiplied vector}
  \param{v}{the result of the multiplication}
  \param{m}{the modulus}
\end{htmlonly}
\begin{code}

   public static void matMatModM (double A[][], double B[][], double C[][],
                                  double m) \begin{hide} {
      int i, j;
      int r = C.length;    //# of rows of C
      int c = C[0].length; //# of columns of C
      double V[] = new double[r];
      double W[][] = new double[r][c];
      for (i = 0; i < c;  ++i) {
         for (j = 0; j < r;  ++j)
            V[j] = B[j][i];
         matVecModM (A, V, V, m);
         for (j = 0; j < r;  ++j)
            W[j][i] = V[j];
      }
      for (i = 0; i < r;  ++i) {
         for (j = 0; j < c;  ++j)
            C[i][j] = W[i][j];
      }
   } \end{hide}
\end{code}
\begin{tabb} Computes $\mathtt{A} \times \mathtt{B} \bmod m$ 
  and puts the result in
  \texttt{C}. Works even if \texttt{A} = \texttt{C}, \texttt{B} = \texttt{C} or
  \texttt{A} = \texttt{B} = \texttt{C}.
\end{tabb}
\begin{htmlonly}
  \param{A}{the first factor of the multiplication}
  \param{B}{the second factor of the multiplication}
  \param{C}{the result of the multiplication}
  \param{m}{the modulus}
\end{htmlonly}
\begin{code}  

   public static void matTwoPowModM (double A[][], double B[][], double m,
                                     int e) \begin{hide} {
      int i, j;
      /* initialize: B = A */
      if (A != B) {
         for (i = 0; i < A.length; i++) {
            for (j = 0; j < A.length;  ++j)  //A is square
               B[i][j] = A[i][j];
         }
      }
      /* Compute B = A^{2^e} */
      for (i = 0; i < e; i++)
         matMatModM (B, B, B, m);
   } \end{hide}
\end{code}
\begin{tabb} Computes $\mathtt{A}^{2^{\mathtt{e}}} \bmod m$ and 
  puts the result in \texttt{B}.
  Works even if \texttt{A} = \texttt{B}.
\end{tabb}
\begin{htmlonly}
  \param{A}{the matrix to raise to a power}
  \param{B}{the result of exponentiation}
  \param{m}{the modulus}
  \param{e}{the $\log_{2}$ of the exponent}
\end{htmlonly}
\begin{code}

   public static void matPowModM (double A[][], double B[][], double m,
                                  int c) \begin{hide} {
      int i, j;
      int n = c;
      int s = A.length;   //we suppose that A is square
      double W[][] = new double[s][s];

      /* initialize: W = A; B = I */
      for (i = 0; i < s; i++) {
         for (j = 0; j < s;  ++j)  {
            W[i][j] = A[i][j];
            B[i][j] = 0.0;
         }
      }
      for (j = 0; j < s;  ++j)
         B[j][j] = 1.0;

      /* Compute B = A^c mod m using the binary decomp. of c */
      while (n > 0) {
         if ((n % 2)==1)
            matMatModM (W, B, B, m);
         matMatModM (W, W, W, m);
         n /= 2;
      }
   } \end{hide}
\end{code}
\begin{tabb} Computes $\mathtt{A}^{c} \bmod m$ 
  and puts the result in \texttt{B}.
  Works even if \texttt{A} = \texttt{B}.
\end{tabb}
\begin{htmlonly}
  \param{A}{the matrix to raise to a power}
  \param{B}{the result of the exponentiation}
  \param{m}{the modulus}
  \param{c}{the exponent}
\end{htmlonly}


%%%%%%%%%%%%%%%%%%%%%%%%%%%%%%%%%

\subsubsection* {Methods using \texttt{int}}
\begin{code}

   public static int multModM (int a, int s, int c, int m) \begin{hide} {
      int r = (int) (((long)a * s + c) % m);
      return r < 0 ? r + m : r;
   } \end{hide}
\end{code}
\begin{tabb} Computes $(a \times s + c) 
  \bmod m$. Works also if \texttt{s} 
  or \texttt{c} are negative.
  The result is always positive (and thus always between 0 and \texttt{m} - 1).
\end{tabb}
\begin{htmlonly}
  \param{a}{the first factor of the multiplication}
  \param{s}{the second factor of the multiplication}
  \param{c}{the second term of the addition}
  \param{m}{the modulus}
  \return{the result of the multiplication and the addition modulo \texttt{m}}
\end{htmlonly}
\begin{code}

   public static void matVecModM (int A[][], int s[], int v[], int m) \begin{hide} {
      int i;
      int x[] = new int[v.length];
      for (i = 0; i < v.length;  ++i) {
         x[i] = 0;
         for(int j = 0; j < s.length; j++)
            x[i] = multModM(A[i][j], s[j], x[i], m);
      }
      for (i = 0; i < v.length;  ++i)
         v[i] = x[i];

   } \end{hide}
\end{code}
\begin{tabb} Exactly like \method{matVecModM}{double[][], double[], 
    double[], double} using \texttt{double}, but with \texttt{int} instead 
  of \texttt{double}.
\end{tabb}
\begin{htmlonly}
  \param{A}{the multiplication matrix}
  \param{s}{the multiplied vector}
  \param{v}{the result of the multiplication}
  \param{m}{the modulus}
\end{htmlonly}
\begin{code}

   public static void matMatModM (int A[][], int B[][], int C[][], int m) \begin{hide} {
      int i, j;
      int r = C.length;    //# of rows of C
      int c = C[0].length; //# of columns of C
      int V[] = new int[r];
      int W[][] = new int[r][c];
      for (i = 0; i < c;  ++i) {
         for (j = 0; j < r;  ++j)
            V[j] = B[j][i];
         matVecModM (A, V, V, m);
         for (j = 0; j < r;  ++j)
            W[j][i] = V[j];
      }
      for (i = 0; i < r;  ++i) {
         for (j = 0; j < c;  ++j)
            C[i][j] = W[i][j];
      }
   } \end{hide}
\end{code}
\begin{tabb} Exactly like \method{matMatModM}{double[][], double[][],
    double[][], double} using \texttt{double}, but with \texttt{int} instead
  of \texttt{double}.
\end{tabb}
\begin{htmlonly}
  \param{A}{the first factor of the multiplication}
  \param{B}{the second factor of the multiplication}
  \param{C}{the result of the multiplication}
  \param{m}{the modulus}
\end{htmlonly}
\begin{code}

   public static void matTwoPowModM (int A[][], int B[][], int m, int e) \begin{hide} {
      int i, j;
      /* initialize: B = A */
      if (A != B) {
         for (i = 0; i < A.length; i++) {
            for (j = 0; j < A.length;  ++j)  //A is square
               B[i][j] = A[i][j];
         }
      }
      /* Compute B = A^{2^e} */
      for (i = 0; i < e; i++)
         matMatModM (B, B, B, m);
   } \end{hide}
\end{code}
\begin{tabb} Exactly like \method{matTwoPowModM}{double[][], double[][],
    double, int} using \texttt{double}, but with \texttt{int} instead of 
  \texttt{double}.
\end{tabb}
\begin{htmlonly}
  \param{A}{the matrix to raise to a power}
  \param{B}{the result of exponentiation}
  \param{m}{the modulus}
  \param{e}{the $\log_{2}$ of the exponent}
\end{htmlonly}
\begin{code}

   public static void matPowModM (int A[][], int B[][], int m, int c) \begin{hide} {
      int i, j;
      int n = c;
      int s = A.length;   //we suppose that A is square (it must be)
      int W[][] = new int[s][s];

      /* initialize: W = A; B = I */
      for (i = 0; i < s; i++) {
         for (j = 0; j < s;  ++j)  {
            W[i][j] = A[i][j];
            B[i][j] = 0;
         }
      }
      for (j = 0; j < s;  ++j)
         B[j][j] = 1;

      /* Compute B = A^c mod m using the binary decomp. of c */
      while (n > 0) {
         if ((n % 2)==1)
            matMatModM (W, B, B, m);
         matMatModM (W, W, W, m);
         n /= 2;
      }
   } \end{hide}
\end{code}
\begin{tabb} Exactly like \method{matPowModM}{double[][], double[][],
    double, int} using \texttt{double}, but with \texttt{int} instead
  of \texttt{double}.
\end{tabb}
\begin{htmlonly}
  \param{A}{the matrix to raise to a power}
  \param{B}{the result of the exponentiation}
  \param{m}{the modulus}
  \param{c}{the exponent}
\end{htmlonly}




%%%%%%%%%%%%%%%%%%%%%%%%%%%%%%%%%%%%%%%%%%%%%%%%%%%%
\subsubsection* {Methods using \texttt{long}}
\begin{code}

   public static long multModM (long a, long s, long c, long m) \begin{hide} {

      /* Suppose que 0 < a < m  et  0 < s < m.   Retourne (a*s + c) % m.
       * Cette procedure est tiree de :
       * L'Ecuyer, P. et Cote, S., A Random Number Package with
       * Splitting Facilities, ACM TOMS, 1991.
       * On coupe les entiers en blocs de d bits. H doit etre egal a 2^d.  */

      final long H = 2147483648L;               // = 2^d  used in MultMod
      long a0, a1, q, qh, rh, k, p;
      if (a < H) {
         a0 = a;
         p = 0;
      } else {
         a1 = a / H;
         a0 = a - H * a1;
         qh = m / H;
         rh = m - H * qh;
         if (a1 >= H) {
            a1 = a1 - H;
            k = s / qh;
            p = H * (s - k * qh) - k * rh;
            if (p < 0)
               p = (p + 1) % m + m - 1;
         } else                      /* p = (A2 * s * h) % m.      */
            p = 0;
         if (a1 != 0) {
            q = m / a1;
            k = s / q;
            p -= k * (m - a1 * q);
            if (p > 0)
               p -= m;
            p += a1 * (s - k * q);
            if (p < 0)
               p = (p + 1) % m + m - 1;
         }                           /* p = ((A2 * h + a1) * s) % m. */
         k = p / qh;
         p = H * (p - k * qh) - k * rh;
         if (p < 0)
            p = (p + 1) % m + m - 1;
      }                               /* p = ((A2 * h + a1) * h * s) % m  */
      if (a0 != 0) {
         q = m / a0;
         k = s / q;
         p -= k * (m - a0 * q);
         if (p > 0)
            p -= m;
         p += a0 * (s - k * q);
         if (p < 0)
            p = (p + 1) % m + m - 1;
      }
      p = (p - m) + c;
      if (p < 0)
         p += m;
      return p;
   } \end{hide}
\end{code}
\begin{tabb} Computes $(a \times s + c) \bmod m$. Works also if \texttt{s} 
  or \texttt{c} are negative.
  The result is always positive (and thus always between 0 and \texttt{m} - 1).
\end{tabb}
\begin{htmlonly}
  \param{a}{the first factor of the multiplication}
  \param{s}{the second factor of the multiplication}
  \param{c}{the second term of the addition}
  \param{m}{the modulus}
  \return{the result of the multiplication and the addition modulo \texttt{m}}
\end{htmlonly}
\begin{code}

   public static void matVecModM (long A[][], long s[], long v[], long m) \begin{hide} {
      int i;
      long x[] = new long[v.length];
      for (i = 0; i < v.length;  ++i) {
         x[i] = 0;
         for(int j = 0; j < s.length; j++)
            x[i] = multModM(A[i][j], s[j], x[i], m);
      }
      for (i = 0; i < v.length;  ++i)
         v[i] = x[i];

   } \end{hide}
\end{code}
\begin{tabb} Exactly like \method{matVecModM}{double[][], double[], 
    double[], double} using \texttt{double}, but with \texttt{long} instead 
  of \texttt{double}.
\end{tabb}
\begin{htmlonly}
  \param{A}{the multiplication matrix}
  \param{s}{the multiplied vector}
  \param{v}{the result of the multiplication}
  \param{m}{the modulus}
\end{htmlonly}
\begin{code}

   public static void matMatModM (long A[][], long B[][], long C[][], long m) \begin{hide} {
      int i, j;
      int r = C.length;    //# of rows of C
      int c = C[0].length; //# of columns of C
      long V[] = new long[r];
      long W[][] = new long[r][c];
      for (i = 0; i < c;  ++i) {
         for (j = 0; j < r;  ++j)
            V[j] = B[j][i];
         matVecModM (A, V, V, m);
         for (j = 0; j < r;  ++j)
            W[j][i] = V[j];
      }
      for (i = 0; i < r;  ++i) {
         for (j = 0; j < c;  ++j)
            C[i][j] = W[i][j];
      }
   } \end{hide}
\end{code}
\begin{tabb} Exactly like \method{matMatModM}{double[][], double[][],
    double[][], double} using \texttt{double}, but with \texttt{long} instead
  of \texttt{double}.
\end{tabb}
\begin{htmlonly}
  \param{A}{the first factor of the multiplication}
  \param{B}{the second factor of the multiplication}
  \param{C}{the result of the multiplication}
  \param{m}{the modulus}
\end{htmlonly}
\begin{code}

   public static void matTwoPowModM (long A[][], long B[][], long m, int e) \begin{hide} {
      int i, j;
      /* initialize: B = A */
      if (A != B) {
         for (i = 0; i < A.length; i++) {
            for (j = 0; j < A.length;  ++j)  //A is square
               B[i][j] = A[i][j];
         }
      }
      /* Compute B = A^{2^e} */
      for (i = 0; i < e; i++)
         matMatModM (B, B, B, m);
   } \end{hide}
\end{code}
\begin{tabb} Exactly like \method{matTwoPowModM}{double[][], double[][],
    double, int} using \texttt{double}, but with \texttt{long} instead of 
  \texttt{double}.
\end{tabb}
\begin{htmlonly}
  \param{A}{the matrix to raise to a power}
  \param{B}{the result of exponentiation}
  \param{m}{the modulus}
  \param{e}{the $\log_{2}$ of the exponent}
\end{htmlonly}
\begin{code}

   public static void matPowModM (long A[][], long B[][], long m, int c) \begin{hide} {
      int i, j;
      int n = c;
      int s = A.length;   //we suppose that A is square (it must be)
      long W[][] = new long[s][s];

      /* initialize: W = A; B = I */
      for (i = 0; i < s; i++) {
         for (j = 0; j < s;  ++j)  {
            W[i][j] = A[i][j];
            B[i][j] = 0;
         }
      }
      for (j = 0; j < s;  ++j)
         B[j][j] = 1;

      /* Compute B = A^c mod m using the binary decomp. of c */
      while (n > 0) {
         if ((n % 2)==1)
            matMatModM (W, B, B, m);
         matMatModM (W, W, W, m);
         n /= 2;
      }
   } \end{hide}
\end{code}
\begin{tabb} Exactly like \method{matPowModM}{double[][], double[][],
    double, int} using \texttt{double}, but with \texttt{long} instead
  of \texttt{double}.
\end{tabb}
\begin{htmlonly}
  \param{A}{the matrix to raise to a power}
  \param{B}{the result of the exponentiation}
  \param{m}{the modulus}
  \param{c}{the exponent}
\end{htmlonly}
\begin{code}
\begin{hide}
}
\end{hide}
\end{code}


