\defclass{ThreadCPUTimeChrono}

Extends the \class{AbstractChrono} 
class to compute the CPU time for a single thread. It is available
only under Java 1.5 which provides platform-independent 
facilities to get the CPU time for a single thread through management API.

Note that this chrono might not work properly on some systems running Linux because
of a bug in Sun's implementation or Linux kernel.
For instance, this class unexpectedly computes the global CPU time
under Fedora 
Core~4, kernel 2.6.17 and JRE version 1.5.0-09.
With Fedora Core~6, kernel 2.6.20, the function is working properly.
As a result, one should not rely on this bug to get the global CPU time.

Note that the above bug does not prevent one from using this chrono to
compute the CPU time for a single-threaded application.  In that case,
the global CPU time corresponds to the CPU time of the current thread.

Running timer fonctions when the associated thread is dead will return 0.

%%%%%%%%%%%%%%%%%%%%%%%%%%%%%%%%%%%%%%%%%%%%%%%%%%%%%%%%%%
\bigskip\hrule
\begin{code}
\begin{hide}
/*
 * Class:        ThreadCPUTimeChrono
 * Description:  Compute the CPU time for a single thread
 * Environment:  Java
 * Software:     SSJ 
 * Copyright (C) 2001  Pierre L'Ecuyer and Université de Montréal
 * Organization: DIRO, Université de Montréal
 * @author       Éric Buist
 * @since

 * SSJ is free software: you can redistribute it and/or modify it under
 * the terms of the GNU General Public License (GPL) as published by the
 * Free Software Foundation, either version 3 of the License, or
 * any later version.

 * SSJ is distributed in the hope that it will be useful,
 * but WITHOUT ANY WARRANTY; without even the implied warranty of
 * MERCHANTABILITY or FITNESS FOR A PARTICULAR PURPOSE.  See the
 * GNU General Public License for more details.

 * A copy of the GNU General Public License is available at
   <a href="http://www.gnu.org/licenses">GPL licence site</a>.
 */
\end{hide}
package umontreal.iro.lecuyer.util;\begin{hide}

import java.lang.management.ManagementFactory;
import java.lang.management.ThreadMXBean;\end{hide}


public class ThreadCPUTimeChrono extends AbstractChrono\begin{hide} {
   private  long           myThreadId;
   static   ThreadMXBean   threadMXBean = null;

   protected void getTime (long[] tab) {
      long rawTime = getTime();
      final long DIV = 1000000000L;
      long seconds = rawTime/DIV;
      long micros = (rawTime % DIV)/1000L;
      tab[0] = seconds;
      tab[1] = micros;
   }

   protected long getTime() {
      if (threadMXBean == null) {
         // We use lazy initialization to avoid a potential exception being wrapped into a confusing
         // ExceptionInInitializerError. That would happen if this initialization was in a static block instead of
         // in this method.
         threadMXBean = ManagementFactory.getThreadMXBean();
         if (!threadMXBean.isThreadCpuTimeEnabled())
            // Call this only when necessary, because this can throw a SecurityException if
            // run under a security manager.
            threadMXBean.setThreadCpuTimeEnabled (true);
      }
      long time = threadMXBean.getThreadCpuTime(myThreadId);
      return time < 0 ? 0 : time;
   }\end{hide}
\end{code}

%%%%%%%%%%%%%%%%%%%%%%%
\subsubsection*{Constructors}

\begin{code}

   public ThreadCPUTimeChrono()\begin{hide} {
      super();
      myThreadId = Thread.currentThread().getId();
      init();
   }\end{hide}
\end{code}
  \begin{tabb} Constructs a \texttt{ThreadCPUTimeChrono} object associated 
  with current thread and initializes it to zero.
  \end{tabb}
\begin{code}

   public ThreadCPUTimeChrono(Thread inThread)\begin{hide} {
      super();
      myThreadId = inThread.getId();
      init();
   }\end{hide}
\end{code}
  \begin{tabb} Constructs a \texttt{ThreadCPUTimeChrono} object associated 
  with the given \class{Thread} variable and initializes it to zero.
  \end{tabb}
\begin{code}
\begin{hide}
}\end{hide}
\end{code}
