\defclass {CategoryChart}

This class provides tools to create charts from data in a simple way. Its main
feature is to produce TikZ/PGF (see WWW link \url{http://sourceforge.net/projects/pgf/}) compatible source code which can be included
in \LaTeX\ documents, but it can also produce charts in other formats.
One can easily create a new chart, and customize its appearance using methods
of this class, with the encapsulated
\externalclass{umontreal.iro.lecuyer.charts}{SSJCategorySeriesCollection} object
representing the data, and an \externalclass{umontreal.iro.lecuyer.charts}{Axis}
object representing the axis.
All these classes depend on the \texttt{JFreeChart} API (see WWW link
\url{http://www.jfree.org/jfreechart/}) which provides tools to build charts with
Java, to draw them, and export them to files. However, only basic features are
used here.

Moreover, \texttt{CategoryChart} provides methods to plot data using a MATLAB friendly
syntax. None of these methods provides new features; they just propose a
different syntax to create charts. Therefore some features are unavailable
when using these methods only.

\bigskip\hrule
\begin{code}
\begin{hide}
/*
 * Class:        CategoryChart
 * Description:  
 * Environment:  Java
 * Software:     SSJ 
 * Copyright (C) 2001  Pierre L'Ecuyer and Université de Montréal
 * Organization: DIRO, Université de Montréal
 * @author       
 * @since

 * SSJ is free software: you can redistribute it and/or modify it under
 * the terms of the GNU General Public License (GPL) as published by the
 * Free Software Foundation, either version 3 of the License, or
 * any later version.

 * SSJ is distributed in the hope that it will be useful,
 * but WITHOUT ANY WARRANTY; without even the implied warranty of
 * MERCHANTABILITY or FITNESS FOR A PARTICULAR PURPOSE.  See the
 * GNU General Public License for more details.

 * A copy of the GNU General Public License is available at
   <a href="http://www.gnu.org/licenses">GPL licence site</a>.
 */
\end{hide}
package umontreal.iro.lecuyer.charts;\begin{hide}

import org.jfree.chart.JFreeChart;
import javax.swing.JFrame;\end{hide}

public abstract class CategoryChart \begin{hide} {

   protected Axis YAxis;
   protected SSJCategorySeriesCollection dataset;
   protected JFreeChart chart;
   protected boolean latexDocFlag = true;

   protected boolean autoRange;
   protected double[] manualRange;

   protected boolean grid = false;
   protected double ystepGrid;

   final protected double BOR = 0.1;\end{hide}
\end{code}

%%%%%%%%%%%%%%%%%%%%%%%%%%%
\subsubsection*{Methods}

\begin{code}

   public JFreeChart getJFreeChart() \begin{hide} {
      return chart;
   }\end{hide}
\end{code}
\begin{tabb}
   Returns the \texttt{JFreeChart} object associated with this chart.
\end{tabb}
\begin{htmlonly}
   \return{the associated JFreeChart object.}
\end{htmlonly}
\begin{code}

   public Axis getYAxis() \begin{hide} {
      return YAxis;
   }\end{hide}
\end{code}
\begin{tabb}
   Returns the chart's range axis ($y$-axis) object.
\end{tabb}
\begin{htmlonly}
   \return{chart's range axis ($y$-axis) object.}
\end{htmlonly}
\begin{code}

   public abstract JFrame view (int width, int height);
\end{code}
\begin{code}

   public String getTitle() \begin{hide} {
      return chart.getTitle().getText();
   }\end{hide}
\end{code}
\begin{tabb}
   Gets the current chart title.
\end{tabb}
\begin{htmlonly}
   \return{Chart title.}
\end{htmlonly}
\begin{code}

   public void setTitle (String title) \begin{hide} {
      chart.setTitle(title);
   }\end{hide}
\end{code}
\begin{tabb}
   Sets a title to this chart. This title will appear on the chart displayed
 by method \method{view}{}.
\end{tabb}
\begin{htmlonly}
   \param{title}{chart title.}
\end{htmlonly}
\begin{code}

   public void setAutoRange () \begin{hide} {
      autoRange = true;

      double BorneMin = Math.abs((dataset.getRangeBounds())[0]);
      double BorneMax = Math.abs((dataset.getRangeBounds())[1]);

      double max = Math.max(BorneMin,BorneMax) * BOR;
      YAxis.getAxis().setLowerBound(BorneMin - max);
      YAxis.getAxis().setUpperBound(BorneMax + max);
      YAxis.setLabelsAuto();
   }\end{hide}
\end{code}
\begin{tabb}
   Sets chart $y$ range to automatic values.
\end{tabb}
\begin{code}

   private void setManualRange (double[] range) \begin{hide}  {
      if(range.length != 2)
         throw new IllegalArgumentException (
             "range must have the format: [ymin, ymax]");
      autoRange = false;
      YAxis.getAxis().setLowerBound(Math.min(range[0],range[1]));
      YAxis.getAxis().setUpperBound(Math.max(range[0],range[1]));
   }\end{hide}
\end{code}
\begin{tabb}
   Sets new $y$-axis bounds, using the format:
\texttt{range} = [\texttt{ymin, ymax}].
\end{tabb}
\begin{htmlonly}
   \param{range}{new axis ranges.}
\end{htmlonly}
\begin{code}

   public void enableGrid (double xstep, double ystep) \begin{hide} {
      this.grid = true;
      this.ystepGrid = ystep;
   }\end{hide}
\end{code}
\begin{tabb}
   Puts a grid on the background. It is important to note that the grid is
   always shifted in such a way that it contains the axes. Thus, the grid does
   not always have an intersection at the corner points; this occurs
   only if the corner points are multiples of the steps: \texttt{xstep}
   and \texttt{ystep} sets the step in each direction.
\end{tabb}
\begin{htmlonly}
   \param{xstep}{sets the step in the x-direction.}
   \param{ystep}{sets the step in the y-direction.}
\end{htmlonly}
\begin{code}

   public void disableGrid() \begin{hide} {
      this.grid = false;
   }\end{hide}
\end{code}
\begin{tabb}
   Disables the background grid.
\end{tabb}


%%%%%%%%%%%%%%%%%%%%%%%%%%%%%%%%%%%%%%%%%%%%%%%%%%%%%%%
\subsubsection*{Latex-specific methods}

\begin{code}

   public abstract String toLatex (double width, double height);
\end{code}
\begin{tabb}
   Transforms the chart into \LaTeX{} form and returns it as a \texttt{String}.
\end{tabb}
\begin{code}

   public void setLatexDocFlag (boolean flag) \begin{hide} {
      latexDocFlag = flag;
   }\end{hide}
\end{code}
\begin{tabb}
Same as in \class{XYChart}.
\end{tabb}
\begin{code}
\begin{hide}

   protected double computeYScale (double position) {
      double[] bounds = new double[2];
      bounds[0] = YAxis.getAxis().getLowerBound();
      bounds[1] = YAxis.getAxis().getUpperBound();

      if (position < bounds[0])
         bounds[0] = position;
      if (position > bounds[1])
         bounds[1] = position;
      bounds[0] -= position;
      bounds[1] -= position;
      return computeScale (bounds);
   }

   protected double computeScale (double[] bounds) {
      int tenPowerRatio = 0;
      // echelle < 1 si les valeurs sont grandes
      while (bounds[1] > 1000 || bounds[0] < -1000) {
         bounds[1] /= 10;
         bounds[0] /= 10;
         tenPowerRatio++;
      }
      // echelle > 1 si les valeurs sont petites
      while (bounds[1] < 100 && bounds[0] > -100) {
         bounds[1] *= 10;
         bounds[0] *= 10;
         tenPowerRatio--;
      }
      return 1/Math.pow(10, tenPowerRatio);
   }
}\end{hide}
\end{code}
