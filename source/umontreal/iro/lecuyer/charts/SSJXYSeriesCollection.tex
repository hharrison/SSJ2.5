\defclass {SSJXYSeriesCollection}

Stores data used in a \texttt{XYChart}.
This class provides tools to manage data sets and rendering options, and modify
plot color, plot style, and marks on points for each series.

\bigskip\hrule
\begin{code}
\begin{hide}
/*
 * Class:        SSJXYSeriesCollection
 * Description:  
 * Environment:  Java
 * Software:     SSJ 
 * Copyright (C) 2001  Pierre L'Ecuyer and Université de Montréal
 * Organization: DIRO, Université de Montréal
 * @author       
 * @since

 * SSJ is free software: you can redistribute it and/or modify it under
 * the terms of the GNU General Public License (GPL) as published by the
 * Free Software Foundation, either version 3 of the License, or
 * any later version.

 * SSJ is distributed in the hope that it will be useful,
 * but WITHOUT ANY WARRANTY; without even the implied warranty of
 * MERCHANTABILITY or FITNESS FOR A PARTICULAR PURPOSE.  See the
 * GNU General Public License for more details.

 * A copy of the GNU General Public License is available at
   <a href="http://www.gnu.org/licenses">GPL licence site</a>.
 */
\end{hide}
package umontreal.iro.lecuyer.charts;\begin{hide}

import   org.jfree.data.xy.XYDataset;
import   org.jfree.chart.renderer.xy.XYItemRenderer;

import   java.util.Locale;
import   java.util.Formatter;
import   java.awt.Color;\end{hide}

public abstract class SSJXYSeriesCollection \begin{hide} {
   protected XYItemRenderer renderer;
   protected XYDataset seriesCollection;
\end{hide}
\end{code}

%%%%%%%%%%%%%%%%%%%%%%%%%%%
\subsubsection*{Data control methods}

\begin{code}

   public double getX (int series, int index) \begin{hide} {
      return seriesCollection.getXValue(series, index);
   }\end{hide}
\end{code}
\begin{tabb}
   Returns the $x$-value at the specified index in the specified series.
\end{tabb}
\begin{htmlonly}
   \param{series}{required series value.}
   \param{index}{value's index.}
   \return{$x$-value at the specified index in the specified series.}
\end{htmlonly}
\begin{code}

   public double getY (int series, int index) \begin{hide} {
      return seriesCollection.getYValue(series, index);
   }\end{hide}
\end{code}
\begin{tabb}
   Returns the $y$-value at the specified index in the specified series.
\end{tabb}
\begin{htmlonly}
   \param{series}{required series value.}
   \param{index}{value's index.}
   \return{$y$-value at the specified index in the specified series.}
\end{htmlonly}
\begin{code}

   public XYDataset getSeriesCollection() \begin{hide} {
      return seriesCollection;
   }\end{hide}
\end{code}
\begin{tabb}
   Returns the \texttt{XYDataset} object associated with the current object.
\end{tabb}
\begin{htmlonly}
   \return{\texttt{XYDataset} object associated with the current variable.}
\end{htmlonly}
\begin{code}

   public double[] getDomainBounds() \begin{hide} {
      double max=-1.0e307, min=1.0e307;

      if(seriesCollection.getSeriesCount() != 0 && seriesCollection.getItemCount(0) != 0)
         max = min = seriesCollection.getXValue(0, 0);

      for(int i = 0; i < seriesCollection.getSeriesCount(); i++) {
         for( int j = 0; j < seriesCollection.getItemCount(i); j++) {
            max = Math.max(max, seriesCollection.getXValue(i, j));
            min = Math.min(min, seriesCollection.getXValue(i, j));
         }
      }

      double[] retour = {min, max};
      return retour;
   }\end{hide}
\end{code}
\begin{tabb}
   Returns domain ($x$-coordinates) min and max values.
\end{tabb}
\begin{htmlonly}
   \return{domain min and max values.}
\end{htmlonly}
\begin{code}

   public double[] getRangeBounds() \begin{hide} {
      double max=-1.7e307, min=1.7e307;

      if(seriesCollection.getSeriesCount() != 0 && seriesCollection.getItemCount(0) != 0)
         max = min = seriesCollection.getYValue(0, 0);

      for(int i = 0; i < seriesCollection.getSeriesCount(); i++) {
         for( int j = 0; j < seriesCollection.getItemCount(i); j++) {
            max = Math.max(max, seriesCollection.getYValue(i, j));
            min = Math.min(min, seriesCollection.getYValue(i, j));
         }
      }

      double[] retour = {min, max};
      return retour;
   }\end{hide}
\end{code}
\begin{tabb}
   Returns range ($y$-coordinates) min and max values.
\end{tabb}
\begin{htmlonly}
   \return{range min and max values.}
\end{htmlonly}
\begin{code}

   public String toString() \begin{hide} {
      Formatter formatter = new Formatter(Locale.US);
      for(int i = 0; i < seriesCollection.getSeriesCount(); i++) {
         formatter.format(" Series " + i + " : %n");
         for(int j = 0; j < seriesCollection.getItemCount(i); j++)
            formatter.format("%15e,%15e%n", getX(i, j), getY(i, j));
      }
      return formatter.toString();
   }\end{hide}
\end{code}
\begin{tabb}
  Returns in a \texttt{String} all data contained in the current object.
\end{tabb}
\begin{htmlonly}
   \return{All data contained in the current object as a \class{String}.}
\end{htmlonly}

%%%%%%%%%%%%%%%%%%%%%%%%%%%
\subsubsection*{Rendering methods}

\begin{code}

   public XYItemRenderer getRenderer() \begin{hide} {
      return renderer;
   }\end{hide}
\end{code}
\begin{tabb}
   Returns the \texttt{XYItemRenderer} object associated with the current object.
\end{tabb}
\begin{htmlonly}
   \return{\texttt{XYItemRenderer} object associated with the current variable.}
\end{htmlonly}
\begin{code}

   public void setRenderer (XYItemRenderer renderer) \begin{hide} {
      this.renderer = renderer;
   }\end{hide}
\end{code}
\begin{tabb}
   Sets the \texttt{XYItemRenderer} object associated with the current variable.
   This object determines the chart JFreeChart look, produced by method
  \texttt{view} in class
   \externalclass{umontreal.iro.lecuyer.charts}{XYChart}.
\end{tabb}
\begin{htmlonly}
   \param{renderer}{ new \texttt{XYItemRenderer} object.}
\end{htmlonly}
\begin{code}

   public Color getColor (int series) \begin{hide} {
      return (Color)renderer.getSeriesPaint(series);
   }\end{hide}
\end{code}
\begin{tabb}
   Gets the current plotting color of the selected series.
\end{tabb}
\begin{htmlonly}
   \return{current plotting color.}
\end{htmlonly}
\begin{code}

   public void setColor (int series, Color color) \begin{hide} {
      renderer.setSeriesPaint(series, color);
   }\end{hide}
\end{code}
\begin{tabb}
   Sets a new plotting color to the series $series$.
\end{tabb}
\begin{htmlonly}
   \param{series}{series index.}
   \param{color}{plotting color.}
\end{htmlonly}
\begin{code}

   public abstract String toLatex (double XScale, double YScale,
                                   double XShift, double YShift,
                                   double xmin, double xmax,
                                   double ymin, double ymax);
\end{code}
\begin{tabb}
   Formats and returns a string containing a \LaTeX-compatible source
   code which represents this data series collection. % and its parameters.
   The original datasets are shifted and scaled with the \texttt{XShift},
   \texttt{YShift}, \texttt{XScale} and \texttt{YScale} parameters.
   \texttt{xmin}, \texttt{xmax}, \texttt{ymin} and \texttt{ymax} represent the chart bounds.
\end{tabb}
\begin{htmlonly}
   \param{XScale}{Domain original data scale.}
   \param{YScale}{Range original data scale.}
   \param{XShift}{Domain original data shift value.}
   \param{YShift}{Range original data shift value.}
   \param{xmin}{Domain min bound.}
   \param{xmax}{Domain nax bound.}
   \param{ymin}{Range min bound.}
   \param{ymax}{Range nax bound.}
   \return{TikZ code.}
\end{htmlonly}
\begin{code}
\begin{hide}

   /**
    * Converts a java Color object into a friendly and readable LaTeX/xcolor string.
    *
    * @param   color    in color.
    * @return           friendly color with string format as possible, null otherwise.
    */
   protected static String detectXColorClassic(Color color) {
      String retour = null;

      int red = color.getRed();
      int green = color.getGreen();
      int blue = color.getBlue();

      // On utilise pas la method Color.equals(Color ) car on ne veut pas tester le parametre de transparence : Alpha
      if (   red == Color.GREEN.getRed()
          && blue == Color.GREEN.getBlue()
          && green == Color.GREEN.getGreen())
         return "green";
      else if (   red == Color.RED.getRed()
               && blue == Color.RED.getBlue()
               && green == Color.RED.getGreen())
         return "red";
      else if (   red == Color.WHITE.getRed()
               && blue == Color.WHITE.getBlue()
               && green == Color.WHITE.getGreen())
         return "white";
      else if (   red == Color.GRAY.getRed()
               && blue == Color.GRAY.getBlue()
               && green == Color.GRAY.getGreen())
          return "gray";
      else if (   red == Color.BLACK.getRed()
               && blue == Color.BLACK.getBlue()
               && green == Color.BLACK.getGreen())
          return "black";
      else if (   red == Color.YELLOW.getRed()
               && blue == Color.YELLOW.getBlue()
               && green == Color.YELLOW.getGreen())
          return "yellow";
      else if (   red == Color.MAGENTA.getRed()
               && blue == Color.MAGENTA.getBlue()
               && green == Color.MAGENTA.getGreen())
          return "magenta";
      else if (   red == Color.CYAN.getRed()
               && blue == Color.CYAN.getBlue()
               && green == Color.CYAN.getGreen())
          return "cyan";
      else if (   red == Color.BLUE.getRed()
               && blue == Color.BLUE.getBlue()
               && green == Color.BLUE.getGreen())
          return "blue";
      else if (   red == Color.DARK_GRAY.getRed()
               && blue == Color.DARK_GRAY.getBlue()
               && green == Color.DARK_GRAY.getGreen())
          return "darkgray";
      else if (   red == Color.LIGHT_GRAY.getRed()
               && blue == Color.LIGHT_GRAY.getBlue()
               && green == Color.LIGHT_GRAY.getGreen())
          return "lightgray";
      else if (   red == Color.ORANGE.getRed()
               && blue == Color.ORANGE.getBlue()
               && green == Color.ORANGE.getGreen())
          return "orange";
      else if (   red == Color.PINK.getRed()
               && blue == Color.PINK.getBlue()
               && green == Color.PINK.getGreen())
          return "pink";


      if (red == 192 && blue == 128 && green == 64)
         return "brown";
     else if (red == 128 && blue == 128 && green == 0)
         return "olive";
      else if (red == 128 && blue == 0 && green == 128)
         return "violet";
      else if (red == 192 && blue == 0 && green ==64)
         return "purple";
      else return null;
   }

   /**
    * Gives the default color associated with a series
    *
    * @param   index Index of the series in the XYDataset object.
    * @return        default color object.
    */
   protected static Color getDefaultColor(int index) {
      if(index%6 == 0)
         return Color.RED;
      else if(index%6 == 1)
         return Color.BLUE;
      else if(index%6 == 2)
         return Color.GREEN;
      else if(index%6 == 3)
         return Color.YELLOW;
      else if(index%6 == 4)
         return Color.MAGENTA;
      else
         return Color.CYAN;
   }

   // Returns maximum value in table t
   protected static double max (double[] t) {
      double aux = t[0];
      for (int i=1 ; i < t.length ; i++)
         if (t[i] > aux)
            aux = t[i];
      return aux ;
   }
   // Returns minimum value in table t
   protected static double min (double[] t) {
      double aux = t[0];
      for (int i=1 ; i < t.length ; i++)
         if (t[i] < aux)
            aux = t[i];
      return aux ;
   }
}\end{hide}
\end{code}
