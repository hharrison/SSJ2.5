\defclass{KorobovLattice}

This class implements \emph{Korobov lattices}, which represents the same point
sets as in class \class{LCGPointSet}, but implemented differently.
The parameters are the modulus $n$ and the multiplier $a$, for arbitrary
integers $1 \le a < n$.
The number of points is $n$, their dimension is 
$s$, and they are defined by
\[
  \mathbf{u}_i = (i/n)(1, a, a^2, \ldots, a^{s-1}) \bmod 1
\]
for $i=0,\dots,n-1$.

It is also possible to build a ``shifted'' Korobov lattice with the first
$t$ coordinates rejected. The $s$-dimensionnal points are then defined as
$$
  \mathbf{u}_i = (i/n)(a^{t}, a^{t+1}, a^{t+2}, \ldots, a^{t+s-1}) \bmod 1
$$
for $i=0,\dots,n-1$ and fixed $t$.

\bigskip\hrule\bigskip

%%%%%%%%%%%%%%%%%%%%%%%%%%%%%%%%%%%%%%%%%%%%%%%%%%%%%%%%%%%%%%%%%%
\begin{code}
\begin{hide}
/*
 * Class:        KorobovLattice
 * Description:  Korobov lattice
 * Environment:  Java
 * Software:     SSJ 
 * Copyright (C) 2001  Pierre L'Ecuyer and Université de Montréal
 * Organization: DIRO, Université de Montréal
 * @author       
 * @since

 * SSJ is free software: you can redistribute it and/or modify it under
 * the terms of the GNU General Public License (GPL) as published by the
 * Free Software Foundation, either version 3 of the License, or
 * any later version.

 * SSJ is distributed in the hope that it will be useful,
 * but WITHOUT ANY WARRANTY; without even the implied warranty of
 * MERCHANTABILITY or FITNESS FOR A PARTICULAR PURPOSE.  See the
 * GNU General Public License for more details.

 * A copy of the GNU General Public License is available at
   <a href="http://www.gnu.org/licenses">GPL licence site</a>.
 */
\end{hide}
package umontreal.iro.lecuyer.hups;
\begin{hide}
import umontreal.iro.lecuyer.util.PrintfFormat;
\end{hide}

public class KorobovLattice extends Rank1Lattice \begin{hide} {
   protected int genA;                        // Multiplier a.

   // Method modPower is inherited from Rank1Lattice.
\end{hide}
\end{code}

%%%%%%%%%%%%%%%%%%%%%%%%%%%%%%%%%
\subsubsection*{Constructors}
\begin{code}
   public KorobovLattice (int n, int a, int s) \begin{hide} {
      super (n, null, 0);
      if (a < 1 || a >= n) 
         throw new IllegalArgumentException 
            ("KorobovLattice must have 1 <= a < n");
      genA = a;
      dim = s;
      v = new double[s];
      long[] B = new long[dim];
      B[0] = 1;
      v[0] = normFactor;
      for (int j = 1; j < dim; j++) {
         B[j] = (a * B[j - 1]) % n;
         v[j] = normFactor * B[j];
      }
   }\end{hide}
\end{code}
 \begin{tabb}
   Instantiates a Korobov lattice point set with modulus $n$ and 
   multiplier $a$ in dimension $s$.
 \end{tabb}
\begin{code}

   public KorobovLattice (int n, int a, int s, int t) \begin{hide} {
      super (n, null, 0);
      if (a < 1 || a >= n) 
         throw new IllegalArgumentException 
            ("KorobovLattice must have 1 <= a < n");
      if (t < 1) 
         throw new IllegalArgumentException 
            ("KorobovLattice: must have 0 < t");
      dim = s;
      genA = a;
      v = new double[s];
      long[] B = new long[dim];
      int j;
      B[0] = a;
      for (j = 1; j < t; j++)
         B[0] *= a;
      v[0] = B[0] * normFactor;
      for (j = 1; j < dim; j++) {
         B[j] = (a * B[j - 1]) % n;
         v[j] = normFactor * B[j];
      }
   }\end{hide}
\end{code}
 \begin{tabb}
   Instantiates a shifted Korobov lattice point set with modulus $n$ and 
   multiplier $a$ in dimension $s$. The first $t$ coordinates of a
   standard Korobov lattice are dropped as described above.
   The case $t=0$ corresponds to the standard  Korobov lattice.
 \end{tabb}

%%%%%%%%%%%%%%%%%%%%%%%%%%%%%%%%%%%%
\subsubsection*{Methods}
\begin{code}

   public int getA() \begin{hide} {
      return genA;
   }\end{hide}
\end{code}
\begin{tabb}
Returns the multiplier $a$ of the lattice.
\end{tabb}
\begin{code}
\begin{hide}

   public String toString() {
      StringBuffer sb = new StringBuffer ("KorobovLattice:" +
                                           PrintfFormat.NEWLINE);
      sb.append ("Multiplier a: " + genA + PrintfFormat.NEWLINE);
      sb.append (super.toString());
      return sb.toString();
   }
}\end{hide}
\end{code}
