\defclass{F2wCycleBasedLFSR}

This class creates a point set based upon a linear feedback shift register
 sequence. The recurrence used to produce the point set is
$$
  m_n = \sum_{i=1}^r b_i m_{n-i}
$$
where $m_n\in \latex{\mathbb{F}}\html{\mathbf{F}}_{2^w}$, $n\geq 0$
  and $b_i\in \latex{\mathbb{F}}\html{\mathbf{F}}_{2^w}$.
There is a polynomial in $\latex{\mathbb{F}}\html{\mathbf{F}}_{2^w}[z]$
 associated with this recurrence called
the \emph{characteristic polynomial}.  It is
$$
  P(z) =  z^r + \sum_{i=1}^r b_i z^{r-i}.
$$
In the implementation, this polynomial is stored in an object
% \externalclass{umontreal.iro.lecuyer.hups}{F2wStructure}.
 \texttt{F2wStructure}.
% See the description of this class for more details.

Let ${\mathbf{x}} = (x^{(0)}, \ldots, x^{(p-1)}) \in
\latex{\mathbb{F}}\html{\mathbf{F}}_{2}^p$
be a $p$-bit vector.
Let us define the function $\phi(\mathbf{x}) = \sum_{i=1}^{p} 2^{-i} x^{(i-1)}$.
The point set in $t$ dimensions produced by this class is
$$
 \left\{ (\phi(\mathbf{y}_0),\phi(\mathbf{y}_s),\ldots,\phi(\mathbf{y}_{s(t-1)}):
  (\mathbf{v}_0,\ldots,\mathbf{v}_{r-1})\in
 \latex{\mathbb{F}}\html{\mathbf{F}}_{2}^{rw}\right\}
$$
where $\mathbf{y}_n = \mbox{trunc}_h(\mathbf{v}_n, \mathbf{v}_{n+1},\ldots)$,
$\mathbf{v}_n$ is the representation of $m_n$ under the polynomial basis of
$\latex{\mathbb{F}}\html{\mathbf{F}}_{2^w}$ over
 $\latex{\mathbb{F}}\html{\mathbf{F}}_2$, and
 $h=w\latex{\lfloor 31/w\rfloor}\html{\mbox{ floor}(31/w)}$.
The parameter $s$ is called the stepping parameter of the recurrence.

\bigskip\hrule\bigskip

%%%%%%%%%%%%%%%%%%%%%%%%%%%%%%%%%%%%%%%%%%%%%%%%%%%%%%%%%%%%%%%%%%
\begin{code}
\begin{hide}
/*
 * Class:        F2wCycleBasedLFSR
 * Description:  point set based upon a linear feedback shift register
                 sequence
 * Environment:  Java
 * Software:     SSJ 
 * Copyright (C) 2001  Pierre L'Ecuyer and Université de Montréal
 * Organization: DIRO, Université de Montréal
 * @author       
 * @since

 * SSJ is free software: you can redistribute it and/or modify it under
 * the terms of the GNU General Public License (GPL) as published by the
 * Free Software Foundation, either version 3 of the License, or
 * any later version.

 * SSJ is distributed in the hope that it will be useful,
 * but WITHOUT ANY WARRANTY; without even the implied warranty of
 * MERCHANTABILITY or FITNESS FOR A PARTICULAR PURPOSE.  See the
 * GNU General Public License for more details.

 * A copy of the GNU General Public License is available at
   <a href="http://www.gnu.org/licenses">GPL licence site</a>.
 */
\end{hide}
package umontreal.iro.lecuyer.hups; \begin{hide}
import cern.colt.list.*;
import umontreal.iro.lecuyer.util.PrintfFormat;
\end{hide}


public class F2wCycleBasedLFSR extends CycleBasedPointSetBase2 \begin{hide} {

   private F2wStructure param;
\end{hide}\end{code}


%%%%%%%%%%%%%%%%%%%%%%%%%%%%
\subsubsection*{Constructors}
\begin{code}

   public F2wCycleBasedLFSR (int w, int r, int modQ, int step, int nbcoeff,
                             int coeff[], int nocoeff[])\begin{hide}
    /**
     * Constructs and stores the set of cycles for an LCG with
     *    modulus <SPAN CLASS="MATH"><I>n</I></SPAN> and multiplier <SPAN CLASS="MATH"><I>a</I></SPAN>.
     *   If pgcd<SPAN CLASS="MATH">(<I>a</I>, <I>n</I>) = 1</SPAN>, this constructs a full-period LCG which has two
     *   cycles, one containing 0 and one, the LCG period.
     *
     * @param n required number of points and modulo of the LCG
     *
     *    @param a generator <SPAN CLASS="MATH"><I>a</I></SPAN> of the LCG
     *
     *
     */
   {
      param = new F2wStructure (w, r, modQ, step, nbcoeff, coeff, nocoeff);
      init ();
   }\end{hide}
\end{code}
 \begin{tabb}
Constructs a point set with $2^{rw}$ points.  See the description of the class
\externalclass{umontreal.iro.lecuyer.hups}{F2wStructure} for the meaning
 of the parameters.
 \end{tabb}
\begin{code}

   public F2wCycleBasedLFSR (String filename, int no) \begin{hide}
   {
      param = new F2wStructure (filename, no);
      init ();
   }\end{hide}
\end{code}
 \begin{tabb}
   Constructs a point set after reading its parameters from
   file \texttt{filename}; the parameters are located at line numbered \texttt{no}
   of  \texttt{filename}.  The available files are listed in the description of class
\externalclass{umontreal.iro.lecuyer.hups}{F2wStructure}.
 \end{tabb}
\begin{code}
\begin{hide}

   private void init ()
   {
      param.initParamLFSR ();
      normFactor = param.normFactor;
      EpsilonHalf = param.EpsilonHalf;
      numBits = param.numBits;
      fillCyclesLFSR ();
   }

   public String toString ()
   {
       String s = "F2wCycleBasedLFSR:" + PrintfFormat.NEWLINE;
       return s + param.toString ();
   }

   private void fillCyclesLFSR ()
   {
      int n = 1 << param.getLog2N ();
      IntArrayList c;             // Array used to store the current cycle.
      int currentState;           // The state currently visited.
      int i;
      boolean stateVisited[] = new boolean[n];

      // Indicates which states have been visited so far.
      for (i = 0; i < n; i++)
         stateVisited[i] = false;
      int startState = 0;    // First state of the cycle currently considered.
      numPoints = 0;
      while (startState < n) {
         stateVisited[startState] = true;
         c = new IntArrayList ();
         param.state = startState;
         param.initF2wLFSR ();
         c.add (param.output);
         param.F2wLFSR ();
         // Warning: watch for overflow !!!
         while (param.state != startState) {
            stateVisited[param.state] = true;
            c.add (param.output);
            param.F2wLFSR ();
         }
         addCycle (c);
         for (i = startState + 1; i < n; i++)
            if (stateVisited[i] == false)
               break;
         startState = i;
      }
   }
}
\end{hide}
\end{code}
