\documentclass[12pt]{article}

\usepackage{amssymb}
\usepackage{alltt}
\usepackage{html}
\usepackage{url}
\usepackage{ssj}

% \def\bq{\mbox{\boldmath $q$}}
% \def\be{\mbox{\boldmath $e$}}
\def\bu{\mbox{\boldmath $u$}}
% \def\bv{\mbox{\boldmath $v$}}
% \def\bC{\mbox{\boldmath $C$}}
% \def\bV{\mbox{\boldmath $V$}}
% \def\cS{\mathcal{S}}


\mytwoheads\dateheadtrue
\detailedfalse

% \includeonly{F2wStructure}

%%%%%%%%%%%%%%%%%%%%%%%%%%%%%%%%%%%%%%%%%%
\begin{document}

\begin{titlepage}

%  \title{hups}{Highly Uniform Point Sets}
\title{hups}{Tools for Quasi-Monte Carlo}

This package provides classes implementing highly uniform point sets
(HUPS) and tools for their randomization.
These point sets can be used for quasi-Monte Carlo integration.
Randomized quasi-Monte Carlo point sets can in fact replace streams
of uniform random numbers in a simulation, for the purpose of
reducing the variance of the estimator.

\vfill
\end{titlepage}

%%%%%%%%%%%%%%%%%%%%%%%%%%%%%%%%%%%%%%%%%%%%%%%%%%%%%

\pagenumbering{roman}
\tableofcontents
\pagenumbering{arabic}

\latex{\section*{Overview}\addcontentsline{toc}{subsection}{Overview}}
\latex{\label {sec:overview}}

Process-oriented simulation is managed through this package.  
A \emph{Process} can be seen as an \emph{active object} whose behavior in
time is described by a method called \texttt{actions()}.
Each process must extend the
\externalclass{umontreal.iro.lecuyer.simprocs}{SimProcess} class and
must implement this \texttt{actions()} method.
Processes are created and can be scheduled to start at a given 
simulation time just like events.
In contrast with the corresponding \texttt{actions()} method of events,
the method of processes is generally not executed instantaneously in the 
simulation time frame.
At any given simulation time, at most one process can be \emph{active}, 
i.e., executing its \texttt{actions()} method.
The active process may create and schedule new processes,
kill suspended processes, and suspend itself.  A process is suspended
for a fixed delay or until a resource becomes available, or a condition
becomes true.  When a process is suspended or finishes its execution,
another process usually starts or resumes.

These processes may represent ``autonomous'' objects 
such as machines and robots in a factory, 
customers in a retail store, 
vehicles in a transportation or delivery system, etc.
The process-oriented paradigm is a natural way of describing complex
systems \cite{sFRA77a,sBIR86a,sKRE86a,sLAW00a} and often leads to more 
compact code than the event-oriented view.
However, it is often preferred to use events only, because this gives a faster
simulation program, by avoiding the process-synchronization overhead.
Most complex discrete-event systems are quite conveniently modeled only 
with events.
In SSJ, events and processes can be mixed freely.
The processes actually use events for their synchronization.

The classes \externalclass{umontreal.iro.lecuyer.simprocs}{Resource},
\externalclass{umontreal.iro.lecuyer.simprocs}{Bin}, and
\externalclass{umontreal.iro.lecuyer.simprocs}{Condition}
provide additional mechanisms for process synchronization.  
A \externalclass{umontreal.iro.lecuyer.simprocs}{Resource} corresponds
to a facility with limited capacity and a waiting queue.
A process can request an arbitrary number of units of a resource,
may have to wait until enough units are available,
can use the resource for a certain time, and eventually releases it.
A \externalclass{umontreal.iro.lecuyer.simprocs}{Bin} supports
producer/consumer relationships between processes.
It corresponds essentially to a pile of free tokens and a queue of 
processes waiting for the tokens.
A \emph{producer} adds tokens to the pile whereas a
\emph{consumer} (a process) can ask for tokens.
When not enough tokens are available, the consumer is blocked 
and placed in the queue.
The class \externalclass{umontreal.iro.lecuyer.simprocs}{Condition}
supports the concept of processes waiting
for a certain boolean condition to be true before continuing their execution.

Two different implementations of processes are available in SSJ, each
one corresponding to a subclass of \texttt{ProcessSimulator}.
The first one, called \texttt{ThreadProcessSimulator}, uses Java
threads as described in Section~4 of \cite{sLEC02a}.
The second one, \texttt{DSOLProcessSimulator}, is taken from DSOL
\cite{iJAC05a,sJAC04a} and was 
provided to us by Peter Jacobs.
Unfortunately, none of these two implementations is fully satisfactory.

Java threads are designed for \emph{real parallelism}, not for the kind of
\emph{simulated} parallelism required in process-oriented simulation.
In the Java Development Kit (JDK) 1.3.1 and earlier, \emph{green threads}
supporting simulated parallelism were available and our original 
implementation of processes\latex{ described in \cite{sLEC02a}} 
is based on them.
But green threads are no longer supported in recent Java runtime 
environments.  
True (native) threads from the operating system are used instead. 
This adds significant overhead and prevents the use of a large number
of processes in the simulation.
\emph{This implementation of processes with threads can be used safely only
with the JDK versions~1.3.1 or earlier}.
A program using the thread-based process view can easily be 10 to 20 
times slower than a similar program using the event view only
(see \cite{sLEC05a} for an example).

%%%  D-SOL.
The second implementation, made by P.{} Jacobs, stays away from threads.
It uses a Java reflection mechanism that interprets the code of processes
at runtime and transforms everything into events.
A program using the process view implemented with the DSOL interpreter
can be 500 to 1000 times slower than the corresponding event-based program
but the number of processes is limited only by the available memory.



\defclass{PointSet}

This abstract class defines the basic methods
for accessing and manipulating point sets.
A point set can be represented as a two-dimensional array, whose element
$(i,j)$ contains $u_{i,j}$, the \emph{coordinate} $j$ of point $i$.
Each coordinate $u_{i,j}$ is assumed to be in the unit interval $[0,1]$.
% and can be accessed directly via the (abstract) method
% \method{getCoordinate}{}.
% (implemented in subclasses).
Whether the values 0 and 1 can occur may depend on the
actual implementation of the point set.

All points have the same number of coordinates (their \emph{dimension})
and this number can be queried by \method{getDimension}{}.
The number of points is queried by \method{getNumPoints}{}.
The points and coordinates are both numbered starting from 0
and their number can actually be infinite.

The \method{iterator}{} method provides a \emph{point set iterator}
which permits one to enumerate the points and their coordinates.
Several iterators over the same point set can coexist at any given time.
These iterators are instances of a hidden inner-class that implements
the \class{PointSetIterator} interface.
The default implementation of iterator provided here relies on
the method \method{getCoordinate}{} to access the coordinates directly.
However, this approach is rarely efficient.
Specialized implementations that dramatically improve the performance
are provided in subclasses of \class{PointSet}.
The \class{PointSetIterator}{} interface actually extends the
\class{RandomStream}{} interface, so that the
iterator can also be seen as a \class{RandomStream} and used wherever
such a stream is required for generating uniform random numbers.
This permits one to easily replace pseudorandom numbers by the
coordinates of a selected set of highly-uniform points, i.e.,
to replace Monte Carlo by quasi-Monte Carlo in a simulation program.

%%%%%%%%
\begin{comment}
The class also offers tools to manipulate a list of randomizations
that can be applied to this point set.
\pierre{So far, the general types of randomizations have been implemented
   as containers.  We may remove this concept of list. }
\richard{La nouvelle randomisation d'Adam rend l'ancienne liste de
 randomisations obsol\`ete: nous n'avons jamais utilis\'e l'ancienne version.}
\end{comment}
%%%%%%

This abstract class has only one abstract method:
\method{getCoordinate}{}.
Providing an implementation for this method is already sufficient
for the subclass to work.
However, in practically all cases, efficiency can be dramatically improved by
overwriting \method{iterator}{} to provide a custom iterator that does not
necessarily rely on \method{getCoordinate}{}.
In fact, direct use of \method{getCoordinate}{} to access the coordinates
is discouraged.
One should access the coordinates only via the iterators.

\begin{detailed}  %%%%%
The built-in range checks require some extra time and also
assumes that nobody ever uses negative indices (Java does not offer unsigned
integers).  If \method{getCoordinate}{} is not accessed directly by the user,
it may be implemented without range checks.
\end{detailed}  %%%

\bigskip\hrule\bigskip
%%%%%%%%%%%%%%%%%%%%%%%%%%%%%%%%%%%%%%%%%%%%%%%%%%%%%%%%%%%%%%%%%%

\begin{code}
\begin{hide}
/*
 * Class:        PointSet
 * Description:  Base class of all point sets
 * Environment:  Java
 * Software:     SSJ 
 * Copyright (C) 2001  Pierre L'Ecuyer and Université de Montréal
 * Organization: DIRO, Université de Montréal
 * @author       
 * @since

 * SSJ is free software: you can redistribute it and/or modify it under
 * the terms of the GNU General Public License (GPL) as published by the
 * Free Software Foundation, either version 3 of the License, or
 * any later version.

 * SSJ is distributed in the hope that it will be useful,
 * but WITHOUT ANY WARRANTY; without even the implied warranty of
 * MERCHANTABILITY or FITNESS FOR A PARTICULAR PURPOSE.  See the
 * GNU General Public License for more details.

 * A copy of the GNU General Public License is available at
   <a href="http://www.gnu.org/licenses">GPL licence site</a>.
 */
\end{hide}
package umontreal.iro.lecuyer.hups;\begin{hide}

import java.util.NoSuchElementException;
import java.util.List;
import java.util.ArrayList;
import umontreal.iro.lecuyer.rng.RandomStream;
import umontreal.iro.lecuyer.util.Num;
import umontreal.iro.lecuyer.util.PrintfFormat;
\end{hide}

public abstract class PointSet \begin{hide} {

   // The maximum number of usable bits (binary digits).
   // Since Java has no unsigned type, the
   // 32nd bit cannot be used efficiently. This mainly affects digit
   // scrambling and bit vectors. This also limits the maximum number
   // of columns for the generating matrices of digital nets in base 2.
   protected static final int MAXBITS = 31;
   // To avoid 0 for nextCoordinate when random shifting 
   protected double EpsilonHalf = 1.0 / Num.TWOEXP[55];  // 1/2^55

   protected int dim = 0;
   protected int numPoints = 0;
   protected int dimShift = 0;            // Current dimension of the shift.
   protected int capacityShift = 0;       // Number of array elements of shift;
                                          // it is always >= dimShift
   protected RandomStream shiftStream;    // Used to generate random shifts.
\end{hide}
\end{code}

%%%%%%%%%%%%%%%%%%%%%%%%%%%%%%%%%
\subsubsection*{Methods}
%%%%%%%%%%%%%%%%%%%%%%%%%%%%%%%%%%%%%%%%%%%%%%%%   Sizes
\begin{code}

   public int getDimension()\begin{hide} {
      return dim;
   }\end{hide}
\end{code}
 \begin{tabb}
   Returns the dimension (number of available coordinates) of the point set.
   If the dimension is actually infinite, \texttt{Integer.MAX\_VALUE} is returned.
 \end{tabb}
\begin{htmlonly}
   \return{the dimension of the point set or \texttt{Integer.MAX\_VALUE}
   if it is infinite}
\end{htmlonly}
\begin{code}

   public int getNumPoints()\begin{hide} {
      return numPoints;
   }\end{hide}
\end{code}
 \begin{tabb}
   Returns the number of points.
   If this number is actually infinite, \texttt{Integer.MAX\_VALUE} is returned.
 \end{tabb}
\begin{htmlonly}
   \return{the number of points in the point set or \texttt{Integer.MAX\_VALUE}
   if the point set has an infinity of points.
\end{htmlonly}

%%%%%%%%%%%%%%%%%%%%%%%%%%%%%%%%%%%%%%%%%%%%%%%%   Coordinates
\begin{code}

   public abstract double getCoordinate (int i, int j);
\end{code}
 \begin{tabb}
   Returns $u_{i,j}$, the coordinate $j$ of the point $i$.
\richard{La m\'ethode \texttt{getCoordinate} de certaines classes ne tient
pas compte du random shift, contrairement \`a l'it\'erateur de la m\^eme classe.
Faut-il que toutes les  \texttt{getCoordinate} impl\'ementent le random shift
quand il existe?}
 \end{tabb}
\begin{htmlonly}
   \param{i}{index of the point to look for}
   \param{j}{index of the coordinate to look for}
   \return{the value of $u_{i,j}$}
\end{htmlonly}
\begin{code}

   public PointSetIterator iterator()\begin{hide} {
      return new DefaultPointSetIterator();
   }\end{hide}
\end{code}
\begin{tabb}
 Constructs and returns a point set iterator.
 The default implementation returns an iterator that uses the method
 \method{getCoordinate}{}~\texttt{(i,j)} to iterate over the
 points and coordinates, but subclasses can reimplement it
 for better efficiency.
\end{tabb}
\begin{htmlonly}
   \return{point set iterator for the point set}
\end{htmlonly}
\begin{code}

   public void setStream (RandomStream stream)\begin{hide} {
      shiftStream = stream;
  }\end{hide}
\end{code}
 \begin{tabb}
   Sets the random stream used to generate random shifts to \texttt{stream}.
 \end{tabb}
\begin{htmlonly}
   \param{stream}{the new random stream}
\end{htmlonly}
\begin{code}

   public RandomStream getStream()\begin{hide} {
      return shiftStream;
  }\end{hide}
\end{code}
 \begin{tabb}
   Returns the random stream used to generate random shifts.
 \end{tabb}
\begin{htmlonly}
   \return{the random stream used}
\end{htmlonly}
\begin{code}

   public void randomize (PointSetRandomization rand) \begin{hide} {
       rand.randomize(this);
   }\end{hide}
\end{code}
\begin{tabb}
   Randomizes the point set using the given \texttt{rand}.
\end{tabb}
\begin{htmlonly}
   \param{rand}{\class{PointSetRandomization} to use}
\end{htmlonly}
\begin{code}

   public void addRandomShift (int d1, int d2, RandomStream stream)\begin{hide} {
//   throw new UnsupportedOperationException
//         ("addRandomShift in PointSet called");
     System.out.println (
        "******* WARNING:  addRandomShift in PointSet does nothing");
   }\end{hide}
\end{code}
\begin{tabb}  This method does nothing for this generic class.
  In some subclasses, it adds a random shift to all the points
  of the point set, using stream \texttt{stream} to generate the random numbers,
  for coordinates \texttt{d1} to \texttt{d2-1}.
\end{tabb}
\begin{code}

   public void addRandomShift (RandomStream stream)\begin{hide} {
      addRandomShift (0, dimShift, stream);
  }\end{hide}
\end{code}
\begin{tabb}  This method does nothing for this generic class.
 Similar to \texttt{addRandomShift (0, d2, stream)},
  with \texttt{d2} the dimension of the current random shift.
\end{tabb}
\begin{hide}\begin{code}

   @Deprecated
   public void addRandomShift (int d1, int d2) {
      addRandomShift (d1, d2, shiftStream);
  }
\end{code}
\begin{tabb} Similar to \texttt{addRandomShift(d1, d2, stream)}, with
  the current random stream.% This method does nothing for this generic class.
\end{tabb}
\begin{code}

   @Deprecated
   public void addRandomShift () {
      addRandomShift (0, dimShift, shiftStream);
   }
\end{code}
\begin{tabb} Similar to \texttt{addRandomShift(0, d2, stream)}
   with  the current random stream and \texttt{d2} the dimension of the current
  random shift.%  This method does nothing for this generic class.
\end{tabb}\end{hide}
\begin{code}

   public void clearRandomShift()\begin{hide} {
      capacityShift = 0;
      dimShift = 0;
//      shiftStream = null;
  }\end{hide}
\end{code}
\begin{tabb}
   Erases the current random shift, if any.
\end{tabb}
\begin{code}

   public void randomize (int d1, int d2, RandomStream stream)\begin{hide} {
      addRandomShift (d1, d2, stream);
   }\end{hide}
\end{code}
\begin{tabb} By default, this method simply calls
  \texttt{addRandomShift(d1, d2, stream)}.
\end{tabb}
\begin{code}

   public void randomize (RandomStream stream)\begin{hide} {
      addRandomShift (stream);
  }\end{hide}
\end{code}
\begin{tabb} By default, this method simply calls
   \texttt{addRandomShift(stream)}.
\end{tabb}
\begin{hide}\begin{code}

   @Deprecated
   public void randomize (int d1, int d2) {
      addRandomShift (d1, d2);
  }
\end{code}
\begin{tabb} By default, this method simply calls \texttt{addRandomShift(d1, d2)}.
\end{tabb}
\begin{code}

   @Deprecated
   public void randomize () {
      addRandomShift();
   }
\end{code}
\begin{tabb} By default, this method simply calls \texttt{addRandomShift()}.
\end{tabb}
\end{hide}\begin{code}

   public void unrandomize()\begin{hide} {
      clearRandomShift();
  }\end{hide}
\end{code}
\begin{tabb} By default, this method simply calls
   \texttt{clearRandomShift()}.
\end{tabb}
%%%%%%%%%%%%%%%%%%%%%%%%%%%%%%%%
\begin{code}

   public String toString() \begin{hide} {
       StringBuffer sb = new StringBuffer ("Number of points: ");
       int x = getNumPoints();
       if (x == Integer.MAX_VALUE)
          sb.append ("infinite");
       else
          sb.append (x);
       sb.append (PrintfFormat.NEWLINE + "Point set dimension: ");
       x = getDimension();
       if (x == Integer.MAX_VALUE)
          sb.append ("infinite");
       else
          sb.append (x);
       return sb.toString();
   }\end{hide}
\end{code}
\begin{tabb}
   Formats a string that contains information about the point set.
\end{tabb}
\begin{htmlonly}
   \return{string representation of the point set information}
\end{htmlonly}
\begin{code}

   public String formatPoints() \begin{hide} {
      PointSetIterator iter = iterator();
      return formatPoints (iter);
   }\end{hide}
\end{code}
\begin{tabb}
   Same as invoking \method{formatPoints}{int,int}\texttt{(n, d)} with $n$ and $d$ equal to the
   number of points and the dimension of this object, respectively.
\end{tabb}
\begin{htmlonly}
   \return{string representation of all the points in the point set}
   \exception{UnsupportedOperationException}{if the number of points
      or dimension of the point set is infinite}
\end{htmlonly}
\begin{code}

   public String formatPoints (int n, int d) \begin{hide} {
      PointSetIterator iter = iterator();
      return formatPoints (iter, n, d);
   }\end{hide}
\end{code}
\begin{tabb}
   Formats a string that displays the same information as returned by
  \method{toString}{}, together with the first $d$ coordinates of the
   first $n$ points. If $n$ is larger than the number of points in the point
  set, it is reset to that number. If $d$ is larger than the dimension of the
 points, it is reset to that dimension. The points are printed in the
  simplest format, separated by spaces,  by calling the default iterator
   repeatedly.
 %  \hpierre{Could perhaps just print what exists.}
\end{tabb}
\begin{htmlonly}
   \param{n}{number of points}
   \param{d}{dimension}
   \return{string representation of first d coordinates of first n points
      in the point set}
\end{htmlonly}
\begin{code}

   public String formatPoints (PointSetIterator iter) \begin{hide} {
      int n = getNumPoints();
      if (n == Integer.MAX_VALUE)
         throw new UnsupportedOperationException (
            "Number of points is infinite");
      int d = getDimension();
      if (d == Integer.MAX_VALUE)
         throw new UnsupportedOperationException ("Dimension is infinite");
      return formatPoints (iter, n, d);
   }\end{hide}
\end{code}
\begin{tabb}
    Same as invoking \method{formatPoints}{PointSetIterator,int,int}\texttt{(iter, n, d)}
    with $n$ and $d$ equal to the number of points and the dimension, respectively.
\end{tabb}
\begin{htmlonly}
  \param{iter}{iterator associated to the point set}
   \return{string representation of all the points in the point set}
   \exception{UnsupportedOperationException}{if the number of points
      or dimension of the point set is infinite}
\end{htmlonly}
\begin{code}

   public String formatPoints (PointSetIterator iter, int n, int d) \begin{hide} {
      if (getNumPoints() < n)
         n = getNumPoints();
      if (getDimension() < d)
         d = getDimension();
      StringBuffer sb = new StringBuffer (toString());
      sb.append (PrintfFormat.NEWLINE + PrintfFormat.NEWLINE
                 + "Points of the point set:" + PrintfFormat.NEWLINE);
      for (int i=0; i<n; i++) {
        for (int j=0; j<d; j++) {
            sb.append ("  ");
            sb.append (iter.nextCoordinate());
         }
         sb.append (PrintfFormat.NEWLINE);
         iter.resetToNextPoint();
      }
      return sb.toString();
   }\end{hide}
\end{code}
\begin{tabb}
  Same as invoking \method{formatPoints}{int,int}\texttt{(n, d)}, but
   prints the points  by calling \texttt{iter} repeatedly. The order of
   the printed points may be different than the one resulting from the
  default iterator.
\end{tabb}
\begin{htmlonly}
  \param{iter}{iterator associated to the point set}
  \param{n}{number of points}
  \param{d}{dimension}
  \return{string representation of first d coordinates of first n points
      in the point set}
\end{htmlonly}
\begin{code}

   public String formatPointsBase (int b) \begin{hide} {
      PointSetIterator iter = iterator();
      return formatPointsBase (iter, b);
   }\end{hide}
\end{code}
\begin{tabb}
Similar to \method{formatPoints}{}\texttt{()}, but the
points coordinates are printed in base $b$.
\end{tabb}
\begin{htmlonly}
  \param{b}{base}
   \return{string representation of all the points in the point set}
   \exception{UnsupportedOperationException}{if the number of points
      or dimension of the point set is infinite}
\end{htmlonly}
\begin{code}

   public String formatPointsBase (int n, int d, int b) \begin{hide} {
      PointSetIterator iter = iterator();
      return formatPointsBase(iter, n, d, b);
   }\end{hide}
\end{code}
\begin{tabb}
Similar to \method{formatPoints}{int,int}\texttt{(n, d)}, but the
 points coordinates are printed in base $b$.
\end{tabb}
\begin{htmlonly}
   \param{n}{number of points}
   \param{d}{dimension}
   \param{b}{base}
   \return{string representation of first d coordinates of first n points
      in the point set}
\end{htmlonly}
\begin{code}

   public String formatPointsBase (PointSetIterator iter, int b) \begin{hide} {
      int n = getNumPoints();
      if (n == Integer.MAX_VALUE)
         throw new UnsupportedOperationException (
            "Number of points is infinite");
      int d = getDimension();
      if (d == Integer.MAX_VALUE)
         throw new UnsupportedOperationException ("Dimension is infinite");
      return formatPointsBase (iter, n, d, b);
   }\end{hide}
\end{code}
\begin{tabb}
Similar to
\method{formatPoints}{PointSetIterator}\texttt{(iter)},
but the points coordinates are printed in base $b$.
\end{tabb}
\begin{htmlonly}
  \param{iter}{iterator associated to the point set}
   \param{b}{base}
   \return{string representation of all the points in the point set}
   \exception{UnsupportedOperationException}{if the number of points
      or dimension of the point set is infinite}
\end{htmlonly}
\begin{code}

   public String formatPointsBase (PointSetIterator iter, int n, int d, int b) \begin{hide} {
      if (getNumPoints() < n)
         n = getNumPoints();
      if (getDimension() < d)
         d = getDimension();
      StringBuffer sb = new StringBuffer (toString());
      sb.append (PrintfFormat.NEWLINE + PrintfFormat.NEWLINE
                 + "Points of the point set:" + PrintfFormat.NEWLINE);
      double x;
      int acc = 10;
      if (b == 2)
         acc = 20;
      else if (b == 3)
         acc = 13;
      else
         acc = 10;
      if (null != shiftStream)
         acc += 6;
      int width = acc + 3;
      String chaine;
      for (int i=0; i<n; i++) {
        for (int j=0; j<d; j++) {
            sb.append ("  ");
            x = iter.nextCoordinate();
            chaine = PrintfFormat.formatBase (-width, acc, b, x);
            sb.append (chaine);
         }
         sb.append (PrintfFormat.NEWLINE);
         iter.resetToNextPoint();
      }
      return sb.toString();
   }\end{hide}
\end{code}
\begin{tabb}
Similar to
\method{formatPoints}{PointSetIterator,int,int}\texttt{(iter, n, d)},
but the points coordinates are printed in base $b$.
\end{tabb}
\begin{htmlonly}
  \param{iter}{iterator associated to the point set}
  \param{n}{number of points}
  \param{d}{dimension}
  \param{b}{base}
  \return{string representation of first d coordinates of first n points
      in the point set}
\end{htmlonly}
\begin{code}

   public String formatPointsNumbered() \begin{hide} {
      int n = getNumPoints();
      if (n == Integer.MAX_VALUE)
         throw new UnsupportedOperationException (
            "Number of points is infinite");
      int d = getDimension();
      if (d == Integer.MAX_VALUE)
         throw new UnsupportedOperationException ("Dimension is infinite");
      return formatPointsNumbered (n, d);
   }\end{hide}
\end{code}
\begin{tabb}
 Same as invoking \method{formatPointsNumbered}{int,int}{\texttt{(n, d)}}
  with $n$ and $d$ equal to the number of points and the dimension,
   respectively.
\end{tabb}
\begin{htmlonly}
   \return{string representation of all the points in the point set}
   \exception{UnsupportedOperationException}{if the number of points
      or dimension of the point set is infinite}
\end{htmlonly}
\begin{code}

   public String formatPointsNumbered (int n, int d) \begin{hide} {
      if (getNumPoints() < n)
         n = getNumPoints();
      if (getDimension() < d)
         d = getDimension();
      StringBuffer sb = new StringBuffer (toString());
      PointSetIterator itr = iterator();
      sb.append (PrintfFormat.NEWLINE + PrintfFormat.NEWLINE
                 + "Points of the point set:");
      for (int i=0; i<n; i++) {
         sb.append (PrintfFormat.NEWLINE + "Point " +
    //                itr.getCurPointIndex() + " = (");
                                           i + "  =  (");
         boolean first = true;
         for (int j=0; j<d; j++) {
            if (first)
               first = false;
            else
               sb.append (", ");
            sb.append (itr.nextCoordinate());
         }
         sb.append (")");
         itr.resetToNextPoint();
      }
      return sb.toString();
   }\end{hide}
\end{code}
\begin{tabb}
   Same as invoking \method{formatPoints}{int,int}{\texttt{(n,d)}}, except that the points are numbered.
\end{tabb}
\begin{htmlonly}
  \param{n}{number of points}
  \param{d}{dimension}
  \return{string representation of first d coordinates of first n points
      in the point set}
\end{htmlonly}

\begin{code}\begin{hide}


// %%%%%%%%%%%%%%%%%%%%%%%%%%%%%%%%%%%%%%%%%%%%%%%%%%%%%%%%%%%%%%%%%%%%%%%%
// This class implements a default point set iterator.
// Since it is inherited by subclasses, it can be used as a base class
// for iterators.
// It is implemented as an inner class because it can then use directly
// the variables of the PointSet class.  It would be more difficult and
// cumbersome to access those variables if it was implemented as a
// separate class.

   protected class DefaultPointSetIterator implements PointSetIterator {

      protected int curPointIndex = 0;      // Index of the current point.
      protected int curCoordIndex = 0;      // Index of the current coordinate.
      protected double EpsilonHalf = 1.0 / Num.TWOEXP[55];
   // protected double EpsilonHalf = PointSet.this.EpsilonHalf;

      protected void outOfBounds () {
         if (getCurPointIndex() >= numPoints)
            throw new NoSuchElementException ("Not enough points available");
         else
            throw new NoSuchElementException ("Not enough coordinates available");
      }

      public void setCurCoordIndex (int j) {
         curCoordIndex = j;
      }

      public void resetCurCoordIndex() {
         setCurCoordIndex (0);
      }

      public int getCurCoordIndex() {
        return curCoordIndex;
      }

      public boolean hasNextCoordinate() {
        return getCurCoordIndex() < getDimension();
      }

      public double nextCoordinate() {
         if (getCurPointIndex() >= numPoints || getCurCoordIndex() >= dim)
            outOfBounds();
         return getCoordinate (curPointIndex, curCoordIndex++);
      }

      public void nextCoordinates (double p[], int d)  {
         if (getCurCoordIndex() + d > getDimension()) outOfBounds();
         for (int j = 0; j < d; j++)
            p[j] = nextCoordinate();
      }

      // This is called with i = numPoints when nextPoint generates the
      // last point, so i = numPoints must be allowed.
      // The "no more point" error will be raised if we ask for
      // a new coordinate or point.
      public void setCurPointIndex (int i) {
         curPointIndex = i;
         resetCurCoordIndex();
      }

      public void resetCurPointIndex() {
         setCurPointIndex (0);
      }

      public int resetToNextPoint() {
         setCurPointIndex (curPointIndex + 1);
         return curPointIndex;
      }

      public int getCurPointIndex() {
        return curPointIndex;
      }

      public boolean hasNextPoint() {
        return getCurPointIndex() < getNumPoints();
      }

      public int nextPoint (double p[], int d) {
         resetCurCoordIndex();
         nextCoordinates (p, d);
         return resetToNextPoint();
      }


      public void resetStartStream() {     // Same as resetCurPointIndex();
         resetCurPointIndex();
      }

      public void resetStartSubstream() {  // Same as resetCurCoordIndex();
         resetCurCoordIndex();
      }

      public void resetNextSubstream() {   // Same as resetToNextPoint();
         resetToNextPoint();
      }

      public void setAntithetic (boolean b) {
         throw new UnsupportedOperationException();
      }

      public double nextDouble() {          // Same as nextCoordinate();
         return nextCoordinate();
      }

      public void nextArrayOfDouble (double[] u, int start, int n) {
         if (n < 0)
            throw new IllegalArgumentException ("n must be positive.");
         for (int i = start; i < start+n; i++)
            u[i] = nextDouble();
      }

      public int nextInt (int i, int j) {
         return (i + (int)(nextDouble() * (j - i + 1.0)));
      }

      public void nextArrayOfInt (int i, int j, int[] u, int start, int n) {
         if (n < 0)
            throw new IllegalArgumentException ("n must be positive.");
         for (int k = start; k < start+n; k++)
            u[k] = nextInt (i, j);
      }

      public String formatState() {
         return "Current point index: " + getCurPointIndex() +
              PrintfFormat.NEWLINE + "Current coordinate index: " +
                  getCurCoordIndex();
      }
   }
}\end{hide}
\end{code}
                        % top class
\defclass{PointSetIterator}

Objects of classes that implement this interface are \emph{iterators} 
that permit one to enumerate
(or observe) the successive points of a point set and the successive 
coordinates of these points.
Each \class{PointSetIterator}{} is associated with a given point set
and maintains a \emph{current point} index $i$ and a \emph{current coordinate}
index $j$, which are both initialized to zero.

Successive coordinates can be accessed one or many at a time by the methods
\method{nextCoor\-di\-nate}{} and \method{nextCoordi\-nates}{}, respectively.
The current coordinate index $j$ can be set explicitely by
\method{setCurCoordIndex}{} and \method{resetCurCoordIndex}{}. 
Similar methods are available for resetting and accessing the current point.
The method \method{nextPoint}{} permits one to 
enumerate the successive points in natural order. 

This class also implements the \class{RandomStream} interface.
This permits one to replace random numbers by the coordinates of
(randomized) quasi-Monte Carlo points without changing the code that calls
the generators in a simulation program.
That is, the same simulation program can be used for both Monte Carlo 
and quasi-Monte Carlo simulations.
The method \method{nextDouble}{} does exactly the same as 
\method{nextCoordinate}{}, it returns the current coordinate of the 
current point and advances the current coordinate by one.  
The substreams correspond to the points, so  
\method{resetStartSubstream}{} resets the current point coordinate to zero, 
\method{resetNextSubstream}{} resets the iterator to the next point, and
\method{resetStartStream}{} resets the iterator to the first point of 
the point set.

There can be several iterators over the same point set.
These iterators are independent from each other.  
Classes that implement this interface must maintain enough information 
so that each iterator is unaffected by other iterator's operations.
However, the iterator does not need to be independent of the underlying 
point set.  If the point set is modified (e.g., randomized), the iterator
may continue to work as usual.

Point set iterators are implemented as inner classes because
this gives a direct access to the private members (or variables) 
of the class.  This is important for efficiency.
They are quite similar to the iterators in Java \emph{collections}.

%%%%%%%%%%%%%%%%%%%%%%%%%%%%%%%%%%%%%%%%%%%%%%%%%%%%%%%%%%%%%%%%%%%%
\bigskip\hrule\bigskip

\begin{code}
\begin{hide}
/*
 * Interface:    PointSetIterator
 * Description:  Iterator over point sets
 * Environment:  Java
 * Software:     SSJ 
 * Copyright (C) 2001  Pierre L'Ecuyer and Université de Montréal
 * Organization: DIRO, Université de Montréal
 * @author       
 * @since

 * SSJ is free software: you can redistribute it and/or modify it under
 * the terms of the GNU General Public License (GPL) as published by the
 * Free Software Foundation, either version 3 of the License, or
 * any later version.

 * SSJ is distributed in the hope that it will be useful,
 * but WITHOUT ANY WARRANTY; without even the implied warranty of
 * MERCHANTABILITY or FITNESS FOR A PARTICULAR PURPOSE.  See the
 * GNU General Public License for more details.

 * A copy of the GNU General Public License is available at
   <a href="http://www.gnu.org/licenses">GPL licence site</a>.
 */
\end{hide}
package umontreal.iro.lecuyer.hups;\begin{hide}

import umontreal.iro.lecuyer.rng.RandomStream;\end{hide}

public interface PointSetIterator extends RandomStream\begin{hide} {\end{hide}

   public void setCurCoordIndex (int j);
\end{code}
 \begin{tabb}
   Sets the current coordinate index to $j$, so that 
   the next calls to \method{nextCoordinate}{} or  \method{nextCoordinates}{}
   will return the values $u_{i,j}, u_{i,j+1}, \dots$, where $i$ is the
   index of the current point.
 \end{tabb}
\begin{htmlonly}
   \param{j}{index of the new current coordinate}
\end{htmlonly}
\begin{code}

   public void resetCurCoordIndex();
\end{code}
 \begin{tabb}
  Equivalent to \method{setCurCoordIndex}{}~\texttt{(0)}.
 \end{tabb}
\begin{code}

   public int getCurCoordIndex();
\end{code}
 \begin{tabb}
   Returns the index $j$ of the current coordinate.  This may be useful,
   e.g., for testing if enough coordinates are still available.
 \end{tabb}
\begin{htmlonly}
   \return{index of the current coordinate}
\end{htmlonly}
\begin{code}

   public boolean hasNextCoordinate();
\end{code}
 \begin{tabb}
   Returns \texttt{true} if the current point has another coordinate.
   This can be useful for testing if coordinates are still available.
 \end{tabb}
\begin{htmlonly}
   \return{\texttt{true} if the current point has another coordinate}
\end{htmlonly}
\begin{code}

   public double nextCoordinate();
\end{code}
 \begin{tabb}
   Returns the current coordinate $u_{i,j}$ and advances to the next one.
   If no current coordinate is available (either because the current
   point index has reached the number of points or because the current
   coordinate index has reached the number of dimensions), it throws a 
   \externalclass{java.util}{NoSuchElementException}.
 \end{tabb}
\begin{htmlonly}
   \return{value of the current coordinate}
   \exception{NoSuchElementException}{if no such coordinate is available}
\end{htmlonly}
\begin{code}

   public void nextCoordinates (double[] p, int d);
\end{code}
 \begin{tabb}
   Returns the next \texttt{d} coordinates of the current point in \texttt{p}
   and advances the current coordinate index by \texttt{d}.
   If the remaining number of coordinates is too small, a
   \texttt{NoSuchElementException} is thrown, as in \method{nextCoordinate}{}.
%   This method does not necessarily check if \texttt{p} is large enough.
 \end{tabb}
\begin{htmlonly}
   \param{p}{array to be filled with the coordinates, starting at index 0}
   \param{d}{number of coordinates to get}
   \exception{NoSuchElementException}{if there are not enough 
             remaining coordinates in the current point}
\end{htmlonly}
\begin{code}

   public void setCurPointIndex (int i);
\end{code}
 \begin{tabb}
   Resets the current point index to $i$ and the current coordinate 
   index to zero.  If \texttt{i} is larger or equal to the number of points,
   an exception will \emph{not} be raised here, but only later if we 
   ask for a new coordinate or point. 
% The next call to either \method{nextCoordinates}{}
% or \method{nextCoordinate}{} will return coordinates starting from $u_{i,0}$.
 \end{tabb}
\begin{htmlonly}
   \param{i}{new index of the current point}
\end{htmlonly}
\begin{code}

   public void resetCurPointIndex();
\end{code}
 \begin{tabb}
   Equivalent to \method{setCurPointIndex}{}~\texttt{(0)}.
 \end{tabb}
\begin{code}

   public int resetToNextPoint();
\end{code}
 \begin{tabb}
   Increases the current point index by 1 and returns its new value.
   If there is no more point, an exception will be raised only if we 
   ask for a new coordinate or point later on. 
%    Equivalent to 
%   \method{setCurPointIndex}{}~\texttt{(}\method{getCurPointIndex()}{}~\texttt{+ 1)}
 \end{tabb}
\begin{htmlonly}
   \return{index of the new current point}
\end{htmlonly}
\begin{code}

   public int getCurPointIndex();
\end{code}
 \begin{tabb}
   Returns the index $i$ of the current point.  
   This can be useful, e.g., for caching point sets.
%   and efficient implementations.
% (see examples in the code).
 \end{tabb}
\begin{htmlonly}
   \return{index of the current point}
\end{htmlonly}
\begin{code}

   public boolean hasNextPoint();
\end{code}
 \begin{tabb}
   Returns \texttt{true} if there is a next point.
   This can be useful for testing if points are still available.
 \end{tabb}
\begin{htmlonly}
   \return{\texttt{true} if a next point is available from the iterated point set}
\end{htmlonly}
\begin{code}

   public int nextPoint (double[] p, int d);\begin{hide}
}\end{hide}
\end{code}
 \begin{tabb}
   Returns the \emph{first} \texttt{d} coordinates of the \emph{current} 
   point in \texttt{p}, advances to the next point, and
   returns the index of the \emph{new} current point.
   Even if the current coordinate index is 0, the point returned
   starts from coordinate 0.
   After obtaining the last point via this method, the current point
   index (returned by the method) is equal to the number of points,
   so it is no longer a valid point index.
   An exception will then be raised if we attempt to generate additional
   points or coordinates.

  Specialized implementations of this method often allow for increased 
  efficiency, e.g., for cycle-based point sets where the cycles 
  (but not the points)
  are stored explicitly or for digital nets  
  by allowing non-incremental point enumerations via Gray-code counters.
 \end{tabb}
\begin{htmlonly}
   \param{p}{array to be filled with the coordinates, 
             starting from array index 0}
   \param{d}{number of coordinates to return}
   \return{index of the new current point}
   \exception{NoSuchElementException}{if there are not enough coordinates 
     available in the current point for filling \texttt{p}}
\end{htmlonly}


% \defclass{Randomization}

Each type of \emph{randomization} that can be applied to a general 
point set should be defined in a class that implements this interface.

\pierre{Specific subclasses have yet to be defined and implemented.
        It is still unclear if we need this in addition to container
        classes that also apply general types of randomizations.}
\hrichard{Nous n'avons jamais utilis\'e cette vieille interface. R\'ecemment,
Adam a cr\'e\'e la nouvelle interface \texttt{PointSetRandomization}.}
\richard{See the new interface \texttt{PointSetRandomization}.}

\bigskip\hrule\bigskip
%%%%%%%%%%%%%%%%%%%%%%%%%%%%%%%%%%%%%%%%%%%%%%%%%%%%%%%%%%%%%%%%%

\begin{code}
\begin{hide}
/*
 * Interface:    Randomization
 * Description:  
 * Environment:  Java
 * Software:     SSJ 
 * Copyright (C) 2001  Pierre L'Ecuyer and Université de Montréal
 * Organization: DIRO, Université de Montréal
 * @author       
 * @since

 * SSJ is free software: you can redistribute it and/or modify it under
 * the terms of the GNU General Public License (GPL) as published by the
 * Free Software Foundation, either version 3 of the License, or
 * any later version.

 * SSJ is distributed in the hope that it will be useful,
 * but WITHOUT ANY WARRANTY; without even the implied warranty of
 * MERCHANTABILITY or FITNESS FOR A PARTICULAR PURPOSE.  See the
 * GNU General Public License for more details.

 * A copy of the GNU General Public License is available at
   <a href="http://www.gnu.org/licenses">GPL licence site</a>.
 */
\end{hide}
package umontreal.iro.lecuyer.hups;\begin{hide}
import umontreal.iro.lecuyer.rng.*;
\end{hide}

@Deprecated
public interface Randomization \begin{hide} {
\end{hide}
\end{code}

%%%%%%%%%%%%%%%%%%%%%%%%%%%%%%%%%
\subsubsection* {Methods}

\begin{code}

   public void apply (RandomStream stream)\begin{hide} ;
     // public double randCoordintate(int i, int j);
}\end{hide}
\end{code}
\begin{tabb}  Applies this randomization to the point set using 
   stream \texttt{stream}.
\end{tabb}
\begin{htmlonly}
   \param{stream}{random number stream used to generate needed uniforms}
\end{htmlonly}

\defclass{PointSetRandomization}

This interface is used to randomize a
\externalclass{umontreal.iro.lecuyer.hups}{PointSet}. One can
implement method \method{randomize}{PointSet} in any way. This
method must use an internal
\externalclass{umontreal.iro.lecuyer.rng}{RandomStream}. This
stream can be set in the constructor, but the methods
\method{getStream}{} and \method{setStream}{RandomStream} must be
implemented.

The method \method{randomize}{PointSet} must be implemented using
combinations of the randomization methods from the point set such
as
\externalmethod{umontreal.iro.lecuyer.hups}{PointSet}{addRandomShift}{},
\externalmethod{umontreal.iro.lecuyer.hups}{DigitalNet}{leftMatrixScramble}{},
\externalmethod{umontreal.iro.lecuyer.hups}{DigitalNet}{striped\-Matrix\-Scramble}{},
\hspace{1pt}\ldots

If more than one \class{PointSetRandomization} is applied to the
same point set, the randomizations will concatenate if they are of
different types, but only the last of each type will remain.

\bigskip\hrule\bigskip
%%%%%%%%%%%%%%%%%%%%%%%%%%%%%%%%%%%%%%%%%%%%%%%%%%%%%%%%%%%%%%%%%%

\begin{code}
\begin{hide}
/*
 * Interface:    PointSetRandomization
 * Description:  Used to randomize a PointSet
 * Environment:  Java
 * Software:     SSJ 
 * Copyright (C) 2001  Pierre L'Ecuyer and Université de Montréal
 * Organization: DIRO, Université de Montréal
 * @author       
 * @since

 * SSJ is free software: you can redistribute it and/or modify it under
 * the terms of the GNU General Public License (GPL) as published by the
 * Free Software Foundation, either version 3 of the License, or
 * any later version.

 * SSJ is distributed in the hope that it will be useful,
 * but WITHOUT ANY WARRANTY; without even the implied warranty of
 * MERCHANTABILITY or FITNESS FOR A PARTICULAR PURPOSE.  See the
 * GNU General Public License for more details.

 * A copy of the GNU General Public License is available at
   <a href="http://www.gnu.org/licenses">GPL licence site</a>.
 */
\end{hide}
package umontreal.iro.lecuyer.hups;
\begin{hide}
import umontreal.iro.lecuyer.rng.RandomStream;

\end{hide}
public interface PointSetRandomization \begin{hide} {
\end{hide}
\end{code}

%%%%%%%%%%%%%%%%%%%%%%%%%%%%
\subsubsection*{Methods}
\begin{code}

   public void randomize (PointSet p);
\end{code}
\begin{tabb}
   This method must randomize \texttt{p}.
\end{tabb}
\begin{htmlonly}
   \param{p}{Point set to randomize}
\end{htmlonly}
\begin{code}

   public void setStream (RandomStream stream);
\end{code}
\begin{tabb}
   Sets the internal
   \externalclass{umontreal.iro.lecuyer.rng}{RandomStream} to
   \texttt{stream}.
\end{tabb}
\begin{htmlonly}
   \param{stream}{stream to use in the randomization}
\end{htmlonly}
\begin{code}

   public RandomStream getStream();
\end{code}
\begin{tabb}
   Returns the internal
   \externalclass{umontreal.iro.lecuyer.rng}{RandomStream}.
\end{tabb}
\begin{htmlonly}
   \return{stream used in the randomization}
\end{htmlonly}
\begin{code}\begin{hide}
}
\end{hide}\end{code}

\include{EmptyRandomization}
\defclass{RandomShift}

This class implements a
\externalclass{umontreal.iro.lecuyer.hups}{PointSetRandomization}.
The \externalclass{umontreal.iro.lecuyer.rng}{RandomStream} is
stored internally. The method \method{randomize}{PointSet} simply
calls
\externalmethod{umontreal.iro.lecuyer.hups}{PointSet}{addRandomShift}{RandomStream}\texttt{(stream)}.

This class can be used as a base class to implement a specific
randomization by overriding method \method{randomize}{PointSet}.

\bigskip\hrule\bigskip
%%%%%%%%%%%%%%%%%%%%%%%%%%%%%%%%%%%%%%%%%%%%%%%%%%%%%%%%%%%%%%%%%%

\begin{code}
\begin{hide}
/*
 * Class:        RandomShift
 * Description:  Applies a random shift on a point set
 * Environment:  Java
 * Software:     SSJ 
 * Copyright (C) 2001  Pierre L'Ecuyer and Université de Montréal
 * Organization: DIRO, Université de Montréal
 * @author       
 * @since

 * SSJ is free software: you can redistribute it and/or modify it under
 * the terms of the GNU General Public License (GPL) as published by the
 * Free Software Foundation, either version 3 of the License, or
 * any later version.

 * SSJ is distributed in the hope that it will be useful,
 * but WITHOUT ANY WARRANTY; without even the implied warranty of
 * MERCHANTABILITY or FITNESS FOR A PARTICULAR PURPOSE.  See the
 * GNU General Public License for more details.

 * A copy of the GNU General Public License is available at
   <a href="http://www.gnu.org/licenses">GPL licence site</a>.
 */
\end{hide}
package umontreal.iro.lecuyer.hups;
\begin{hide}
 import umontreal.iro.lecuyer.rng.RandomStream;

\end{hide}
public class RandomShift implements PointSetRandomization\begin{hide} {
   protected RandomStream stream;
\end{hide}
\end{code}
%%%%%%%%%%%%%%%%%%%%%%%%%%%%
\subsubsection*{Constructors}
\begin{code}

   public RandomShift() \begin{hide} {
   }
   \end{hide}
\end{code}
\begin{tabb}
   Empty constructor.
\end{tabb}
\begin{code}

   public RandomShift (RandomStream stream) \begin{hide} {
       this.stream = stream;
   }
   \end{hide}
\end{code}
\begin{tabb}
   Sets the internal
  \externalclass{umontreal.iro.lecuyer.rng}{RandomStream} to  \texttt{stream}.
\end{tabb}
\begin{htmlonly}
   \param{stream}{stream to use in the randomization}
\end{htmlonly}

%%%%%%%%%%%%%%%%%%%%%%%%%%%%
\subsubsection*{Methods}
\begin{code}

   public void randomize (PointSet p) \begin{hide} {
      p.addRandomShift(stream);
   } \end{hide}
\end{code}
\begin{tabb}
   This method calls
   \externalmethod{umontreal.iro.lecuyer.hups}{PointSet}
   {addRandomShift}{RandomStream}~\texttt{(stream)}.
\end{tabb}
\begin{htmlonly}
   \param{p}{Point set to randomize}
\end{htmlonly}
\begin{code}

   public void setStream (RandomStream stream) \begin{hide} {
      this.stream = stream;
   } \end{hide}
\end{code}
\begin{tabb}
   Sets the internal
   \externalclass{umontreal.iro.lecuyer.rng}{RandomStream} to
   \texttt{stream}.
\end{tabb}
\begin{htmlonly}
   \param{stream}{stream to use in the randomization}
\end{htmlonly}
\begin{code}

   public RandomStream getStream() \begin{hide} {
      return stream;
   } \end{hide}
\end{code}
\begin{tabb}
   Returns the internal
   \externalclass{umontreal.iro.lecuyer.rng}{RandomStream}.
\end{tabb}
\begin{htmlonly}
   \return{stream used in the randomization}
\end{htmlonly}
\begin{code}\begin{hide}
}
\end{hide}\end{code}

\defclass{LMScrambleShift}

This class implements a
\externalclass{umontreal.iro.lecuyer.hups}{PointSetRandomization}
that performs a left matrix scrambling and adds a random digital
shift. Point set must be a
\externalclass{umontreal.iro.lecuyer.hups}{DigitalNet} or an
\externalclass{java.lang}{IllegalArgumentException} is thrown.

\bigskip\hrule\bigskip
%%%%%%%%%%%%%%%%%%%%%%%%%%%%%%%%%%%%%%%%%%%%%%%%%%%%%%%%%%%%%%%%%%

\begin{code}
\begin{hide}
/*
 * Class:        LMScrambleShift
 * Description:  performs a left matrix scramble and adds a random digital shift
 * Environment:  Java
 * Software:     SSJ 
 * Copyright (C) 2001  Pierre L'Ecuyer and Université de Montréal
 * Organization: DIRO, Université de Montréal
 * @author       
 * @since

 * SSJ is free software: you can redistribute it and/or modify it under
 * the terms of the GNU General Public License (GPL) as published by the
 * Free Software Foundation, either version 3 of the License, or
 * any later version.

 * SSJ is distributed in the hope that it will be useful,
 * but WITHOUT ANY WARRANTY; without even the implied warranty of
 * MERCHANTABILITY or FITNESS FOR A PARTICULAR PURPOSE.  See the
 * GNU General Public License for more details.

 * A copy of the GNU General Public License is available at
   <a href="http://www.gnu.org/licenses">GPL licence site</a>.
 */
\end{hide}
package umontreal.iro.lecuyer.hups;
\begin{hide}
 import umontreal.iro.lecuyer.rng.RandomStream;
 import java.lang.IllegalArgumentException;
\end{hide}

public class LMScrambleShift extends RandomShift \begin{hide} {
\end{hide}
\end{code}

%%%%%%%%%%%%%%%%%%%%%%%%%%%%
\subsubsection*{Constructors}
\begin{code}

   public LMScrambleShift() \begin{hide} {
   }
   \end{hide}
\end{code}
\begin{tabb}
   Empty constructor.
\end{tabb}
\begin{code}

   public LMScrambleShift (RandomStream stream) \begin{hide} {
       super(stream);
   }
   \end{hide}
\end{code}
\begin{tabb}
   Sets internal variable \texttt{stream} to the given
   \texttt{stream}.
\end{tabb}
\begin{htmlonly}
   \param{stream}{stream to use in the randomization}
\end{htmlonly}

%%%%%%%%%%%%%%%%%%%%%%%%%%%%
\subsubsection*{Methods}
\begin{code}

   public void randomize (PointSet p) \begin{hide} {
      if (p instanceof DigitalNet) {
         ((DigitalNet)p).leftMatrixScramble (stream);
         ((DigitalNet)p).addRandomShift (stream);
      } else {
         throw new IllegalArgumentException("LMScrambleShift"+
                                            " can only randomize a DigitalNet");
      }
   }
   \end{hide}
\end{code}
\begin{tabb}
   This method calls
   \externalmethod{umontreal.iro.lecuyer.hups}{DigitalNet}{leftMatrixScramble}{RandomStream},
   then
   \externalmethod{umontreal.iro.lecuyer.hups}{DigitalNet}{addRandomShift}{RandomStream}.
   If \texttt{p} is not a
   \externalclass{umontreal.iro.lecuyer.hups}{DigitalNet}, an
\externalclass{java.lang}{IllegalArgumentException} is thrown.
\end{tabb}
\begin{htmlonly}
   \param{p}{Point set to randomize}
\end{htmlonly}
\begin{code}\begin{hide}
}
\end{hide}\end{code}

\include{SMScrambleShift}
\defclass{RandomStart}

This class implements a
\externalclass{umontreal.iro.lecuyer.hups}{PointSetRandomization}
that randomizes a sequence with a random starting point.
The point set must be an instance of 
\externalclass{umontreal.iro.lecuyer.hups}{HaltonSequence} or an
\externalclass{java.lang}{IllegalArgumentException} is thrown.
For now, only the Halton sequence is allowed, but there may be others
later.

\bigskip\hrule\bigskip
%%%%%%%%%%%%%%%%%%%%%%%%%%%%%%%%%%%%%%%%%%%%%%%%%%%%%%%%%%%%%%%%%%

\begin{code}
\begin{hide}
/*
 * Class:        RandomStart
 * Description:  Randomizes a sequence with a random starting point
 * Environment:  Java
 * Software:     SSJ 
 * Copyright (C) 2001  Pierre L'Ecuyer and Université de Montréal
 * Organization: DIRO, Université de Montréal
 * @author       
 * @since

 * SSJ is free software: you can redistribute it and/or modify it under
 * the terms of the GNU General Public License (GPL) as published by the
 * Free Software Foundation, either version 3 of the License, or
 * any later version.

 * SSJ is distributed in the hope that it will be useful,
 * but WITHOUT ANY WARRANTY; without even the implied warranty of
 * MERCHANTABILITY or FITNESS FOR A PARTICULAR PURPOSE.  See the
 * GNU General Public License for more details.

 * A copy of the GNU General Public License is available at
   <a href="http://www.gnu.org/licenses">GPL licence site</a>.
 */
\end{hide}
package umontreal.iro.lecuyer.hups;
\begin{hide}
 import umontreal.iro.lecuyer.rng.RandomStream;
 import java.lang.IllegalArgumentException;
\end{hide}

public class RandomStart implements PointSetRandomization \begin{hide} {

   protected RandomStream stream;
\end{hide}
\end{code}

%%%%%%%%%%%%%%%%%%%%%%%%%%%%
\subsubsection*{Constructors}
\begin{code}

   public RandomStart() \begin{hide} {
   }
   \end{hide}
\end{code}
\begin{tabb}
   Empty constructor.
\end{tabb}
\begin{code}

   public RandomStart (RandomStream stream) \begin{hide} {
       this.stream = stream;
   }
   \end{hide}
\end{code}
\begin{tabb}
   Sets internal variable \texttt{stream} to the given
   \texttt{stream}.
\end{tabb}
\begin{htmlonly}
   \param{stream}{stream to use in the randomization}
\end{htmlonly}

%%%%%%%%%%%%%%%%%%%%%%%%%%%%
\subsubsection*{Methods}
\begin{code}

   public void randomize (PointSet p) \begin{hide} {
      if (p instanceof HaltonSequence) {
         double[] x0 = new double[p.getDimension()];
         stream.nextArrayOfDouble(x0, 0, x0.length);
         ((HaltonSequence)p).setStart (x0);
      } else {
         throw new IllegalArgumentException("RandomStart" +
                     " can only randomize a HaltonSequence");
      }
   }
   \end{hide}
\end{code}
\begin{tabb}
   This method calls
   \externalmethod{umontreal.iro.lecuyer.hups}{HaltonSequence}{init}{double[]}.
   If \texttt{p} is not a
   \externalclass{umontreal.iro.lecuyer.hups}{HaltonSequence}, an
\externalclass{java.lang}{IllegalArgumentException} is thrown.
\end{tabb}
\begin{htmlonly}
   \param{p}{Point set to randomize}
\end{htmlonly}
\begin{code}

   public void setStream (RandomStream stream) \begin{hide} {
      this.stream = stream;
   } \end{hide}
\end{code}
\begin{tabb}
   Sets the internal
   \externalclass{umontreal.iro.lecuyer.rng}{RandomStream} to
   \texttt{stream}.
\end{tabb}
\begin{htmlonly}
   \param{stream}{stream to use in the randomization}
\end{htmlonly}
\begin{code}

   public RandomStream getStream() \begin{hide} {
      return stream;
   } \end{hide}
\end{code}
\begin{tabb}
   Returns the internal
   \externalclass{umontreal.iro.lecuyer.rng}{RandomStream}.
\end{tabb}
\begin{htmlonly}
   \return{stream used in the randomization}
\end{htmlonly}
\begin{code}\begin{hide}
}
\end{hide}\end{code}


\defclass{ContainerPointSet}

This acts as a generic base class for all \emph{container
classes} that contain a point set and apply some kind of
transformation to the coordinates to define a new point set.
One example of such transformation is the \emph{antithetic} map,
applied by the container class \class{AntitheticPointSet},
where each output coordinate $u_{i,j}$ is transformed into $1-u_{i,j}$.
Another example is \class{RandShiftedPointSet}.

The class implements a specialized type of iterator for container
point sets.  This type of iterator contains itself an iterator for
the contained point set and uses it to access the points and coordinates
internally, instead of maintaining itself indices for the current point
and current coordinate.


\bigskip\hrule\bigskip
%%%%%%%%%%%%%%%%%%%%%%%%%%%%%%%%%%%%%%%%%%%%%%%%%%%%%%%%%%%%%%%%%%

\begin{code}
\begin{hide}
/*
 * Class:        ContainerPointSet
 * Description:  
 * Environment:  Java
 * Software:     SSJ 
 * Copyright (C) 2001  Pierre L'Ecuyer and Université de Montréal
 * Organization: DIRO, Université de Montréal
 * @author       
 * @since

 * SSJ is free software: you can redistribute it and/or modify it under
 * the terms of the GNU General Public License (GPL) as published by the
 * Free Software Foundation, either version 3 of the License, or
 * any later version.

 * SSJ is distributed in the hope that it will be useful,
 * but WITHOUT ANY WARRANTY; without even the implied warranty of
 * MERCHANTABILITY or FITNESS FOR A PARTICULAR PURPOSE.  See the
 * GNU General Public License for more details.

 * A copy of the GNU General Public License is available at
   <a href="http://www.gnu.org/licenses">GPL licence site</a>.
 */
\end{hide}
package umontreal.iro.lecuyer.hups;\begin{hide}

import umontreal.iro.lecuyer.util.PrintfFormat;
import umontreal.iro.lecuyer.rng.RandomStream;
\end{hide}

public abstract class ContainerPointSet extends PointSet \begin{hide} {
   protected PointSet P;                 // contained point set
\end{hide}
\end{code}

%%%%%%%%%%%%%%%%%%%%%%%%%%%%
% \subsubsection*{Constructor}
\begin{code}

   protected void init (PointSet P0) \begin{hide} {
      P = P0;
//      this.dim = P.getDimension();
//      this.numPoints = P.getNumPoints();
   }\end{hide}
\end{code}
 \begin{tabb}
   Initializes the container point set which will contain point set \texttt{P0}.
   This method must be called by the constructor of any class inheriting from
   \class{ContainerPointSet}.
 \hpierre{Since this is an abstract class, this constructor should be
  replaced by a method \texttt{init()}, if really needed.}
 \end{tabb}
\begin{htmlonly}
   \param{P}{contained point set}
\end{htmlonly}
\begin{code}

   public PointSet getOriginalPointSet() \begin{hide} {
      return P;
   }\end{hide}
\end{code}
\begin{tabb}
   Returns the (untransformed) point set inside this container.
\end{tabb}
\begin{htmlonly}
   \return{the point set inside this container}
\end{htmlonly}
\begin{code}

   public int getDimension() \begin{hide} {
      return P.getDimension();
   }\end{hide}
\end{code}
\begin{tabb}
   Returns the dimension of the contained point set.
\end{tabb}
\begin{htmlonly}
   \return{the dimension of the contained point set}
\end{htmlonly}
\begin{code}

   public int getNumPoints()\begin{hide} {
      return P.getNumPoints();
   }\end{hide}
\end{code}
\begin{tabb}
   Returns the number of points of the contained point set.
\end{tabb}
\begin{htmlonly}
   \return{the number of points of the contained point set}
\end{htmlonly}
\begin{code}\begin{hide}

   public double getCoordinate(int i, int j) {
      return P.getCoordinate (i, j);
   }

   public PointSetIterator iterator(){
      return new ContainerPointSetIterator();
   }\end{hide}

   public void randomize (PointSetRandomization rand) \begin{hide} {
       P.randomize(rand);
   }\end{hide}
\end{code}
\begin{tabb}
   Randomizes the contained point set using \texttt{rand}.
\end{tabb}
\begin{htmlonly}
   \param{rand}{\class{PointSetRandomization} to use}
\end{htmlonly}
\begin{code}

   public void addRandomShift (int d1, int d2, RandomStream stream)\begin{hide} {
      P.addRandomShift (d1, d2, stream);
   }\end{hide}
\end{code}
\begin{tabb}
  Calls \texttt{addRandomShift(d1, d2, stream)} of the contained point set.
\end{tabb}
\begin{htmlonly}
   \param{d1}{lower dimension of the random shift}
   \param{d2}{upper dimension of the random shift}
   \param{stream}{the random stream}
\end{htmlonly}
\begin{code}

   public void addRandomShift (RandomStream stream)\begin{hide} {
      P.addRandomShift (stream);
   }\end{hide}
\end{code}
\begin{tabb}
  Calls \texttt{addRandomShift(stream)} of the contained point set.
\end{tabb}
\begin{htmlonly}
   \param{stream}{the random stream}
\end{htmlonly}
\begin{code}

   public void clearRandomShift()\begin{hide} {
      P.clearRandomShift ();
   }\end{hide}
\end{code}
\begin{tabb}
  Calls \texttt{clearRandomShift()} of the contained point set.
\end{tabb}
\begin{code}\begin{hide}

   public String toString() {
      return "Container point set of: {" + PrintfFormat.NEWLINE
              + P.toString() + PrintfFormat.NEWLINE + "}";
   }


   // ********************************************************
   protected class ContainerPointSetIterator extends DefaultPointSetIterator {

      protected PointSetIterator innerIterator = P.iterator();

      public void setCurCoordIndex (int j) {
         innerIterator.setCurCoordIndex (j);
      }

      public void resetCurCoordIndex() {
         innerIterator.resetCurCoordIndex();
      }

      public boolean hasNextCoordinate() {
         return innerIterator.hasNextCoordinate();
      }

      public double nextCoordinate() {
         return innerIterator.nextCoordinate();
      }

      public void setCurPointIndex (int i) {
         innerIterator.setCurPointIndex(i);
      }

      public void resetCurPointIndex() {
         innerIterator.resetCurPointIndex();
      }

      public int resetToNextPoint() {
         return innerIterator.resetToNextPoint();
      }

      public boolean hasNextPoint() {
        return innerIterator.hasNextPoint();
      }

   }
}\end{hide}
\end{code}

\defclass{CachedPointSet}

This container class caches a point set by precomputing
and storing its points locally in an array.
This can be used to speed up computations when using
a small low-dimensional point set more than once.

\begin{detailed} %%
After the points are stored in the array, this class uses
the default methods and the default iterator type provided by
the base class \class{PointSet}{}.
This is one of the rare cases where direct use of the
\method{getCoordinate}{} method is efficient.
\pierre {We could also implement an iterator that directly returns
  \texttt{x[i][j]} instead of calling \texttt{getCoordinate}, for slightly
  better efficiency.  On the other hand, even better efficiency can
  be achieved by getting an entire point at a time in an array.
} However, it might require too much memory for a large point set.
\end{detailed} %%

% No array index range check is performed in this class,
% neither for the dimension nor for the number of points.
% However, this is done by Java.

\bigskip\hrule\bigskip
%%%%%%%%%%%%%%%%%%%%%%%%%%%%%%%%%%%%%%%%%%%%%%%%%%%%%%%%%%%%%%%%%%

\begin{code}
\begin{hide}
/*
 * Class:        CachedPointSet
 * Description:  
 * Environment:  Java
 * Software:     SSJ 
 * Copyright (C) 2001  Pierre L'Ecuyer and Université de Montréal
 * Organization: DIRO, Université de Montréal
 * @author       
 * @since

 * SSJ is free software: you can redistribute it and/or modify it under
 * the terms of the GNU General Public License (GPL) as published by the
 * Free Software Foundation, either version 3 of the License, or
 * any later version.

 * SSJ is distributed in the hope that it will be useful,
 * but WITHOUT ANY WARRANTY; without even the implied warranty of
 * MERCHANTABILITY or FITNESS FOR A PARTICULAR PURPOSE.  See the
 * GNU General Public License for more details.

 * A copy of the GNU General Public License is available at
   <a href="http://www.gnu.org/licenses">GPL licence site</a>.
 */
\end{hide}
package umontreal.iro.lecuyer.hups;\begin{hide}

import umontreal.iro.lecuyer.util.PrintfFormat;
\end{hide}
    import umontreal.iro.lecuyer.rng.RandomStream;


public class CachedPointSet extends PointSet \begin{hide} {
   protected PointSet P;        // Original PointSet which is cached here.
   protected double x[][];      // Cached points.
   protected CachedPointSet() {}
\end{hide}
\end{code}

%%%%%%%%%%%%%%%%%%%%%%%%%%%%
\subsubsection*{Constructors}
\begin{code}

   public CachedPointSet (PointSet P, int n, int dim) \begin{hide} {
      if (P.getNumPoints() < n)
         throw new IllegalArgumentException(
            "Cannot cache more points than in point set P.");
      if (P.getDimension() < dim)
         throw new IllegalArgumentException(
            "Cannot cache points with more coordinates than the dimension.");
      numPoints = n;
      this.dim = dim;
      this.P = P;
      init ();
   }

   protected void init () {
      PointSetIterator itr = P.iterator();
      x = new double[numPoints][dim];
      for (int i = 0; i < numPoints; i++)
         itr.nextPoint (x[i], dim);
   }\end{hide}
\end{code}
 \begin{tabb}
   Creates a new \texttt{PointSet} object that contains an array storing
   the first \texttt{dim} coordinates of the first \texttt{n} points of \texttt{P}.
   The original point set \texttt{P} itself is not modified.
%, except for its point and coordinate iterators.
 \end{tabb}
\begin{htmlonly}
   \param{P}{point set to be cached}
   \param{n}{number of points}
   \param{dim}{number of dimensions of the points}
\end{htmlonly}
\begin{code}

   public CachedPointSet (PointSet P) \begin{hide} {
      numPoints = P.getNumPoints();
      dim = P.getDimension();
      if (numPoints == Integer.MAX_VALUE)
         throw new IllegalArgumentException(
            "Cannot cache infinite number of points");
      if (dim == Integer.MAX_VALUE)
         throw new IllegalArgumentException(
            "Cannot cache infinite dimensional points");
      this.P = P;
      init ();
   }\end{hide}
\end{code}
 \begin{tabb}
   Creates a new \texttt{PointSet} object that contains an array storing
   the points of \texttt{P}.
   The number of points and their dimension are the same as in the
   original point set.  Both must be finite.
%  The point set \texttt{P} itself is not modified.
 \end{tabb}
\begin{htmlonly}
   \param{P}{point set to be cached}
\end{htmlonly}


%%%%%%%%%%%%%%%%%%%%%%%%%%%%
\subsubsection*{Methods}
\begin{code}

   public void addRandomShift(int d1, int d2, RandomStream stream)\begin{hide} {
        P.addRandomShift(d1, d2, stream);
        init();
   }\end{hide}
\end{code}
\begin{tabb}
Add the shift to the contained point set and recaches the points. See the doc of the
overridden method \externalmethod{umontreal.iro.lecuyer.hups}{PointSet}{addRandomShift}{(int, int, RandomStream)}\texttt{(d1, d2, stream)} in \class{PointSet}.
\end{tabb}
\begin{code}

   public void randomize (PointSetRandomization rand)\begin{hide} {
      P.randomize(rand);
      init();
   }\end{hide}
\end{code}
\begin{tabb}
Randomizes the underlying point set using \texttt{rand} and
recaches the points.
\end{tabb}
\begin{code}\begin{hide}

   public String toString() {
     StringBuffer sb = new StringBuffer ("Cached point set" +
          PrintfFormat.NEWLINE);
     sb.append (super.toString());
     sb.append (PrintfFormat.NEWLINE + "Cached point set information {"
                + PrintfFormat.NEWLINE);
     sb.append (P.toString());
     sb.append (PrintfFormat.NEWLINE + "}");
     return sb.toString();
   }

   public double getCoordinate (int i, int j) {
      return x[i][j];
   }

}
\end{hide}
\end{code}

% \include{SortedPointSet}
\include{SubsetOfPointSet}
\include{PaddedPointSet}
\defclass{AntitheticPointSet}

This container class provides antithetic points.
That is, $1 - u_{i,j}$ is returned in place of coordinate $u_{i,j}$.
To generate regular and antithetic variates with a point
set \texttt{p}, e.g., for variance reduction, one can define an
\class{AntitheticPointSet} object \texttt{pa} that contains \texttt{p},
and then generate the regular variates with \texttt{p} and the
antithetic variates with \texttt{pa}.


\hpierre{Perhaps we should have a container of a stream rather
        than a container of a point set.}

\bigskip\hrule\bigskip

%%%%%%%%%%%%%%%%%%%%%%%%%%%%%%%%%%%%%%%%%%%%%%%%%%%%%%%%%%%%%%%%%%
\begin{code}
\begin{hide}
/*
 * Class:        AntitheticPointSet
 * Description:  
 * Environment:  Java
 * Software:     SSJ 
 * Copyright (C) 2001  Pierre L'Ecuyer and Université de Montréal
 * Organization: DIRO, Université de Montréal
 * @author       
 * @since

 * SSJ is free software: you can redistribute it and/or modify it under
 * the terms of the GNU General Public License (GPL) as published by the
 * Free Software Foundation, either version 3 of the License, or
 * any later version.

 * SSJ is distributed in the hope that it will be useful,
 * but WITHOUT ANY WARRANTY; without even the implied warranty of
 * MERCHANTABILITY or FITNESS FOR A PARTICULAR PURPOSE.  See the
 * GNU General Public License for more details.

 * A copy of the GNU General Public License is available at
   <a href="http://www.gnu.org/licenses">GPL licence site</a>.
 */
\end{hide}
package umontreal.iro.lecuyer.hups;\begin{hide}

import umontreal.iro.lecuyer.util.PrintfFormat;
\end{hide}

public class AntitheticPointSet extends ContainerPointSet \begin{hide} {
\end{hide}
\end{code}

%%%%%%%%%%%%%%%%%%%%%%%%%%%%
\subsubsection*{Constructor}
\begin{code}

   public AntitheticPointSet (PointSet P) \begin{hide} {
      init (P);
   }\end{hide}
\end{code}
 \begin{tabb}
   Constructs an antithetic point set from the given point set \texttt{P}.
 \end{tabb}
\begin{htmlonly}
   \param{P}{point set for which we want antithetic version}
\end{htmlonly}
\begin{code}
\begin{hide}

   public double getCoordinate (int i, int j) {
      return 1.0 - P.getCoordinate (i, j);
   }

   public PointSetIterator iterator(){
      return new AntitheticPointSetIterator();
   }

   public String toString() {
      return "Antithetic point set of: {" + PrintfFormat.NEWLINE +
              P.toString() + PrintfFormat.NEWLINE + "}";
   }


   // ***************************************************************

   protected class AntitheticPointSetIterator
                   extends ContainerPointSetIterator {

      public double nextCoordinate() {
         return 1.0 - innerIterator.nextCoordinate();
      }

      public double nextDouble() {
         return 1.0 - innerIterator.nextCoordinate();
      }

      public void nextCoordinates (double p[], int d)  {
         innerIterator.nextCoordinates (p, d);
         for (int j = 0; j < d; j++)
            p[j] = 1.0 - p[j];
      }

      public int nextPoint (double p[], int d)  {
         innerIterator.nextPoint (p, d);
         for (int j = 0; j < d; j++)
            p[j] = 1.0 - p[j];
         return getCurPointIndex();
      }

   }
}\end{hide}
\end{code}

\defclass{RandShiftedPointSet}

\adam{Cette classe reprogramme un \texttt{addRandomShift} explicitement sur
l'ensemble $P$, alors
que le \texttt{ContainerPointSet} applique le \texttt{addRandomShift} sur le
contenu $P$,  ce qui est beaucoup plus propre et plus g\'en\'eral.
Faut-il \'eliminer cette classe? Est-elle vraiment n\'ecessaire?}
%
This container class embodies a point set to which a random shift
modulo 1 is applied (i.e., a single uniform random point is added
to all points, modulo 1, to randomize the inner point set).
When calling \method{addRandomShift}{}, a new random shift will be generated.
This shift is represented by a vector of $d$ uniforms over $(0,1)$,
where $d$ is the current dimension of the shift.

% The shift is executed \emph{after} all other randomizations in the
% randomization list, if any.


\bigskip\hrule\bigskip

%%%%%%%%%%%%%%%%%%%%%%%%%%%%%%%%%%%%%%%%%%%%%%%%%%%%%%%%%%%%%%%%%%%%%%%%%%%%%%%
\begin{code}
\begin{hide}
/*
 * Class:        RandShiftedPointSet
 * Description:  Point set to which a random shift modulo 1 is applied
 * Environment:  Java
 * Software:     SSJ 
 * Copyright (C) 2001  Pierre L'Ecuyer and Université de Montréal
 * Organization: DIRO, Université de Montréal
 * @author       
 * @since

 * SSJ is free software: you can redistribute it and/or modify it under
 * the terms of the GNU General Public License (GPL) as published by the
 * Free Software Foundation, either version 3 of the License, or
 * any later version.

 * SSJ is distributed in the hope that it will be useful,
 * but WITHOUT ANY WARRANTY; without even the implied warranty of
 * MERCHANTABILITY or FITNESS FOR A PARTICULAR PURPOSE.  See the
 * GNU General Public License for more details.

 * A copy of the GNU General Public License is available at
   <a href="http://www.gnu.org/licenses">GPL licence site</a>.
 */
\end{hide}
package umontreal.iro.lecuyer.hups;\begin{hide}

import umontreal.iro.lecuyer.util.PrintfFormat;
import umontreal.iro.lecuyer.rng.*;
\end{hide}

public class RandShiftedPointSet extends ContainerPointSet \begin{hide} {

   protected double[] shift;           // The random shift.
   protected int dimShift = 0;         // Current dimension of the shift.
   protected int capacityShift = 0;    // Number of array elements;
                                       // always >= dimShift.
   protected RandomStream shiftStream; // Used to generate random shifts.
\end{hide}
\end{code}

%%%%%%%%%%%%%%%%%%%%%%%%%%%%%%%%%%%%%%%%%%%%%%%%%%%%%%%%%%%%%%%%%
\subsubsection*{Constructor}

\begin{code}

   public RandShiftedPointSet (PointSet P, int dimShift, RandomStream stream)\begin{hide} {
      init (P);
      if (dimShift <= 0) {
         throw new IllegalArgumentException (
            "Cannot construct RandShiftedPointSet with dimShift <= 0");
      }
      shiftStream = stream;
      shift = new double [dimShift];
      capacityShift = this.dimShift = dimShift;
   }\end{hide}
\end{code}
 \begin{tabb}
  Constructs a structure to contain a randomly shifted version of \texttt{P}.
  The random shifts will be generated in up to \texttt{dimShift} dimensions,
  using stream \texttt{stream}.
 \end{tabb}
\begin{htmlonly}
   \param{P}{point set being randomized}
   \param{dimShift}{dimension of the initial shift}
   \param{stream}{stream used for generating random shifts}
\end{htmlonly}

%%%%%%%%%%%%%%%%%%%%%%%%%%%%%%%%%
\subsubsection*{Methods}
\begin{code}

   public int getShiftDimension()\begin{hide} {
      return dimShift;
   }\end{hide}
\end{code}
 \begin{tabb}
  Returns the number of dimensions of the current random shift.
 \end{tabb}
\begin{code}

   public void addRandomShift (int d1, int d2, RandomStream stream) \begin{hide} {
      if (null == stream)
         throw new IllegalArgumentException (
              PrintfFormat.NEWLINE +
                  "   Calling addRandomShift with null stream");
      if (stream != shiftStream)
         shiftStream = stream;
      addRandomShift (d1, d2);
   }\end{hide}
\end{code}
 \begin{tabb}
  Changes the stream used for the random shifts to \texttt{stream}, then
  refreshes the shift for coordinates \texttt{d1} to \texttt{d2-1}.
\richard{Il y a 4 m\'ethodes \texttt{addRandomShift}. Peut-\^etre faudrait-il
en \'eliminer 2, comme dans \texttt{PointSet}.}
 \end{tabb}
\begin{code}

   public void addRandomShift (RandomStream stream) \begin{hide} {
      if (stream != shiftStream)
         shiftStream = stream;
      addRandomShift (0, dimShift);
   }\end{hide}
\end{code}
 \begin{tabb}
  Changes the stream used for the random shifts to \texttt{stream}, then
  refreshes all coordinates of the random shift, up to its current dimension.
 \end{tabb}
\begin{hide}
\begin{code}

   @Deprecated
   public void addRandomShift (int d1, int d2)\begin{hide} {
      if (d1 < 0 || d1 > d2)
         throw new IllegalArgumentException ("illegal parameter d1 or d2");
      if (d2 > capacityShift) {
         int d3 = Math.max (4, capacityShift);
         while (d2 > d3)
            d3 *= 2;
         double[] temp = new double[d3];
         capacityShift = d3;
         for (int i = 0; i < d1; i++)
            temp[i] = shift[i];
         shift = temp;
      }
      dimShift = d2;
      for (int i = d1; i < d2; i++)
         shift[i] = shiftStream.nextDouble();

 // Just for testing, to see the single uniform random point
 //     for(int k = 0; k < d2; k++)
 //       System.out.println ("shift " + k + " = " + shift[k]);
 //     System.out.println();

   }\end{hide}
\end{code}
 \begin{tabb}
  Refreshes the random shift (generates new uniform values for the
  random shift coordinates) for coordinates \texttt{d1} to \texttt{d2-1}.
 \end{tabb}
\begin{code}

   @Deprecated
   public void addRandomShift() \begin{hide} {
      addRandomShift (0, dimShift);
   }\end{hide}
\end{code}
 \begin{tabb}
  Refreshes all coordinates of the random shift, up to its current dimension.
 \end{tabb}
\end{hide}
\begin{code}\begin{hide}

   public String toString() {
      return "RandShiftedPointSet of: {" + PrintfFormat.NEWLINE
              + P.toString() + PrintfFormat.NEWLINE + "}";
   }

   public PointSetIterator iterator() {
      return new RandShiftedPointSetIterator();
   }

   // ***************************************************************

   private class RandShiftedPointSetIterator
                 extends ContainerPointSetIterator {

      public double getCoordinate (int i, int j) {
         int d1 = innerIterator.getCurCoordIndex();
         if (dimShift <= d1)
            // Must extend randomization.
            addRandomShift (dimShift, 1 + d1);
         double u = P.getCoordinate(i, j) + shift[j];
         if (u >= 1.0)
            u -= 1.0;
         if (u > 0.0)
            return u;
         return EpsilonHalf;  // avoid u = 0
      }

      public double nextCoordinate() {
         int d1 = innerIterator.getCurCoordIndex();
         if (dimShift <= d1)
            addRandomShift (dimShift,1 + d1 );
         double u = shift [d1];
         u += innerIterator.nextCoordinate();
         if (u >= 1.0)
            u -= 1.0;
         if (u > 0.0)
            return u;
         return EpsilonHalf;  // avoid u = 0
      }

   }
}\end{hide}
\end{code}

%% \include{RandXoredPointSet}  % remove ?
\include{BakerTransformedPointSet}

\include{CycleBasedPointSet}
\defclass {LCGPointSet}

Implements a recurrence-based point set defined via a linear 
congruential recurrence of the form $x_i = a x_{i-1} \mod n$
and $u_i = x_i / n$.  The implementation is done by storing the values
of $u_i$ over the set of all cycles of the recurrence.
 

\bigskip\hrule\bigskip

%%%%%%%%%%%%%%%%%%%%%%%%%%%%%%%%%%%%%%%%%%%%%%%%%%%%%%%%%%%%%%%%%%%%%%%%%%%%%%%
\begin{code}
\begin{hide}
/*
 * Class:        LCGPointSet
 * Description:  point set defined via a linear congruential recurrence
 * Environment:  Java
 * Software:     SSJ 
 * Copyright (C) 2001  Pierre L'Ecuyer and Université de Montréal
 * Organization: DIRO, Université de Montréal
 * @author       
 * @since

 * SSJ is free software: you can redistribute it and/or modify it under
 * the terms of the GNU General Public License (GPL) as published by the
 * Free Software Foundation, either version 3 of the License, or
 * any later version.

 * SSJ is distributed in the hope that it will be useful,
 * but WITHOUT ANY WARRANTY; without even the implied warranty of
 * MERCHANTABILITY or FITNESS FOR A PARTICULAR PURPOSE.  See the
 * GNU General Public License for more details.

 * A copy of the GNU General Public License is available at
   <a href="http://www.gnu.org/licenses">GPL licence site</a>.
 */
\end{hide}
package umontreal.iro.lecuyer.hups;\begin{hide}
import umontreal.iro.lecuyer.util.PrintfFormat;
import cern.colt.list.*;
\end{hide}

public class LCGPointSet extends CycleBasedPointSet \begin{hide} {

      private int a;                      // Multiplier.

\end{hide}
\end{code}

%%%%%%%%%%%%%%%%%%%%%%%%%%%%
\subsubsection* {Constructors}
\begin{code}

   public LCGPointSet (int n, int a) \begin{hide} {
      this.a = a;
      double invn = 1.0 / (double)n;   // 1/n
      DoubleArrayList c;  // Array used to store the current cycle.
      long currentState;  // The state currently visited. 
      int i;
      boolean stateVisited[] = new boolean[n];  
         // Indicates which states have been visited so far.
      for (i = 0; i < n; i++)
         stateVisited[i] = false;
      int startState = 0;    // First state of the cycle currently considered.
      numPoints = 0;
      while (startState < n) {
         stateVisited[startState] = true;
         c = new DoubleArrayList();
         c.add (startState * invn);
         // We use the fact that a "long" has 64 bits in Java.
         currentState = (startState * (long)a) % (long)n;
         while (currentState != startState) {
            stateVisited[(int)currentState] = true;
            c.add (currentState * invn);
            currentState = (currentState * (long)a) % (long)n;
            }
         addCycle (c);
         for (i = startState+1; i < n; i++)
            if (stateVisited[i] == false)
                break;
         startState = i;
         }
      }\end{hide}
\end{code}
 \begin{tabb} Constructs and stores the set of cycles for an LCG with
   modulus $n$ and multiplier $a$.
  If the LCG has full period length $n-1$,
% pgcd$(a,n)=1$, 
  there are two cycles, the first one containing only 0 
  and the second one of period length $n-1$.
 \end{tabb}
\begin{htmlonly}
   \param{n}{required number of points and modulus of the LCG}
   \param{a}{generator $a$ of the LCG}
\end{htmlonly}
\begin{code}

   public LCGPointSet (int b, int e, int c, int a) \begin{hide} {
      this (computeModulus (b, e, c), a);
   }\end{hide}
\end{code}
 \begin{tabb} Constructs and stores the set of cycles for an LCG with
   modulus $n = b^e + c$ and multiplier $a$.
 \end{tabb}
\begin{code}\begin{hide}
   private static int computeModulus (int b, int e, int c) {
      int n;
      int i;
      if (b == 2) 
         n = (1 << e);
      else {
         for (i = 1, n = b;  i < e;  i++)  n *= b;
         }
      n += c;
      return n;
   }


   public String toString() {
      StringBuffer sb = new StringBuffer ("LCGPointSet:" +
                                           PrintfFormat.NEWLINE);
      sb.append (super.toString());
      sb.append (PrintfFormat.NEWLINE + "Multiplier a: ");
      sb.append (a);
      return sb.toString();
   }
\end{hide}

   public int geta () \begin{hide} {
      return a;
   }
}\end{hide}
\end{code}
 \begin{tabb} Returns the value of the multiplier $a$.
 \end{tabb}

\defclass {CycleBasedPointSetBase2}

Similar to \class{CycleBasedPointSet}, except that the successive
values in the cycles are stored as integers in the range
$\{0,\dots,2^k-1\}$, where $1\le k \le 31$.
The output values $u_{i,j}$ are obtained by dividing these integer
values by $2^k$.  Point sets where the successive coordinates of each
point are obtained via linear recurrences modulo 2 (e.g., linear feedback
shift registers or Korobov-type polynomial lattice rules)
are naturally expressed in this form.
Storing the integers $2^k u_{i,j}$ instead of the $u_{i,j}$ themselves
makes it easier to apply randomizations such as digital random shifts
in base 2, which are applied to the bits \emph{before} transforming
the value to a real number $u_{i,j}$. When a random digital shift is
performed, it applies a bitwise exclusive-or of all the points with a single
  random point.
% \richard{Ne faudrait-il pas un constructeur explicite pour cette classe?}

\bigskip\hrule\bigskip

%%%%%%%%%%%%%%%%%%%%%%%%%%%%%%%%%%%%%%%%%%%%%%%%%%%%%%%%%%%%%%%%%%%%%%%%%%%%
\begin{code}
\begin{hide}
/*
 * Class:        CycleBasedPointSetBase2
 * Description:  
 * Environment:  Java
 * Software:     SSJ 
 * Copyright (C) 2001  Pierre L'Ecuyer and Université de Montréal
 * Organization: DIRO, Université de Montréal
 * @author       
 * @since

 * SSJ is free software: you can redistribute it and/or modify it under
 * the terms of the GNU General Public License (GPL) as published by the
 * Free Software Foundation, either version 3 of the License, or
 * any later version.

 * SSJ is distributed in the hope that it will be useful,
 * but WITHOUT ANY WARRANTY; without even the implied warranty of
 * MERCHANTABILITY or FITNESS FOR A PARTICULAR PURPOSE.  See the
 * GNU General Public License for more details.

 * A copy of the GNU General Public License is available at
   <a href="http://www.gnu.org/licenses">GPL licence site</a>.
 */
\end{hide}
package umontreal.iro.lecuyer.hups;\begin{hide}

import umontreal.iro.lecuyer.util.PrintfFormat;
import umontreal.iro.lecuyer.rng.RandomStream;
import cern.colt.list.*;
\end{hide}

public abstract class CycleBasedPointSetBase2 extends CycleBasedPointSet\begin{hide} {

// dim = Integer.MAX_VALUE;     // Dimension is infinite.
   private int[] digitalShift;  // Digital shift, initially zero (null).
                                // Entry j is for dimension j.
   protected int numBits;       // Number of bits in stored values.
   protected double normFactor; // To convert output to (0,1); 1/2^numBits.

\end{hide}
\end{code}

%%%%%%%%%%%%%%%%%%%%%%%%%%%%
% \subsubsection* {Methods}

\begin{code}
\begin{hide}

   public double getCoordinate (int i, int j) {
      // Find cycle that contains point i, then index in cycle.
      int l = 0;         // Length of next cycle.
      int n = 0;         // Total length of cycles added so far.
      int k;
      for (k = 0;  n <= i;  k++)
         n += l = ((AbstractList) cycles.get (k)).size();
      AbstractList curCycle = (AbstractList) cycles.get (k-1);
      int[] curCycleI = ((IntArrayList) curCycle).elements();
      int coordinate = (i - n + l + j) % curCycle.size();
      int shift = 0;
      if (digitalShift != null) {
         shift = digitalShift[j];
         return (shift ^ curCycleI[coordinate]) * normFactor + EpsilonHalf;
      } else
         return (shift ^ curCycleI[coordinate]) * normFactor;
   }

   public PointSetIterator iterator() {
      return new CycleBasedPointSetBase2Iterator ();
   }
\end{hide}

   public void addRandomShift (int d1, int d2, RandomStream stream) \begin{hide} {
      if (null == stream)
         throw new IllegalArgumentException (
              PrintfFormat.NEWLINE +
              "   Calling addRandomShift with null stream");
      if (0 == d2)
         d2 = Math.max (1, dim);
      if (digitalShift == null) {
         digitalShift = new int[d2];
         capacityShift = d2;
      } else if (d2 > capacityShift) {
         int d3 = Math.max (4, capacityShift);
         while (d2 > d3)
            d3 *= 2;
         int[] temp = new int[d3];
         capacityShift = d3;
         for (int i = 0; i < dimShift; i++)
            temp[i] = digitalShift[i];
         digitalShift = temp;
      }
      dimShift = d2;
      int maxj;
      if (numBits < 31) {
         maxj = (1 << numBits) - 1;
      } else {
         maxj = 2147483647;
      }
      for (int i = d1; i < d2; i++)
         digitalShift[i] = stream.nextInt (0, maxj);
      shiftStream = stream;

   }\end{hide}
\end{code}
\begin{tabb}  Adds a random digital shift in base 2 to all the points
  of the point set, using stream \texttt{stream} to generate the random numbers,
  for coordinates \texttt{d1} to \texttt{d2 - 1}.
  This applies a bitwise exclusive-or of all the points with a single
  random point.
\end{tabb}
\begin{hide}\begin{code}

   public void clearRandomShift() {
      super.clearRandomShift();
      digitalShift = null;
   }
\end{code}
\begin{tabb}
   Erases the current digital random shift, if any.
\end{tabb}\end{hide}
\begin{code}
\begin{hide}

   public String formatPoints() {
      StringBuffer sb = new StringBuffer (toString());
      for (int c = 0; c < numCycles; c++) {
         AbstractList curCycle = (AbstractList)cycles.get (c);
         int[] cycle = ((IntArrayList)curCycle).elements();
         sb.append (PrintfFormat.NEWLINE + "Cycle " + c + ": (");
         boolean first = true;
         for (int e = 0; e < curCycle.size(); e++) {
            if (first)
               first = false;
            else
               sb.append (", ");
            sb.append (cycle[e]);
         }
         sb.append (")");
      }
      return sb.toString();
   }

   // %%%%%%%%%%%%%%%%%%%%%%%%%%%%%%%%%%%%%%%%%%%%%%%%%%%%%%%%%%%%%%%%

   public class CycleBasedPointSetBase2Iterator
                   extends CycleBasedPointSetIterator {

      protected int[] curCycleI;          // The array for current cycle

      public CycleBasedPointSetBase2Iterator () {
         super ();
         resetCurCycle (0);
      }

      protected void init() { }

      public void resetCurCycle (int index) {
         curCycleIndex = index;
         curCycle = (AbstractList) cycles.get (index);
         curCycleI = ((IntArrayList) curCycle).elements();
      }

      public double nextCoordinate() {
          // First, verify if there are still points....
          if (curPointIndex >= numPoints)
             outOfBounds();
          int x = curCycleI [curCoordInCycle];
          if (digitalShift != null) {
             if (curCoordIndex >= dimShift)   // Extend the shift.
                addRandomShift (dimShift, curCoordIndex + 1, shiftStream);
             x ^= digitalShift[curCoordIndex];
          }
          curCoordIndex++;
          curCoordInCycle++;
          if (curCoordInCycle >= curCycle.size())
             curCoordInCycle = 0;
          if (digitalShift == null)
             return x * normFactor;
          else
             return x * normFactor + EpsilonHalf;
     }

      public void nextCoordinates (double p[], int dim) {
         // First, verify if there are still points....
         if (curPointIndex >= numPoints)
            outOfBounds();
         if (curCoordIndex + dim >= dimShift)
            addRandomShift (dimShift, curCoordIndex + dim + 1, shiftStream);
         int j = curCoordInCycle;
         int maxj = curCycle.size();
         int x;
         for (int i = 0; i < dim; i++) {
            x = curCycleI [curCoordInCycle++];
            if (curCoordInCycle >= maxj) curCoordInCycle = 0;
            if (digitalShift == null)
               p[i] = x * normFactor;
            else
               p[i] = (digitalShift[curCoordIndex + i] ^ x) * normFactor + EpsilonHalf;
         }
         curCoordIndex += dim;
      }

      public int nextPoint (double p[], int dim) {
         if (getCurPointIndex() >= getNumPoints ())
            outOfBounds();
         curCoordIndex = 0;
         curCoordInCycle = startPointInCycle;
         nextCoordinates (p, dim);
         resetToNextPoint();
         return curPointIndex;
      }
   }
}
\end{hide}
\end{code}

\defclass{F2wStructure}

This class implements methods and fields needed by the classes
 \externalclass{umontreal.iro.lecuyer.hups}{F2wNetLFSR},
 \externalclass{umontreal.iro.lecuyer.hups}{F2wNetPolyLCG},
 \externalclass{umontreal.iro.lecuyer.hups}{F2wCycleBasedLFSR} and
 \externalclass{umontreal.iro.lecuyer.hups}{F2wCycleBasedPolyLCG}.
It also stores the parameters of these point sets which will contain
$2^{rw}$ points (see the meaning of $r$ and $w$ below).
The parameters can be stored as a polynomial $P(z)$ over
 $\latex{\mathbb{F}}\html{\mathbf{F}}_{2^w}[z]$
$$
P(z) = z^{r} + \sum_{i=1}^{r} b_i z^{r-i}
$$
where $b_i\in \latex{\mathbb{F}}\html{\mathbf{F}}_{2^w}$ for $i=1,\ldots,r$.
 Let $\zeta$ be the root of an irreducible polynomial
 $Q(z)\in \latex{\mathbb{F}}\html{\mathbf{F}}_2[z]$.  It is well known
that $\zeta$ is a generator of the finite field
 $\latex{\mathbb{F}}\html{\mathbf{F}}_{2^w}$.
The elements of $\latex{\mathbb{F}}\html{\mathbf{F}}_{2^w}$ are
 represented using the polynomial ordered
 basis $(1,\zeta,\ldots,\zeta^{w-1})$.

In this class, only the non-zero coefficients of $P(z)$ are stored.
 It is stored as
$$
P(z) = z^{\mathtt{r}} + \sum_{i=0}^{\mathtt{nbcoeff}} {\mathtt{coeff[}}i{\mathtt{]}}
   z^{{\mathtt{nocoeff[}}i{\mathtt{]}}}
$$
where the coefficients in \texttt{coeff[]} represent the non-zero
 coefficients $b_i$ of $P(z)$ using the polynomial basis.
The finite field $\latex{\mathbb{F}}\html{\mathbf{F}}_{2^w}$ used is
 defined by the polynomial
$$
Q(z) = z^{w} +  \sum_{i=1}^{w} a_i z^{w-i}
$$
where $a_i\in \latex{\mathbb{F}}\html{\mathbf{F}}_{2}$,
 for $i=1,\ldots,w$. Polynomial $Q$ is
 stored as the bit vector {\texttt{modQ}} = $(a_w,\ldots,a_1)$.

The class also stores the parameter \texttt{step} that is used by the classes
\externalclass{umontreal.iro.lecuyer.hups}{F2wNetLFSR},
 \externalclass{umontreal.iro.lecuyer.hups}{F2wNetPolyLCG},
 \externalclass{umontreal.iro.lecuyer.hups}{F2wCycleBasedLFSR} and
 \externalclass{umontreal.iro.lecuyer.hups}{F2wCycleBasedPolyLCG}.
This parameter is such that the implementation of the recurrence
 will output a value  at every {\texttt{step}} iterations.

\bigskip\hrule\bigskip
%%%%%%%%%%%%%%%%%%%%%%%%%%%%%%%%%%%%%%%%%%%%%%%%%%%%%%%%%%%%%%%%%%%%%%%%%%%%

\begin{code}
\begin{hide}
/*
 * Class:        F2wStructure
 * Description:  Tools for point sets and sequences based on field F_{2^w}
 * Environment:  Java
 * Software:     SSJ 
 * Copyright (C) 2001  Pierre L'Ecuyer and Université de Montréal
 * Organization: DIRO, Université de Montréal
 * @author       
 * @since

 * SSJ is free software: you can redistribute it and/or modify it under
 * the terms of the GNU General Public License (GPL) as published by the
 * Free Software Foundation, either version 3 of the License, or
 * any later version.

 * SSJ is distributed in the hope that it will be useful,
 * but WITHOUT ANY WARRANTY; without even the implied warranty of
 * MERCHANTABILITY or FITNESS FOR A PARTICULAR PURPOSE.  See the
 * GNU General Public License for more details.

 * A copy of the GNU General Public License is available at
   <a href="http://www.gnu.org/licenses">GPL licence site</a>.
 */
\end{hide}
package umontreal.iro.lecuyer.hups;\begin{hide}
import java.io.*;
import java.util.*;
\end{hide}

public class F2wStructure \begin{hide} {

   private final int ALLONES = 2147483647; // 2^31-1 --> 01111...1
   int w;
   int r;
   int numBits;
   private int modQ;
   private int step;
   private int[] coeff;
   private int[] nocoeff;
   private int nbcoeff;
   int S;
   private int maskw;
   private int maskrw;
   private int maskZrm1;
   private int mask31;
   private int t;
   private int masktrw;
   private int[] maskv;
   int state;
   int output;            // augmented state
   double normFactor;
   double EpsilonHalf;
   final static int MBL = 140; //maximum of bytes in 1 line
   //92 bytes for a number of coeff = 15


   private void init (int w, int r, int modQ, int step,
      int nbcoeff, int coeff[], int nocoeff[])
   {
      normFactor = 1.0 / (1L << 31); // 4.65661287307739258e-10;
      EpsilonHalf = 0.5*normFactor;
      numBits = 31;
      this.step = step;
      this.w = w;
      this.r = r;
      S = 31 - r * w;
      mask31 = ~(1 << 31);
      maskw = (1 << w) - 1;
      maskrw = ((1 << (r * w)) - 1) << S;
      maskZrm1 = (ALLONES >> (r * w)) ^ (ALLONES >> ((r - 1) * w));
      this.modQ = modQ;
      this.nbcoeff = nbcoeff;
      this.nocoeff = new int[nbcoeff];
      this.coeff = new int[nbcoeff];
      for (int j = 0; j < nbcoeff; j++) {
         this.nocoeff[j] = nocoeff[j];
         this.coeff[j] = coeff[j];
      }
   }

   void initParamLFSR ()
   {
      t = (31 - r * w) / w;
      masktrw = (~0) << (31 - (t + r) * w) & mask31;
      maskv = new int[r];
      for (int j = 0; j < r; j++) {
         maskv[j] = maskw << (S + ((r - 1 - j) * w));
      }
   }
\end{hide}
\end{code}

%%%%%%%%%%%%%%%%%%%%%%%%%%%%%%%%%
\subsubsection* {Constructors}
\begin{code}


   F2wStructure (int w, int r, int modQ, int step, int nbcoeff,
                 int coeff[], int nocoeff[]) \begin{hide}
   {
      init (w, r, modQ, step, nbcoeff, coeff, nocoeff);
   }
\end{hide}
\end{code}
\begin{tabb}
  Constructs a \texttt{F2wStructure} object that contains  the parameters of a
  polynomial in $\latex{\mathbb{F}}\html{\mathbf{F}}_{2^w}[z]$,
 as well as a stepping parameter.
\end{tabb}
\begin{code}

   F2wStructure (String filename, int no)\begin{hide}
   {
     // If filename can be found starting from the program's directory,
     // it will be used; otherwise, the filename in the Jar archive will
     // be used.
     BufferedReader input;
     try {
       if ((new File (filename)).exists()) {
          input = new BufferedReader (new FileReader (filename));
       } else {
          DataInputStream dataInput;
          dataInput = new DataInputStream (
             F2wStructure.class.getClassLoader().getResourceAsStream (
                 "umontreal/iro/lecuyer/hups/dataF2w/" + filename));
          input = new BufferedReader (new InputStreamReader (dataInput));
       }
       initFromReader (filename, input, no);
       input.close ();

     } catch (Exception e) {
       System.out.println ("IO Error: problems finding file " + filename);
       System.exit (1);
     }
   }\end{hide}
\end{code}
 \begin{tabb}
   Constructs a polynomial in $\latex{\mathbb{F}}\html{\mathbf{F}}_{2^w}[z]$
   after reading its parameters from file {\texttt{filename}};
   the parameters of this polynomial are stored  at line number
   {\texttt{no}} of {\texttt{filename}}.
   The files are kept in different
   directories depending on the criteria used in the searches for the
   parameters defining the polynomials. The different criteria for the
   searches and the theory behind it are described in \cite{rPAN04d,rPAN04t}.
   The existing files and the number of polynomials they contain are
   given in the following tables.
   The first table below contains files in subdirectory
    \texttt{LFSR\_equid\_max}. The name of each
   file indicates the value of $r$ and $w$ for the polynomials.
   For example, file \texttt{f2wR2\_W5.dat} in directory
   \texttt{LFSR\_equid\_max} contains the parameters of 2358
   polynomials with $r=2$ and $w=5$. For example, to use the 5\textit{-th}
    polynomial of file \texttt{f2wR2\_W5.dat}, one may call
   \texttt{F2wStructure("f2wR2\_W5.dat", 5)}.
   The files of parameters have been stored at the address
   \url{http://www.iro.umontreal.ca/~simardr/ssj-1/dataF2w/}.

 \end{tabb}
\begin{code}
\begin{hide}

   private int multiplyz (int a, int k)
   {
      int i;
      if (k == 0)
         return a & maskw;
      else {
         for (i = 0; i < k; i++) {
            if ((1 & a) == 1) {
               a = (a >> 1) ^ modQ;
            } else
               a = a >> 1;
         }
         return a & maskw;
      }
   }\end{hide}
\end{code}
%%%%%%%%%%%%%%%%%%%%%%%%%%%%%%%%%
\subsubsection* {Methods}
\begin{code}

   int getLog2N ()\begin{hide}
   {
      return r * w;
   }\end{hide}
\end{code}
 \begin{tabb}
  This method returns the product $rw$.
 \end{tabb}
\begin{code}

   int multiply (int a, int b)\begin{hide}

   {
      int i;
      int res = 0, verif = 1;
      for (i = 0; i < w; i++) {
         if ((b & verif) == verif)
            res ^= multiplyz (a, w - 1 - i);
         verif <<= 1;
      }
      return res & maskw;
   }\end{hide}
\end{code}
 \begin{tabb}
  Method that multiplies two elements in
 $\latex{\mathbb{F}}\html{\mathbf{F}}_{2^w}$.
 \end{tabb}
\begin{code} \begin{hide}

   void initF2wLFSR ()     // Initialisation de l'etat d'un LFSR
   {
      int v, d = 0;
      int tempState;

      tempState = state << S;
      output = tempState;
      for (int i = 1; i <= t; i++) {
         d = 0;
         for (int j = 0; j < nbcoeff; j++) {
            v = (tempState & maskv[nocoeff[j]]) >>
                 (S + (r - 1 - nocoeff[j]) * w);
            d ^= multiply (coeff[j], v);
         }
         output |= d << (S - i * w);
         tempState = (output << (i * w)) & maskrw;
      }
   }


   void F2wLFSR ()       // Une iteration d'un LFSR
   {
      int v, d = 0;
      int tempState;
      for (int i = 0; i < step; i++) {
         tempState = (output << (t * w)) & maskrw;
         d = 0;
         for (int j = 0; j < nbcoeff; j++) {
            v = (tempState & maskv[nocoeff[j]]) >>
                (S + (r - 1 - nocoeff[j]) * w);
            d ^= multiply (coeff[j], v);
         }
         output = ((output << w) & masktrw) |
                  (d << (31 - (r + t) * w));
      }
      state = (output & maskrw) >> S;
   }


   int F2wPolyLCG ()    // Une iteration d'un PolyLCG
   {
      int Zrm1, d;
      for (int i = 0; i < step; i++) {
         Zrm1 = (state & maskZrm1) >> S;
         state = (state >> w) & maskrw;
         for (int j = 0; j < nbcoeff; j++)
            state ^=
               multiply (coeff[j], Zrm1) << (S + (r - 1 - nocoeff[j]) * w);
      }
      return state;
   }\end{hide}

   public static void print (String filename)\begin{hide}
   {
     BufferedReader input;
     try {
       if ((new File (filename)).exists()) {
          input = new BufferedReader (new FileReader (filename));
       } else {
          DataInputStream dataInput;
          dataInput = new DataInputStream (
             F2wStructure.class.getClassLoader().getResourceAsStream (
                 "umontreal/iro/lecuyer/hups/dataF2w/" + filename));
          input = new BufferedReader (new InputStreamReader (dataInput));
       }

     String s;
     for (int i = 0; i < 4; i++)
        input.readLine ();
     while ((s = input.readLine ()) != null)
        System.out.println (s);
     input.close ();

     } catch (Exception e) {
       System.out.println ("IO Error: problems reading file " + filename);
       System.exit (1);
     }
   }\end{hide}
\end{code}
 \begin{tabb}
    Prints the content of file \texttt{filename}. See the constructor
    above for the conditions on \texttt{filename}.
 \end{tabb}
\begin{code}

   public String toString ()\begin{hide}
   {
      StringBuffer sb = new StringBuffer ("z^");
      sb.append (r);
      for (int j = nbcoeff - 1; j >= 0; j--)
         sb.append (" + (" + coeff[j] + ") z^" + nocoeff[j]);
      sb.append ("   modQ = " + modQ + "    w = " + w + "   step = " + step);
      return sb.toString ();
   }
\end{hide}
\end{code}
\begin{tabb}
  This method returns a string containing the polynomial $P(z)$ and
 the stepping parameter.
\end{tabb}

\begin{code} \begin{hide}
    private void initFromReader (String filename, BufferedReader input, int no)
    {
      int w, r, modQ, step, nbcoeff;
      int coeff[], nocoeff[];
      StringTokenizer line;
      int nl = no + 4;

      try {
        for (int j = 1; j < nl ; j++)
          input.readLine ();

        line = new StringTokenizer (input.readLine ());
        w = Integer.parseInt (line.nextToken ());
        r = Integer.parseInt (line.nextToken ());
        modQ = Integer.parseInt (line.nextToken ());
        step = Integer.parseInt (line.nextToken ());
        nbcoeff = Integer.parseInt (line.nextToken ());
        nocoeff = new int[nbcoeff];
        coeff = new int[nbcoeff];
        for (int i = 0; i < nbcoeff; i++) {
          coeff[i] = Integer.parseInt (line.nextToken ());
          nocoeff[i] = Integer.parseInt (line.nextToken ());
        }
        init (w, r, modQ, step, nbcoeff, coeff, nocoeff);
        input.close ();

      } catch (Exception e) {
        System.out.println ("Input Error: problems reading file " + filename);
        System.exit (1);
      }
    }
  }
  \end{hide}
\end{code}

\tt
\begin{minipage}{7cm}
   \begin {tabular}{|c|c|}
   \multicolumn{2}{c} {{\rm Directory} LFSR\_equid\_max} \\
\hline
  Filename     &  Num of poly.  \\
\hline
  f2wR2\_W5.dat  & 2358   \\
  f2wR2\_W6.dat  & 1618   \\
  f2wR2\_W7.dat  & 507    \\
  f2wR2\_W8.dat  & 26     \\
  f2wR2\_W9.dat  & 3      \\
  f2wR3\_W4.dat  & 369    \\
  f2wR3\_W5.dat  & 26     \\
  f2wR3\_W6.dat  & 1      \\
  f2wR4\_W3.dat  & 117    \\
  f2wR4\_W4.dat  & 1      \\
  f2wR5\_W2.dat  & 165    \\
  f2wR5\_W3.dat  & 1      \\
  f2wR6\_W2.dat  & 36     \\
  f2wR6\_W3.dat  & 1      \\
  f2wR7\_W2.dat  & 10     \\
  f2wR8\_W2.dat  & 11     \\
  f2wR9\_W2.dat  & 1      \\
\hline
\end {tabular}
\end{minipage}
\hfill
\begin{minipage}{7cm}
\begin {tabular}{|c|c|}
 \multicolumn{2}{c} {{\rm  Directory} LFSR\_equid\_sum} \\
\hline
  Filename     &  Num of poly.  \\
\hline
 f2wR2\_W5.dat  & 2276     \\
 f2wR2\_W6.dat  & 1121     \\
 f2wR2\_W7.dat  & 474      \\
 f2wR2\_W8.dat  & 37       \\
 f2wR2\_W9.dat  & 6        \\
 f2wR3\_W4.dat  & 381      \\
 f2wR3\_W5.dat  & 65       \\
 f2wR3\_W6.dat  & 7        \\
 f2wR4\_W3.dat  & 154      \\
 f2wR4\_W4.dat  & 2        \\
 f2wR5\_W2.dat  & 688      \\
 f2wR5\_W3.dat  & 5        \\
 f2wR6\_W2.dat  & 70       \\
 f2wR6\_W3.dat  & 1        \\
 f2wR7\_W2.dat  & 9        \\
 f2wR8\_W2.dat  & 3        \\
 f2wR9\_W2.dat  & 3        \\
\hline
\end {tabular}
\end{minipage}
\bigskip

\begin{minipage}{7cm}
\begin {tabular}{|c|c|}
 \multicolumn{2}{c} {{\rm Directory} LFSR\_mindist\_max} \\
\hline
  Filename     &  Num of poly.  \\
\hline
 f2wR2\_W5.dat  & 1    \\
 f2wR2\_W6.dat  & 1    \\
 f2wR2\_W7.dat  & 2    \\
 f2wR2\_W8.dat  & 2    \\
 f2wR2\_W9.dat  & 1    \\
 f2wR3\_W4.dat  & 2    \\
 f2wR3\_W5.dat  & 2    \\
 f2wR3\_W6.dat  & 1    \\
 f2wR4\_W3.dat  & 1    \\
 f2wR4\_W4.dat  & 1    \\
 f2wR5\_W2.dat  & 2    \\
 f2wR5\_W3.dat  & 1    \\
 f2wR6\_W2.dat  & 4    \\
 f2wR6\_W3.dat  & 1    \\
 f2wR7\_W2.dat  & 1    \\
 f2wR8\_W2.dat  & 1    \\
 f2wR9\_W2.dat  & 1    \\
\hline
\end {tabular}
\end{minipage}
\hfill
\begin{minipage}{7cm}
\begin {tabular}{|c|c|}
\multicolumn{2}{c} {{\rm Directory} LFSR\_mindist\_sum} \\
\hline
  Filename     &  Num of poly.  \\
\hline
  f2wR2\_W5.dat  & 1   \\
  f2wR2\_W6.dat  & 1   \\
  f2wR2\_W7.dat  & 1   \\
  f2wR2\_W8.dat  & 1   \\
  f2wR2\_W9.dat  & 1   \\
  f2wR3\_W4.dat  & 1   \\
  f2wR3\_W5.dat  & 1   \\
  f2wR3\_W6.dat  & 1   \\
  f2wR4\_W3.dat  & 1   \\
  f2wR4\_W4.dat  & 2   \\
  f2wR5\_W2.dat  & 2   \\
  f2wR5\_W3.dat  & 2   \\
  f2wR6\_W2.dat  & 1   \\
  f2wR6\_W3.dat  & 1   \\
  f2wR7\_W2.dat  & 2   \\
  f2wR8\_W2.dat  & 1   \\
  f2wR9\_W2.dat  & 2   \\
\hline
\end {tabular}
\end{minipage}

\bigskip
\begin{minipage}{7cm}
\begin {tabular}{|c|c|}
   \multicolumn{2}{c} {{\rm  Directory} LFSR\_tvalue\_max} \\
\hline
  Filename     &  Num of poly.  \\
\hline
 f2wR2\_W5.dat  & 7     \\
 f2wR2\_W6.dat  & 1     \\
 f2wR2\_W7.dat  & 1     \\
 f2wR2\_W8.dat  & 1     \\
 f2wR2\_W9.dat  & 1     \\
 f2wR3\_W4.dat  & 1     \\
 f2wR3\_W5.dat  & 1     \\
 f2wR3\_W6.dat  & 1     \\
 f2wR4\_W3.dat  & 2     \\
 f2wR4\_W4.dat  & 1     \\
 f2wR5\_W2.dat  & 14    \\
 f2wR5\_W3.dat  & 1     \\
 f2wR6\_W2.dat  & 2     \\
 f2wR6\_W3.dat  & 1     \\
 f2wR7\_W2.dat  & 1     \\
 f2wR8\_W2.dat  & 1     \\
 f2wR9\_W2.dat  & 1     \\
\hline
\end {tabular}
\end{minipage}
\hfill
\begin{minipage}{7cm}
\begin {tabular}{|c|c|}
   \multicolumn{2}{c} {{\rm  Directory} LFSR\_tvalue\_sum} \\
\hline
  Filename     &  Num of poly.  \\
\hline
 f2wR2\_W5.dat  & 15    \\
 f2wR2\_W6.dat  & 1     \\
 f2wR2\_W7.dat  & 1     \\
 f2wR2\_W8.dat  & 2     \\
 f2wR2\_W9.dat  & 1     \\
 f2wR3\_W4.dat  & 1     \\
 f2wR3\_W5.dat  & 1     \\
 f2wR3\_W6.dat  & 1     \\
 f2wR4\_W3.dat  & 2     \\
 f2wR4\_W4.dat  & 1     \\
 f2wR5\_W2.dat  & 13    \\
 f2wR5\_W3.dat  & 2     \\
 f2wR6\_W2.dat  & 12    \\
 f2wR6\_W3.dat  & 1     \\
 f2wR7\_W2.dat  & 1     \\
 f2wR8\_W2.dat  & 1     \\
 f2wR9\_W2.dat  & 1     \\
\hline
\end {tabular}
\end{minipage}
\rm

\defclass{F2wCycleBasedLFSR}

This class creates a point set based upon a linear feedback shift register
 sequence. The recurrence used to produce the point set is
$$
  m_n = \sum_{i=1}^r b_i m_{n-i}
$$
where $m_n\in \latex{\mathbb{F}}\html{\mathbf{F}}_{2^w}$, $n\geq 0$
  and $b_i\in \latex{\mathbb{F}}\html{\mathbf{F}}_{2^w}$.
There is a polynomial in $\latex{\mathbb{F}}\html{\mathbf{F}}_{2^w}[z]$
 associated with this recurrence called
the \emph{characteristic polynomial}.  It is
$$
  P(z) =  z^r + \sum_{i=1}^r b_i z^{r-i}.
$$
In the implementation, this polynomial is stored in an object
% \externalclass{umontreal.iro.lecuyer.hups}{F2wStructure}.
 \texttt{F2wStructure}.
% See the description of this class for more details.

Let ${\mathbf{x}} = (x^{(0)}, \ldots, x^{(p-1)}) \in
\latex{\mathbb{F}}\html{\mathbf{F}}_{2}^p$
be a $p$-bit vector.
Let us define the function $\phi(\mathbf{x}) = \sum_{i=1}^{p} 2^{-i} x^{(i-1)}$.
The point set in $t$ dimensions produced by this class is
$$
 \left\{ (\phi(\mathbf{y}_0),\phi(\mathbf{y}_s),\ldots,\phi(\mathbf{y}_{s(t-1)}):
  (\mathbf{v}_0,\ldots,\mathbf{v}_{r-1})\in
 \latex{\mathbb{F}}\html{\mathbf{F}}_{2}^{rw}\right\}
$$
where $\mathbf{y}_n = \mbox{trunc}_h(\mathbf{v}_n, \mathbf{v}_{n+1},\ldots)$,
$\mathbf{v}_n$ is the representation of $m_n$ under the polynomial basis of
$\latex{\mathbb{F}}\html{\mathbf{F}}_{2^w}$ over
 $\latex{\mathbb{F}}\html{\mathbf{F}}_2$, and
 $h=w\latex{\lfloor 31/w\rfloor}\html{\mbox{ floor}(31/w)}$.
The parameter $s$ is called the stepping parameter of the recurrence.

\bigskip\hrule\bigskip

%%%%%%%%%%%%%%%%%%%%%%%%%%%%%%%%%%%%%%%%%%%%%%%%%%%%%%%%%%%%%%%%%%
\begin{code}
\begin{hide}
/*
 * Class:        F2wCycleBasedLFSR
 * Description:  point set based upon a linear feedback shift register
                 sequence
 * Environment:  Java
 * Software:     SSJ 
 * Copyright (C) 2001  Pierre L'Ecuyer and Université de Montréal
 * Organization: DIRO, Université de Montréal
 * @author       
 * @since

 * SSJ is free software: you can redistribute it and/or modify it under
 * the terms of the GNU General Public License (GPL) as published by the
 * Free Software Foundation, either version 3 of the License, or
 * any later version.

 * SSJ is distributed in the hope that it will be useful,
 * but WITHOUT ANY WARRANTY; without even the implied warranty of
 * MERCHANTABILITY or FITNESS FOR A PARTICULAR PURPOSE.  See the
 * GNU General Public License for more details.

 * A copy of the GNU General Public License is available at
   <a href="http://www.gnu.org/licenses">GPL licence site</a>.
 */
\end{hide}
package umontreal.iro.lecuyer.hups; \begin{hide}
import cern.colt.list.*;
import umontreal.iro.lecuyer.util.PrintfFormat;
\end{hide}


public class F2wCycleBasedLFSR extends CycleBasedPointSetBase2 \begin{hide} {

   private F2wStructure param;
\end{hide}\end{code}


%%%%%%%%%%%%%%%%%%%%%%%%%%%%
\subsubsection*{Constructors}
\begin{code}

   public F2wCycleBasedLFSR (int w, int r, int modQ, int step, int nbcoeff,
                             int coeff[], int nocoeff[])\begin{hide}
    /**
     * Constructs and stores the set of cycles for an LCG with
     *    modulus <SPAN CLASS="MATH"><I>n</I></SPAN> and multiplier <SPAN CLASS="MATH"><I>a</I></SPAN>.
     *   If pgcd<SPAN CLASS="MATH">(<I>a</I>, <I>n</I>) = 1</SPAN>, this constructs a full-period LCG which has two
     *   cycles, one containing 0 and one, the LCG period.
     *
     * @param n required number of points and modulo of the LCG
     *
     *    @param a generator <SPAN CLASS="MATH"><I>a</I></SPAN> of the LCG
     *
     *
     */
   {
      param = new F2wStructure (w, r, modQ, step, nbcoeff, coeff, nocoeff);
      init ();
   }\end{hide}
\end{code}
 \begin{tabb}
Constructs a point set with $2^{rw}$ points.  See the description of the class
\externalclass{umontreal.iro.lecuyer.hups}{F2wStructure} for the meaning
 of the parameters.
 \end{tabb}
\begin{code}

   public F2wCycleBasedLFSR (String filename, int no) \begin{hide}
   {
      param = new F2wStructure (filename, no);
      init ();
   }\end{hide}
\end{code}
 \begin{tabb}
   Constructs a point set after reading its parameters from
   file \texttt{filename}; the parameters are located at line numbered \texttt{no}
   of  \texttt{filename}.  The available files are listed in the description of class
\externalclass{umontreal.iro.lecuyer.hups}{F2wStructure}.
 \end{tabb}
\begin{code}
\begin{hide}

   private void init ()
   {
      param.initParamLFSR ();
      normFactor = param.normFactor;
      EpsilonHalf = param.EpsilonHalf;
      numBits = param.numBits;
      fillCyclesLFSR ();
   }

   public String toString ()
   {
       String s = "F2wCycleBasedLFSR:" + PrintfFormat.NEWLINE;
       return s + param.toString ();
   }

   private void fillCyclesLFSR ()
   {
      int n = 1 << param.getLog2N ();
      IntArrayList c;             // Array used to store the current cycle.
      int currentState;           // The state currently visited.
      int i;
      boolean stateVisited[] = new boolean[n];

      // Indicates which states have been visited so far.
      for (i = 0; i < n; i++)
         stateVisited[i] = false;
      int startState = 0;    // First state of the cycle currently considered.
      numPoints = 0;
      while (startState < n) {
         stateVisited[startState] = true;
         c = new IntArrayList ();
         param.state = startState;
         param.initF2wLFSR ();
         c.add (param.output);
         param.F2wLFSR ();
         // Warning: watch for overflow !!!
         while (param.state != startState) {
            stateVisited[param.state] = true;
            c.add (param.output);
            param.F2wLFSR ();
         }
         addCycle (c);
         for (i = startState + 1; i < n; i++)
            if (stateVisited[i] == false)
               break;
         startState = i;
      }
   }
}
\end{hide}
\end{code}

\defclass{F2wCycleBasedPolyLCG}

This class creates a point set based upon
a linear congruential sequence in the finite field
 $\latex{\mathbb{F}}\html{\mathbf{F}}_{2^w}[z]/P(z)$.
The recurrence is
$$
q_n(z) = z^s q_{n-1}(z)\ \mod P(z)
$$
where $P(z)\in \latex{\mathbb{F}}\html{\mathbf{F}}_{2^w}[z]$ has degree $r$ and
$q_n(z) = q_{n,1} z^{r-1} + \cdots + q_{n,r} \in
 \latex{\mathbb{F}}\html{\mathbf{F}}_{2^w}[z]/P(z)$.
The parameter $s$ is
called the stepping parameter of the recurrence.
The polynomial $P(z)$ is not necessarily the characteristic polynomial
of this recurrence, but it can still be used to store the parameters of
 the recurrence.
In the implementation, it is stored in an object of the class
 \externalclass{umontreal.iro.lecuyer.hups}{F2wStructure}.  See the description
of this class for more details on how the polynomial
is stored.


Let $\mathbf{x} = (x^{(0)}, \ldots, x^{(p-1)}) \in
 \latex{\mathbb{F}}\html{\mathbf{F}}_{2}^p$ be a $p$-bit vector.
Let us define the function $\phi(\mathbf{x}) =
\sum_{i=1}^{p} 2^{-i} x^{(i-1)}$.
The point set in $t$ dimensions produced by this class is
$$
\{ (\phi(\mathbf{y}_0),\phi(\mathbf{y}_1),\ldots,\phi(\mathbf{y}_{t-1}):
 (\mathbf{q}_{0,1},\ldots,\mathbf{q}_{0,r-1})\in
 \latex{\mathbb{F}}\html{\mathbf{F}}_{2}^{rw}\}
$$
where $\mathbf{y}_n = (\mathbf{q}_{n,1},\ldots,\mathbf{q}_{n,r})$,
 $\mathbf{q}_{n,i}$
 is the representation of $q_{n,i}$ under the polynomial basis of
$\latex{\mathbb{F}}\html{\mathbf{F}}_{2^w}$ over
 $\latex{\mathbb{F}}\html{\mathbf{F}}_2$.

\bigskip\hrule\bigskip

%%%%%%%%%%%%%%%%%%%%%%%%%%%%%%%%%%%%%%%%%%%%%%%%%%%%%%%%%%%%%%%%%%
\begin{code}
\begin{hide}
/*
 * Class:        F2wCycleBasedPolyLCG
 * Description:  point set based upon a linear congruential sequence in a
                 finite field
 * Environment:  Java
 * Software:     SSJ 
 * Copyright (C) 2001  Pierre L'Ecuyer and Université de Montréal
 * Organization: DIRO, Université de Montréal
 * @author       
 * @since

 * SSJ is free software: you can redistribute it and/or modify it under
 * the terms of the GNU General Public License (GPL) as published by the
 * Free Software Foundation, either version 3 of the License, or
 * any later version.

 * SSJ is distributed in the hope that it will be useful,
 * but WITHOUT ANY WARRANTY; without even the implied warranty of
 * MERCHANTABILITY or FITNESS FOR A PARTICULAR PURPOSE.  See the
 * GNU General Public License for more details.

 * A copy of the GNU General Public License is available at
   <a href="http://www.gnu.org/licenses">GPL licence site</a>.
 */
\end{hide}
package umontreal.iro.lecuyer.hups; \begin{hide}
import umontreal.iro.lecuyer.util.PrintfFormat;
import umontreal.iro.lecuyer.rng.*;
import cern.colt.list.*;
\end{hide}


public class F2wCycleBasedPolyLCG extends CycleBasedPointSetBase2 \begin{hide} {

   private F2wStructure param;
\end{hide}\end{code}


%%%%%%%%%%%%%%%%%%%%%%%%%%%%
\subsubsection*{Constructors}
\begin{code}

   public F2wCycleBasedPolyLCG (int w, int r, int modQ, int step, int nbcoeff,
                                int coeff[], int nocoeff[]) \begin{hide}
    /**
     * Constructs and stores the set of cycles for an LCG with
     *    modulus <SPAN CLASS="MATH"><I>n</I></SPAN> and multiplier <SPAN CLASS="MATH"><I>a</I></SPAN>.
     *   If pgcd<SPAN CLASS="MATH">(<I>a</I>, <I>n</I>) = 1</SPAN>, this constructs a full-period LCG which has two
     *   cycles, one containing 0 and one, the LCG period.
     *
     * @param n required number of points and modulo of the LCG
     *
     *    @param a generator <SPAN CLASS="MATH"><I>a</I></SPAN> of the LCG
     *
     *
     */
   {
      param = new F2wStructure (w, r, modQ, step, nbcoeff, coeff, nocoeff);
      numBits = param.numBits;
      normFactor = param.normFactor;
      EpsilonHalf = param.EpsilonHalf;
      fillCyclesPolyLCG ();
   }\end{hide}
\end{code}
 \begin{tabb}
 Constructs a point set with $2^{rw}$ points.  See the description of the class
 \externalclass{umontreal.iro.lecuyer.hups}{F2wStructure}
 for the meaning of the parameters.
 \end{tabb}
\begin{code}

   public F2wCycleBasedPolyLCG (String filename, int no) \begin{hide}
   {
      param = new F2wStructure (filename, no);
      numBits = param.numBits;
      normFactor = param.normFactor;
      fillCyclesPolyLCG ();
   }\end{hide}
\end{code}
 \begin{tabb}
   Constructs a point set after reading its parameters from
   file \texttt{filename}; the parameters are located at line numbered \texttt{no}
   of \texttt{filename}. The available files are listed in the description of class
\externalclass{umontreal.iro.lecuyer.hups}{F2wStructure}.
 \end{tabb}
\begin{code}
\begin{hide}

   public String toString ()
   {
       String s = "F2wCycleBasedPolyLCG:" + PrintfFormat.NEWLINE;
       return s + param.toString ();
   }

   private void fillCyclesPolyLCG ()
   {
      int n = 1 << param.getLog2N();
      int i;
      int mask1 = (1 << (31 - param.r * param.w)) - 1;
      int mask2 = ~mask1;
      RandomStream random = new MRG32k3a();
      IntArrayList c;             // Array used to store the current cycle.
      int currentState;           // The state currently visited.

      boolean stateVisited[] = new boolean[n];
      // Indicates which states have been visited so far.
      for (i = 0; i < n; i++)
         stateVisited[i] = false;
      int startState = 0;  // First state of the cycle currently considered.
      numPoints = 0;
      while (startState < n) {
         stateVisited[startState] = true;
         c = new IntArrayList ();
         param.state = startState << param.S;
         c.add (param.state);
         // c.add ((state & mask2) | (mask1 &
         // (random.nextInt(0,2147483647))));
         param.output = param.F2wPolyLCG ();
         // Warning: watch for overflow !!!
         while (param.state != (startState << param.S)) {
            stateVisited[param.state >> param.S] = true;
            // c.add ((param.state&mask2) | (mask1 &
            // (random.nextInt(0,2147483647))));
            c.add (param.state);
            param.output = param.F2wPolyLCG ();
         }
         addCycle (c);
         for (i = startState + 1; i < n; i++)
            if (stateVisited[i] == false)
               break;
         startState = i;
      }
   }
}
\end{hide}
\end{code}


\defclass{DigitalNet}

This class provides the basic structures for storing and
manipulating \emph{linear digital nets in base $b$}, for an arbitrary
base $b\ge 2$.  We recall that a net contains $n = b^k$ points in
$s$ dimensions, where the $i$th point $\mathbf{u}_i$,
for $i=0,\dots,b^k-1$, is defined as follows:
\begin{latexonly}
\begin{eqnarray*}
  i &=& \sum_{\ell=0}^{k-1} a_{i,\ell} b^\ell, \\ %  \eqlabel{eq:digital-i} \\
 \pmatrix{u_{i,j,1}\cr u_{i,j,2}\cr \vdots \cr }
    &=& \bC_j \pmatrix{a_{i,0}\cr a_{i,1}\cr \vdots \cr a_{i,k-1}\cr},
                                           \\ %  \eqlabel{eq:digital-Cj} \\
 u_{i,j} &=& \sum_{\ell=1}^\infty u_{i,j,\ell} b^{-\ell},
                                 \\[7pt] %  \eqlabel{eq:digital-uij} \\
  \bu_i &=& (u_{i,0},\dots,u_{i,s-1}).     \\ %  \eqlabel{eq:digital-ui}
\end{eqnarray*}
\end{latexonly}
\begin{htmlonly}
\begin{eqnarray}
  i &\quad=\quad& \sum_{r=0}^{k-1} a_{i,r} b^r, \\
  (u_{i,j,1}\  u_{i,j,2}\  \ldots )^T
    &\quad=\quad& \mathbf{C}_j\ (a_{i,0}\  a_{i,1}\  \ldots \ a_{i,k-1})^T, \\
 u_{i,j} &\quad=\quad& \sum_{r=1}^\infty u_{i,j,r} b^{-r},  \\
  \mathbf{u}_i &\quad=\quad& (u_{i,0},\ \ldots,\ u_{i,s-1}).
\end{eqnarray}
\end{htmlonly}
In our implementation, the matrices $\mathbf{C}_j$ are $r\times k$,
so the expansion of $u_{i,j}$ is truncated to its first $r$ terms.
The points are stored implicitly by storing the generator matrices
$\mathbf{C}_j$ in a large two-dimensional array of integers,
 with $srk$ elements.
\begin{detailed}
For general $b$, the element $(l,c)$ of $\mathbf{C}_j$ (counting elements
 from 0)  is stored at position $[jk+c][l]$ in this array.
\end{detailed}

The points $\mathbf{u}_i$ are enumerated using the Gray code
technique\latex{ as proposed in \cite{rANT79a,rTEZ95a}
(see also \cite{fGLA04a,vHON03a})}.
With this technique, the $b$-ary representation of $i$,
$\mathbf{a}_i = (a_{i,0}, \dots, a_{i,k-1})$, is replaced \latex{in Equation
(\ref{eq:digital-Cj})} by a Gray code representation of $i$,
$\mathbf{g}_i = (g_{i,0}, \dots, g_{i,k-1})$. The Gray code $\mathbf{g}_i$
 used here is defined by $g_{i,k-1} = a_{i,k-1}$ and
\latex{
$g_{i,\ell} = (a_{i,\ell} - a_{i,\ell+1}) \mod b$ for $\ell = 0,\dots,k-2$.}
\html{$g_{i,\nu} = (a_{i,\nu} - a_{i,\nu+1}) \mod b$ for $\nu = 0,\dots,k-2$.}
It has the property that
$\mathbf{g}_i = (g_{i,0}, \dots, g_{i,k-1})$ and
$\mathbf{g}_{i+1} = (g_{i+1,0}, \dots, g_{i+1,k-1})$
differ only in the position of the smallest index
\latex{$\ell$}\html{$\nu$} such that
\latex{$a_{i,\ell} < b - 1$, and we have $g_{i+1,\ell} = (g_{i,\ell}+1) \mod b$}\html{$a_{i,\nu} < b - 1$, and we have $g_{i+1,\nu} = (g_{i,\nu}+1) \mod b$}
in that position.

This Gray code representation permits a more efficient enumeration
of the points by the iterators.
It changes the order in which the points $\mathbf{u}_i$ are enumerated,
but the first $b^m$ points remain the same for every integer $m$.
The $i$th point of the sequence with the Gray enumeration
is the $i'$th point of the original enumeration, where $i'$ is the
integer whose $b$-ary representation $\mathbf{a}_{i'}$ is given by the Gray
code $\mathbf{g}_i$.
To enumerate all the points successively, we never need to compute
the Gray codes explicitly.
It suffices to know the position \latex{$\ell$}\html{$\nu$} of the Gray code digit
that changes at each step, and this can be found quickly from the
$b$-ary representation $\mathbf{a}_i$.
The digits of each coordinate $j$ of the current point can be updated by
adding column \latex{$\ell$}\html{$\nu$} of the generator matrix $\mathbf{C}_j$ to the old digits,
modulo $b$.

One should avoid using the method \method{getCoordinate}{}{(i, j)}
for arbitrary values of \texttt{i} and \texttt{j}, because this is much
slower than using an iterator to access successive coordinates.

Digital nets can be randomized in various ways
\cite{mMAT99a,rFAU02a,vLEC02a,vOWE03a}.
Several types of randomizations specialized for nets are implemented
directly in this class.

A simple but important randomization is the \emph{random digital shift}
in base $b$, defined as follows: replace each digit $u_{i,j,\latex{\ell}\html{\nu}}$ in \html{the third equation above}
\latex{(\ref{eq:digital-uij})} by $(u_{i,j,\latex{\ell}\html{\nu}} + d_{j,\latex{\ell}\html{\nu}}) \mod b$,
where the $d_{j,\latex{\ell}\html{\nu}}$'s are i.i.d.\ uniform over $\{0,\dots,b-1\}$.
This is equivalent to applying a single random shift to all the points
in a formal series representation of their coordinates \cite{vLEC02a,vLEM03a}.
In practice, the digital shift is truncated to $w$ digits,
for some integer $w\ge r$.
Applying a digital shift does not change the equidistribution
and $(t,m,s)$-net properties of a point set \cite{vHON03a,vLEC99a,vLEM03a}.
Moreover, with the random shift, each point is uniformly distributed over
the unit hypercube (but the points are not independent, of course).

A second class of randomizations specialized for digital nets
are the \emph{linear matrix scrambles}
\cite{mMAT99a,rFAU02a,vHON03a,vOWE03a}, which multiply the matrices
 $\mathbf{C}_j$
by a random invertible matrix $\mathbf{M}_j$, modulo $b$.
There are several variants, depending on how $\mathbf{M}_j$ is generated, and on
whether $\mathbf{C}_j$ is multiplied on the left or on the right.
In our implementation, the linear matrix scrambles are incorporated directly
into the matrices $\mathbf{C}_j$\latex{ (as in \cite{vHON03a})}, so they
 do not slow down  the enumeration of points.
Methods are available for applying
linear matrix scrambles and for removing these randomizations.
These methods generate the appropriate random numbers and make
the corresponding changes to the $\mathbf{C}_j$'s.
A copy of the original $\mathbf{C}_j$'s is maintained, so the point
set can be returned to its original unscrambled state at any time.
When a new linear matrix scramble is applied, it is always applied to
the \emph{original} generator matrices.
% (so the new matrix scramble is combined with the previous ones).
The method \method{resetGeneratorMatrices}{} removes the current matrix
scramble by resetting the generator matrices to their original state.
On the other hand, the method \method{eraseOriginalGeneratorMatrices}{}
replaces the original generator matrices by the current
ones, making the changes permanent.
This is useful if one wishes to apply two or more linear matrix
scrambles on top of each other.

Linear matrix scrambles are usually combined with a random digital shift;
this combination is called an \emph{affine matrix scramble} \cite{vOWE03a}.
These two randomizations are applied via separate methods.
The linear matrix scrambles are incorporated into the matrices $\mathbf{C}_j$
whereas the digital random shift is stored and applied separately,
independently of the other scramblings.

Applying a digital shift or a linear matrix scramble to a digital net
invalidates all iterators for that randomized point,
because each iterator uses a
\emph{cached} copy of the current point, which is updated only when
the current point index of that iterator changes, and the update also
depends on the cached copy of the previous point.
After applying any kind of scrambling, the iterators must be
reinitialized to the \emph{initial point} by invoking
\externalmethod{}{PointSetIterator}{resetCurPointIndex}{}
or reinstantiated by the \method{iterator}{} method
(this is not done automatically).


\bigskip\hrule\bigskip
%%%%%%%%%%%%%%%%%%%%%%%%%%%%%%%%%%%%%%%%%%%%%%%%%%%%%%%%%%%%%%%%%%%%%%%%%%%

\begin{code}
\begin{hide}
/*
 * Class:        DigitalNet
 * Description:  
 * Environment:  Java
 * Software:     SSJ 
 * Copyright (C) 2001  Pierre L'Ecuyer and Université de Montréal
 * Organization: DIRO, Université de Montréal
 * @author       
 * @since

 * SSJ is free software: you can redistribute it and/or modify it under
 * the terms of the GNU General Public License (GPL) as published by the
 * Free Software Foundation, either version 3 of the License, or
 * any later version.

 * SSJ is distributed in the hope that it will be useful,
 * but WITHOUT ANY WARRANTY; without even the implied warranty of
 * MERCHANTABILITY or FITNESS FOR A PARTICULAR PURPOSE.  See the
 * GNU General Public License for more details.

 * A copy of the GNU General Public License is available at
   <a href="http://www.gnu.org/licenses">GPL licence site</a>.
 */
\end{hide}
package umontreal.iro.lecuyer.hups;\begin{hide}

import umontreal.iro.lecuyer.util.PrintfFormat;
import umontreal.iro.lecuyer.rng.*;
\end{hide}

public class DigitalNet extends PointSet \begin{hide} {

   // Variables to be initialized by the constructor subclasses.
   protected int b = 0;         // Base.
   protected int numCols = 0;   // The number of columns in each C_j. (= k)
   protected int numRows = 0;   // The number of rows in each C_j. (= r)
   protected int outDigits = 0; // Number of output digits (= w)
   private int[][] originalMat; // Original gen. matrices without randomizat.
   protected int[][] genMat;    // The current generator matrices.
                                // genMat[j*numCols+c][l] contains column c
                                // and row l of C_j.
   protected int[][] digitalShift; // The digital shift, initially zero (null).
                                // Entry [j][l] is for dimension j, digit l,
                                // for 0 <= l < outDigits.
   protected double normFactor; // To convert output to (0,1); 1/b^outDigits.
   protected double[] factor;   // Lookup table in ascending order: factor[i]
                                // = 1/b^{i+1} for 0 <= i < outDigits.

   // primes gives the first index in array FaureFactor
   // for the prime p. If primes[i] = p, then
   // FaureFactor[p][j] contains the Faure ordered factors of base p.
   private int[] primes = {2, 3, 5, 7, 11, 13, 17, 19, 23, 29, 31, 37, 41,
      43, 47, 53, 59, 61, 67};

   // Factors on the diagonal corresponding to base b = prime[i] ordered by
   //  increasing Bounds.
   private int[][] FaureFactor = {{1}, {1, 2}, {2, 3, 1, 4},
      { 2, 3, 4, 5, 1, 6}, {3, 4, 7, 8, 2, 5, 6, 9, 1, 10},
      { 5, 8, 3, 4, 9, 10, 2, 6, 7, 11, 1, 12},
      { 5, 7, 10, 12, 3, 6, 11, 14, 4, 13, 2, 8, 9, 15, 1, 16},
      { 7, 8, 11, 12, 4, 5, 14, 15, 3, 6, 13, 16, 2, 9, 10, 17, 1, 18},
      { 5, 9, 14, 18, 7, 10, 13, 16, 4, 6, 17, 19, 3, 8, 15, 20, 2, 11, 12,
        21, 1, 22},
      { 8, 11, 18, 21, 12, 17, 9, 13, 16, 20, 5, 6, 23, 24, 4, 7, 22, 25, 3,
        10, 19, 26, 2, 14, 15, 27, 1, 28},
      { 12, 13, 18, 19, 11, 14, 17, 20, 7, 9, 22, 24, 4, 8, 23, 27, 5, 6, 25,
        26, 3, 10, 21, 28, 2, 15, 16, 29, 1, 30},
      { 8, 14, 23, 29, 10, 11, 26, 27, 13, 17, 20, 24, 7, 16, 21, 30, 5, 15,
        22, 32, 6, 31, 4, 9, 28, 33, 3, 12, 25, 34, 2, 18, 19, 35, 1, 36},
      { 16, 18, 23, 25, 11, 15, 26, 30, 12, 17, 24, 29, 9, 32, 13, 19, 22,
        28, 6, 7, 34, 35, 5, 8, 33, 36, 4, 10, 31, 37, 3, 14, 27, 38, 2, 20,
        21, 39, 1, 40},
      { 12, 18, 25, 31, 9, 19, 24, 34, 8, 16, 27, 35, 10, 13, 30, 33, 15, 20,
        23, 28, 5, 17, 26, 38, 6, 7, 36, 37, 4, 11, 32, 39, 3, 14, 29, 40, 2,
        21, 22, 41, 1, 42},
      { 13, 18, 29, 34, 11, 17, 30, 36, 10, 14, 33, 37, 7, 20, 27, 40, 9, 21,
        26, 38, 15, 22, 25, 32, 6, 8, 39, 41, 5, 19, 28, 42, 4, 12, 35, 43,
        3, 16, 31, 44, 2, 23, 24, 45, 1, 46},
      { 14, 19, 34, 39, 23, 30, 12, 22, 31, 41, 8, 11, 20, 24, 29, 33, 42, 45,
        10, 16, 37, 43, 7, 15, 38, 46, 17, 25, 28, 36, 5, 21, 32, 48, 6, 9,
        44, 47, 4, 13, 40, 49, 3, 18, 35, 50, 2, 26, 27, 51, 1, 52},
      { 25, 26, 33, 34, 18, 23, 36, 41, 14, 21, 38, 45, 24, 27, 32, 35, 11,
        16, 43, 48, 9, 13, 46, 50, 8, 22, 37, 51, 7, 17, 42, 52, 19, 28, 31,
        40, 6, 10, 49, 53, 5, 12, 47, 54, 4, 15, 44, 55, 3, 20, 39, 56, 2,
        29, 30, 57, 1, 58},
      { 22, 25, 36, 39, 17, 18, 43, 44, 24, 28, 33, 37, 13, 14, 47, 48, 16,
        19, 42, 45, 9, 27, 34, 52, 8, 23, 38, 53, 11, 50, 7, 26, 35, 54, 21,
        29, 32, 40, 6, 10, 51, 55, 5, 12, 49, 56, 4, 15, 46, 57, 3, 20, 41,
        58, 2, 30, 31, 59, 1, 60},
      { 18, 26, 41, 49, 14, 24, 43, 53, 12, 28, 39, 55, 29, 30, 37, 38, 10,
        20, 47, 57, 16, 21, 46, 51, 8, 25, 42, 59, 13, 31, 36, 54, 9, 15,
        52, 58, 7, 19, 48, 60, 23, 32, 35, 44, 5, 27, 40, 62, 6, 11, 56,
        61, 4, 17, 50, 63, 3, 22, 45, 64, 2, 33, 34, 65, 1, 66}
};


   public double getCoordinate (int i, int j) {
      // convert i to Gray representation, put digits in gdigit[].
      int l, c, sum;
      int[] bdigit = new int[numCols];
      int[] gdigit = new int[numCols];
      int idigits = intToDigitsGray (b, i, numCols, bdigit, gdigit);
      double result = 0;
      if (digitalShift != null && dimShift < j)
         addRandomShift (dimShift, j, shiftStream);
      for (l = 0; l < outDigits; l++) {
         if (digitalShift == null)
            sum = 0;
         else
            sum = digitalShift[j][l];
         if (l < numRows)
            for (c = 0; c < idigits; c++)
               sum += genMat[j*numCols+c][l] * gdigit[c];
         result += (sum % b) * factor[l];
      }
      if (digitalShift != null)
         result += EpsilonHalf;
      return result;
   }

   public PointSetIterator iterator() {
      return new DigitalNetIterator();
   }\end{hide}
\end{code}

%%%%%%%%%%%%%%%%%%%%%%%%%%%%
\subsubsection* {Constructor}
\begin{code}

   public DigitalNet ()\begin{hide} {
     }\end{hide}
\end{code}
\begin{tabb} Empty constructor.
\end{tabb}


%%%%%%%%%%%%%%%%%%%%%%%%%%%%
\subsubsection* {Methods}
\begin{code}

   public double getCoordinateNoGray (int i, int j)\begin{hide} {
      // convert i to b-ary representation, put digits in bdigit[].
      int l, c, sum;
      int[] bdigit = new int[numCols];
      int idigits = 0;
      for (c = 0; i > 0; c++) {
         idigits++;
         bdigit[c] = i % b;
         i = i / b;
      }
      if (digitalShift != null && dimShift < j)
         addRandomShift (dimShift, j, shiftStream);
      double result = 0;
      for (l = 0; l < outDigits; l++) {
         if (digitalShift == null)
            sum = 0;
         else
            sum = digitalShift[j][l];
         if (l < numRows)
            for (c = 0; c < idigits; c++)
               sum += genMat[j*numCols+c][l] * bdigit[c];
         result += (sum % b) * factor[l];
      }
      if (digitalShift != null)
         result += EpsilonHalf;
      return result;
   }\end{hide}
\end{code}
 \begin{tabb}
   Returns $u_{i,j}$, the coordinate $j$ of point $i$, the points
   being enumerated in the standard order (no Gray code).
 \end{tabb}
\begin{htmlonly}
   \param{i}{point index}
   \param{j}{coordinate index}
   \return{the value of $u_{i,j}$}
\end{htmlonly}
\begin{code}

   public PointSetIterator iteratorNoGray()\begin{hide} {
      return new DigitalNetIteratorNoGray();
   }\end{hide}
\end{code}
\begin{tabb}
  This iterator does not use the Gray code. Thus the points are enumerated
  in the order of their first coordinate before randomization.
\end{tabb}
\begin{code}

   public void addRandomShift (int d1, int d2, RandomStream stream)\begin{hide} {
      if (null == stream)
         throw new IllegalArgumentException (
              PrintfFormat.NEWLINE +
              "   Calling addRandomShift with null stream");
      if (0 == d2)
         d2 = Math.max (1, dim);
      if (digitalShift == null) {
         digitalShift = new int[d2][outDigits];
         capacityShift = d2;
      } else if (d2 > capacityShift) {
         int d3 = Math.max (4, capacityShift);
         while (d2 > d3)
            d3 *= 2;
         int[][] temp = new int[d3][outDigits];
         capacityShift = d3;
         for (int i = 0; i < d1; i++)
            for (int j = 0; j < outDigits; j++)
               temp[i][j] = digitalShift[i][j];
         digitalShift = temp;
      }
      for (int i = d1; i < d2; i++)
         for (int j = 0; j < outDigits; j++)
            digitalShift[i][j] = stream.nextInt (0, b - 1);
      dimShift = d2;
      shiftStream = stream;
     }\end{hide}
\end{code}
\begin{tabb}  Adds a random digital shift to all the points of the point set,
  using stream \texttt{stream} to generate the random numbers.
  For each coordinate $j$ from \texttt{d1} to \texttt{d2-1},
  the shift vector $(d_{j,0},\dots,d_{j,k-1})$
  is generated uniformly over $\{0,\dots,b-1\}^k$ and added modulo $b$ to
  the digits of all the points.
  After adding a digital shift, all iterators must be reconstructed or
  reset to zero.
\end{tabb}
\begin{htmlonly}
   \param{stream}{random number stream used to generate uniforms}
\end{htmlonly}
\begin{code}

   public void addRandomShift (RandomStream stream)\begin{hide} {
      addRandomShift (0, dim, stream);
     }\end{hide}
\end{code}
\begin{tabb}  Same as \method{addRandomShift}{}\texttt{(0, dim, stream)},
  where \texttt{dim} is the dimension of the digital net.
\end{tabb}
\begin{htmlonly}
   \param{stream}{random number stream used to generate uniforms}
\end{htmlonly}
\begin{hide}\begin{code}

   public void clearRandomShift() {
      super.clearRandomShift();
      digitalShift = null;
   }
\end{code}
\begin{tabb}
   Erases the current digital random shift, if any.
\end{tabb}\end{hide}
\begin{code}\begin{hide}

   public String toString() {
      StringBuffer sb = new StringBuffer (100);
      if (b > 0) {
         sb.append ("Base = ");   sb.append (b);
         sb.append (PrintfFormat.NEWLINE);
      }
      sb.append ("Num cols = ");   sb.append (numCols);
      sb.append (PrintfFormat.NEWLINE + "Num rows = ");
      sb.append (numRows);
      sb.append (PrintfFormat.NEWLINE + "outDigits = ");
      sb.append (outDigits);
      return sb.toString();
   }


   // Print matrices M for dimensions 0 to N-1.
   private void printMat (int N, int[][][] A, String name) {
      for (int i = 0; i < N; i++) {
         System.out.println ("-------------------------------------" +
            PrintfFormat.NEWLINE + name + "   dim = " + i);
         int l, c;   // row l, column c, dimension i for A[i][l][c].
         for (l = 0; l < numRows; l++) {
            for (c = 0; c < numCols; c++) {
               System.out.print (A[i][l][c] + "  ");
            }
            System.out.println ("");
         }
      }
      System.out.println ("");
   }


   // Print matrix M
   private void printMat0 (int[][] A, String name) {
         System.out.println ("-------------------------------------" +
                             PrintfFormat.NEWLINE + name);
         int l, c;   // row l, column c for A[l][c].
         for (l = 0; l < numCols; l++) {
            for (c = 0; c < numCols; c++) {
               System.out.print (A[l][c] + "  ");
            }
            System.out.println ("");
         }
      System.out.println ("");
   }


   // Left-multiplies lower-triangular matrix Mj by original C_j,
   // where original C_j is in originalMat and result is in genMat.
   // This implementation is safe only if (numRows*(b-1)^2) is a valid int.
   private void leftMultiplyMat (int j, int[][] Mj) {
      int l, c, i, sum;   // Dimension j, row l, column c for new C_j.
      for (l = 0; l < numRows ; l++) {
         for (c = 0; c < numCols; c++) {
            // Multiply row l of M_j by column c of C_j.
            sum = 0;
            for (i = 0; i <= l; i++)
               sum += Mj[l][i] * originalMat[j*numCols+c][i];
            genMat[j*numCols+c][l] = sum % b;
         }
      }
   }


   // Left-multiplies diagonal matrix Mj by original C_j,
   // where original C_j is in originalMat and result is in genMat.
   // This implementation is safe only if (numRows*(b-1)^2) is a valid int.
   private void leftMultiplyMatDiag (int j, int[][] Mj) {
      int l, c, sum;   // Dimension j, row l, column c for new C_j.
      for (l = 0; l < numRows ; l++) {
         for (c = 0; c < numCols; c++) {
            // Multiply row l of M_j by column c of C_j.
            sum = Mj[l][l] * originalMat[j*numCols+c][l];
            genMat[j*numCols+c][l] = sum % b;
         }
      }
   }


   // Right-multiplies original C_j by upper-triangular matrix Mj,
   // where original C_j is in originalMat and result is in genMat.
   // This implementation is safe only if (numCols*(b-1)^2) is a valid int.
   private void rightMultiplyMat (int j, int[][] Mj) {
      int l, c, i, sum;   // Dimension j, row l, column c for new C_j.
      for (l = 0; l < numRows ; l++) {
         for (c = 0; c < numCols; c++) {
            // Multiply row l of C_j by column c of M_j.
            sum = 0;
            for (i = 0; i <= c; i++)
               sum += originalMat[j*numCols+i][l] * Mj[i][c];
            genMat[j*numCols+c][l] = sum % b;
         }
      }
   }


   private int getFaureIndex (String method, int sb, int flag) {
      // Check for errors in ...FaurePermut. Returns the index ib of the
      // base b in primes, i.e.  b = primes[ib].
      if (sb >= b)
         throw new IllegalArgumentException (PrintfFormat.NEWLINE +
            "   sb >= base in " + method);
      if (sb < 1)
         throw new IllegalArgumentException (PrintfFormat.NEWLINE +
            "   sb = 0 in " + method);
      if ((flag > 2) || (flag < 0))
         throw new IllegalArgumentException (
             PrintfFormat.NEWLINE + "   lowerFlag not in {0, 1, 2} in "
             + method);

      // Find index ib of base in array primes
      int ib = 0;
      while ((ib < primes.length) && (primes[ib] < b))
          ib++;
      if (ib >= primes.length)
         throw new IllegalArgumentException ("base too large in " + method);
      if (b != primes[ib])
         throw new IllegalArgumentException (
            "Faure factors are not implemented for this base in " + method);
      return ib;
      }
\end{hide}

   public void leftMatrixScramble (RandomStream stream) \begin{hide} {
      int j, l, c;  // dimension j, row l, column c.

      // If genMat contains the original gen. matrices, copy to originalMat.
      if (originalMat == null) {
         originalMat = genMat;
         genMat = new int[dim * numCols][numRows];
      }

      // Constructs the lower-triangular scrambling matrices M_j, r by r.
      int[][][] scrambleMat = new int[dim][numRows][numRows];
      for (j = 0 ; j < dim; j++) {
         for (l = 0; l < numRows; l++) {
            for (c = 0; c < numRows; c++) {
               if (c == l)                   // No zero on the diagonal.
                  scrambleMat[j][l][c] = stream.nextInt(1, b - 1) ;
               else if (c < l)
                  scrambleMat[j][l][c] = stream.nextInt(0, b - 1);
               else
                  scrambleMat[j][l][c] = 0;  // Zeros above the diagonal;
            }
         }
      }

      // Multiply M_j by the generator matrix C_j for each j.
      for (j = 0 ; j < dim; j++) leftMultiplyMat (j, scrambleMat[j]);
   }\end{hide}
\end{code}
 \begin{tabb}
   Applies a linear scramble by multiplying each $\mathbf{C}_j$ on the left
   by a $w\times w$ nonsingular lower-triangular matrix $\mathbf{M}_j$,
   as suggested by Matou\v{s}ek\latex{ \cite{mMAT99a}} and implemented
   by Hong and Hickernell\latex{ \cite{vHON03a}}. The diagonal entries of
   each matrix $\mathbf{M}_j$ are generated uniformly over
   $\{1,\dots,b-1\}$, the entries below the diagonal are generated uniformly
   over $\{0,\dots,b-1\}$, and all these entries are generated independently.
   This means that in base $b=2$, all diagonal elements are equal to 1.
\richard{Les matrices de \texttt{leftMatrixScramble} sont carr\'ees et
  triangulaires inf\'erieures. PL pense qu'il faut consid\'erer la
  possibilit\'e de rajouter des lignes \`a ces matrices pour pouvoir
  randomiser plus les derniers chiffres ou les derniers bits.}
\end{tabb}
\begin{htmlonly}
   \param{stream}{random number stream used to generate the randomness}
\end{htmlonly}
\begin{code}

   public void leftMatrixScrambleDiag (RandomStream stream) \begin{hide} {
      int j, l, c;  // dimension j, row l, column c.

      // If genMat contains the original gen. matrices, copy to originalMat.
      if (originalMat == null) {
         originalMat = genMat;
         genMat = new int[dim * numCols][numRows];
      }

      // Constructs the diagonal scrambling matrices M_j, r by r.
      int[][][] scrambleMat = new int[dim][numRows][numRows];
      for (j = 0 ; j < dim; j++) {
         for (l = 0; l < numRows; l++) {
            for (c = 0; c < numRows; c++) {
               if (c == l)                   // No zero on the diagonal.
                  scrambleMat[j][l][c] = stream.nextInt(1, b - 1) ;
               else
                  scrambleMat[j][l][c] = 0;  // Diagonal matrix;
            }
         }
      }

      // Multiply M_j by the generator matrix C_j for each j.
      for (j = 0 ; j < dim; j++) leftMultiplyMatDiag (j, scrambleMat[j]);
   }\end{hide}
\end{code}
 \begin{tabb}
   Similar to \method{leftMatrixScramble}{} except that all the
   off-diagonal elements of the $\mathbf{M}_j$ are 0.
\end{tabb}
\begin{htmlonly}
   \param{stream}{random number stream used to generate the randomness}
\end{htmlonly}
\begin{code}\begin{hide}

   private void LMSFaurePermut (String method, RandomStream stream, int sb,
      int lowerFlag) {
/*
   If \texttt{lowerFlag = 2}, the off-diagonal elements below the diagonal
   of $\mathbf{M}_j$ are chosen as in \method{leftMatrixScramble}{}.
   If \texttt{lowerFlag = 1}, the off-diagonal elements below the diagonal of
   $\mathbf{M}_j$ are also chosen from the restricted set. If
    \texttt{lowerFlag = 0}, the off-diagonal elements of $\mathbf{M}_j$ are all 0.
*/
      int ib = getFaureIndex (method, sb, lowerFlag);

      // If genMat contains the original gen. matrices, copy to originalMat.
      if (originalMat == null) {
         originalMat = genMat;
         genMat = new int[dim * numCols][numRows];
      }

      // Constructs the lower-triangular scrambling matrices M_j, r by r.
      int jb;
      int j, l, c;  // dimension j, row l, column c.
      int[][][] scrambleMat = new int[dim][numRows][numRows];
      for (j = 0 ; j < dim; j++) {
         for (l = 0; l < numRows; l++) {
            for (c = 0; c < numRows; c++) {
               if (c == l) {
                  jb = stream.nextInt(0, sb - 1);
                  scrambleMat[j][l][c] = FaureFactor[ib][jb];
               } else if (c < l) {
                  if (lowerFlag == 2) {
                     scrambleMat[j][l][c] = stream.nextInt(0, b - 1);
                  } else if (lowerFlag == 1) {
                     jb = stream.nextInt(0, sb - 1);
                     scrambleMat[j][l][c] = FaureFactor[ib][jb];
                  } else {   // lowerFlag == 0
                     scrambleMat[j][l][c] = 0;
                  }
               } else
                  scrambleMat[j][l][c] = 0;  // Zeros above the diagonal;
            }
         }
      }
      // printMat (dim, scrambleMat, method);

      // Multiply M_j by the generator matrix C_j for each j.
      if (lowerFlag == 0)
         for (j = 0 ; j < dim; j++) leftMultiplyMatDiag (j, scrambleMat[j]);
      else
         for (j = 0 ; j < dim; j++) leftMultiplyMat (j, scrambleMat[j]);
   }\end{hide}

   public void leftMatrixScrambleFaurePermut (RandomStream stream, int sb)\begin{hide} {
       LMSFaurePermut ("leftMatrixScrambleFaurePermut", stream, sb, 2);
   }\end{hide}
\end{code}
 \begin{tabb}
   Similar to \method{leftMatrixScramble}{} except that the diagonal elements
   of each matrix $\mathbf{M}_j$ are chosen from a restricted set of the best
   integers as calculated by Faure \cite{rFAU02a}. They are generated
   uniformly over the first \texttt{sb} elements of array $F$, where $F$ is
   made up of a permutation of the integers $[1..(b-1)]$. These integers are
   sorted by increasing  order of the upper bounds of the extreme discrepancy
   for the given integer.
\end{tabb}
\begin{htmlonly}
   \param{stream}{random number stream used to generate the randomness}
   \param{sb}{Only the first $sb$ elements of $F$ are used}
\end{htmlonly}
\begin{code}

   public void leftMatrixScrambleFaurePermutDiag (RandomStream stream,
                                                  int sb)\begin{hide} {
       LMSFaurePermut ("leftMatrixScrambleFaurePermutDiag",
                              stream, sb, 0);
   }\end{hide}
\end{code}
 \begin{tabb}
   Similar to \method{leftMatrixScrambleFaurePermut}{} except that all
   off-diagonal elements are 0.
\end{tabb}
\begin{htmlonly}
   \param{stream}{random number stream used to generate the randomness}
   \param{sb}{Only the first $sb$ elements of $F$ are used}
\end{htmlonly}
\begin{code}

   public void leftMatrixScrambleFaurePermutAll (RandomStream stream,
                                                 int sb)\begin{hide} {
       LMSFaurePermut ("leftMatrixScrambleFaurePermutAll",
                              stream, sb, 1);
   }\end{hide}
\end{code}
 \begin{tabb}
   Similar to \method{leftMatrixScrambleFaurePermut}{} except that the
   elements under the diagonal are also
   chosen from the same restricted set as the diagonal elements.
\end{tabb}
\begin{htmlonly}
   \param{stream}{random number stream used to generate the randomness}
   \param{sb}{Only the first $sb$ elements of $F$ are used}
\end{htmlonly}
\begin{code}

   public void iBinomialMatrixScramble (RandomStream stream)\begin{hide} {
      int j, l, c;  // dimension j, row l, column c.
      int diag;     // random entry on the diagonal;
      int col1;     // random entries below the diagonal;

      // If genMat is original generator matrices, copy it to originalMat.
      if (originalMat == null) {
         originalMat = genMat;
         genMat = new int[dim * numCols][numRows];
      }

      // Constructs the lower-triangular scrambling matrices M_j, r by r.
      int[][][] scrambleMat = new int[dim][numRows][numRows];
      for (j = 0 ; j < dim; j++) {
         diag = stream.nextInt(1, b - 1);
         for (l = 0; l < numRows; l++) {
            // Single random nonzero element on the diagonal.
            scrambleMat[j][l][l] = diag;
            for (c = l+1; c < numRows; c++) scrambleMat[j][l][c] = 0;
         }
         for (l = 1; l < numRows; l++) {
            col1 = stream.nextInt(0, b - 1);
            for (c = 0; l+c < numRows; c++) scrambleMat[j][l+c][c] = col1;
         }
      }
      // printMat (dim, scrambleMat,  "iBinomialMatrixScramble");
      for (j = 0 ; j < dim; j++) leftMultiplyMat (j, scrambleMat[j]);
   }\end{hide}
\end{code}
 \begin{tabb}
   Applies the $i$-binomial matrix scramble proposed by Tezuka \cite{rTEZ02a}
   \latex{ (see also \cite{vOWE03a})}.
   This multiplies each $\mathbf{C}_j$ on the left
   by a $w\times w$ nonsingular lower-triangular matrix $\mathbf{M}_j$ as in
   \method{leftMatrixScramble}{}, but with the additional constraint that
   all entries on any given diagonal or subdiagonal of $\mathbf{M}_j$ are identical.
\end{tabb}
\begin{htmlonly}
   \param{stream}{random number stream used as generator of the randomness}
\end{htmlonly}
\begin{code}\begin{hide}

   private void iBMSFaurePermut (String method, RandomStream stream,
                                 int sb, int lowerFlag) {
      int ib = getFaureIndex (method, sb, lowerFlag);

      // If genMat is original generator matrices, copy it to originalMat.
      if (originalMat == null) {
         originalMat = genMat;
         genMat = new int[dim * numCols][numRows];
      }

      // Constructs the lower-triangular scrambling matrices M_j, r by r.
      int j, l, c;  // dimension j, row l, column c.
      int diag;     // random entry on the diagonal;
      int col1;     // random entries below the diagonal;
      int jb;
      int[][][] scrambleMat = new int[dim][numRows][numRows];
      for (j = 0 ; j < dim; j++) {
         jb = stream.nextInt(0, sb - 1);
         diag = FaureFactor[ib][jb];
         for (l = 0; l < numRows; l++) {
            // Single random nonzero element on the diagonal.
            scrambleMat[j][l][l] = diag;
            for (c = l+1; c < numRows; c++) scrambleMat[j][l][c] = 0;
         }
         for (l = 1; l < numRows; l++) {
            if (lowerFlag == 2) {
               col1 = stream.nextInt(0, b - 1);
            } else if (lowerFlag == 1) {
               jb = stream.nextInt(0, sb - 1);
               col1 = FaureFactor[ib][jb];
            } else {   // lowerFlag == 0
               col1 = 0;
            }
            for (c = 0; l+c < numRows; c++) scrambleMat[j][l+c][c] = col1;
         }
      }
      // printMat (dim, scrambleMat, method);

      if (lowerFlag > 0)
         for (j = 0 ; j < dim; j++) leftMultiplyMat (j, scrambleMat[j]);
      else
         for (j = 0 ; j < dim; j++) leftMultiplyMatDiag (j, scrambleMat[j]);
   }\end{hide}

   public void iBinomialMatrixScrambleFaurePermut (RandomStream stream,
                                                   int sb)\begin{hide} {
       iBMSFaurePermut ("iBinomialMatrixScrambleFaurePermut",
                              stream, sb, 2);
   }\end{hide}
\end{code}
 \begin{tabb}
   Similar to \method{iBinomialMatrixScramble}{} except that the diagonal
   elements of each matrix $\mathbf{M}_j$ are chosen as in
  \method{leftMatrixScrambleFaurePermut}{}.
\end{tabb}
\begin{htmlonly}
   \param{stream}{random number stream used to generate the randomness}
   \param{sb}{Only the first $sb$ elements of $F$ are used}
\end{htmlonly}
\begin{code}

   public void iBinomialMatrixScrambleFaurePermutDiag (RandomStream stream,
                                                       int sb)\begin{hide} {
       iBMSFaurePermut ("iBinomialMatrixScrambleFaurePermutDiag",
                               stream, sb, 0);
   }\end{hide}
\end{code}
 \begin{tabb}
   Similar to \method{iBinomialMatrixScrambleFaurePermut}{} except that all the
   off-diagonal elements are 0.
\end{tabb}
\begin{htmlonly}
   \param{stream}{random number stream used to generate the randomness}
   \param{sb}{Only the first $sb$ elements of $F$ are used}
\end{htmlonly}
\begin{code}

   public void iBinomialMatrixScrambleFaurePermutAll (RandomStream stream,
                                                      int sb)\begin{hide} {
       iBMSFaurePermut ("iBinomialMatrixScrambleFaurePermutAll",
                               stream, sb, 1);
   }\end{hide}
\end{code}
 \begin{tabb}
   Similar to \method{iBinomialMatrixScrambleFaurePermut}{} except that the
   elements under the diagonal are also
   chosen from the same restricted set as the diagonal elements.
\end{tabb}
\begin{htmlonly}
   \param{stream}{random number stream used to generate the randomness}
   \param{sb}{Only the first $sb$ elements of $F$ are used}
\end{htmlonly}
\begin{code}

   public void stripedMatrixScramble (RandomStream stream)\begin{hide} {
      int j, l, c;  // dimension j, row l, column c.
      int diag;     // random entry on the diagonal;
      int col1;     // random entries below the diagonal;

      // If genMat contains original gener. matrices, copy it to originalMat.
      if (originalMat == null) {
         originalMat = genMat;
         genMat = new int[dim * numCols][numRows];
      }

      // Constructs the lower-triangular scrambling matrices M_j, r by r.
      int[][][] scrambleMat = new int[dim][numRows][numRows];
      for (j = 0 ; j < dim; j++) {
         for (c = 0; c < numRows; c++) {
            diag = stream.nextInt (1, b - 1);   // Random entry in this column.
            for (l = 0; l < c; l++)         scrambleMat[j][l][c] = 0;
            for (l = c; l < numRows; l++) scrambleMat[j][l][c] = diag;
         }
      }
      // printMat (dim, scrambleMat,  "stripedMatrixScramble");
      for (j = 0 ; j < dim; j++) leftMultiplyMat (j, scrambleMat[j]);
   }\end{hide}
\end{code}
 \begin{tabb}
   Applies the \emph{striped matrix scramble} proposed by Owen \cite{vOWE03a}.
   It multiplies each $\mathbf{C}_j$ on the left
   by a $w\times w$ nonsingular lower-triangular matrix $\mathbf{M}_j$ as in
   \method{leftMatrixScramble}{}, but with the additional constraint that
   in any column, all entries below the diagonal are equal to the
   diagonal entry, which is generated randomly over $\{1,\dots,b-1\}$.
   Note that for $b=2$, the matrices $\mathbf{M}_j$ become deterministic, with
   all entries on and below the diagonal equal to 1.
\end{tabb}
\begin{htmlonly}
   \param{stream}{random number stream used as generator of the randomness}
\end{htmlonly}
\begin{code}

   public void stripedMatrixScrambleFaurePermutAll (RandomStream stream,
                                                    int sb)\begin{hide} {
      int ib = getFaureIndex ("stripedMatrixScrambleFaurePermutAll", sb, 1);

      // If genMat contains original gener. matrices, copy it to originalMat.
      if (originalMat == null) {
         originalMat = genMat;
         genMat = new int[dim * numCols][numRows];
      }

      // Constructs the lower-triangular scrambling matrices M_j, r by r.
      int j, l, c;  // dimension j, row l, column c.
      int diag;     // random entry on the diagonal;
      int col1;     // random entries below the diagonal;
      int jb;
      int[][][] scrambleMat = new int[dim][numRows][numRows];
      for (j = 0 ; j < dim; j++) {
         for (c = 0; c < numRows; c++) {
            jb = stream.nextInt(0, sb - 1);
            diag = FaureFactor[ib][jb];  // Random entry in this column.
            for (l = 0; l < c; l++)         scrambleMat[j][l][c] = 0;
            for (l = c; l < numRows; l++) scrambleMat[j][l][c] = diag;
         }
      }
      // printMat (dim, scrambleMat,  "stripedMatrixScrambleFaurePermutAll");
      for (j = 0 ; j < dim; j++) leftMultiplyMat (j, scrambleMat[j]);
   }\end{hide}
\end{code}
 \begin{tabb}
   Similar to \method{stripedMatrixScramble}{} except that the
   elements on and under the diagonal of each matrix $\mathbf{M}_j$ are
   chosen as in \method{leftMatrixScrambleFaurePermut}{}.
\end{tabb}
\begin{htmlonly}
   \param{stream}{random number stream used as generator of the randomness}
   \param{sb}{Only the first $sb$ elements of $F$ are used}
\end{htmlonly}
\begin{code}

   public void rightMatrixScramble (RandomStream stream) \begin{hide} {
      int j, c, l, i, sum;  // dimension j, row l, column c, of genMat.

      // SaveOriginalMat();
      if (originalMat == null) {
         originalMat = genMat;
         genMat = new int[dim * numCols][numRows];
      }

      // Generate an upper triangular matrix for Faure-Tezuka right-scramble.
      // Entry [l][c] is in row l, col c.
      int[][] scrambleMat = new int[numCols][numCols];
      for (c = 0; c < numCols; c++) {
         for (l = 0; l < c; l++) scrambleMat[l][c] = stream.nextInt (0, b - 1);
         scrambleMat[c][c] = stream.nextInt (1, b - 1);
         for (l = c+1; l < numCols; l++) scrambleMat[l][c] = 0;
      }

      // printMat0 (scrambleMat,  "rightMatrixScramble");
      // Right-multiply the generator matrices by the scrambling matrix.
      for (j = 0 ; j < dim; j++) rightMultiplyMat (j, scrambleMat);
   }\end{hide}
\end{code}
 \begin{tabb}
   Applies a linear scramble by multiplying each $\mathbf{C}_j$ on the right
   by a single $k\times k$ nonsingular upper-triangular matrix $\mathbf{M}$,
   as suggested by Faure and Tezuka\latex{ \cite{rFAU02a} (see also
   \cite{vHON03a})}.
   The diagonal entries of the matrix $\mathbf{M}$ are generated uniformly over
   $\{1,\dots,b-1\}$, the entries above the diagonal are generated uniformly
   over $\{0,\dots,b-1\}$, and all the entries are generated independently.
   The effect of this scramble is only to change the order in which the
   points are generated.  If one computes the average value of a
   function over \emph{all} the points of a given digital net, or over
   a number of points that is a power of the basis, then this
   scramble makes no difference.
\end{tabb}
\begin{htmlonly}
   \param{stream}{random number stream used as generator of the randomness}
\end{htmlonly}
\begin{hide} %%%%
\begin{code}

   public void unrandomize() {
      resetGeneratorMatrices();
      digitalShift = null;
   }
\end{code}
\begin{tabb}
   Restores the original generator matrices and
   removes the random shift.
\end{tabb}
\end{hide}
\begin{code}

   public void resetGeneratorMatrices()\begin{hide} {
      if (originalMat != null) {
         genMat = originalMat;
         originalMat = null;
      }
   }\end{hide}
\end{code}
\begin{tabb}
   Restores the original generator matrices.
   This removes the current linear matrix scrambles.
\end{tabb}
\begin{code}

   public void eraseOriginalGeneratorMatrices()\begin{hide} {
      originalMat = null;
   }\end{hide}
\end{code}
\begin{tabb}
   Erases the original generator matrices and replaces them by
   the current ones.  The current linear matrix scrambles thus become
   \emph{permanent}.  This is useful if we want to apply several
   scrambles in succession to a given digital net.
\end{tabb}
\begin{code}

   public void printGeneratorMatrices (int s) \begin{hide} {
      // row l, column c, dimension j.
      for (int j = 0; j < s; j++) {
         System.out.println ("dim = " + (j+1) + PrintfFormat.NEWLINE);
         for (int l = 0; l < numRows; l++) {
            for (int c = 0; c < numCols; c++)
               System.out.print (genMat[j*numCols+c][l] + "  ");
            System.out.println ("");
         }
         System.out.println ("----------------------------------");
      }
   }\end{hide}
\end{code}
\begin{tabb}
  Prints the generator matrices in standard form for dimensions 1 to $s$.
\end{tabb}

\begin{code}\begin{hide}

      // Computes the digital expansion of $i$ in base $b$, and the digits
      // of the Gray code of $i$.
      // These digits are placed in arrays \texttt{bary} and \texttt{gray},
      // respectively, and the method returns the number of digits in the
      // expansion.  The two arrays are assumed to be of sizes larger or
      // equal to this new number of digits.
      protected int intToDigitsGray (int b, int i, int numDigits,
                                            int[] bary, int[] gray) {
         if (i == 0)
            return 0;
         int idigits = 0; // Num. of digits in b-ary and Gray repres.
         int c;
         // convert i to b-ary representation, put digits in bary[].
         for (c = 0; i > 0; c++) {
            idigits++;
            bary[c] = i % b;
            i = i / b;
         }
         // convert b-ary representation to Gray code.
         gray[idigits-1] = bary[idigits-1];
         int diff;
         for (c = 0;  c < idigits-1; c++) {
            diff = bary[c] - bary[c+1];
            if (diff < 0)
               gray[c] = diff + b;
            else
               gray[c] = diff;
         }
         for (c = idigits; c < numDigits; c++) gray[c] = bary[c] = 0;
         return idigits;
      }


// ************************************************************************

   protected class DigitalNetIterator extends
          PointSet.DefaultPointSetIterator {

      protected int idigits;        // Num. of digits in current code.
      protected int[] bdigit;       // b-ary code of current point index.
      protected int[] gdigit;       // Gray code of current point index.
      protected int dimS;           // = dim except in the shift iterator
                                    // children where it is = dim + 1.
      protected int[] cachedCurPoint; // Digits of coords of the current point,
                           // with digital shift already applied.
                           // Digit l of coord j is at pos. [j*outDigits + l].
                           // Has one more dimension for the shift iterators.

      public DigitalNetIterator() {
 //        EpsilonHalf = 0.5 / Math.pow ((double) b, (double) outDigits);
         bdigit = new int[numCols];
         gdigit = new int[numCols];
         dimS = dim;
         cachedCurPoint = new int[(dim + 1)* outDigits];
         init();  // This call is important so that subclasses don't
                  // automatically call 'resetCurPointIndex' at the time
                  // of construction as this may cause a subtle
                  // 'NullPointerException'
      }

      public void init() { // See constructor
         resetCurPointIndex();
      }

      // We want to avoid generating 0 or 1
      public double nextDouble() {
         return nextCoordinate() + EpsilonHalf;
      }

      public double nextCoordinate() {
         if (curPointIndex >= numPoints || curCoordIndex >= dimS)
            outOfBounds();
         int start = outDigits * curCoordIndex++;
         double sum = 0;
         // Can always have up to outDigits digits, because of digital shift.
         for (int k = 0; k < outDigits; k++)
            sum += cachedCurPoint [start+k] * factor[k];
         if (digitalShift != null)
            sum += EpsilonHalf;
        return sum;
      }

      public void resetCurPointIndex() {
         if (digitalShift == null)
            for (int i = 0; i < cachedCurPoint.length; i++)
               cachedCurPoint[i] = 0;
         else {
            if (dimShift < dimS)
               addRandomShift (dimShift, dimS, shiftStream);
            for (int j = 0; j < dimS; j++)
               for (int k = 0; k < outDigits; k++)
                  cachedCurPoint[j*outDigits + k] = digitalShift[j][k];
         }
         for (int i = 0; i < numCols; i++) bdigit[i] = 0;
         for (int i = 0; i < numCols; i++) gdigit[i] = 0;
         curPointIndex = 0;
         curCoordIndex = 0;
         idigits = 0;
      }

      public void setCurPointIndex (int i) {
         if (i == 0) {
            resetCurPointIndex();
            return;
         }
         curPointIndex = i;
         curCoordIndex = 0;
         if (digitalShift != null && dimShift < dimS)
             addRandomShift (dimShift, dimS, shiftStream);

         // Digits of Gray code, used to reconstruct cachedCurPoint.
         idigits = intToDigitsGray (b, i, numCols, bdigit, gdigit);
         int c, j, l, sum;
         for (j = 0; j < dimS; j++) {
            for (l = 0; l < outDigits; l++) {
               if (digitalShift == null)
                  sum = 0;
               else
                  sum = digitalShift[j][l];
               if (l < numRows)
                  for (c = 0; c < idigits; c++)
                     sum += genMat[j*numCols+c][l] * gdigit[c];
               cachedCurPoint [j*outDigits+l] = sum % b;
            }
         }
      }

      public int resetToNextPoint() {
         // incremental computation.
         curPointIndex++;
         curCoordIndex = 0;
         if (curPointIndex >= numPoints)
            return curPointIndex;

         // Update the digital expansion of i in base b, and find the
         // position of change in the Gray code. Set all digits == b-1 to 0
         // and increase the first one after by 1.
         int pos;      // Position of change in the Gray code.
         for (pos = 0; gdigit[pos] == b-1; pos++)
            gdigit[pos] = 0;
         gdigit[pos]++;

         // Update the cachedCurPoint by adding the column of the gener.
         // matrix that corresponds to the Gray code digit that has changed.
         // The digital shift is already incorporated in the cached point.
         int c, j, l;
         int lsup = numRows;        // Max index l
         if (outDigits < numRows)
            lsup = outDigits;
         for (j = 0; j < dimS; j++) {
            for (l = 0; l < lsup; l++) {
               cachedCurPoint[j*outDigits + l] += genMat[j*numCols + pos][l];
               cachedCurPoint[j * outDigits + l] %= b;
            }
         }
         return curPointIndex;
      }
   }


// ************************************************************************

   protected class DigitalNetIteratorNoGray extends DigitalNetIterator {

      public DigitalNetIteratorNoGray() {
         super();
      }

      public void setCurPointIndex (int i) {
         if (i == 0) {
            resetCurPointIndex();
            return;
         }
         curPointIndex = i;
         curCoordIndex = 0;
         if (dimShift < dimS)
             addRandomShift (dimShift, dimS, shiftStream);

         // Convert i to b-ary representation, put digits in bdigit.
         idigits = intToDigitsGray (b, i, numCols, bdigit, gdigit);
         int c, j, l, sum;
         for (j = 0; j < dimS; j++) {
            for (l = 0; l < outDigits; l++) {
               if (digitalShift == null)
                  sum = 0;
               else
                  sum = digitalShift[j][l];
               if (l < numRows)
                  for (c = 0; c < idigits; c++)
                     sum += genMat[j*numCols+c][l] * bdigit[c];
               cachedCurPoint [j*outDigits+l] = sum % b;
            }
         }
      }

      public int resetToNextPoint() {
         curPointIndex++;
         curCoordIndex = 0;
         if (curPointIndex >= numPoints)
            return curPointIndex;

         // Find the position of change in the digits of curPointIndex in base
         // b. Set all digits = b-1 to 0; increase the first one after by 1.
         int pos;
         for (pos = 0; bdigit[pos] == b-1; pos++)
            bdigit[pos] = 0;
         bdigit[pos]++;

         // Update the digital expansion of curPointIndex in base b.
         // Update the cachedCurPoint by adding 1 unit at the digit pos.
         // If pos > 0, remove b-1 units in the positions < pos. Since
         // calculations are mod b, this is equivalent to adding 1 unit.
         // The digital shift is already incorporated in the cached point.
         int c, j, l;
         int lsup = numRows;        // Max index l
         if (outDigits < numRows)
            lsup = outDigits;
         for (j = 0; j < dimS; j++) {
            for (l = 0; l < lsup; l++) {
               for (c = 0; c <= pos; c++)
                  cachedCurPoint[j*outDigits + l] += genMat[j*numCols + c][l];
               cachedCurPoint[j * outDigits + l] %= b;
            }
         }
         return curPointIndex;
      }
   }

}\end{hide}
\end{code}

\include{DigitalSequence}
\defclass {DigitalNetFromFile}

This class allows us to read the parameters defining a digital net either
from a file, or from a URL address on the World Wide Web.
The parameters used in building the net are those defined in class
\externalclass{umontreal.iro.lecuyer.hups}{DigitalNet}.
The format of the data files must be the following:
\begin{htmlonly} (see the format in \texttt{guidehups.pdf})\end{htmlonly}

\begin{figure}
\begin{center}
\tt
\fbox {
\begin {tabular}{llll}
 \multicolumn{4}{l}{// Any number of comment lines starting with //} \\
     $b$      & &      & //  $\mbox{Base}$  \\
     $k$      & &      & //    Number of columns   \\
     $r$      & &      & //    Maximal number of rows  \\
     $n$      & &      & //    Number of points = $b^k$  \\
     $s$      & &      & //    Maximal dimension of points \\
\\
 \multicolumn{4}{l}{// dim = 1} \\
  $c_{11}$ & $c_{21}$ & $\cdots$ & $c_{r1}$ \\
  $c_{12}$ & $c_{22}$ & $\cdots$ & $c_{r2}$ \\
    &  $\vdots$ && \\
  $c_{1k}$ & $c_{2k}$ & $\cdots$ & $c_{rk}$ \\
\\
 \multicolumn{4}{l}{// dim = 2} \\
    &  $\vdots$ && \\
\\
 \multicolumn{4}{l}{// dim = $s$} \\
  $c_{11}$ & $c_{21}$ & $\cdots$ & $c_{r1}$ \\
  $c_{12}$ & $c_{22}$ & $\cdots$ & $c_{r2}$ \\
    &  $\vdots$ && \\
  $c_{1k}$ & $c_{2k}$ & $\cdots$ & $c_{rk}$ \\
\end {tabular}
}
\end{center}
%\caption { General format of the parameter file for
%  \externalclass{umontreal.iro.lecuyer.hups}{DigitalNetFromFile}.
%\label{formatdon}}
\end{figure}

The figure above gives the general format of the data file
needed by \texttt{DigitalNetFromFile}.
The values of the parameters on the left must appear in the file
as integers. On the right of each parameter, there is an optional
 comment that is disregarded by the reader program. In general, the
 Java line comments  \texttt{//} are accepted anywhere and will
ensure that the rest of the line is dropped by the reader. Blank lines
are also disregarded by the reader program. For each dimension, there must
 be a $k\times r$ matrix of integers in $\{0, 1, \ldots, b-1\}$ (note that
the matrices must appear in transposed form).

The predefined files of parameters are kept in different directories,
depending on the criteria used in the searches for the parameters defining
the digital net. These files have all been stored at the address
 \url{http://www.iro.umontreal.ca/~simardr/ssj/data}.
 Each file contains the parameters for a specific digital net.
% One may get a list of all available files in a directory by using
% method \method{listDir}{} below.
The name of the files gives information about the main parameters of
the digital net. For example, the file named \texttt{Edel/OOA2/B3S13R9C9St6}
 contains the parameters for a digital net proposed by Yves Edel
(see \url{http://www.mathi.uni-heidelberg.de/~yves/index.html}) based
on ordered orthogonal arrays; the digital net has base \texttt{B = 3},
dimension \texttt{S = 13}, the generating matrices have \texttt{R = 9} rows
and \texttt{C = 9} columns, and the strength of the net is \texttt{St = 6}.
%At the moment, there are no existing subdirectories of predefined files in SSJ.
\iffalse
At the moment, the existing subdirectories of predefined files in SSJ
are the following
 (for details on the available files,
 see \url{http://www.iro.umontreal.ca/~simardr/ssj/data})
(in OOA, O is the letter O, not the number 0):
\begin {table}[htb]
\begin{center}
% \caption {\label {tab:datadir1}}
\begin {tabular}{|l|l|l|}
\hline
  $\mbox{Directory}$  &   $\mbox{Remark}$  & $\mbox{Reference}$  \\
\hline
 \texttt{Edel/OOA2/} & Based on orthogonal ordered arrays & Yves Edel  \\
 \texttt{Edel/OOA3/} & Based on orthogonal ordered arrays & Yves Edel  \\
 \texttt{Edel/OOA4/} & Based on orthogonal ordered arrays & Yves Edel  \\
 \texttt{Edel/RSNet/} & Maximally equidistributed-collision free
   \cite{rLEC99a} & Yves Edel  \\
\hline
\end {tabular}
\label {tab:datadir1}
\end{center}
\end {table}
\fi


\bigskip\hrule\bigskip
%%%%%%%%%%%%%%%%%%%%%%%%%%%%%%%%%%%%%%%%%%%%%%%%%%%%%%%%%%%%%%%%%%%%%%%%%%%

\begin{code}
\begin{hide}
/*
 * Class:        DigitalNetFromFile
 * Description:  read the parameters defining a digital net from a file
                 or from a URL address
 * Environment:  Java
 * Software:     SSJ 
 * Copyright (C) 2001  Pierre L'Ecuyer and Université de Montréal
 * Organization: DIRO, Université de Montréal
 * @author       
 * @since

 * SSJ is free software: you can redistribute it and/or modify it under
 * the terms of the GNU General Public License (GPL) as published by the
 * Free Software Foundation, either version 3 of the License, or
 * any later version.

 * SSJ is distributed in the hope that it will be useful,
 * but WITHOUT ANY WARRANTY; without even the implied warranty of
 * MERCHANTABILITY or FITNESS FOR A PARTICULAR PURPOSE.  See the
 * GNU General Public License for more details.

 * A copy of the GNU General Public License is available at
   <a href="http://www.gnu.org/licenses">GPL licence site</a>.
 */
\end{hide}
package umontreal.iro.lecuyer.hups;\begin{hide}

import java.io.*;
import java.util.*;
import java.net.URL;
import java.net.MalformedURLException;
import umontreal.iro.lecuyer.util.PrintfFormat;
\end{hide}

public class DigitalNetFromFile extends DigitalNet \begin{hide} {
   private String filename;

   private void readMatrices (StreamTokenizer st,
                              int r, int k, int dim)
      throws IOException, NumberFormatException {
      // Read dim matrices with r rows and k columns.
      // dim is the dimension of the digital net.
      genMat = new int[dim * k][r];
      for (int i = 0; i < dim; i++)
         for (int c = 0; c < k; c++) {
             for (int j = 0; j < r; j++) {
                 st.nextToken ();
                 genMat[i*numCols + c][j]  = (int) st.nval;
             }
             // If we do not use all the rows, drop the unused ones.
             for (int j = r; j < numRows; j++) {
                 st.nextToken ();
             }
         }
   }


   void readData (StreamTokenizer st) throws
                                      IOException, NumberFormatException
   {
      // Read beginning of data file, but do not read the matrices
      st.eolIsSignificant (false);
      st.slashSlashComments (true);
      int i = st.nextToken ();
      if (i != StreamTokenizer.TT_NUMBER)
         throw new NumberFormatException(" readData: cannot read base");
      b = (int) st.nval;
      st.nextToken ();   numCols = (int) st.nval;
      st.nextToken ();   numRows = (int) st.nval;
      st.nextToken ();   numPoints = (int) st.nval;
      st.nextToken ();   dim = (int) st.nval;
      if (dim < 1)
         throw new IllegalArgumentException (" dimension dim <= 0");
   }


   static BufferedReader openURL (String filename)
                                  throws MalformedURLException, IOException {
      try {
         URL url = new URL (filename);
         BufferedReader input = new BufferedReader (
                                    new InputStreamReader (
                                        url.openStream()));
         return input;

      } catch (MalformedURLException e) {
         System.err.println (e + "   Invalid URL address:   " + filename);
         throw e;

      }  catch (IOException e) {
          // This can receive a FileNotFoundException
         System.err.println (e + " in openURL with " + filename);
         throw e;
      }
   }

   static BufferedReader openFile (String filename) throws
            IOException {
      try {
         BufferedReader input;
         File f = new File (filename);

         // If file with relative path name exists, read it
         if (f.exists()) {
            if (f.isDirectory())
               throw new IOException (filename + " is a directory");
            input = new BufferedReader (new FileReader (filename));
         } else {              // else read it from ssj.jar
            String pseudo = "umontreal/iro/lecuyer/hups/data/";
            StringBuffer pathname = new StringBuffer (pseudo);
            for (int ci = 0; ci < filename.length(); ci++) {
               char ch = filename.charAt (ci);
               if (ch == File.separatorChar)
                  pathname.append ('/');
               else
                  pathname.append (ch);
            }
            InputStream dataInput =
                DigitalNetFromFile.class.getClassLoader().getResourceAsStream (
                  pathname.toString());
            if (dataInput == null)
               throw new FileNotFoundException();
            input = new BufferedReader (new InputStreamReader (dataInput));
         }
         return input;

       } catch (FileNotFoundException e) {
         System.err.println (e + " *** cannot find  " + filename);
         throw e;

      } catch (IOException e) {
         // This will never catch FileNotFoundException since there
         // is a catch clause above.
         System.err.println (e + " cannot read from  " + filename);
         throw e;
      }
   }

\end{hide}
\end{code}
%%%%%%%%%%%%%%%%%%%%%%%%%%%%%%%%%
\subsubsection* {Constructors}
\begin{code}

   public DigitalNetFromFile (String filename, int r1, int w, int s1)
          throws MalformedURLException, IOException \begin{hide}
   {
      super ();
      BufferedReader input = null;
      StreamTokenizer st = null;
      try {
         if (filename.startsWith("http:") || filename.startsWith("ftp:"))
            input = openURL(filename);
         else
            input = openFile(filename);
         st = new StreamTokenizer (input);
         readData (st);

      } catch (MalformedURLException e) {
         System.err.println ("   Invalid URL address:   " + filename);
         throw e;
      } catch (FileNotFoundException e) {
         System.err.println ("   Cannot find  " + filename);
         throw e;
      } catch (NumberFormatException e) {
         System.err.println ("   Cannot read number from " + filename);
         throw e;
      }  catch (IOException e) {
         System.err.println ("   IOException:   " + filename);
         throw e;
      }

      if (b == 2) {
         System.err.println ("   base = 2, use DigitalNetBase2FromFile");
         throw new IllegalArgumentException
             ("base = 2, use DigitalNetBase2FromFile");
      }
      if ((double)numCols * Math.log ((double)b) > (31.0 * Math.log (2.0)))
         throw new IllegalArgumentException
            ("DigitalNetFromFile:   too many points" + PrintfFormat.NEWLINE);
      if (r1 > numRows)
         throw new IllegalArgumentException
            ("DigitalNetFromFile:   One must have   r1 <= Max num rows" +
                PrintfFormat.NEWLINE);
      if (s1 > dim)
         throw new IllegalArgumentException
            ("DigitalNetFromFile:   One must have   s1 <= Max dimension" +
                 PrintfFormat.NEWLINE);
      if (w < 0) {
         r1 = w = numRows;
         s1 = dim;
      }
      if (w < numRows)
         throw new IllegalArgumentException
            ("DigitalNetFromFile:   One must have   w >= numRows" +
              PrintfFormat.NEWLINE);

      try {
         readMatrices (st, r1, numCols, s1);
      } catch (NumberFormatException e) {
         System.err.println (e + "   cannot read matrices from " + filename);
         throw e;
      }  catch (IOException e) {
         System.err.println (e + "   cannot read matrices from  " + filename);
         throw e;
      }
      input.close();

      this.filename = filename;
      numRows = r1;
      dim = s1;
      outDigits = w;
      int x = b;
      for (int i=1; i<numCols; i++) x *= b;
      if (x != numPoints) {
         System.out.println ("DigitalNetFromFile:   numPoints != b^k");
         throw new IllegalArgumentException (" numPoints != b^k");
      }

      // Compute the normalization factors.
      normFactor = 1.0 / Math.pow ((double) b, (double) outDigits);
      double invb = 1.0 / b;
      factor = new double[outDigits];
      factor[0] = invb;
      for (int j = 1; j < outDigits; j++)
         factor[j] = factor[j-1] * invb;
  }\end{hide}
\end{code}
\begin{tabb}
    Constructs a digital net after reading its parameters from file
    {\texttt{filename}}. If a file named \texttt{filename}
   can be found relative to the program's directory, then the parameters
   will be read from this file; otherwise, they will be read from the file
   named  \texttt{filename} in the \texttt{ssj.jar} archive.
   If {\texttt{filename}} is a URL string, it will be read on
   the World Wide Web.
   For example, to construct a digital net from the parameters in file
   \texttt{B3S13R9C9St6} in the current directory,  one must give the string
   \texttt{"B3S13R9C9St6"} as argument to the constructor.
   As an example of a file read from the WWW, one may give
   as argument to the constructor the string
   \texttt{
  "http://www.iro.umontreal.ca/\~{}simardr/ssj/data/Edel/OOA3/B3S13R6C6St4"}.
   Parameter \texttt{w} gives the number of digits of resolution, \texttt{r1} is
   the number of rows, and \texttt{s1} is the dimension.
   Restrictions: \texttt{s1} must be less than the maximal dimension, and
   \texttt{r1} less than the maximal number of rows in the data file.
   Also \texttt{w} $\ge$ \texttt{r1}.
\end{tabb}
\begin{htmlonly}
   \param{filename}{Name of the file to be read}
   \param{r1}{Number of rows for the generating matrices}
   \param{w}{Number of digits of resolution}
   \param{s1}{Number of dimensions}
\end{htmlonly}
\begin{code}

   public DigitalNetFromFile (String filename, int s)
          throws MalformedURLException, IOException \begin{hide}
   {
       this (filename, -1, -1, s);
   }

   DigitalNetFromFile ()
   {
       super ();
   }\end{hide}
\end{code}
\begin{tabb}
   Same as \method{DigitalNetFromFile}{}\texttt{(filename, r, r, s)} where
   \texttt{s} is the dimension and  \texttt{r} is given in data file \texttt{filename}.
\end{tabb}
\begin{htmlonly}
   \param{filename}{Name of the file to be read}
   \param{s}{Number of dimensions}
\end{htmlonly}



%%%%%%%%%%%%%%%%%%%%%%%%%%%%%%%%%
\subsubsection*{Methods}

\begin{code}\begin{hide}

   public String toString() {
      StringBuffer sb = new StringBuffer ("File:   " + filename +
         PrintfFormat.NEWLINE);
      sb.append (super.toString());
      return sb.toString();
   }\end{hide}

   public String toStringDetailed() \begin{hide} {
      StringBuffer sb = new StringBuffer (toString());
      sb.append (PrintfFormat.NEWLINE + "n = " + numPoints  +
                 PrintfFormat.NEWLINE);
      sb.append ("dim = " + dim  + PrintfFormat.NEWLINE);
      for (int i = 0; i < dim; i++) {
         sb.append (PrintfFormat.NEWLINE + " // dim = " + (1 + i) +
                    PrintfFormat.NEWLINE);
         for (int c = 0; c < numCols; c++) {
            for (int r = 0; r < numRows; r++)
                sb.append (genMat[i*numCols + c][r] + " ");
            sb.append (PrintfFormat.NEWLINE);
         }
      }
      return sb.toString ();
   }\end{hide}
\end{code}
\begin{tabb}
    Writes the parameters and the generating matrices of this digital net
    to a string. This is useful to check that the file parameters have been
    read correctly.
\end{tabb}
\begin{code} \begin{hide}

   static class NetComparator implements Comparator {
      // Used to sort list of nets. Sort first by base, then by dimension,
      // then by the number of rows. Don't forget that base = 4 are in files
      // named B4_2* and that the computations are done in base 2.
      public int compare (Object o1, Object o2) {
         DigitalNetFromFile net1 = (DigitalNetFromFile) o1;
         DigitalNetFromFile net2 = (DigitalNetFromFile) o2;
         if (net1.b < net2.b)
            return -1;
         if (net1.b > net2.b)
            return 1;
         if (net1.filename.indexOf("_") >= 0 &&
             net2.filename.indexOf("_") < 0 )
            return 1;
         if (net2.filename.indexOf("_") >= 0 &&
             net1.filename.indexOf("_") < 0 )
            return -1;
         if (net1.dim < net2.dim)
            return -1;
         if (net1.dim > net2.dim)
            return 1;
         if (net1.numRows < net2.numRows)
            return -1;
         if (net1.numRows > net2.numRows)
            return 1;
         return 0;
      }
   }


   private static List getListDir (String dirname) throws IOException {
      try {
         String pseudo = "umontreal/iro/lecuyer/hups/data/";
         StringBuffer pathname = new StringBuffer (pseudo);
         for (int ci = 0; ci < dirname.length(); ci++) {
            char ch = dirname.charAt (ci);
            if (ch == File.separatorChar)
               pathname.append ('/');
            else
               pathname.append (ch);
         }
         URL url = DigitalNetFromFile.class.getClassLoader().getResource (
                      pathname.toString());
         File dir = new File (url.getPath());
         if (!dir.isDirectory())
            throw new IllegalArgumentException (
               dirname + " is not a directory");
         File[] files = dir.listFiles();
         List alist = new ArrayList (200);
         if (!dirname.endsWith (File.separator))
            dirname += File.separator;
         for (int i = 0; i < files.length; i++) {
            if (files[i].isDirectory())
               continue;
            if (files[i].getName().endsWith ("gz") ||
                files[i].getName().endsWith ("zip"))
               continue;
            DigitalNetFromFile net = new DigitalNetFromFile();
            BufferedReader input = net.openFile(dirname + files[i].getName());
            StreamTokenizer st = new StreamTokenizer (input);
            net.readData (st);
            net.filename = files[i].getName();
            alist.add (net);
         }
         if (alist != null && !files[0].isDirectory())
            Collections.sort (alist, new NetComparator ());
         return alist;

      } catch (NullPointerException e) {
         System.err.println ("getListDir: cannot find directory   " + dirname);
         throw e;

      } catch (NumberFormatException e) {
         System.err.println (e + "***   cannot read number ");
         throw e;

      }  catch (IOException e) {
         System.err.println (e);
         throw e;
      }
   }


   private static String listFiles (String dirname) {
      try {
         String pseudo = "umontreal/iro/lecuyer/hups/data/";
         StringBuffer pathname = new StringBuffer (pseudo);
         for (int ci = 0; ci < dirname.length(); ci++) {
            char ch = dirname.charAt (ci);
            if (ch == File.separatorChar)
               pathname.append ('/');
            else
               pathname.append (ch);
         }
         URL url = DigitalNetFromFile.class.getClassLoader().getResource (
                      pathname.toString());
         File dir = new File (url.getPath());
         File[] list = dir.listFiles();
         List alist = new ArrayList (200);
         final int NPRI = 3;
         StringBuffer sb = new StringBuffer(1000);
         for (int i = 0; i < list.length; i++) {
            if (list[i].isDirectory()) {
               sb.append (PrintfFormat.s(-2, list[i].getName()));
               sb.append (File.separator + PrintfFormat.NEWLINE);
            } else {
               sb.append (PrintfFormat.s(-25, list[i].getName()));
               if (i % NPRI == 2)
                  sb.append (PrintfFormat.NEWLINE);
            }
         }
         if (list.length % NPRI > 0)
            sb.append (PrintfFormat.NEWLINE);
         return sb.toString();

      } catch (NullPointerException e) {
         System.err.println ("listFiles: cannot find directory   " + dirname);
         throw e;
      }
   }\end{hide}

   public static String listDir (String dirname) throws IOException \begin{hide} {
      try {
         List list = getListDir (dirname);
         if (list == null || list.size() == 0)
            return listFiles (dirname);
         StringBuffer sb = new StringBuffer(1000);

         sb.append ("Directory:   " + dirname  + PrintfFormat.NEWLINE +
                    PrintfFormat.NEWLINE);
         sb.append (PrintfFormat.s(-25, "     File") +
                    PrintfFormat.s(-15, "       Base") +
                    PrintfFormat.s(-10, "Dimension") +
                    PrintfFormat.s(-10, " numRows") +
                    PrintfFormat.s(-10, "numColumns" +
                    PrintfFormat.NEWLINE));
         int base = 0;
         for (int i = 0; i < list.size(); i++) {
            DigitalNet net = (DigitalNet) list.get(i);
            int j = ((DigitalNetFromFile)net).filename.lastIndexOf
                (File.separator);
            if (net.b != base) {
               sb.append (
      "----------------------------------------------------------------------"
            + PrintfFormat.NEWLINE);
            base = net.b;
            }
            String name = ((DigitalNetFromFile)net).filename.substring(j+1);
            sb.append (PrintfFormat.s(-25, name) +
                       PrintfFormat.d(10, net.b) +
                       PrintfFormat.d(10, net.dim) +
                       PrintfFormat.d(10, net.numRows) +
                       PrintfFormat.d(10, net.numCols) +
                       PrintfFormat.NEWLINE);
         }
         return sb.toString();

      } catch (NullPointerException e) {
         System.err.println (
            "formatPlain: cannot find directory   " + dirname);
         throw e;
      }
   }\end{hide}
\end{code}
\begin{tabb}
  Lists all files (or directories) in directory \texttt{dirname}. Only relative
  pathnames should be used. The files are  parameter files used in defining
  digital nets.  For example, calling \texttt{listDir("")} will give the list
  of the main data directory in SSJ, while calling \texttt{listDir("Edel/OOA2")}
  will give the list of all files in directory \texttt{Edel/OOA2}.
 \end{tabb}
\begin{code}

   public static void listDirHTML (String dirname, String filename)
          throws IOException \begin{hide} {
      String list = listDir(dirname);
      StreamTokenizer st = new StreamTokenizer (new StringReader(list));
      st.eolIsSignificant(true);
      st.ordinaryChar('/');
      st.ordinaryChar('_');
      st.ordinaryChar('-');
      st.wordChars('-', '-');
      st.wordChars('_', '_');
      st.slashSlashComments(false);
      st.slashStarComments(false);
      PrintWriter out = new PrintWriter (
                            new BufferedWriter (
                               new FileWriter (filename)));
      out.println ("<html>" + PrintfFormat.NEWLINE +
          "<head>" + PrintfFormat.NEWLINE + "<title>");
      while (st.nextToken () != st.TT_EOL)
         ;
      out.println ( PrintfFormat.NEWLINE + "</title>" +
           PrintfFormat.NEWLINE + "</head>");
//      out.println ("<body background bgcolor=#e1eae8 vlink=#c00000>");
      out.println ("<body>");
      out.println ("<table border>");
      out.println ("<caption> Directory: " + dirname + "</caption>");
      st.nextToken(); st.nextToken();
      while (st.sval.compareTo ("File") != 0)
         st.nextToken();
      out.print ("<tr align=center><th>" + st.sval + "</th>");
      while (st.nextToken () != st.TT_EOL) {
         out.print ("<th>" + st.sval + "</th>" );
      }
      out.println ("</tr>" + PrintfFormat.NEWLINE);
      while (st.nextToken () != st.TT_EOF) {
          switch(st.ttype) {
          case StreamTokenizer.TT_EOL:
             out.println ("</tr>");
             break;
          case StreamTokenizer.TT_NUMBER:
             out.print ("<td>" + (int) st.nval + "</td>" );
             break;
          case StreamTokenizer.TT_WORD:
             if (st.sval.indexOf ("---") >= 0) {
                st.nextToken ();
                continue;
             }
             out.print ("<tr align=center><td>" + st.sval + "</td>");
             break;
          default:
             out.print (st.sval);
             break;
        }
      }

      out.println ("</table>");
      out.println ("</body>" + PrintfFormat.NEWLINE + "</html>");
      out.close();
}\end{hide}
\end{code}
\begin{tabb}
 Creates a list of all data files in directory \texttt{dirname} and writes
that list in format HTML in output file \texttt{filename}.
Each data file contains the parameters required to build a digital net.
The resulting list contains a line for each data file giving the
name of the file, the base, the dimension, the number of rows and
the number of columns of the corresponding digital net.
 \end{tabb}
\begin{code}
\begin{hide}
}
\end{hide}
\end{code}

\defclass{FaureSequence}

This class implements digital nets or digital sequences formed by the
 first $n = b^k$ points of the Faure sequence in base $b$.
Values of $n$ up to $2^{31}$ are allowed.
% When the base $b$ is not specified, it is taken as the smallest
% prime number greater or equal to the selected dimension.
One has $r = k$.
The generator matrices are
\eq
  \mathbf{C}_j = \mathbf{P}^j\ \mod b
\endeq
for $j=0,\dots,s-1$, where $\mathbf{P}$ is a $k\times k$ upper
 triangular matrix whose entry $(l,c)$ is the number of combinations
of $l$ objects among $c$\latex{, ${c\choose l}$},
for $l\le c$ and is 0 for $l > c$.
% This matrix $\mathbf{P}$ is the transpose of the $k\times k$ \emph{Pascal matrix}.
The matrix $\mathbf{C}_0$ is the identity, $\mathbf{C}_1 = \mathbf{P}$,
and the other $\mathbf{C}_j$'s can be defined recursively via
$\mathbf{C}_j = \mathbf{P} \mathbf{C}_{j-1} \mod b$.
Our implementation uses the recursion
\begin{equation}
\begin{latexonly}
  {c \choose l} = {{c-1} \choose l} + {{c-1} \choose {l-1}}
\end{latexonly}
\begin{htmlonly}
 \mbox{Combination}(c, l)\ = \ \mbox{Combination}(c-1, l) \; + \;
  \mbox{Combination}(c-1, l-1)
\end{htmlonly}
\end{equation}
to evaluate the binomial coefficients in the matrices $\mathbf{C}_j$,
as suggested by Fox\latex{ \cite{rFOX86a} (see also \cite{fGLA04a}, page 301)}.
The entries $x_{j,l,c}$ of $\mathbf{C}_j$ are computed as follows:
\[
\begin{array}{lcll}
 x_{j,c,c} &=& 1             &\quad\mbox{ for } c=0,\dots,k-1,\\[4pt]
 x_{j,0,c} &=& j x_{j,0,c-1} &\quad\mbox{ for } c=1,\dots,k-1, \\[4pt]
 x_{j,l,c} &=& x_{j,l-1,c-1} + j x_{j,l,c-1}
                      &\quad\mbox{ for } 2\le c < l \le k-1, \\[4pt]
 x_{j,l,c} &=& 0      &\quad\mbox{ for } c>l \mbox{ or } l \ge k.
\end{array}
\]

For any integer $m > 0$ and $\nu\ge 0$, if we look at the
vector $(u_{i,j,1},\dots,u_{i,j,m})$ (the first $m$ digits
of coordinate $j$ of the output) when $i$ goes from
$\nu b^m$ to $(\nu+1)b^m - 1$, this vector takes each of its $b^m$
possible values exactly once.
In particular, for $\nu = 0$, $u_{i,j}$ visits each value in the
set $\{0, 1/b^m, 2/b^m, \dots, (b^m-1)/b^m\}$ exactly once, so all
one-dimensional projections of the point set are identical.
However, the values are visited in a different order for
the different values of $j$ (otherwise all coordinates would be identical).
For $j=0$, they are visited in the same
order as in the van der Corput sequence in base $b$.

An important property of Faure nets is that for any integers $m > 0$
and $\nu\ge 0$, the point set
$\{\bu_i$ for $i = \nu b^m,\dots, (\nu+1)b^m -1\}$
is a $(0,m,s)$-net in base $b$.
In particular, for $n = b^k$, the first $n$ points form a
 $(0,k,s)$-net in base $b$.
The Faure nets are also projection-regular and dimension-stationary\latex{
(see \cite{vLEC02a} for definitions of these properties)}.

To obtain digital nets from the \emph{generalized Faure sequence}
\latex{ \cite{rTEZ95a}}, where $\mathbf{P}_j$ is left-multiplied by some
invertible matrix $\mathbf{A}_j$, it suffices to apply an appropriate
matrix scramble (e.g., via \method{leftMatrixScramble}{}).
This changes the order in which $u_{i,j}$ visits its different
values, for each coordinate $j$, but does not change the set of values
that are visited.  The $(0,m,s)$-net property stated above remains valid.


%%%%%%%%%%%%%%%%%%%%%%%%%%%%%%%%%%%%%%%%
\bigskip\hrule\bigskip

\begin{code}
\begin{hide}
/*
 * Class:        FaureSequence
 * Description:  
 * Environment:  Java
 * Software:     SSJ 
 * Copyright (C) 2001  Pierre L'Ecuyer and Université de Montréal
 * Organization: DIRO, Université de Montréal
 * @author       
 * @since

 * SSJ is free software: you can redistribute it and/or modify it under
 * the terms of the GNU General Public License (GPL) as published by the
 * Free Software Foundation, either version 3 of the License, or
 * any later version.

 * SSJ is distributed in the hope that it will be useful,
 * but WITHOUT ANY WARRANTY; without even the implied warranty of
 * MERCHANTABILITY or FITNESS FOR A PARTICULAR PURPOSE.  See the
 * GNU General Public License for more details.

 * A copy of the GNU General Public License is available at
   <a href="http://www.gnu.org/licenses">GPL licence site</a>.
 */
\end{hide}
package umontreal.iro.lecuyer.hups;\begin{hide}
import umontreal.iro.lecuyer.util.PrintfFormat;

\end{hide}

public class FaureSequence extends DigitalSequence \begin{hide} {

    // Maximum dimension for the case where b is not specified.
    // Can be extended by extending the precomputed array prime[].
    private static final int MAXDIM = 500;

    // For storing the generator matrices for given dim and numPoints.
    private int[][][] v;
\end{hide}
\end{code}
%%%%%%%%%%%%%%%%%%%%%%%%%%%%%%%%%
\subsubsection* {Constructors}
\begin{code}

   public FaureSequence (int b, int k, int r, int w, int dim) \begin{hide} {
      init (b, k, r, w, dim);
   }

   private void init (int b, int k, int r, int w, int dim) {
      if (dim < 1)
         throw new IllegalArgumentException
            ("Dimension for FaureSequence must be > 1");
      if ((double)k * Math.log ((double)b) > (31.0 * Math.log (2.0)))
         throw new IllegalArgumentException
            ("Trying to construct a FaureSequence with too many points");
      if (r < k || w < r)
         throw new IllegalArgumentException
            ("One must have k <= r <= w for FaureSequence");
      this.b    = b;
      numCols   = k;
      numRows   = r;
      outDigits = w;
      this.dim  = dim;

      int i, j;
      numPoints = b;
      for (i=1; i<k; i++) numPoints *= b;

      // Compute the normalization factors.
      normFactor = 1.0 / Math.pow ((double) b, (double) outDigits);
//      EpsilonHalf = 0.5*normFactor;
      double invb = 1.0 / b;
      factor = new double[outDigits];
      factor[0] = invb;
      for (j = 1; j < outDigits; j++)
         factor[j] = factor[j-1] * invb;

      genMat = new int[dim * numCols][numRows];
      initGenMat();
   }
\end{hide}
\end{code}
\begin{tabb}
    Constructs a digital net in base $b$,
    with $n = b^k$ points and $w$ output digits,
    in \texttt{dim} dimensions.
    The points are the first $n$ points of the Faure sequence.
    The generator matrices $\mathbf{C}_j$ are $r\times k$.
    Unless, one plans to apply a randomization on more than $k$ digits
    (e.g., a random digital shift for $w > k$ digits, or a linear
    scramble yielding $r > k$ digits), one should
    take $w = r = k$ for better computational efficiency.
    Restrictions: \texttt{dim} $\le 500$ and $b^k \le 2^{31}$.
\end{tabb}
\begin{htmlonly}
   \param{b}{base}
   \param{k}{there will be b^k points}
   \param{r}{number of rows in the generator matrices}
   \param{w}{number of output digits}
   \param{dim}{dimension of the point set}
\end{htmlonly}
\begin{code}

   public FaureSequence (int n, int dim) \begin{hide} {
      if ((dim < 1) || (dim > MAXDIM))
         throw new IllegalArgumentException
            ("Dimension for Faure net must be > 1 and < " + MAXDIM);
      b = getSmallestPrime (dim);
      numCols = (int) Math.ceil (Math.log ((double) n)
                                 / Math.log ((double) b));
      outDigits = (int) Math.floor (Math.log ((double)(1 << (MAXBITS - 1)))
                                 / Math.log ((double)b));
      outDigits = Math.max (outDigits, numCols);
      numRows = outDigits;
      init (b, numCols, numRows, outDigits, dim);
   }
\end{hide}
\end{code}
\begin{tabb}
  Same as \method{FaureSequence}{}\texttt{(b, k, w, w, dim)}
  with base $b$ equal to the smallest prime larger or equal to \texttt{dim},
  and with \emph{at least} \texttt{n} points.
\hrichard{Ce constructeur devrait-il dispara\^\i tre?}
  The values of $k$, $r$, and $w$ are taken as
  $k = \lceil \log_b n\rceil$ and
  $r = w = \max(k, \lfloor 30 / \log_2 b\rfloor)$.
\end{tabb}
\begin{htmlonly}
   \param{n}{minimal number of points}
   \param{dim}{dimension of the point set}
\end{htmlonly}
\begin{code}\begin{hide}

   public String toString() {
      StringBuffer sb = new StringBuffer ("Faure sequence:" +
                  PrintfFormat.NEWLINE);
      sb.append (super.toString());
      return sb.toString();
   }


   public void extendSequence (int k) {
      init (b, k, numRows, outDigits, dim);
   }


   // Fills up the generator matrices in genMat for a Faure sequence.
   // See Glasserman (2004), \cite{fGLA04a}, page 301.
   private void initGenMat() {
      int j, c, l;
      // Initialize C_0 to the identity (for first coordinate).
      for (c = 0; c < numCols; c++) {
         for (l = 0; l < numRows; l++)
            genMat[c][l] = 0;
         genMat[c][c] = 1;
      }
      // Compute C_1, ... ,C_{dim-1}.
      for (j = 1; j < dim; j++) {
         genMat[j*numCols][0] = 1;
         for (c = 1; c < numCols; c++) {
            genMat[j*numCols+c][c] = 1;
            genMat[j*numCols+c][0] = (j * genMat[j*numCols+c-1][0]) % b;
         }
         for (c = 2; c < numCols; c++) {
            for (l = 1; l < c; l++)
               genMat[j*numCols+c][l] = (genMat[j*numCols+c-1][l-1]
                                        + j * genMat[j*numCols+c-1][l]) % b;
         }
         for (c = 0; c < numCols; c++)
            for (l = c+1; l < numRows; l++)
               genMat[j*numCols+c][l] = 0;
      }
/*
      for (j = 0; j < dim; j++) {
     for (l = 0; l < numRows; l++) {
         for (c = 0; c < numCols; c++)
            System.out.print ("  " + genMat[j*numCols+c][l]);
       System.out.println ("");
      }
        System.out.println ("");
  }
*/
   }

/*
   // Fills up the generator matrices in genMat for a Faure net.
   // See Glasserman (2004), \cite{fGLA04a}, page 301.
   protected void initGenMatNet() {
      int j, c, l, start;
      // Initialize C_0 to the reflected identity (for first coordinate).
      for (c = 0; c < numCols; c++) {
         for (l = 0; l < numRows; l++)
            genMat[c][l] = 0;
         genMat[c][numCols-c-1] = 1;
      }
      // Initialize C_1 to the identity (for second coordinate).
      for (c = 0; c < numCols; c++) {
         for (l = 0; l < numRows; l++)
            genMat[numCols+c][l] = 0;
         genMat[numCols+c][c] = 1;
      }
      // Compute C_2, ... ,C_{dim-1}.
      for (j = 2; j < dim; j++) {
         start = j * numCols;
         genMat[start][0] = 1;
         for (c = 1; c < numCols; c++) {
            genMat[start+c][c] = 1;
            genMat[start+c][0] = ((j-1) * genMat[start+c-1][0]) % b;
         }
         for (c = 2; c < numCols; c++) {
            for (l = 1; l < c; l++)
               genMat[start+c][l] = (genMat[start+c-1][l-1]
                                     + (j-1) * genMat[start+c-1][l]) % b;
         }
         for (c = 0; c < numCols; c++)
            for (l = c+1; l < numRows; l++)
               genMat[start+c][l] = 0;
      }
   }
*/

   // Returns the smallest prime larger or equal to d.
   private int getSmallestPrime (int d) {
      return primes[d-1];
   }

   // Gives the appropriate prime base for each dimension.
   // Perhaps should be internal to getPrime, and non-static, to avoid
   // wasting time and memory when this array is not needed ???
   static final int primes[] =
      {2,2,3,5,5,7,7,11,11,11,11,13,13,17,17,17,17,19,19,23,
     23,23,23,29,29,29,29,29,29,31,31,37,37,37,37,37,37,41,41,41,
     41,43,43,47,47,47,47,53,53,53,53,53,53,59,59,59,59,59,59,61,
     61,67,67,67,67,67,67,71,71,71,71,73,73,79,79,79,79,79,79,83,
     83,83,83,89,89,89,89,89,89,97,97,97,97,97,97,97,97,101,101,101,
     101,103,103,107,107,107,107,109,109,113,113,113,113,127,127,127,127,127,127,127,
     127,127,127,127,127,127,127,131,131,131,131,137,137,137,137,137,137,139,139,149,
     149,149,149,149,149,149,149,149,149,151,151,157,157,157,157,157,157,163,163,163,
     163,163,163,167,167,167,167,173,173,173,173,173,173,179,179,179,179,179,179,181,
     181,191,191,191,191,191,191,191,191,191,191,193,193,197,197,197,197,199,199,211,
     211,211,211,211,211,211,211,211,211,211,211,223,223,223,223,223,223,223,223,223,
     223,223,223,227,227,227,227,229,229,233,233,233,233,239,239,239,239,239,239,241,
     241,251,251,251,251,251,251,251,251,251,251,257,257,257,257,257,257,263,263,263,
     263,263,263,269,269,269,269,269,269,271,271,277,277,277,277,277,277,281,281,281,
     281,283,283,293,293,293,293,293,293,293,293,293,293,307,307,307,307,307,307,307,
     307,307,307,307,307,307,307,311,311,311,311,313,313,317,317,317,317,331,331,331,
     331,331,331,331,331,331,331,331,331,331,331,337,337,337,337,337,337,347,347,347,
     347,347,347,347,347,347,347,349,349,353,353,353,353,359,359,359,359,359,359,367,
     367,367,367,367,367,367,367,373,373,373,373,373,373,379,379,379,379,379,379,383,
     383,383,383,389,389,389,389,389,389,397,397,397,397,397,397,397,397,401,401,401,
     401,409,409,409,409,409,409,409,409,419,419,419,419,419,419,419,419,419,419,421,
     421,431,431,431,431,431,431,431,431,431,431,433,433,439,439,439,439,439,439,443,
     443,443,443,449,449,449,449,449,449,457,457,457,457,457,457,457,457,461,461,461,
     461,463,463,467,467,467,467,479,479,479,479,479,479,479,479,479,479,479,479,487,
     487,487,487,487,487,487,487,491,491,491,491,499,499,499,499,499,499,499,499,503};

}
\end{hide}
\end{code}


\include{DigitalNetBase2}
\include{DigitalSequenceBase2}
\defclass {DigitalNetBase2FromFile}

This class allows us to read the parameters defining a digital net 
 {\em in base 2\/} either from a file, or from a URL address on the
 World Wide Web. See the documentation in 
\externalclass{umontreal.iro.lecuyer.hups}{DigitalNetFromFile}.
The parameters used in building the net are those defined in class
\externalclass{umontreal.iro.lecuyer.hups}{DigitalNetBase2}.
 The format of the data files must be the following (where $B$ is any $C_j$):
\begin{htmlonly} (see the format in \texttt{guidehups.pdf})\end{htmlonly}

\begin{figure}
\begin{center}
\tt
\fbox {
\begin {tabular}{ll}
 \multicolumn{2}{l}{// Any number of comment lines starting with //} \\
     $2$      & //    Base  \\
     $k$      & //    Number of columns   \\
     $r$      & //    Number of rows  \\
     $n$      & //    Number of points = $2^k$  \\
     $s$      & //    Dimension of points \\
\\
 \multicolumn{2}{l}{// dim = 1} \\
  $a_{1}$ &  //  $= 2^{30}B_{11} + 2^{29}B_{21} + \cdots + 2^{31 - r}B_{r1}$ \\
  $a_{2}$ &  //  $= 2^{30}B_{12} + 2^{29}B_{22} + \cdots + 2^{31 - r}B_{r2}$ \\
  $\vdots$ & \\
  $a_{k}$ & \\
\\
 \multicolumn{2}{l}{// dim = 2} \\
   $\vdots$ & \\
\\
 \multicolumn{2}{l}{// dim = $s$} \\
  $a_{1}$ &  \\
  $a_{2}$ &  \\
  $\vdots$ &\\
  $a_{k}$ & \\
\end {tabular}
}
\end{center}
\end{figure}

 For each dimension $j$, there must be a $k$-vector of 32-bit integers
 (the $a_i$) corresponding to the columns of $\mathbf{C}_j$. The 
  correspondance is such that integer
  $a_i = 2^{30}(\mathbf{C}_j)_{1i} + 2^{29}(\mathbf{C}_j)_{2i} +
   \cdots +    2^{31 - r}(\mathbf{C}_j)_{ri}$.


\bigskip\hrule\bigskip
%%%%%%%%%%%%%%%%%%%%%%%%%%%%%%%%%%%%%%%%%%%%%%%%%%%%%%%%%%%%%%%%%%%%%%%%%%%

\begin{code}
\begin{hide}
/*
 * Class:        DigitalNetBase2FromFile
 * Description:  read the parameters defining a digital net in base 2
                 from a file or from a URL address
 * Environment:  Java
 * Software:     SSJ 
 * Copyright (C) 2001  Pierre L'Ecuyer and Université de Montréal
 * Organization: DIRO, Université de Montréal
 * @author       
 * @since

 * SSJ is free software: you can redistribute it and/or modify it under
 * the terms of the GNU General Public License (GPL) as published by the
 * Free Software Foundation, either version 3 of the License, or
 * any later version.

 * SSJ is distributed in the hope that it will be useful,
 * but WITHOUT ANY WARRANTY; without even the implied warranty of
 * MERCHANTABILITY or FITNESS FOR A PARTICULAR PURPOSE.  See the
 * GNU General Public License for more details.

 * A copy of the GNU General Public License is available at
   <a href="http://www.gnu.org/licenses">GPL licence site</a>.
 */
\end{hide}
package umontreal.iro.lecuyer.hups;\begin{hide}

import java.io.*;
import java.net.URL;
import java.net.MalformedURLException;
import umontreal.iro.lecuyer.util.PrintfFormat;
\end{hide}

public class DigitalNetBase2FromFile extends DigitalNetBase2 \begin{hide} {
   private String filename;

   // s is the effective dimension if > 0, otherwise it is dim
   private void readData (Reader re, int r1, int s1)
       throws IOException, NumberFormatException
   {
      try {
         StreamTokenizer st = new StreamTokenizer (re);
         if (st == null) return;
         st.eolIsSignificant (false);
         st.slashSlashComments (true);

         int i = st.nextToken ();
         if (i != StreamTokenizer.TT_NUMBER)
            throw new NumberFormatException();
         b = (int) st.nval;
         st.nextToken ();   numCols = (int) st.nval;
         st.nextToken ();   numRows = (int) st.nval;
         st.nextToken ();   numPoints = (int) st.nval;
         st.nextToken ();   dim = (int) st.nval;
         if (dim < 1) {
            System.err.println (PrintfFormat.NEWLINE + 
                "DigitalNetBase2FromFile:   dimension dim <= 0");
            throw new IllegalArgumentException ("dimension dim <= 0");
         }
         if (r1 > numRows) 
            throw new IllegalArgumentException (
            "DigitalNetBase2FromFile:   One must have   r1 <= Max num rows");
         if (s1 > dim) {
            throw new IllegalArgumentException ("s1 is too large");
         }
         if (s1 > 0)
            dim = s1;
         if (r1 > 0)
            numRows = r1;

         if (b != 2) {
            System.err.println (
              "***** DigitalNetBase2FromFile:    only base 2 allowed");
            throw new IllegalArgumentException ("only base 2 allowed");
         }
         genMat = new int[dim * numCols];
         for (i = 0; i < dim; i++)
            for (int c = 0; c < numCols; c++) {
                st.nextToken ();
                genMat[i*numCols + c] = (int) st.nval;
            }

      } catch (NumberFormatException e) {
         System.err.println (
            "   DigitalNetBase2FromFile:   not a number  " + e);
         throw e;
      }
   }


    private void maskRows (int r, int w) {
       // Keep only the r most significant bits and set the others to 0.
       int mask = (int) ((1L << r) - 1);
       mask <<= MAXBITS - r;
       for (int i = 0; i < dim; i++)
          for (int c = 0; c < numCols; c++) {
              genMat[i*numCols + c] &= mask;
              genMat[i*numCols + c] >>= MAXBITS - w;
          }
    }
\end{hide}
\end{code}
%%%%%%%%%%%%%%%%%%%%%%%%%%%%%%%%%
\subsubsection* {Constructor}
\begin{code} 

   public DigitalNetBase2FromFile (String filename, int r1, int w, int s1) 
         throws IOException, MalformedURLException\begin{hide} 
   {
      super ();
      if (w < r1 || w > MAXBITS) 
         throw new IllegalArgumentException (" Must have numRows <= w <= 31");

      BufferedReader input;
      if (filename.startsWith("http:") || filename.startsWith("ftp:"))
         input = DigitalNetFromFile.openURL(filename);
      else
         input = DigitalNetFromFile.openFile(filename);

      try {
         readData (input, r1, s1);
      } catch (NumberFormatException e) {
         System.err.println (
            "   DigitalNetBase2FromFile:   cannot read from   " + filename);
         throw e;

      }  catch (IOException e) {
         System.err.println (
            "   DigitalNetBase2FromFile:  cannot read from  " + filename);
         throw e;
      }
      input.close();
      maskRows (numRows, w);
      outDigits = w;
      if (numCols >= MAXBITS) 
         throw new IllegalArgumentException (" Must have numCols < 31");

      this.filename = filename;
      int x = (1 << numCols);
      if (x != numPoints) {
         System.out.println ("numPoints != 2^k");
         throw new IllegalArgumentException ("numPoints != 2^k");
      }
      // Compute the normalization factors.  
      normFactor = 1.0 / ((double) (1L << (outDigits)));

  }\end{hide}
\end{code} 
\begin{tabb}
    Constructs a digital net in base 2 after reading its parameters from file
    {\texttt{filename}}. See the documentation in
  \externalclass{umontreal.iro.lecuyer.hups}{DigitalNetFromFile}.
   Parameter \texttt{w} gives the number of bits of resolution, \texttt{r1} is
   the number of rows, and \texttt{s1} is the dimension.
   Restrictions: \texttt{s1} must be less than the maximal dimension, and
   \texttt{r1} less than the maximal number of rows in the data file.
   Also \texttt{w} $\ge$ \texttt{r1}.
\end{tabb}
\begin{htmlonly}
   \param{filename}{Name of the file to be read}
   \param{r1}{Number of rows for the generating matrices}
   \param{w}{Number of bits of resolution}
   \param{s1}{Number of dimensions}
\end{htmlonly}
\begin{code} 

   public DigitalNetBase2FromFile (String filename, int s1) 
        throws IOException, MalformedURLException\begin{hide}
   {
       this (filename, -1, 31, s1);
   }\end{hide}
\end{code} 
\begin{tabb}
   Same as \method{DigitalNetBase2FromFile}{}\texttt{(filename, r, 31, s1)} where
   \texttt{s1} is the dimension and \texttt{r} is given in data file \texttt{filename}.
\end{tabb}
\begin{htmlonly}
   \param{filename}{Name of the file to be read}
   \param{s1}{Number of dimensions}
\end{htmlonly}


%%%%%%%%%%%%%%%%%%%%%%%%%%%%%%%%%
\subsubsection*{Methods}

\begin{code}\begin{hide} 

   public String toString() {
      StringBuffer sb = new StringBuffer ("File:  " + filename  +
         PrintfFormat.NEWLINE);
      sb.append (super.toString());
      return sb.toString();
   }\end{hide}

   public String toStringDetailed() \begin{hide} {
      StringBuffer sb = new StringBuffer (toString() + PrintfFormat.NEWLINE);
      sb.append ("dim = " + dim + PrintfFormat.NEWLINE);
      for (int i = 0; i < dim; i++) {
         sb.append (PrintfFormat.NEWLINE + "// dim = " + (1 + i) +
              PrintfFormat.NEWLINE);
         for (int c = 0; c < numCols; c++)
            sb.append  (genMat[i*numCols + c]  + PrintfFormat.NEWLINE);
      }
      sb.append ("--------------------------------" + PrintfFormat.NEWLINE);
      return sb.toString ();
   }\end{hide}
\end{code} 
\begin{tabb}
    Writes the parameters and the generating matrices of this digital net
    to a string.
    This is useful to check that the file parameters have been read correctly.
\end{tabb}
\begin{code}

   public static String listDir (String dirname) throws IOException \begin{hide} {
      return DigitalNetFromFile.listDir(dirname);
   }\end{hide}
\end{code} 
\begin{tabb}
  Lists all files (or directories) in directory \texttt{dirname}. Only relative
  pathnames should be used. The files are  parameter files used in defining
  digital nets.  For example, calling \texttt{listDir("")} will give the list
  of the main data directory in SSJ, while calling \texttt{listDir("Edel/OOA2")}
  will give the list of all files in directory \texttt{Edel/OOA2}. 
 \end{tabb}
\begin{code}
\begin{hide} 
}
\end{hide}
\end{code}

\include{SobolSequence}
\defclass {NiedSequenceBase2}

This class implements digital sequences constructed from the
Niederreiter sequence in base 2.
\latex{For details on these point sets, see \cite{rBRA92a}.}
\hpierre{The generator matrices do not seem to be computed correctly
        in this implementation!}
\hrichard{Corrig\'e: elles le sont maintenant.}

%%%%%%%%%%%%%%%%%%%%%%%

\bigskip\hrule
\begin{code}
\begin{hide}
/*
 * Class:        NiedSequenceBase2
 * Description:  digital Niederreiter sequences in base 2.
 * Environment:  Java
 * Software:     SSJ 
 * Copyright (C) 2001  Pierre L'Ecuyer and Université de Montréal
 * Organization: DIRO, Université de Montréal
 * @author       
 * @since

 * SSJ is free software: you can redistribute it and/or modify it under
 * the terms of the GNU General Public License (GPL) as published by the
 * Free Software Foundation, either version 3 of the License, or
 * any later version.

 * SSJ is distributed in the hope that it will be useful,
 * but WITHOUT ANY WARRANTY; without even the implied warranty of
 * MERCHANTABILITY or FITNESS FOR A PARTICULAR PURPOSE.  See the
 * GNU General Public License for more details.

 * A copy of the GNU General Public License is available at
   <a href="http://www.gnu.org/licenses">GPL licence site</a>.
 */
\end{hide}
package umontreal.iro.lecuyer.hups;\begin{hide}

import java.io.Serializable;
import java.io.ObjectInputStream;
import java.io.InputStream;
import java.io.FileNotFoundException;
import java.io.IOException;
import umontreal.iro.lecuyer.util.PrintfFormat;
\end{hide}

public class NiedSequenceBase2 extends DigitalSequenceBase2\begin{hide} { 

   private static final int MAXDIM = 318;  // Maximum dimension.
   private static final int NUMCOLS = 30;  // Maximum number of columns.
\end{hide} 
\end{code}
%%%%%%%%%%%%%%%%%%%%%%%%%%%%%%%%%
\subsubsection* {Constructor}
\begin{code} 

   public NiedSequenceBase2 (int k, int w, int dim) \begin{hide} {
      init (k, w, w, dim);
   } 
\end{hide}
\end{code} 
\begin{tabb}
    Constructs a new digital sequence in base 2 from the first $n=2^k$ points 
    of the Niederreiter sequence,
    with $w$ output digits, in \texttt{dim} dimensions.
    The generator matrices $\mathbf{C}_j$ are $w\times k$.
%    Unless, one plans to apply a randomization on more than $k$ digits
%    (e.g., a random digital shift for $w > k$ digits, or a linear
%    scramble yielding $r > k$ digits), one should 
%    take $w = r = k$ for better computational efficiency.
    Restrictions: $0\le k\le 30$, $k\le w$, and \texttt{dim} $\le 318$.
\end{tabb}
\begin{htmlonly}
   \param{k}{there will be 2^k points}
   \param{w}{number of output digits}
   \param{dim}{dimension of the point set}
\end{htmlonly}
\begin{code}\begin{hide}

   public String toString() {
      StringBuffer sb = new StringBuffer ("Niederreiter sequence:" +
                                           PrintfFormat.NEWLINE);
      sb.append (super.toString());
      return sb.toString();
   }

   private void init (int k, int r, int w, int dim) {
      if ((dim < 1) || (dim > MAXDIM))
         throw new IllegalArgumentException 
            ("Dimension for NiedSequenceBase2 must be > 1 and <= " + MAXDIM);
      if (r < k || w < r || w > MAXBITS || k >= MAXBITS) 
         throw new IllegalArgumentException
            ("One must have k < 31 and k <= r <= w <= 31 for NiedSequenceBase2");
      numCols   = k;
      numRows   = r;   // Unused!
      outDigits = w;
      numPoints = (1 << k);
      this.dim  = dim;
      normFactor = 1.0 / ((double) (1L << (outDigits)));
      genMat = new int[dim * numCols];
      initGenMat();
   }


   public void extendSequence (int k) {
      init (k, numRows, outDigits, dim);
   }


   // Initializes the generator matrices for a sequence. 
   /* I multiply by 2 because the relevant columns are in the 30 least 
      significant bits of NiedMat, but if I understand correctly,
      SSJ assumes that they are in bits [31, ..., 1]. 
      Then I shift right if w < 31. */
   private void initGenMat ()  {
      for (int j = 0; j < dim; j++)
         for (int c = 0; c < numCols; c++) {
            genMat[j*numCols + c] = NiedMat[j*NUMCOLS + c] << 1;
            genMat[j*numCols + c] >>= MAXBITS - outDigits;
         }
   }

/*
   // Initializes the generator matrices for a net. 
   protected void initGenMatNet()  {
      int j, c;

      // the first dimension, j = 0.
      for (c = 0; c < numCols; c++)
         genMat[c] = (1 << (outDigits-numCols+c));

      for (j = 1; j < dim; j++)
         for (c = 0; c < numCols; c++)
            genMat[j*numCols + c] = 2 * NiedMat[(j - 1)*NUMCOLS + c];
   }
*/

   // ****************************************************************** 
   // Generator matrices of Niederreiter sequence. 
   // This array stores explicitly NUMCOLS columns in 318 dimensions.

   private static int[] NiedMat;
   private static final int MAXN = 9540;

   static {
      NiedMat = new int[MAXN];

      try {
         InputStream is = 
            NiedSequenceBase2.class.getClassLoader().getResourceAsStream (
            "umontreal/iro/lecuyer/hups/dataSer/Nieder/NiedSequenceBase2.ser");
         if (is == null)
            throw new FileNotFoundException (
               "Cannot find NiedSequenceBase2.ser");
         ObjectInputStream ois = new ObjectInputStream(is);
         NiedMat = (int[]) ois.readObject();
         ois.close();

      } catch(FileNotFoundException e) {
         e.printStackTrace();
         System.exit(1);

      } catch(IOException e) {
         e.printStackTrace();
         System.exit(1);

      } catch(ClassNotFoundException e) {
         e.printStackTrace();
         System.exit(1);
      }
   }

}
\end{hide}
\end{code}






\defclass {NiedXingSequenceBase2}

This class implements digital sequences based on the
 Niederreiter-Xing sequence in base 2.
\latex{For details on these point sets, see \cite{rNIE98a,rPIR01c}.}
\hpierre{The current numbers used to fill up the generating matrices
   were taken from the web page of Gottlieb Pirsic some time ago
   and contain errors. We should check for an update on his web page.}
\hrichard{Les nouveaux nombres ont \'et\'e recopi\'es le 15 juillet 2004.}
\hrichard{Nouvelle r\'ef\'erence int\'eressante? \cite{rCLA99a}}

%%%%%%%%%%%%%%%%%%%%%%%

\bigskip\hrule
\begin{code}
\begin{hide}
/*
 * Class:        NiedXingSequenceBase2
 * Description:  Niederreiter-Xing sequences in base 2
 * Environment:  Java
 * Software:     SSJ 
 * Copyright (C) 2001  Pierre L'Ecuyer and Université de Montréal
 * Organization: DIRO, Université de Montréal
 * @author       
 * @since

 * SSJ is free software: you can redistribute it and/or modify it under
 * the terms of the GNU General Public License (GPL) as published by the
 * Free Software Foundation, either version 3 of the License, or
 * any later version.

 * SSJ is distributed in the hope that it will be useful,
 * but WITHOUT ANY WARRANTY; without even the implied warranty of
 * MERCHANTABILITY or FITNESS FOR A PARTICULAR PURPOSE.  See the
 * GNU General Public License for more details.

 * A copy of the GNU General Public License is available at
   <a href="http://www.gnu.org/licenses">GPL licence site</a>.
 */
\end{hide}
package umontreal.iro.lecuyer.hups; \begin{hide} 

import java.io.Serializable;
import java.io.ObjectInputStream;
import java.io.InputStream;
import java.io.FileNotFoundException;
import java.io.IOException;
import umontreal.iro.lecuyer.util.PrintfFormat;
\end{hide}

public class NiedXingSequenceBase2 extends DigitalSequenceBase2 \begin{hide} { 

   private static final int MAXDIM  = 32;  // Maximum dimension.
   private static final int NUMCOLS = 30;  // Maximum number of columns.
   private static final boolean isTrans = true;
\end{hide} 
\end{code}
%%%%%%%%%%%%%%%%%%%%%%%%%%%%%%%%%
\subsubsection* {Constructors}
\begin{code} 

   public NiedXingSequenceBase2 (int k, int w, int dim) \begin{hide} {
      init (k, w, w, dim);
   } 
\end{hide}
\end{code} 
\begin{tabb}
    Constructs a new Niederreiter-Xing digital sequence in base 2 
    with $n = 2^k$ points and $w$ output digits, in \texttt{dim} dimensions.
    The generator matrices $\mathbf{C}_j$ are $w\times k$ and
    the numbers making the bit matrices are taken from
   \htmladdnormallink{Pirsic's site}{http://www.ricam.oeaw.ac.at/people/page/pirsic/niedxing/index.html}.
  The bit matrices from Pirsic's site are transposed to be consistent with SSJ,
  and at most 30 bits of the matrices are used. 
%    Unless, one plans to apply a randomization on more than $k$ digits
%     (e.g., a random digital shift for $w > k$ digits, or a linear
%     scramble yielding $r > k$ digits), one should 
%     take $w = r = k$ for better computational efficiency.
    Restrictions: $0\le k\le 30$, $k\le w$, and \texttt{dim} $\le 32$.
\end{tabb}
\begin{htmlonly}
   \param{k}{there will be 2^k points}
   \param{w}{number of output digits}
   \param{dim}{dimension of the point set}
\end{htmlonly}
\begin{code}\begin{hide}

   public String toString() {
      StringBuffer sb = new StringBuffer ("Niederreiter-Xing sequence:" +
                                           PrintfFormat.NEWLINE);
      sb.append (super.toString());
      return sb.toString();
   }

   private void init (int k, int r, int w, int dim) {
      if ((dim < 1) || (dim > MAXDIM))
         throw new IllegalArgumentException 
            ("Dimension for NiedXingSequenceBase2 must be > 1 and <= " + MAXDIM);
      if (r < k || w < r || w > MAXBITS || k >= MAXBITS) 
         throw new IllegalArgumentException
        ("One must have k < 31 and k <= r <= w <= 31 for NiedXingSequenceBase2");
      numCols   = k;
      numRows   = r;
      outDigits = w;
      numPoints = (1 << k);
      this.dim  = dim;
      // 1L otherwise gives wrong results for outDigits >= 31
      normFactor = 1.0 / ((double) (1L << (outDigits)));
      genMat = new int[dim * numCols];
      initGenMat();
   } 


   public void extendSequence (int k) {
      init (k, numRows, outDigits, dim);
   }


   // Initializes the generator matrices for a sequence.
   private void initGenMat ()  {
      // Compute where we should start reading matrices.
      /* Pirsic gives matrices starting at dimension 4; there are j matrices
         of dimension j. Thus if we want to use matrices in dimension
         dim, we must jump over all matrices of dimension < dim. If they
         started at dimension 1, there would be dim*(dim-1)/2 elements to
         disregard. But the first 3 dimensions are not there, thus subtract
         3*(3 - 1)/2 = 6 to get the starting element of dimension dim.
      I also multiply by 2 because the relevant columns are in the 30 least 
         significant bits of NiedXingMat, but if I understand correctly,
         SSJ assumes that they are in bits [31, ..., 1].
      At last, I shift right if w < 31 bits. */

      int start;
      if (dim <= 4) 
         start = 0;
      else
         start = ((dim*(dim-1)/2)-6)*NUMCOLS;

      long x;
      if (isTrans) {
         for (int j = 0; j < dim; j++)
            for (int c = 0; c < numCols; c++) {
               x = NiedXingMatTrans[start + j*NUMCOLS + c];
               x <<= 1;
               genMat[j*numCols + c] = (int) (x >> (MAXBITS - outDigits));
             }
      } else {
         for (int j = 0; j < dim; j++)
            for (int c = 0; c < numCols; c++) {
               x = NiedXingMat[start + j*NUMCOLS + c];
               x <<= 1;
               genMat[j*numCols + c] = (int) (x >> (MAXBITS - outDigits));
            }
      }
   }


   // ****************************************************************** 
   // Generator matrices of Niederreite-Xing sequences.
   // This array stores explicitly NUMCOLS columns in MAXDIM dimensions.
   // The implemented generator matrices are provided by Gottlieb Pirsic
   // and were downloaded on 15 July 2004. (RS)
   // These are the numbers given by Pirsic: I kept the first 30 of each
   // row vector in each dimension and shifted them to get the 30 most 
   // significant bits. These numbers considered as 30 X 30 bit matrices 
   // are given in matrix NiedXingMat below. The same numbers but transposed
   // as 30 X 30 bit matrices are given in matrix NiedXingMatTrans below.
   // According to Yves Edel, the correct matrices for Niederreiter-Xing
   // are NiedXingMatTrans.
   //
   // The matrices were given up to MAXDIM = 32, but the javac compiler
   // cannot compile code  that is too big. So I serialized them in
   // file NiedXingSequenceBase2.ser. (RS)

   private static int[] NiedXingMat;
   private static int[] NiedXingMatTrans;
   private static final int MAXN = 15660;

   static {
//      NiedXingMat = new int[MAXN];
      NiedXingMatTrans = new int[MAXN];

      try {
/*
         InputStream is = 
            NiedXingSequenceBase2.class.getClassLoader().getResourceAsStream (
         "umontreal/iro/lecuyer/hups/dataSer/Nieder/NiedXingSequenceBase2.ser");
         if (is == null)
            throw new FileNotFoundException (
               "Cannot find NiedXingSequenceBase2.ser");
         ObjectInputStream ois = new ObjectInputStream(is);
         NiedXingMat = (int[]) ois.readObject();
         ois.close();
*/
         InputStream is =
            NiedXingSequenceBase2.class.getClassLoader().getResourceAsStream(
      "umontreal/iro/lecuyer/hups/dataSer/Nieder/NiedXingSequenceBase2Trans.ser");
         if (is == null)
            throw new FileNotFoundException (
               "Cannot find NiedXingSequenceBase2Trans.ser");
         ObjectInputStream ois = new ObjectInputStream(is);
         NiedXingMatTrans = (int[]) ois.readObject();
         ois.close();

      } catch(FileNotFoundException e) {
         e.printStackTrace();
         System.exit(1);

      } catch(IOException e) {
         e.printStackTrace();
         System.exit(1);

      } catch(ClassNotFoundException e) {
         e.printStackTrace();
         System.exit(1);
      }
   }
}
\end{hide}
\end{code}


\include{F2wNetLFSR}
\defclass{F2wNetPolyLCG}


This class implements a digital net in base 2 starting from a
polynomial LCG in $\latex{\mathbb{F}}\html{\mathbf{F}}_{2^w}[z]/P(z)$.
 It is exactly the same
point set as the one defined in the class 
 \externalclass{umontreal.iro.lecuyer.hups}{F2wCycleBasedPolyLCG}.
 See the description
of this class for more details on the way the point set is constructed.

Constructing a point set using this class, instead of using
 \externalclass{umontreal.iro.lecuyer.hups}{F2wCycleBasedPolyLCG},
makes SSJ construct a digital net in base 2.  This is useful if one
wants to randomize using one of the randomizations included in the class
 \externalclass{umontreal.iro.lecuyer.hups}{DigitalNet}.

\textbf{Note: This class in not operational yet!}

\bigskip\hrule\bigskip

%%%%%%%%%%%%%%%%%%%%%%%%%%%%%%%%%%%%%%%%%%%%%%%%%%%%%%%%%%%%%%%%%%
\begin{code}
\begin{hide}
/*
 * Class:        F2wNetPolyLCG
 * Description:  digital nets in base 2 starting from a polynomial LCG 
 * Environment:  Java
 * Software:     SSJ 
 * Copyright (C) 2001  Pierre L'Ecuyer and Université de Montréal
 * Organization: DIRO, Université de Montréal
 * @author       
 * @since

 * SSJ is free software: you can redistribute it and/or modify it under
 * the terms of the GNU General Public License (GPL) as published by the
 * Free Software Foundation, either version 3 of the License, or
 * any later version.

 * SSJ is distributed in the hope that it will be useful,
 * but WITHOUT ANY WARRANTY; without even the implied warranty of
 * MERCHANTABILITY or FITNESS FOR A PARTICULAR PURPOSE.  See the
 * GNU General Public License for more details.

 * A copy of the GNU General Public License is available at
   <a href="http://www.gnu.org/licenses">GPL licence site</a>.
 */
\end{hide}
package umontreal.iro.lecuyer.hups; \begin{hide} 

import umontreal.iro.lecuyer.util.PrintfFormat;
\end{hide}

public class F2wNetPolyLCG extends DigitalNetBase2 \begin{hide} 
{
   private F2wStructure param;

    /**
     * Constructs and stores the set of cycles for an LCG with
     *    modulus <SPAN CLASS="MATH"><I>n</I></SPAN> and multiplier <SPAN CLASS="MATH"><I>a</I></SPAN>.
     *   If pgcd<SPAN CLASS="MATH">(<I>a</I>, <I>n</I>) = 1</SPAN>, this constructs a full-period LCG which has two
     *   cycles, one containing 0 and one, the LCG period.
     *
     * @param n required number of points and modulo of the LCG
     *
     *    @param a generator <SPAN CLASS="MATH"><I>a</I></SPAN> of the LCG
     *
     *
     */
\end{hide}
\end{code}

%%%%%%%%%%%%%%%%%%%%%%%%%%%%
\subsubsection*{Constructors}
\begin{code}

   public F2wNetPolyLCG (int type, int w, int r, int modQ, int step,
                         int nbcoeff, int coeff[], int nocoeff[], int dim) \begin{hide} 
   {
      param = new F2wStructure (w, r, modQ, step, nbcoeff, coeff, nocoeff);
      initNet (r, w, dim);
   }
\end{hide}
\end{code}
 \begin{tabb}
Constructs a point set with $2^{rw}$ points.  See the description of the class
\externalclass{umontreal.iro.lecuyer.hups}{F2wStructure} for the meaning of the 
 parameters.
 \end{tabb}
\begin{code}

   public F2wNetPolyLCG (String filename, int no, int dim) \begin{hide} 
   {
      param = new F2wStructure (filename, no);
      initNet (param.r, param.w, dim);
   }\end{hide}
\end{code}
 \begin{tabb}
   Constructs a point set after reading its parameters from
   file \texttt{filename}; the parameters are located at line numbered \texttt{no}
   of \texttt{filename}. The available files are listed in the description of class
\externalclass{umontreal.iro.lecuyer.hups}{F2wStructure}.
 \end{tabb}
\begin{code}
\begin{hide} 

   public String toString ()
   {
       String s = "F2wNetPolyLCG:" + PrintfFormat.NEWLINE;
       return s + param.toString ();
   }


   private void initNet (int r, int w, int dim)
   {
      normFactor = param.normFactor;
   }
}
\end{hide}
\end{code}


\include{RadicalInverse}
\defclass{HammersleyPointSet}

This class implements \emph{Hammersley point sets}, 
which are defined as follows.
Let $2 = b_1 < b_2 < \cdots$ denote the sequence of all prime 
numbers by increasing order.
The Hammersley point set with $n$ points in $s$ dimensions contains
the points
\eq
 \mathbf{u}_i = (i/n,\psi_{b_1}(i),\psi_{b_2}(i),\dots, \psi_{b_{s-1}}(i)),
                                         \eqlabel{eq:Hammersley-point2}
\endeq
for $i=0,\dots,n-1$, where $\psi_b$ is the radical inverse function
in base $b$, defined in \class{RadicalInverse}.
This class is not a subclass of \class{DigitalNet}, because the basis
is not the same for all coordinates.
We do obtain a net in a generalized sense if 
$n = b_1^{k_1} = b_2^{k_2} = \cdots = b_{s-1}^{k_{s-1}}$
for some integers $k_1,\dots,k_{s-1}$.

The points of a Hammersley point set can be ``scrambled'' by applying a 
permutation to the digits of $i$ before computing each coordinate\latex{
via (\ref{eq:Hammersley-point})}.  If 
\[
  i = a_0 + a_1 b_j + \dots + a_{k_j-1} b_j^{k_j-1},
\]
and $\pi_j$ is a permutation of the digits $\{0,\dots,b_j-1\}$, then
\[
 \psi_{b_j}(i) = \sum_{r=0}^{k_j-1} a_r b_j^{-r-1} 
\]
is replaced \latex{in (\ref{eq:Hammersley-point})} by
\[
 u_{i,j}= \sum_{r=0}^{k_j-1} \pi_j[a_r] b_j^{-r-1}.
\]
The permutations $\pi_j$ can be deterministic or random.
One (deterministic) possibility implemented here is to use
the Faure permutation of $\{0,\dots,b_j\}$ for $\pi_j$, for each 
coordinate $j > 0$.


\bigskip\hrule\bigskip
%%%%%%%%%%%%%%%%%%%%%%%%%%%%%%%%%%%%%%%%%%%%%%%%%%%%%%%%%%%%%%%%%%

\begin{code}
\begin{hide}
/*
 * Class:        HammersleyPointSet
 * Description:  
 * Environment:  Java
 * Software:     SSJ 
 * Copyright (C) 2001  Pierre L'Ecuyer and Université de Montréal
 * Organization: DIRO, Université de Montréal
 * @author       
 * @since

 * SSJ is free software: you can redistribute it and/or modify it under
 * the terms of the GNU General Public License (GPL) as published by the
 * Free Software Foundation, either version 3 of the License, or
 * any later version.

 * SSJ is distributed in the hope that it will be useful,
 * but WITHOUT ANY WARRANTY; without even the implied warranty of
 * MERCHANTABILITY or FITNESS FOR A PARTICULAR PURPOSE.  See the
 * GNU General Public License for more details.

 * A copy of the GNU General Public License is available at
   <a href="http://www.gnu.org/licenses">GPL licence site</a>.
 */
\end{hide}
package umontreal.iro.lecuyer.hups;


public class HammersleyPointSet extends PointSet\begin{hide} {
   private int[] base;           // Vector of prime bases.
   private int[][] permutation;  // Digits permutation, for each dimension.
   private boolean permuted;     // Permute digits?
\end{hide}
\end{code}

%%%%%%%%%%%%%%%%%%%%%%%%%%%%%%%%%
\subsubsection*{Constructor}

\begin{code}

   public HammersleyPointSet (int n, int dim) \begin{hide} {
      if (n < 0 || dim < 1)
         throw new IllegalArgumentException
            ("Hammersley point sets must have positive dimension and n >= 0");
      numPoints = n;
      this.dim  = dim;
      if (dim > 1)
         base = RadicalInverse.getPrimes (dim - 1);
   }\end{hide}
\end{code}
 \begin{tabb}
   Constructs a new Hammersley point set with \texttt{n} points in \texttt{dim}
   dimensions.
 \end{tabb}
\begin{htmlonly}
   \param{n}{number of points}
   \param{dim}{dimension of the point set}
\end{htmlonly}

%%%%%%%%%%%%%%%%%%%%%%%%%%%%%%%%%
\subsubsection*{Methods}

\begin{code}

   public void addFaurePermutations()\begin{hide} {
      if (dim > 1) {
         permutation = new int[dim][];
         for (int i = 0; i < dim - 1; i++) {
            permutation[i] = new int[base[i]];
            RadicalInverse.getFaurePermutation (base[i], permutation[i]);
         }
      }
      permuted = true;
   }\end{hide}
\end{code}
 \begin{tabb}
  Permutes the digits using Faure permutations for all coordinates.
  After the method is called, the coordinates $u_{i,j}$ are generated via
\[
  u_{i,j} = \sum_{r=0}^{k-1} \pi_j[a_r] b_j^{-r-1},
\]
 for $j=1,\dots,s-1$ and $u_{i,0}=i/n$,
 where $\pi_j$ is the Faure permutation of $\{0,\dots,b_j-1\}$.
 \end{tabb}
\begin{code}

   public void ErasePermutations()\begin{hide} {
      permuted = false;
      permutation = null;
   }
\end{hide}
\end{code}
 \begin{tabb}
  Erases the Faure permutations: from now on, the digits will not be
  Faure permuted.
 \end{tabb}
\begin{code}
\begin{hide}

   public double getCoordinate (int i, int j) {
      if (j == 0)
         return (double) i / (double) numPoints;
      if (permuted)
         return RadicalInverse.permutedRadicalInverse 
                   (base[j - 1], permutation[j - 1], i);
      else 
         return RadicalInverse.radicalInverse (base[j - 1], i);
   }
}
\end{hide}
\end{code}

\defclass{HaltonSequence}

This class implements the sequence of Halton \cite{rHAL60a},
which is essentially a modification of Hammersley nets for producing 
an infinite sequence of points having low discrepancy.
The $i$th point in $s$ dimensions is 
\eq
 \bu_i = (\psi_{b_1}(i),\psi_{b_2}(i),\dots, \psi_{b_s}(i)),
                                            \eqlabel{eq:Halton-point2}
\endeq
for $i=0,1,2,\dots$, where $\psi_b$ is the radical inverse function
in base $b$, defined in class \class{RadicalInverse}, and where
$2 = b_1 < \cdots < b_s$ are the $s$ smallest prime numbers in 
increasing order.

A fast method is implemented to generate randomized Halton sequences\latex{
\cite{rSTR95a,vWAN99a}}, starting from an arbitrary point $x_0$.

The points can be ``scrambled'' by applying a permutation to the 
digits of $i$ before computing each coordinate\latex{
via (\ref{eq:Halton-point})}, in the same way as for the class
\class{HammersleyPointSet}, for all coordinates $j\ge 0$.

\bigskip\hrule\bigskip
%%%%%%%%%%%%%%%%%%%%%%%%%%%%%%%%%%%%%%%%%%%%%%%%%%%%%%%%%%%%%%%%%%

\begin{code}

\begin{hide}
/*
 * Class:        HaltonSequence
 * Description:  
 * Environment:  Java
 * Software:     SSJ 
 * Copyright (C) 2001  Pierre L'Ecuyer and Université de Montréal
 * Organization: DIRO, Université de Montréal
 * @author       
 * @since

 * SSJ is free software: you can redistribute it and/or modify it under
 * the terms of the GNU General Public License (GPL) as published by the
 * Free Software Foundation, either version 3 of the License, or
 * any later version.

 * SSJ is distributed in the hope that it will be useful,
 * but WITHOUT ANY WARRANTY; without even the implied warranty of
 * MERCHANTABILITY or FITNESS FOR A PARTICULAR PURPOSE.  See the
 * GNU General Public License for more details.

 * A copy of the GNU General Public License is available at
   <a href="http://www.gnu.org/licenses">GPL licence site</a>.
 */
\end{hide}
package umontreal.iro.lecuyer.hups;


public class HaltonSequence extends PointSet\begin{hide} { 
   private int[] base;           // Vector of prime bases.
   private int[][] permutation;  // Digits permutation, for each dimension.
   private boolean permuted;     // Permute digits?
   private RadicalInverse[] radinv; // Vector of RadicalInverse's.
   private int[] start;          // starting indices
   private final static int positiveBitMask = ~Integer.reverse(1);
\end{hide}
\end{code}

%%%%%%%%%%%%%%%%%%%%%%%%%%%%%%%%%
\subsubsection*{Constructor}

\begin{code}
   public HaltonSequence (int dim) \begin{hide} {
      if (dim < 1)
         throw new IllegalArgumentException
            ("Halton sequence must have positive dimension dim");
      this.dim  = dim;
      numPoints = Integer.MAX_VALUE;
      base = RadicalInverse.getPrimes (dim);
      start = new int[dim];
      java.util.Arrays.fill(start, 0);
   }\end{hide}
\end{code}
 \begin{tabb}
   Constructs a new Halton sequence %% of $n$ points
    in \texttt{dim} dimensions.
%%   The number of points is infinite.
 \end{tabb}
\begin{htmlonly}
   \param{dim}{dimension}
\end{htmlonly}


%%%%%%%%%%%%%%%%%%%%%%%%%%%%%%%%%
\subsubsection*{Methods}

\begin{code}

   public void setStart (double[] x0) \begin{hide} {
      for (int i = 0; i < dim; i++)
         start[i] = RadicalInverse.radicalInverseInteger(base[i], x0[i]);
   }\end{hide}
\end{code}
 \begin{tabb}
   Initializes the Halton sequence starting at point \texttt{x0}.
   For each coordinate $j$, the sequence starts at index $i_j$ such that 
   \texttt{x0[$j$]} is the radical inverse of $i_j$.
   The dimension of \texttt{x0} must be at least as large as the dimension
   of this object.
 \end{tabb}
\begin{htmlonly}
   \param{x0}{starting point of the Halton sequence}
\end{htmlonly}
\begin{code}

   public void init (double[] x0) \begin{hide} {
      radinv = new RadicalInverse[dim];
      for (int i = 0; i < dim; i++)
         radinv[i] = new RadicalInverse (base[i], x0[i]);
   }\end{hide}
\end{code}
 \begin{tabb}
   Initializes the Halton sequence starting at point \texttt{x0}.
   The dimension of \texttt{x0} must be at least as large as the dimension
   of this object.
%%   The number of points is infinite.
 \end{tabb}
\begin{htmlonly}
   \param{x0}{starting point of the Halton sequence}
\end{htmlonly}
\begin{code}

   public void addFaureLemieuxPermutations()\begin{hide} {
      permutation = new int[dim][];
      for (int i = 0; i < dim; i++) {
         permutation[i] = new int[base[i]];
         RadicalInverse.getFaureLemieuxPermutation (i, permutation[i]);
      }
      permuted = true;
   }
\end{hide}
\end{code}
 \begin{tabb}
Permutes the digits using permutations from \cite{vFAU09a} for all coordinates.
After the method is called, the coordinates $u_{i,j}$ are generated via
\[
  u_{i,j} = \sum_{r=0}^{k-1} \pi_j[a_r] b_j^{-r-1},
\]
 for $j=0,\dots,s-1$,
 where $\pi_j$ is the Faure-Lemieux (2008) permutation of $\{0,\dots,b_j-1\}$.
 \end{tabb}
\begin{code}

   public void addFaurePermutations()\begin{hide} {
      permutation = new int[dim][];
      for (int i = 0; i < dim; i++) {
         permutation[i] = new int[base[i]];
         RadicalInverse.getFaurePermutation (base[i], permutation[i]);
      }
      permuted = true;
   }
\end{hide}
\end{code}
 \begin{tabb}
  Permutes the digits using Faure permutations for all coordinates.
  After the method is called, the coordinates $u_{i,j}$ are generated via
\[
  u_{i,j} = \sum_{r=0}^{k-1} \pi_j[a_r] b_j^{-r-1},
\]
 for $j=0,\dots,s-1$,
 where $\pi_j$ is the Faure permutation of $\{0,\dots,b_j-1\}$.
 \end{tabb}
\begin{code}

   public void ErasePermutations()\begin{hide} {
      permuted = false;
      permutation = null;
   }
\end{hide}
\end{code}
 \begin{tabb}
  Erases the permutations: from now on, the digits will not be
  permuted.
 \end{tabb}
\begin{code}
\begin{hide}
    
   public int getNumPoints () {
      return Integer.MAX_VALUE;
   }

   public double getCoordinate (int i, int j) {
      if (radinv != null) {
         if (!permuted) {
            return radinv[j].nextRadicalInverse ();
         } else {
            throw new UnsupportedOperationException (
            "Fast radical inverse is not implemented in case of permutation");
         }
      } else {
         int k = start[j] + i;
         // if overflow, restart at first nonzero point
         // (Struckmeier restarts at zero)
         if (k < 0)
            k = (k & positiveBitMask) + 1;
         if (permuted)
            return RadicalInverse.permutedRadicalInverse 
            (base[j], permutation[j], k);
         else 
            return RadicalInverse.radicalInverse (base[j], k);
      }
   }
}
\end{hide}
\end{code}


\defclass{Rank1Lattice}

This class implements point sets defined by integration
lattices of rank 1, defined as follows \cite{vSLO94a}.
One selects an arbitrary positive integer $n$ and a $s$-dimensional
integer vector $(a_0,\dots,a_{s-1})$, where $0 \le a_j < n$ for each $j$.
Usually, $a_0=1$.  The points are defined by
\eq
  \mathbf{u}_i = (i/n)(a_0, a_1, \ldots, a_{s-1}) \bmod 1
\endeq
for $i=0,\dots,n-1$.
These $n$ points are distinct provided that $n$ and the $a_j$'s have
no common factor.


\bigskip\hrule\bigskip

%%%%%%%%%%%%%%%%%%%%%%%%%%%%%%%%%%%%%%%%%%%%%%%%%%%%%%%%%%%%%%%%%%
\begin{code}
\begin{hide}
/*
 * Class:        Rank1Lattice
 * Description:  Rank-1 lattice
 * Environment:  Java
 * Software:     SSJ 
 * Copyright (C) 2001  Pierre L'Ecuyer and Université de Montréal
 * Organization: DIRO, Université de Montréal
 * @author       
 * @since

 * SSJ is free software: you can redistribute it and/or modify it under
 * the terms of the GNU General Public License (GPL) as published by the
 * Free Software Foundation, either version 3 of the License, or
 * any later version.

 * SSJ is distributed in the hope that it will be useful,
 * but WITHOUT ANY WARRANTY; without even the implied warranty of
 * MERCHANTABILITY or FITNESS FOR A PARTICULAR PURPOSE.  See the
 * GNU General Public License for more details.

 * A copy of the GNU General Public License is available at
   <a href="http://www.gnu.org/licenses">GPL licence site</a>.
 */
\end{hide}
package umontreal.iro.lecuyer.hups;\begin{hide}
import umontreal.iro.lecuyer.util.PrintfFormat;
import umontreal.iro.lecuyer.rng.RandomStream;
\end{hide}

public class Rank1Lattice extends PointSet \begin{hide} {

   protected int[] genAs;          // Lattice generator:  a[i]
   protected double[] v;           // Lattice vector:  v[i] = a[i]/n
   protected double normFactor;    // 1/n.
   protected double[] shift;       // Random shift, initially null.

\end{hide}
\end{code}

%%%%%%%%%%%%%%%%%%%%%%%%%%%%%%%%%
\subsubsection*{Constructor}
\begin{code}

   public Rank1Lattice (int n, int[] a, int s) \begin{hide} {
      dim = s;
      numPoints = n;
      normFactor = 1.0 / (double) n;
      v = new double[s];
      genAs = new int[s];
      for (int j = 0; j < s; j++) {
         if (a[j] < 0 || a[j] >= n)
            throw new IllegalArgumentException
               ("Rank1Lattice must have 0 <= a[j] < n");
         v[j] = normFactor * a[j];
         genAs[j] = a[j];
      }
   }\end{hide}
\end{code}
 \begin{tabb}
   Instantiates a \class{Rank1Lattice}{} with $n$ points and lattice
   vector $a$ of dimension $s$.
 \end{tabb}
\begin{htmlonly}
   \param{n}{there are n points}
   \param{a}{the lattice vector}
   \param{s}{dimension of the lattice vector a}
\end{htmlonly}

%%%%%%%%%%%%%%%%%%%%%%%%%%%%%%%%%%%%
\subsubsection*{Methods}
\begin{code}

   public int[] getAs() \begin{hide} {
      return genAs;
   }\end{hide}
\end{code}
\begin{tabb}
Returns the generator $a_j$ of the lattice. Its components
  are returned as \texttt{a[$j$]}, for $j = 0, 1, \ldots, (s-1)$.
\end{tabb}
\begin{code}

   public void addRandomShift (int d1, int d2, RandomStream stream) \begin{hide} {
      if (null == stream)
         throw new IllegalArgumentException (
              PrintfFormat.NEWLINE +
                  "   Calling addRandomShift with null stream");
      if (0 == d2)
         d2 = Math.max (1, dim);
      if (shift == null) {
         shift = new double[d2];
         capacityShift = d2;
      } else if (d2 > capacityShift) {
         int d3 = Math.max (4, capacityShift);
         while (d2 > d3)
            d3 *= 2;
         double[] temp = new double[d3];
         capacityShift = d3;
         for (int i = 0; i < d1; i++)
            temp[i] = shift[i];
         shift = temp;
      }
      dimShift = d2;
      for (int i = d1; i < d2; i++)
         shift[i] = stream.nextDouble ();
      shiftStream = stream;
   }\end{hide}
\end{code}
\begin{tabb}  Adds a random shift to all the points of the point set,
  using stream \texttt{stream} to generate the random numbers.
  For each coordinate $j$ from \texttt{d1} to \texttt{d2-1},
  the shift $d_{j}$ is generated uniformly over $[0, 1)$ and added modulo $1$ to
  all the coordinates of all the points.
%  After adding a digital shift, all iterators must be reconstructed or
%  reset to zero.
\end{tabb}
\begin{htmlonly}
   \param{d1}{lower dimension of shift}
   \param{d2}{upper dimension of shift is d2 - 1}
   \param{stream}{random number stream used to generate uniforms}
\end{htmlonly}
\begin{code}

   public void clearRandomShift() \begin{hide} {
      super.clearRandomShift();
      shift = null;
   }\end{hide}
\end{code}
\begin{tabb}  Clears the random shift.
\end{tabb}
\begin{code}
 \begin{hide}

   public String toString() {
      StringBuffer sb = new StringBuffer ("Rank1Lattice:" +
                                           PrintfFormat.NEWLINE);
      sb.append (super.toString());
      return sb.toString();
   }


   public double getCoordinate (int i, int j) {
      double x = (v[j] * i) % 1.0;
      if (shift != null) {
         if (j >= dimShift)   // Extend the shift.
            addRandomShift (dimShift, j + 1, shiftStream);
         x += shift[j];
         if (x >= 1.0)
            x -= 1.0;
         if (x <= 0.0)
            x = EpsilonHalf;  // avoid x = 0
       }
      return x;
   }


   // Recursive method that computes a^e mod m.
   protected long modPower (long a, int e, int m) {
      // If parameters a and m == numPoints could be omitted, then
      // the routine would run much faster due to reduced stack usage.
      // Note that a can be larger than m, e.g. in lattice sequences !

      if (e == 0)
         return 1;
      else if (e == 1)
         return a % m;
      else if ((e & 1) == 1)
         return (a * modPower(a, e - 1, m)) % m;
      else {
         long p = modPower(a, e / 2, m);
         return (p * p) % m;
      }
   }

   protected double radicalInverse (int base, int i) {
      double digit, radical, inverse;
      digit = radical = 1.0 / (double) base;
      for (inverse = 0.0; i > 0; i /= base) {
         inverse += digit * (double) (i % base);
         digit *= radical;
      }
      return inverse;
   }

   public PointSetIterator iterator() {
      return new Rank1LatticeIterator();
   }

// ************************************************************************

   protected class Rank1LatticeIterator extends PointSet.DefaultPointSetIterator
   {
      public double nextCoordinate() {
         // I tried with long's and with double's. The double version is
         // 4.5 times faster than the long version.
         if (curPointIndex >= numPoints || curCoordIndex >= dim)
            outOfBounds();
//      return (curPointIndex * v[curCoordIndex++]) % 1.0;
         double x = (curPointIndex * v[curCoordIndex]) % 1.0;
         if (shift != null) {
             if (curCoordIndex >= dimShift)   // Extend the shift.
                addRandomShift (dimShift, curCoordIndex + 1, shiftStream);
             x += shift[curCoordIndex];
             if (x >= 1.0)
                x -= 1.0;
             if (x <= 0.0)
                x = EpsilonHalf;  // avoid x = 0
         }
         curCoordIndex++;
         return x;
      }
   }
}\end{hide}
\end{code}

\defclass{KorobovLattice}

This class implements \emph{Korobov lattices}, which represents the same point
sets as in class \class{LCGPointSet}, but implemented differently.
The parameters are the modulus $n$ and the multiplier $a$, for arbitrary
integers $1 \le a < n$.
The number of points is $n$, their dimension is 
$s$, and they are defined by
\[
  \mathbf{u}_i = (i/n)(1, a, a^2, \ldots, a^{s-1}) \bmod 1
\]
for $i=0,\dots,n-1$.

It is also possible to build a ``shifted'' Korobov lattice with the first
$t$ coordinates rejected. The $s$-dimensionnal points are then defined as
$$
  \mathbf{u}_i = (i/n)(a^{t}, a^{t+1}, a^{t+2}, \ldots, a^{t+s-1}) \bmod 1
$$
for $i=0,\dots,n-1$ and fixed $t$.

\bigskip\hrule\bigskip

%%%%%%%%%%%%%%%%%%%%%%%%%%%%%%%%%%%%%%%%%%%%%%%%%%%%%%%%%%%%%%%%%%
\begin{code}
\begin{hide}
/*
 * Class:        KorobovLattice
 * Description:  Korobov lattice
 * Environment:  Java
 * Software:     SSJ 
 * Copyright (C) 2001  Pierre L'Ecuyer and Université de Montréal
 * Organization: DIRO, Université de Montréal
 * @author       
 * @since

 * SSJ is free software: you can redistribute it and/or modify it under
 * the terms of the GNU General Public License (GPL) as published by the
 * Free Software Foundation, either version 3 of the License, or
 * any later version.

 * SSJ is distributed in the hope that it will be useful,
 * but WITHOUT ANY WARRANTY; without even the implied warranty of
 * MERCHANTABILITY or FITNESS FOR A PARTICULAR PURPOSE.  See the
 * GNU General Public License for more details.

 * A copy of the GNU General Public License is available at
   <a href="http://www.gnu.org/licenses">GPL licence site</a>.
 */
\end{hide}
package umontreal.iro.lecuyer.hups;
\begin{hide}
import umontreal.iro.lecuyer.util.PrintfFormat;
\end{hide}

public class KorobovLattice extends Rank1Lattice \begin{hide} {
   protected int genA;                        // Multiplier a.

   // Method modPower is inherited from Rank1Lattice.
\end{hide}
\end{code}

%%%%%%%%%%%%%%%%%%%%%%%%%%%%%%%%%
\subsubsection*{Constructors}
\begin{code}
   public KorobovLattice (int n, int a, int s) \begin{hide} {
      super (n, null, 0);
      if (a < 1 || a >= n) 
         throw new IllegalArgumentException 
            ("KorobovLattice must have 1 <= a < n");
      genA = a;
      dim = s;
      v = new double[s];
      long[] B = new long[dim];
      B[0] = 1;
      v[0] = normFactor;
      for (int j = 1; j < dim; j++) {
         B[j] = (a * B[j - 1]) % n;
         v[j] = normFactor * B[j];
      }
   }\end{hide}
\end{code}
 \begin{tabb}
   Instantiates a Korobov lattice point set with modulus $n$ and 
   multiplier $a$ in dimension $s$.
 \end{tabb}
\begin{code}

   public KorobovLattice (int n, int a, int s, int t) \begin{hide} {
      super (n, null, 0);
      if (a < 1 || a >= n) 
         throw new IllegalArgumentException 
            ("KorobovLattice must have 1 <= a < n");
      if (t < 1) 
         throw new IllegalArgumentException 
            ("KorobovLattice: must have 0 < t");
      dim = s;
      genA = a;
      v = new double[s];
      long[] B = new long[dim];
      int j;
      B[0] = a;
      for (j = 1; j < t; j++)
         B[0] *= a;
      v[0] = B[0] * normFactor;
      for (j = 1; j < dim; j++) {
         B[j] = (a * B[j - 1]) % n;
         v[j] = normFactor * B[j];
      }
   }\end{hide}
\end{code}
 \begin{tabb}
   Instantiates a shifted Korobov lattice point set with modulus $n$ and 
   multiplier $a$ in dimension $s$. The first $t$ coordinates of a
   standard Korobov lattice are dropped as described above.
   The case $t=0$ corresponds to the standard  Korobov lattice.
 \end{tabb}

%%%%%%%%%%%%%%%%%%%%%%%%%%%%%%%%%%%%
\subsubsection*{Methods}
\begin{code}

   public int getA() \begin{hide} {
      return genA;
   }\end{hide}
\end{code}
\begin{tabb}
Returns the multiplier $a$ of the lattice.
\end{tabb}
\begin{code}
\begin{hide}

   public String toString() {
      StringBuffer sb = new StringBuffer ("KorobovLattice:" +
                                           PrintfFormat.NEWLINE);
      sb.append ("Multiplier a: " + genA + PrintfFormat.NEWLINE);
      sb.append (super.toString());
      return sb.toString();
   }
}\end{hide}
\end{code}

\defclass{KorobovLatticeSequence}

\pierre{This class is not yet fully implemented}

This class implements Korobov lattice sequences, defined as follows.
One selects a \emph{basis} $b$ and a (large) multiplier $a$.
For each integer $k\ge 0$, we may consider the
$n$-point Korobov lattice with modulus $n = b^k$ and multiplier
$\mbox{\~a} = a \bmod n$.
Its points have the form
\eq
 \mathbf{u}_i = (a^i (1, a, a^2, \ldots) \bmod n) / n
       = (\mbox{\~a}^i (1, \mbox{\~a}, \mbox{\~a}^2, \ldots) \bmod n) / n
                                        \eqlabel{eq:Korobov-seq1}
\endeq
for $i=0,\dots,n-1$.
For $k = 0,1,\dots$, we have an increasing sequence of lattices
contained in one another.

These embedded lattices contain an infinite sequence of points that 
can be enumerated as follows \cite{vHIC01a}:
\eq
 \mathbf{u}_i = \psi_b(i) \left(1, a, a^2, \ldots \right) \bmod 1.
                                        \eqlabel{eq:Korobov-seq2}
\endeq
where $\psi_b(i)$ is the radical inverse function in base $b$,
defined in \class{RadicalInverse}.
The first $n=b^k$ points in this sequence are exactly the same as
the $n$ points in (\ref{eq:Korobov-seq1}), for each $k\ge 0$.

\bigskip\hrule\bigskip
%%%%%%%%%%%%%%%%%%%%%%%%%%%%%%%%%%%%%%%%%%%%%%%%%%%%%%%%%%%%%%%%%%

\begin{code}

\begin{hide}
/*
 * Class:        KorobovLatticeSequence
 * Description:  
 * Environment:  Java
 * Software:     SSJ 
 * Copyright (C) 2001  Pierre L'Ecuyer and Université de Montréal
 * Organization: DIRO, Université de Montréal
 * @author       
 * @since

 * SSJ is free software: you can redistribute it and/or modify it under
 * the terms of the GNU General Public License (GPL) as published by the
 * Free Software Foundation, either version 3 of the License, or
 * any later version.

 * SSJ is distributed in the hope that it will be useful,
 * but WITHOUT ANY WARRANTY; without even the implied warranty of
 * MERCHANTABILITY or FITNESS FOR A PARTICULAR PURPOSE.  See the
 * GNU General Public License for more details.

 * A copy of the GNU General Public License is available at
   <a href="http://www.gnu.org/licenses">GPL licence site</a>.
 */
\end{hide}
package umontreal.iro.lecuyer.hups;


public class KorobovLatticeSequence extends KorobovLattice \begin{hide} { 
   int base;         // Base for radical inversion
   int inverse;      // global variables for radical inverssion,
   int n;            // since bloody JAVA cannot pass references

   // Method modPower is inherited from Rank1Lattice.
\end{hide}
\end{code}

%%%%%%%%%%%%%%%%%%%%%%%%%%%%%%%%%
\subsubsection*{Constructor}

\begin{code}

   public KorobovLatticeSequence (int b, int a) \begin{hide} {
// Pas termine: ne fonctionne pas
      super (2, 3, 1);
      if (a < 1)
         throw new IllegalArgumentException
             ("KorobovLatticeSequence:   Multiplier a must be >= 1");
//      dim       = Integer.MAX_VALUE;
//      numPoints = Integer.MAX_VALUE;
      base = b;
throw new UnsupportedOperationException ("NOT FINISHED");
   } \end{hide}
\end{code}
 \begin{tabb}
  Constructs a new lattice sequence with base \texttt{b} and 
 \texttt{generator} $ = a$.
 \end{tabb}
\begin{htmlonly}
   \param{b}{number of points (modulus) is a power of b}
   \param{a}{multiplier $a$ of this lattice sequence}
\end{htmlonly}
\begin{code}\begin{hide} 

   // A very inefficient way of generating the points!
   public double getCoordinate (int i, int j) {
      int n;
      int inverse;
      if (i == 0)
         return 0.0;
      else if (j == 0)
         return radicalInverse (base, i);
      else {
         // integerRadicalInverse (i);
         n = 1;
         for (inverse = 0; i > 0; i /= base) {
            inverse = inverse * base + (i % base);
            n *= base;
         }
         return (double) ((inverse * modPower (genA, j, n)) % n) / (double) n;
      }
   }

   // ... has been unrolled in getCoordinate.
   private void integerRadicalInverse (int i) {
      // Attention: returns results in variables n and inverse.
      n = 1;
      for (inverse = 0; i > 0; i /= base) {
         inverse = inverse * base + (i % base);
         n *= base;
      }
   }
 
}\end{hide}
\end{code}


\bibliographystyle{plain}
\bibliography{stat,random,vrt,simul,math,ift,fin}

\end{document}
%%%%%%%%%%%%%%%%%%%%%%%%%%%%%
