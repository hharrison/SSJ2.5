\defclass {DigitalNetBase2FromFile}

This class allows us to read the parameters defining a digital net 
 {\em in base 2\/} either from a file, or from a URL address on the
 World Wide Web. See the documentation in 
\externalclass{umontreal.iro.lecuyer.hups}{DigitalNetFromFile}.
The parameters used in building the net are those defined in class
\externalclass{umontreal.iro.lecuyer.hups}{DigitalNetBase2}.
 The format of the data files must be the following (where $B$ is any $C_j$):
\begin{htmlonly} (see the format in \texttt{guidehups.pdf})\end{htmlonly}

\begin{figure}
\begin{center}
\tt
\fbox {
\begin {tabular}{ll}
 \multicolumn{2}{l}{// Any number of comment lines starting with //} \\
     $2$      & //    Base  \\
     $k$      & //    Number of columns   \\
     $r$      & //    Number of rows  \\
     $n$      & //    Number of points = $2^k$  \\
     $s$      & //    Dimension of points \\
\\
 \multicolumn{2}{l}{// dim = 1} \\
  $a_{1}$ &  //  $= 2^{30}B_{11} + 2^{29}B_{21} + \cdots + 2^{31 - r}B_{r1}$ \\
  $a_{2}$ &  //  $= 2^{30}B_{12} + 2^{29}B_{22} + \cdots + 2^{31 - r}B_{r2}$ \\
  $\vdots$ & \\
  $a_{k}$ & \\
\\
 \multicolumn{2}{l}{// dim = 2} \\
   $\vdots$ & \\
\\
 \multicolumn{2}{l}{// dim = $s$} \\
  $a_{1}$ &  \\
  $a_{2}$ &  \\
  $\vdots$ &\\
  $a_{k}$ & \\
\end {tabular}
}
\end{center}
\end{figure}

 For each dimension $j$, there must be a $k$-vector of 32-bit integers
 (the $a_i$) corresponding to the columns of $\mathbf{C}_j$. The 
  correspondance is such that integer
  $a_i = 2^{30}(\mathbf{C}_j)_{1i} + 2^{29}(\mathbf{C}_j)_{2i} +
   \cdots +    2^{31 - r}(\mathbf{C}_j)_{ri}$.


\bigskip\hrule\bigskip
%%%%%%%%%%%%%%%%%%%%%%%%%%%%%%%%%%%%%%%%%%%%%%%%%%%%%%%%%%%%%%%%%%%%%%%%%%%

\begin{code}
\begin{hide}
/*
 * Class:        DigitalNetBase2FromFile
 * Description:  read the parameters defining a digital net in base 2
                 from a file or from a URL address
 * Environment:  Java
 * Software:     SSJ 
 * Copyright (C) 2001  Pierre L'Ecuyer and Université de Montréal
 * Organization: DIRO, Université de Montréal
 * @author       
 * @since

 * SSJ is free software: you can redistribute it and/or modify it under
 * the terms of the GNU General Public License (GPL) as published by the
 * Free Software Foundation, either version 3 of the License, or
 * any later version.

 * SSJ is distributed in the hope that it will be useful,
 * but WITHOUT ANY WARRANTY; without even the implied warranty of
 * MERCHANTABILITY or FITNESS FOR A PARTICULAR PURPOSE.  See the
 * GNU General Public License for more details.

 * A copy of the GNU General Public License is available at
   <a href="http://www.gnu.org/licenses">GPL licence site</a>.
 */
\end{hide}
package umontreal.iro.lecuyer.hups;\begin{hide}

import java.io.*;
import java.net.URL;
import java.net.MalformedURLException;
import umontreal.iro.lecuyer.util.PrintfFormat;
\end{hide}

public class DigitalNetBase2FromFile extends DigitalNetBase2 \begin{hide} {
   private String filename;

   // s is the effective dimension if > 0, otherwise it is dim
   private void readData (Reader re, int r1, int s1)
       throws IOException, NumberFormatException
   {
      try {
         StreamTokenizer st = new StreamTokenizer (re);
         if (st == null) return;
         st.eolIsSignificant (false);
         st.slashSlashComments (true);

         int i = st.nextToken ();
         if (i != StreamTokenizer.TT_NUMBER)
            throw new NumberFormatException();
         b = (int) st.nval;
         st.nextToken ();   numCols = (int) st.nval;
         st.nextToken ();   numRows = (int) st.nval;
         st.nextToken ();   numPoints = (int) st.nval;
         st.nextToken ();   dim = (int) st.nval;
         if (dim < 1) {
            System.err.println (PrintfFormat.NEWLINE + 
                "DigitalNetBase2FromFile:   dimension dim <= 0");
            throw new IllegalArgumentException ("dimension dim <= 0");
         }
         if (r1 > numRows) 
            throw new IllegalArgumentException (
            "DigitalNetBase2FromFile:   One must have   r1 <= Max num rows");
         if (s1 > dim) {
            throw new IllegalArgumentException ("s1 is too large");
         }
         if (s1 > 0)
            dim = s1;
         if (r1 > 0)
            numRows = r1;

         if (b != 2) {
            System.err.println (
              "***** DigitalNetBase2FromFile:    only base 2 allowed");
            throw new IllegalArgumentException ("only base 2 allowed");
         }
         genMat = new int[dim * numCols];
         for (i = 0; i < dim; i++)
            for (int c = 0; c < numCols; c++) {
                st.nextToken ();
                genMat[i*numCols + c] = (int) st.nval;
            }

      } catch (NumberFormatException e) {
         System.err.println (
            "   DigitalNetBase2FromFile:   not a number  " + e);
         throw e;
      }
   }


    private void maskRows (int r, int w) {
       // Keep only the r most significant bits and set the others to 0.
       int mask = (int) ((1L << r) - 1);
       mask <<= MAXBITS - r;
       for (int i = 0; i < dim; i++)
          for (int c = 0; c < numCols; c++) {
              genMat[i*numCols + c] &= mask;
              genMat[i*numCols + c] >>= MAXBITS - w;
          }
    }
\end{hide}
\end{code}
%%%%%%%%%%%%%%%%%%%%%%%%%%%%%%%%%
\subsubsection* {Constructor}
\begin{code} 

   public DigitalNetBase2FromFile (String filename, int r1, int w, int s1) 
         throws IOException, MalformedURLException\begin{hide} 
   {
      super ();
      if (w < r1 || w > MAXBITS) 
         throw new IllegalArgumentException (" Must have numRows <= w <= 31");

      BufferedReader input;
      if (filename.startsWith("http:") || filename.startsWith("ftp:"))
         input = DigitalNetFromFile.openURL(filename);
      else
         input = DigitalNetFromFile.openFile(filename);

      try {
         readData (input, r1, s1);
      } catch (NumberFormatException e) {
         System.err.println (
            "   DigitalNetBase2FromFile:   cannot read from   " + filename);
         throw e;

      }  catch (IOException e) {
         System.err.println (
            "   DigitalNetBase2FromFile:  cannot read from  " + filename);
         throw e;
      }
      input.close();
      maskRows (numRows, w);
      outDigits = w;
      if (numCols >= MAXBITS) 
         throw new IllegalArgumentException (" Must have numCols < 31");

      this.filename = filename;
      int x = (1 << numCols);
      if (x != numPoints) {
         System.out.println ("numPoints != 2^k");
         throw new IllegalArgumentException ("numPoints != 2^k");
      }
      // Compute the normalization factors.  
      normFactor = 1.0 / ((double) (1L << (outDigits)));

  }\end{hide}
\end{code} 
\begin{tabb}
    Constructs a digital net in base 2 after reading its parameters from file
    {\texttt{filename}}. See the documentation in
  \externalclass{umontreal.iro.lecuyer.hups}{DigitalNetFromFile}.
   Parameter \texttt{w} gives the number of bits of resolution, \texttt{r1} is
   the number of rows, and \texttt{s1} is the dimension.
   Restrictions: \texttt{s1} must be less than the maximal dimension, and
   \texttt{r1} less than the maximal number of rows in the data file.
   Also \texttt{w} $\ge$ \texttt{r1}.
\end{tabb}
\begin{htmlonly}
   \param{filename}{Name of the file to be read}
   \param{r1}{Number of rows for the generating matrices}
   \param{w}{Number of bits of resolution}
   \param{s1}{Number of dimensions}
\end{htmlonly}
\begin{code} 

   public DigitalNetBase2FromFile (String filename, int s1) 
        throws IOException, MalformedURLException\begin{hide}
   {
       this (filename, -1, 31, s1);
   }\end{hide}
\end{code} 
\begin{tabb}
   Same as \method{DigitalNetBase2FromFile}{}\texttt{(filename, r, 31, s1)} where
   \texttt{s1} is the dimension and \texttt{r} is given in data file \texttt{filename}.
\end{tabb}
\begin{htmlonly}
   \param{filename}{Name of the file to be read}
   \param{s1}{Number of dimensions}
\end{htmlonly}


%%%%%%%%%%%%%%%%%%%%%%%%%%%%%%%%%
\subsubsection*{Methods}

\begin{code}\begin{hide} 

   public String toString() {
      StringBuffer sb = new StringBuffer ("File:  " + filename  +
         PrintfFormat.NEWLINE);
      sb.append (super.toString());
      return sb.toString();
   }\end{hide}

   public String toStringDetailed() \begin{hide} {
      StringBuffer sb = new StringBuffer (toString() + PrintfFormat.NEWLINE);
      sb.append ("dim = " + dim + PrintfFormat.NEWLINE);
      for (int i = 0; i < dim; i++) {
         sb.append (PrintfFormat.NEWLINE + "// dim = " + (1 + i) +
              PrintfFormat.NEWLINE);
         for (int c = 0; c < numCols; c++)
            sb.append  (genMat[i*numCols + c]  + PrintfFormat.NEWLINE);
      }
      sb.append ("--------------------------------" + PrintfFormat.NEWLINE);
      return sb.toString ();
   }\end{hide}
\end{code} 
\begin{tabb}
    Writes the parameters and the generating matrices of this digital net
    to a string.
    This is useful to check that the file parameters have been read correctly.
\end{tabb}
\begin{code}

   public static String listDir (String dirname) throws IOException \begin{hide} {
      return DigitalNetFromFile.listDir(dirname);
   }\end{hide}
\end{code} 
\begin{tabb}
  Lists all files (or directories) in directory \texttt{dirname}. Only relative
  pathnames should be used. The files are  parameter files used in defining
  digital nets.  For example, calling \texttt{listDir("")} will give the list
  of the main data directory in SSJ, while calling \texttt{listDir("Edel/OOA2")}
  will give the list of all files in directory \texttt{Edel/OOA2}. 
 \end{tabb}
\begin{code}
\begin{hide} 
}
\end{hide}
\end{code}
