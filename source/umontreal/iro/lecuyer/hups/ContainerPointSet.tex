\defclass{ContainerPointSet}

This acts as a generic base class for all \emph{container
classes} that contain a point set and apply some kind of
transformation to the coordinates to define a new point set.
One example of such transformation is the \emph{antithetic} map,
applied by the container class \class{AntitheticPointSet},
where each output coordinate $u_{i,j}$ is transformed into $1-u_{i,j}$.
Another example is \class{RandShiftedPointSet}.

The class implements a specialized type of iterator for container
point sets.  This type of iterator contains itself an iterator for
the contained point set and uses it to access the points and coordinates
internally, instead of maintaining itself indices for the current point
and current coordinate.


\bigskip\hrule\bigskip
%%%%%%%%%%%%%%%%%%%%%%%%%%%%%%%%%%%%%%%%%%%%%%%%%%%%%%%%%%%%%%%%%%

\begin{code}
\begin{hide}
/*
 * Class:        ContainerPointSet
 * Description:  
 * Environment:  Java
 * Software:     SSJ 
 * Copyright (C) 2001  Pierre L'Ecuyer and Université de Montréal
 * Organization: DIRO, Université de Montréal
 * @author       
 * @since

 * SSJ is free software: you can redistribute it and/or modify it under
 * the terms of the GNU General Public License (GPL) as published by the
 * Free Software Foundation, either version 3 of the License, or
 * any later version.

 * SSJ is distributed in the hope that it will be useful,
 * but WITHOUT ANY WARRANTY; without even the implied warranty of
 * MERCHANTABILITY or FITNESS FOR A PARTICULAR PURPOSE.  See the
 * GNU General Public License for more details.

 * A copy of the GNU General Public License is available at
   <a href="http://www.gnu.org/licenses">GPL licence site</a>.
 */
\end{hide}
package umontreal.iro.lecuyer.hups;\begin{hide}

import umontreal.iro.lecuyer.util.PrintfFormat;
import umontreal.iro.lecuyer.rng.RandomStream;
\end{hide}

public abstract class ContainerPointSet extends PointSet \begin{hide} {
   protected PointSet P;                 // contained point set
\end{hide}
\end{code}

%%%%%%%%%%%%%%%%%%%%%%%%%%%%
% \subsubsection*{Constructor}
\begin{code}

   protected void init (PointSet P0) \begin{hide} {
      P = P0;
//      this.dim = P.getDimension();
//      this.numPoints = P.getNumPoints();
   }\end{hide}
\end{code}
 \begin{tabb}
   Initializes the container point set which will contain point set \texttt{P0}.
   This method must be called by the constructor of any class inheriting from
   \class{ContainerPointSet}.
 \hpierre{Since this is an abstract class, this constructor should be
  replaced by a method \texttt{init()}, if really needed.}
 \end{tabb}
\begin{htmlonly}
   \param{P}{contained point set}
\end{htmlonly}
\begin{code}

   public PointSet getOriginalPointSet() \begin{hide} {
      return P;
   }\end{hide}
\end{code}
\begin{tabb}
   Returns the (untransformed) point set inside this container.
\end{tabb}
\begin{htmlonly}
   \return{the point set inside this container}
\end{htmlonly}
\begin{code}

   public int getDimension() \begin{hide} {
      return P.getDimension();
   }\end{hide}
\end{code}
\begin{tabb}
   Returns the dimension of the contained point set.
\end{tabb}
\begin{htmlonly}
   \return{the dimension of the contained point set}
\end{htmlonly}
\begin{code}

   public int getNumPoints()\begin{hide} {
      return P.getNumPoints();
   }\end{hide}
\end{code}
\begin{tabb}
   Returns the number of points of the contained point set.
\end{tabb}
\begin{htmlonly}
   \return{the number of points of the contained point set}
\end{htmlonly}
\begin{code}\begin{hide}

   public double getCoordinate(int i, int j) {
      return P.getCoordinate (i, j);
   }

   public PointSetIterator iterator(){
      return new ContainerPointSetIterator();
   }\end{hide}

   public void randomize (PointSetRandomization rand) \begin{hide} {
       P.randomize(rand);
   }\end{hide}
\end{code}
\begin{tabb}
   Randomizes the contained point set using \texttt{rand}.
\end{tabb}
\begin{htmlonly}
   \param{rand}{\class{PointSetRandomization} to use}
\end{htmlonly}
\begin{code}

   public void addRandomShift (int d1, int d2, RandomStream stream)\begin{hide} {
      P.addRandomShift (d1, d2, stream);
   }\end{hide}
\end{code}
\begin{tabb}
  Calls \texttt{addRandomShift(d1, d2, stream)} of the contained point set.
\end{tabb}
\begin{htmlonly}
   \param{d1}{lower dimension of the random shift}
   \param{d2}{upper dimension of the random shift}
   \param{stream}{the random stream}
\end{htmlonly}
\begin{code}

   public void addRandomShift (RandomStream stream)\begin{hide} {
      P.addRandomShift (stream);
   }\end{hide}
\end{code}
\begin{tabb}
  Calls \texttt{addRandomShift(stream)} of the contained point set.
\end{tabb}
\begin{htmlonly}
   \param{stream}{the random stream}
\end{htmlonly}
\begin{code}

   public void clearRandomShift()\begin{hide} {
      P.clearRandomShift ();
   }\end{hide}
\end{code}
\begin{tabb}
  Calls \texttt{clearRandomShift()} of the contained point set.
\end{tabb}
\begin{code}\begin{hide}

   public String toString() {
      return "Container point set of: {" + PrintfFormat.NEWLINE
              + P.toString() + PrintfFormat.NEWLINE + "}";
   }


   // ********************************************************
   protected class ContainerPointSetIterator extends DefaultPointSetIterator {

      protected PointSetIterator innerIterator = P.iterator();

      public void setCurCoordIndex (int j) {
         innerIterator.setCurCoordIndex (j);
      }

      public void resetCurCoordIndex() {
         innerIterator.resetCurCoordIndex();
      }

      public boolean hasNextCoordinate() {
         return innerIterator.hasNextCoordinate();
      }

      public double nextCoordinate() {
         return innerIterator.nextCoordinate();
      }

      public void setCurPointIndex (int i) {
         innerIterator.setCurPointIndex(i);
      }

      public void resetCurPointIndex() {
         innerIterator.resetCurPointIndex();
      }

      public int resetToNextPoint() {
         return innerIterator.resetToNextPoint();
      }

      public boolean hasNextPoint() {
        return innerIterator.hasNextPoint();
      }

   }
}\end{hide}
\end{code}
