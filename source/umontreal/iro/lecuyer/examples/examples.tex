\documentclass[twoside,12pt]{article}

\usepackage{ssj}
\usepackage[procnames]{listings}
\usepackage{lstpatch}
\usepackage{url}
\usepackage{color}
\usepackage{crayola}
\usepackage{amsfonts}
% \usepackage{mycal}
% \usepackage{mybold}
% \usepackage{mymathbb}
\usepackage{pictexwd}


\lstloadlanguages{Java}
\lstset{language=Java,
float=tbhp,
captionpos=t,
frame=trbl,
abovecaptionskip=1.5em,
belowskip=2em,
basicstyle=\small\ttfamily,
stringstyle=\color{OliveGreen},
commentstyle=\color{red},
identifierstyle=\color{Bittersweet},
basewidth={0.5em},
showstringspaces=false,
framerule=0.8pt,
procnamestyle=\bfseries\color{blue},
emphstyle=\bfseries\color{Cerulean},
procnamekeys={class,extends,interface,implements}
}

% listings lacks a label feature to refer to listings.
% The code was inspired from the \label LaTeX command definition.
% \makeatletter
% \newcommand{\lstlabel}[1]{{
% \let\@currentlabel=\thelstlisting
% \label{#1}}}
% \makeatother

\def\g   {{\;\leftarrow\;}}
\def\tbu {\tilde{\mbox{\boldmath $u$}}}
\def\bu  {\mbox{\boldmath $u$}}

%%%%%%%%%%%%%%
%begin{latexonly}
%\def\fiverm {}
%\input prepictex.tex  \input pictex.tex  \input postpictex.tex
%\valuestolabelleading = -0.1\baselineskip
%end{latexonly}
\vbadness=10000

%%%%%%%%%%%%%%%%%%%%%%%%%%%%%%%%%%%%%%%%%%%%%%%%%%%%%%%
\begin{document}

\begin{titlepage}

\null\vfill
\begin {center}
{\Large\bf SSJ User's Guide } \\[12pt] 
{\Large Overview and Examples} \\[20pt]
% of Simulation programs using SSJ} \\[20pt]
 Version: \today \\
\vfill
 {\sc Pierre L'Ecuyer}
\footnote {\normalsize 
 SSJ was designed and implemented in the Simulation laboratory of the
 D\'epartement d'Informa\-tique et de Recherche
 Op\'erationnelle (DIRO), at the Universit\'e de Montr\'eal,
 under the supervision of 
 Pierre L'Ecuyer, with the contribution of
% (until August 2004):  
%\begin{verse}
 Mathieu Bague,
 Sylvain Bonnet,
 \'Eric Buist, 
% Chiheb Dkhil,
 Yves Edel,
 Regina H.{} S.{} Hong, 
 Alexander Keller,
 Pierre L'Ecuyer, 
 \'Etienne Marcotte,
 Lakhdar Meliani, 
 Abdelazziz Milib, 
 Fran\c{c}ois Panneton,
 Richard Simard,
 Pierre-Alexandre Tremblay, 
 and Jean Vaucher.
%\end{verse}
%
Its development has been supported by NSERC-Canada grant No.\ ODGP0110050,
NATEQ-Qu\'ebec grant No.\ 02ER3218, a Killam fellowship,
and a Canada Research Chair to P.~L'Ecuyer.

%  listed at the end of the \emph{SSJ overview} document.
% The first implementation of SSJ was done by Lakhdar Meliani   
% for his master's thesis.
} \\[10pt]
  Canada Research Chair in Stochastic Simulation and Optimization \\
% Chaire du Canada en simulation et optimisation stochastiques \\
  D\'epartement d'Informatique et de Recherche Op\'erationnelle \\
  Universit\'e de Montr\'eal, Canada
\vfill\vfill
\end {center}

\begin {quotation}
\noindent 
SSJ stands for \emph{stochastic simulation in Java}.
The SSJ library provides facilities for generating uniform and nonuniform random 
variates, computing different measures related to probability 
distributions, performing goodness-of-fit tests, applying 
quasi-Monte Carlo methods, collecting statistics,
and programming discrete-event simulations with both events and processes.
This document provides a very brief overview of this library
and presents several examples of small simulation programs in Java,
based on this library.  The examples are commented in detail.
They can serve as a good starting point for learning how to use SSJ.
The first part of the guide gives very simple examples that do not
need event or process scheduling.
The second part contains examples of discrete-event simulation 
programs implemented with an \emph{event view},
% using the package \texttt{simevents}.
while the third part gives examples of implementations based 
on the \emph{process view}.
% supported by the package \texttt{simprocs}.
\end {quotation}

\vfill
\end{titlepage}

\pagenumbering{roman}
\tableofcontents
\pagenumbering{arabic}
\section{Introduction}
% \addcontentsline{toc}{section}{Introduction}}

The aim of this document is to provide an introduction to SSJ
via a brief overview and a series of examples.
The examples are collected in three groups:
\begin{itemize}
\itemsep=0pt
\item[(1)] 
 those that need no event or process scheduling;
\item[(2)] 
 those based on the discrete-event simulation paradigm 
 and implemented with an \emph{event view} using the package 
 \texttt{simevents};
\item[(3)] 
 those implemented with the \emph{process view}, 
 supported by the package \texttt{simprocs}.
\end{itemize}
Sections~\ref{sec:simple} to \ref{sec:process} 
of this guide correspond to these three groups.
Some examples (e.g., the single-server queue) are carried across
two or three sections to illustrate different ways of implementing
the same model.
The Java code of all these examples is available on-line from the 
SSJ web page (just type ``SSJ iro'' in Google).

While studying the examples, the reader can refer to the functional 
definitions (the APIs) of the SSJ classes and methods in the guides of the 
corresponding packages.
Each package in SSJ has its own user's guide in the form of a \texttt{.pdf}
document that contains the detailed API and complete documentation,
and starts with an overview of one or two pages.
We strongly recommend reading each of these overviews.
We also recommend to refer to the \texttt{.pdf} versions of the guides,
because they contain a more detailed and complete documentation
than the \texttt{.html} versions, which are better suited for quick
on-line referencing for those who are already familiar with SSJ.


%%%%%%%%%%%%%%%%%%%%
\begin{comment}

In Section~\ref{sec:queue}, we start with a very simple classical 
example: a single queue.
We give different variants of this example, illustrating the mixture
of processes and events.
%
In Section~\ref{sec:preypred}, we give a small example of a deterministic
continuous simulation.
%
In Sections~\ref{sec:jobshop} and \ref{sec:timeshared},
we give examples of a job shop model and a time-shared computer model,
adapted from \cite{sLAW00a}.
%
A queuing model of a bank, taken from \cite{sBRA87a}, is programmed in
Section~\ref{sec:bank}, with both the process and event views.
%
In Section~\ref{sec:visits}, we simulate a model of guided tours
for groups of people, where the process synchronization is slightly
more complicated than for the earlier models.
%
In Section~\ref{sec:ingots}, we give an example
of a mixed discrete-continuous simulation.
%
In Section~\ref{sec:robot}, we give a more elaborate example,
for a model discussed in \cite{sLEC91a}, where a robot maintains
a series of machines subject to random failures.
It illustrates the idea of modular design. 
% for large simulation models.

\end{comment}

\latex{\section*{Overview}\addcontentsline{toc}{subsection}{Overview}}
\latex{\label {sec:overview}}

Process-oriented simulation is managed through this package.  
A \emph{Process} can be seen as an \emph{active object} whose behavior in
time is described by a method called \texttt{actions()}.
Each process must extend the
\externalclass{umontreal.iro.lecuyer.simprocs}{SimProcess} class and
must implement this \texttt{actions()} method.
Processes are created and can be scheduled to start at a given 
simulation time just like events.
In contrast with the corresponding \texttt{actions()} method of events,
the method of processes is generally not executed instantaneously in the 
simulation time frame.
At any given simulation time, at most one process can be \emph{active}, 
i.e., executing its \texttt{actions()} method.
The active process may create and schedule new processes,
kill suspended processes, and suspend itself.  A process is suspended
for a fixed delay or until a resource becomes available, or a condition
becomes true.  When a process is suspended or finishes its execution,
another process usually starts or resumes.

These processes may represent ``autonomous'' objects 
such as machines and robots in a factory, 
customers in a retail store, 
vehicles in a transportation or delivery system, etc.
The process-oriented paradigm is a natural way of describing complex
systems \cite{sFRA77a,sBIR86a,sKRE86a,sLAW00a} and often leads to more 
compact code than the event-oriented view.
However, it is often preferred to use events only, because this gives a faster
simulation program, by avoiding the process-synchronization overhead.
Most complex discrete-event systems are quite conveniently modeled only 
with events.
In SSJ, events and processes can be mixed freely.
The processes actually use events for their synchronization.

The classes \externalclass{umontreal.iro.lecuyer.simprocs}{Resource},
\externalclass{umontreal.iro.lecuyer.simprocs}{Bin}, and
\externalclass{umontreal.iro.lecuyer.simprocs}{Condition}
provide additional mechanisms for process synchronization.  
A \externalclass{umontreal.iro.lecuyer.simprocs}{Resource} corresponds
to a facility with limited capacity and a waiting queue.
A process can request an arbitrary number of units of a resource,
may have to wait until enough units are available,
can use the resource for a certain time, and eventually releases it.
A \externalclass{umontreal.iro.lecuyer.simprocs}{Bin} supports
producer/consumer relationships between processes.
It corresponds essentially to a pile of free tokens and a queue of 
processes waiting for the tokens.
A \emph{producer} adds tokens to the pile whereas a
\emph{consumer} (a process) can ask for tokens.
When not enough tokens are available, the consumer is blocked 
and placed in the queue.
The class \externalclass{umontreal.iro.lecuyer.simprocs}{Condition}
supports the concept of processes waiting
for a certain boolean condition to be true before continuing their execution.

Two different implementations of processes are available in SSJ, each
one corresponding to a subclass of \texttt{ProcessSimulator}.
The first one, called \texttt{ThreadProcessSimulator}, uses Java
threads as described in Section~4 of \cite{sLEC02a}.
The second one, \texttt{DSOLProcessSimulator}, is taken from DSOL
\cite{iJAC05a,sJAC04a} and was 
provided to us by Peter Jacobs.
Unfortunately, none of these two implementations is fully satisfactory.

Java threads are designed for \emph{real parallelism}, not for the kind of
\emph{simulated} parallelism required in process-oriented simulation.
In the Java Development Kit (JDK) 1.3.1 and earlier, \emph{green threads}
supporting simulated parallelism were available and our original 
implementation of processes\latex{ described in \cite{sLEC02a}} 
is based on them.
But green threads are no longer supported in recent Java runtime 
environments.  
True (native) threads from the operating system are used instead. 
This adds significant overhead and prevents the use of a large number
of processes in the simulation.
\emph{This implementation of processes with threads can be used safely only
with the JDK versions~1.3.1 or earlier}.
A program using the thread-based process view can easily be 10 to 20 
times slower than a similar program using the event view only
(see \cite{sLEC05a} for an example).

%%%  D-SOL.
The second implementation, made by P.{} Jacobs, stays away from threads.
It uses a Java reflection mechanism that interprets the code of processes
at runtime and transforms everything into events.
A program using the process view implemented with the DSOL interpreter
can be 500 to 1000 times slower than the corresponding event-based program
but the number of processes is limited only by the available memory.



\section {Some Elementary Examples}
\label {sec:simple}

We start with elementary examples that illustrate how to
generate uniform and nonuniform random numbers,
construct probability distributions,
collect elementary statistics,
and compute confidence intervals,
compare similar systems,
and use randomized quasi-Monte Carlo point sets,
with SSJ.

The models considered here are quite simple and some of the
performance measures can be computed by (more accurate) numerical
methods rather than by simulation.
The fact that we use these models to give a first tasting of SSJ
should not be interpreted to mean that
simulation is necessarily the best tool for them.


%%%%%%%%%%%%%%%%%%%%%%%%%%%%%%%%%%%%%%%%%%%%%%%%%%%%%%%%%
\subsection{Collisions in a hashing system}
\label{sec:collision}

% In this small example,
We want to estimate the expected number of collisions in a hashing system.
There are $k$ locations (or addresses) and $m$ distinct items.
Each item is assigned a random location, independently of the other items.
A \emph{collision} occurs each time an item is assigned a location already
occupied.  Let $C$ be the number of collisions.
We want to estimate $\mathbb{E}[C]$, the expected number of collisions, by simulation.
A theoretical result states that when $k\to\infty$ while $\lambda = m^2/(2k)$ is fixed,
$C$ converges in distribution to a Poisson random variable with mean $\lambda$.
For finite values of $k$ and $m$, we may want to approximate the distribution of $C$
by the Poisson distribution with mean $\lambda$, and
use Monte Carlo simulation to assess the quality of this approximation.
To do that, we can generate $n$ independent realizations of $C$, say $C_1,\dots,C_n$,
compute their empirical distribution and empirical mean, and compare with the
Poisson distribution.

The Java program in Listing~\ref{lst:Collision} simulates $C_1,\dots,C_n$
and computes a 95\%{} confidence interval on $\mathbb{E}[C]$.
The results for $k = 10000$, $m = 500$, and $n = 100000$, are in Listing~\ref{res:Collision}.
The reported confidence interval is $(12.25,\, 12.29)$, whereas $\lambda = 12.5$.
This indicates that the asymptotic result underestimates $\mathbb{E}[C]$ by nearly 2\%.


The Java program imports the SSJ packages \texttt{rng} and \texttt{stat}.
It uses only two types of objects from SSJ:
a \texttt{RandomStream} object, defined in the \texttt{rng} package,
that generates a stream of independent random numbers from the uniform distribution,
and a \texttt{Tally} object, from the \texttt{stat} package,
to collect statistical observations and produce the report.
In SSJ, \texttt{RandomStream} is actually just an interface that specifies all
the methods that must be provided by its different implementations, which
correspond to different brands of random streams (i.e., different types of
uniform random number generators).
The class \texttt{MRG32k3a}, whose constructor is invoked in the main program,
is one such implementation of \texttt{RandomStream}.  This is the one we use here.
The class \texttt{Tally} provides the simplest type of statistical collector.
It receives observations one by one, and after each new observation,
it updates the number, average,
variance, minimum, and maximum of the observations.
At any time, it can return these statistics or compute a confidence interval
for the theoretical mean of these observations, assuming that they are
independent and identically distributed with the normal distribution.
Other types of collectors that memorize the observations are also available in SSJ.


\lstinputlisting[label=lst:Collision,
caption={Simulating the number of collisions in a hashing system},%
lineskip=-1pt,%
emph={generateC,simulateRuns,main}
]
{Collision.java}

The class \texttt{Collision} offers the facilities to simulate copies of $C$.
Its constructor specifies $k$ and $m$, computes $\lambda$, and constructs
a boolean array of size $k$ to memorize the locations used so far,
in order to detect the collisions.
The method \texttt{generateC} initializes the boolean array to \texttt{false},
generates the $m$ locations, and computes $C$.
The method \texttt{simulateRuns} first resets the statistical collector \texttt{statC},
then generates $n$ independent copies of $C$ and pass these $n$ observations
to the collector via the method \texttt{add}.
The method \texttt{statC.report} computes a confidence interval from
these $n$ observations and returns a statistical report in the form of a character string.
This report is printed, together with the value of $\lambda$.


\lstinputlisting[label=res:Collision,
caption={Results of the program {\tt Collision}},%
float=hbt,lineskip=-1pt]{Collision.res}


%%%%%%%%%%%%%%%%%%%%%%%%%%%%%%%%%%%%%%%%%%%%%%%%%%%%%%%%%
\subsection {Nonuniform variate generation and simple quantile estimates}
\label {sec:nonuniform}

The program of Listing~\ref{lst:Nonuniform} simulates the following artificial model.
Define the random variable
\[
  X = Y_1 + \cdots + Y_N + W_1 + \dots + W_M,
\]
where $N$ is Poisson with mean $\lambda$, $M$ is geometric with parameter $p$,
the $Y_j$'s are gamma with parameters $(\alpha, \beta)$,
the $W_j$'s are lognormal with parameters $(\mu,\sigma)$,
and all these random variables are independent.
We want to generate $n$ copies of $X$, say $X_1,\dots,X_n$, and estimate
the 0.10, 0.50, 0.90, and 0.99 quantiles of the distribution of $X$,
simply from the quantiles of the empirical distribution.

The method \texttt{simulateRuns} generates $n$ copies of $X$ and pass
them to a statistical collector of class \texttt{TallyStore},
that stores the individual observations. % in a special type of array
% called \texttt{DoubleArrayList}, provided by the COLT library.
These observations are sorted in increasing order by invoking \texttt{quickSort},
and the appropriate empirical quantiles are printed,
together with a short report.


\lstinputlisting[label=lst:Nonuniform,
caption={Simulating nonuniform variates and observing quantiles},%
lineskip=-1pt,%
emph={generateX,simulateRuns,main}
]
{Nonuniform.java}

\lstinputlisting[label=res:Nonuniform,
caption={Results of the program {\tt Nonuniform}},%
float=hbt,lineskip=-1pt]{Nonuniform.res}


To simplify the program, all the parameters are fixed as constants at the
beginning of the class. This is simpler, but not recommended in general
because it does not permit one to perform experiments with different parameter
sets in a single program.
Passing the parameters to the constructor as in Listing~\ref{lst:Collision}
would require more lines of code, but would provide more flexibility.

The class initialization also constructs a \texttt{RandomStream} of type
\texttt{LFSR113} (this is a faster uniform generator that \texttt{MRG32k3a}),
used to generate all the random numbers.
For the generation of $N$, we construct a Poisson distribution
with mean $\lambda$ (without giving it a name),
and pass it together with the random stream to
the constructor of class \texttt{PoissonGen}.
The returned object \texttt{genN} is random number generator that generate
Poisson random variables with mean $\lambda$, via inversion.
As similar procedure is used to construct \texttt{genY} and \texttt{genW},
which generate gamma and lognormal random variates, respectively.
Note that a \texttt{RandomVariateGenInt} produces integer-valued random variates,
while a \texttt{RandomVariateGen} produces real-valued random variates.
For the gamma distribution, we use a special type of random number generator
based on a rejection method, which is faster than inversion.
These constructors precompute some (hidden) constants once for all,
to speedup the random variate generation.
For the Poisson distribution with mean $\lambda$, the constructor of \texttt{PoissonDist}
actually precomputes the distribution function in a table, and uses
this table to compute the inverse distribution function each time a Poisson
random variate needs to be generated with this particular distribution.
This is possible because all Poisson random variates have the same parameter $\lambda$.
If a different $\lambda$ was used for each variate, then we would use
the static method of \texttt{PoissonDist} instead of constructing a Poisson distribution,
otherwise we would have to reconstruct the distribution each time.
The static method reconstructs part of the table each time, with the given $\lambda$,
so it is slower if we want to generate several Poisson variates with the same $\lambda$.
As an illustration, we use the static method to generate the geometric random variates
(in \texttt{generateX}), instead of constructing a geometric distribution and variate generator.
To generate $M$, we invoke the static method \texttt{inverseF} of the class
\texttt{GeometricDist}, which evaluates the inverse geometric distribution function
for a given parameter $p$ and a given uniform random variate.

The results of this program, with $n = 10000$, are in Listing~\ref{res:Nonuniform}.
We see that $X$ has a coefficient of variation larger than 1,
and the quantiles indicate that the distribution is skewed,
with a long thick tail to the right.
We have $X < 553$ about half the time, but values over several thousands are not uncommon.
This probably happens when $N$ or $M$ takes a large value.
There are also cases where $N=M=0$, in which case $X=0$.



%%%%%%%%%%%%%%%%%%%%%%%%%%%%%%%%%%%%%%%%%%%%%%%%%%%%%%%%%
\subsection {A discrete-time inventory system}
\label {sec:inventory}

Consider a simple inventory system where the demands for a given product
on successive days are independent Poisson random variables with mean
$\lambda$.  If $X_j$ is the stock level at the beginning of day $j$ and
$D_j$ is the demand on that day, then there are $\min(D_j, X_j)$ sales,
$\max(0, D_j - X_j)$ lost sales, and the stock at the end of the day
is $Y_j = \max(0, X_j - D_j)$.
There is a revenue $c$ for each sale and a cost $h$ for each unsold
item at the end of the day.
The inventory is controlled using a $(s,S)$ policy:
If $Y_j < s$, order $S - Y_j$ items, otherwise do not order.
When an order is made in the evening, with probability $p$
it arrives during the night and can be used for the next day,
and with probability $1-p$ it never arrives (in which case a new order
will have to be made the next evening).
When the order arrives, there is a fixed cost $K$ plus a marginal cost
of $k$ per item.
The stock at the beginning of the first day is $X_0 = S$.

We want to simulate this system for $m$ days, for a given set of
parameters and a given control policy $(s,S)$, and replicate this
simulation $n$ times independently to estimate the
expected profit per day over a time horizon of $m$ days.
Eventually, we might want to \emph{optimize} the values of the decision
parameters $(s,S)$ via simulation, but we do not do that here.
(In practice, this is usually done for more complicated models.)


\lstinputlisting[label=lst:Inventory,
caption={A simulation program for the simple inventory system},%
lineskip=-1pt,%
emph={simulateOneRun,simulateRuns,main}
]
{Inventory.java}

%\bigskip

Listing~\ref{lst:Inventory} gives a Java program, based on the SSJ
library, that performs the required simulation for $n=500$, $m=2000$,
$s=80$, $S=200$, $\lambda=100$, $c=2$, $h=0.1$, $K=10$, $k=1$, and $p=0.95$.

The \texttt{import} statements at the beginning of the program
retrieve the SSJ packages/classes that are needed.
The \texttt{Inventory} class has a constructor that initializes the
model parameters (saving their values in class variables) and constructs
the required random number generators and the statistical collector.
To generate the demands $D_j$ on successive days, we create
(in the last line of the constructor) a random number stream and
a Poisson distribution with mean $\lambda$, and then a Poisson random
variate generator \texttt{genDemand} that uses this stream and this
distribution.  This mechanism will (automatically) precompute tables
to ensure that the Poisson variate generation is efficient.
This can be done because the value of $\lambda$ does not change during
the simulation.
The random number stream \texttt{streamOrder}, used to decide which
orders are received, and the statistical collector \texttt{statProfit},
are also created when the \texttt{Inventory} constructor is invoked.
The code that invokes their constructors is outside the
\texttt{Inventory} constructor, but it could have been inside as well.
On the other hand, \texttt{genDemand} must be constructed inside
the \texttt{Inventory} constructor, because the value of $\lambda$
is not yet defined outside.
The \emph{random number streams} can be viewed as virtual random number
generators that generate random numbers in the interval $[0,1)$
according to the uniform probability distribution.

The method \texttt{simulateOneRun} simulates the system for $m$ days,
with a given policy, and returns the average profit per day.
For each day, we generate the demand $D_j$, compute the stock $Y_j$
at the end of the day, and add the sales revenues minus the leftover
inventory costs to the profit.  If $Y_j < s$, we generate a uniform
random variable $U$ over the interval $(0,1)$ and an order of size
$S - Y_j$ is received the next morning if $U < p$ (that is, with
probability $p$).  In case of a successful order, we pay for it and
the stock level is reset to $S$.

The method \texttt{simulateRuns} performs $n$ independent
simulation runs of this system and returns a report that contains
a 90\%{} confidence interval for the expected profit.
The main program constructs an \texttt{Inventory} object with the
desired parameters, asks for $n$ simulation runs, and prints the report.
It also creates a timer that computes the total CPU time to execute
the program, and prints it.
The results are in Listing~\ref{res:Inventory}.
The average profit per day is approximately 85.
It took 0.39 seconds (on a 2.4 GHz computer running Linux)
to simulate the system for 2000 days,
compute the statistics, and print the results.

\lstinputlisting[label=res:Inventory,
caption={Results of the program \texttt{Inventory}},%
float=hbt,lineskip=-1pt]{Inventory.res}


\iffalse  %%%%%%%%%%
\setbox1=\vbox{\hsize=5.8in
\begin{verbatim}

REPORT on Tally stat. collector ==> stats on profit
         min        max      average    standard dev.  num. obs
      185.701     188.606    187.114        0.510          500
  90.0% confidence interval for mean (student): (   187.076,   187.151 )

Total CPU time: 0:0:0.60

\end{verbatim}
}

\begin{figure}[htb]
\centerline {\boxit{\box1}}
\
\label {fig:Inventory}
\end{figure}
\fi  %%%%%%%%%%%%

% \bigskip

In Listing~\ref{lst:InventoryCRN}, we extend the \texttt{Inventory} class
to a class \texttt{InventoryCRN} that compares two sets of values
of the inventory control policy parameters $(s,S)$.

% \bigskip

\lstinputlisting[label=lst:InventoryCRN,
caption={Comparing two inventory policies with common random numbers},%
lineskip=-1pt,%
emph={simulateDiffCRN,simulateDiff,main}]
{InventoryCRN.java}

The method \texttt{simulateDiff} simulates the system with policies
$(s_1, S_1)$ and $(s_2, S_2)$ independently, computes the difference
in profits, and repeats this $n$ times.  These $n$ differences are
tallied in statistical collector \texttt{statDiff}, to estimate the
expected difference in average daily profits between the two policies.

The method \texttt{simulateDiffCRN} does the same, but using
\emph{common random numbers} across pairs of simulation runs.
After running the simulation with policy $(s_1, S_1)$, the two random
number streams are reset to the start of their current substream,
so that they produce exactly the same sequence of random numbers
when the simulation is run with policy $(s_2, S_2)$.
Then the difference in profits is given to the statistical collector
\texttt{statDiff} as before and the two streams are reset to a
new substream for the next pair of simulations.

Why not use the same stream for both the demands and orders?
In this example, we need one random number to generate the demand each day,
and also one random number to know if the order arrives, but only on the days
where we make an order.  These days where we make an order are not
necessarily the same for the two policies.
So if we use a single stream for both the demands and orders,
the random numbers will not necessarily be used for the same purpose across
the two policies: a random number used to decide if the order arrives in one case
may end up being used to generate a demand in the other case.
This can greatly diminish the power of the common random numbers technology.
Using two different streams as in Listing~\ref{lst:InventoryCRN} ensures
at least that the random numbers are used for the same purpose for the
two policies.  For more explanations and examples about common random numbers,
see \cite{sLAW00a,sLEC09a}.

The main program estimates the expected difference in average
daily profits for policies $(s_1, S_1) = (80, 198)$ and
$(s_2, S_2) = (80, 200)$, first with independent random numbers,
then with common random numbers.
The other parameters are the same as before.
The results are in Listing~\ref{res:InventoryCRN}.
We see that use of common random numbers reduces the variance by
a factor of approximately 19 in this case.

\lstinputlisting[label=res:InventoryCRN,
caption={Results of the program \texttt{InventoryCRN}},%
lineskip=-1pt]{InventoryCRN.res}


%%%%%%%%%%%%%%%%%%%%%%%%%%%%%%%%%%%%%%%%%%%%%%%%%%%%%%%%%%%%%%
\subsection {A single-server queue with Lindley's recurrence}
\label {sec:queue-lindley}

We consider here a {\em single-server queue}, where customers
arrive randomly and are served one by one in their order of arrival,
i.e., {\em first in, first out\/} (FIFO).
We suppose that the times between successive arrivals are exponential
random variables with mean $1/\lambda$, that the service times are exponential
random variables with mean $1/\mu$, and that all these random variables are
mutually independent.
The customers arriving while the server is busy must join the queue.
The system initially starts empty.  We want to simulate the first
$m$ customers in the system and compute the mean waiting time per customer.

This simple model is well-known in queuing theory: It is called
an $M/M/1$ queue.  Simple formulas are available for this model to
compute the average waiting time per customer, average queue length,
etc., over an {\em infinite\/} time horizon \cite{pKLE75a}.
For a finite number of customers or a finite time horizon,
these expectations can also be computed by numerical methods,
but here we just want to show how it can be simulated.

In a single-server queue,
if $W_i$ and $S_i$ are the waiting time and service time of the
$i$th customer, and $A_i$ is the time between the arrivals of
the $i$th and $(i+1)$th customers, we have $W_1=0$ and the $W_i$'s
follow the recurrence
\eq
  W_{i+1} = \max(0,\; W_i + S_i - A_i),               \label {eq:lindley}
\endeq
known as \emph{Lindley's equation} \cite {pKLE75a}.


\lstinputlisting[label=lst:QueueLindley,
caption={A simulation based on Lindley's recurrence},
emph={simulateOneRun,simulateRuns,main}
]{QueueLindley.java}

% \bigskip

The program of Listing~\ref{lst:QueueLindley} exploits (\ref{eq:lindley})
to compute the average waiting time of the first $m$ customers in
the queue, repeats it $n$ times independently, and prints a summary of
the results.
Here, for a change, we pass the model parameters to the methods
instead of to the constructor, and
the random variates are generated by static methods
instead of via a \texttt{RandomVariateGen} object as in the
\emph{Inventory} class (previous example).
This illustrates various ways of doing the same thing.
The instruction ``\texttt{Wi += \dots}'' could also be replaced by
\begin{code}
      Wi += - Math.log (1.0 - streamServ.nextDouble()) / mu
            + Math.log (1.0 - streamArr.nextDouble()) / lambda;
\end{code}
which directly implements inversion of the exponential distribution.



%%%%%%%%%%%%%%%%%%%%%%%%%%%%%%%%%%%%%%%%%%%%%%%%%%%
\subsection{Using the observer design pattern}
\label {sec:observer}

Listing~\ref{lst:QueueObs} adds a few ingredients to the program
\texttt{QueueLindley}, in order to illustrate the \emph{observer}
design pattern implemented in package \texttt{stat}.
This mechanism permits one to separate data generation from data
processing.  It can be very helpful in large simulation programs or
libraries, where different objects may need to process the same data
in different ways.  These objects may have the task of storing observations
or displaying statistics in different formats, for example, and they are
not necessarily fixed in advance.

\iffalse
The \emph{observer} pattern, supported by the \texttt{Observable}
class and \texttt{Observer} interface in Java,
offers the appropriate flexibility for that kind of situation.
An \texttt{Observable} object acts in a sense like a \emph{broadcasting}
\emph{distribution agency} that maintains a list of registered
\texttt{Observer} objects, and sends information to all its registered
observers whenever appropriate.
\fi

The \emph{observer} pattern, supported by the
\externalclass{umontreal.iro.lecuyer.stat}{ObservationListener}
 interface in SSJ,
offers the appropriate flexibility for that kind of situation.
A statistical probe maintains a list of registered
\externalclass{umontreal.iro.lecuyer.stat}{ObservationListener}
objects, and broadcasts information to all its registered
observers whenever appropriate. Any object that implements the interface
\externalclass{umontreal.iro.lecuyer.stat}{ObservationListener}
can register as an observer.

\texttt{StatProbe} in package \texttt{stat}, as well as its subclasses
\texttt{Tally} and \texttt{Accumulate}, contains a list of
 \texttt{ObservationListener}'s.
Whenever they receive a new statistical observation, e.g.,  via
\texttt{Tally.add} or \texttt{Accumulate.update}, they send the new value
to all registered observers.
To register as an observer, an object must implement the interface
\externalclass{umontreal.iro.lecuyer.stat}{ObservationListener}
This implies that it must provide an implementation of the method
\texttt{newObservation}, whose purpose is to recover the information
that the object has registered for.

In the example, the statistical collector \texttt{waitingTimes}
 transmits to all its registered
listeners each new statistical observation that it receives via
its \texttt{add} method.
More specifically, each call to \texttt{waitingTimes.add(x)} generates
in the background a call to \texttt{o.newObservation(waitingTimes, x)}
for all registered observers \texttt{o}.

\begin{comment}
The method \texttt{notifyObs} is used to
turn the tally into such an agency.  In fact, the collector is both a
tally and a distribution agency, but its tally functionality can be
disabled using the \texttt{stopCollectStat} method.  This can be useful when
the registered observers already perform statistical collection.
\end{comment}

Two observers register to receive observations from
\texttt{waitingTimes} in the example.
They are anonymous objects of classes \texttt{ObservationTrace} and
\texttt{LargeWaitsCollector}, respectively. Each one is informed of any new
observation $W_i$ via its \texttt{newObservation} method.
The task of the \texttt{ObservationTrace} observer is to
print the waiting times $W_5$, $W_{10}$, $W_{15}$, \dots, whereas
the \texttt{LargeWaitsCollector} observer stores in an array all waiting
times that exceed 2.
The statistical collector \texttt{waitingTimes} itself also stores
appropriate information to be able to provide a statistical report
when required.

The \texttt{ObservationListener} interface specifies that \texttt{newObservation}
 must have two formal parameters, of classes \texttt{StatProbe} and
\texttt{double}, respectively. The second parameter is the value of the
observation.
In the case where the observer registers to several \texttt{ObservationListener}
objects, the first parameter of \texttt{newObservation} tells it which one
is sending the information, so it can adopt the correct behavior for
this sender.


\lstinputlisting[label=lst:QueueObs,
caption={A simulation of Lindley's recurrence using observers},
emph={simulateOneRun,simulateRuns,updateLargeWaitsCollector,formatLargeWaits,update,main}
]{QueueObs.java}



%%%%%%%%%%%%%%%%%%%%%%%%%%%%%%%%%%%%%%%%%%%%%%%%%%%%%%%
\subsection {Pricing an Asian option}
\label {sec:asian}

A \emph{geometric Brownian motion} (GBM) $\{S(\zeta),\,\zeta\ge 0\}$ satisfies
\[
  S(\zeta) = S(0) \exp\left[(r - \sigma^2/2)\zeta + \sigma B(\zeta)\right]
\]
where $r$ is the \emph{risk-free appreciation rate},
$\sigma$ is the \emph{volatility parameter}, and
$B$ is a standard Brownian motion, i.e.,
a stochastic process whose increments over disjoint intervals
are independent normal random variables, with mean 0 and variance
$\delta$ over an interval of length $\delta$ (see, e.g., \cite{fGLA04a}).
The GBM process is a popular model for the evolution in time of the
market price of financial assets.
A discretely-monitored  \emph{Asian option} on the arithmetic
average of a given asset has discounted payoff
\eq                                            \label{eq:payasian}
 X = e^{-rT} \max[\bar S - K,\, 0]
\endeq
where $K$ is a constant called the \emph{strike price} and
\begin{equation}                        \label{eq:arithmetic-average}
 \bar S = \frac{1}{t} \sum_{j=1}^t S(\zeta_j),
\end{equation}
for some fixed observation times $0 < \zeta_1 < \cdots < \zeta_t = T$.
The value (or fair price) of the Asian option is $c = E[X]$ where
the expectation is taken under the so-called risk-neutral measure
(which means that the parameters $r$ and $\sigma$ have to be selected
in a particular way; see \cite{fGLA04a}).

This value $c$ can be estimated by simulation as follows.
Generate $t$ independent and identically distributed
(i.i.d.) $N(0,1)$ random variables $Z_1,\dots,Z_t$
and put $B(\zeta_j) = B(\zeta_{j-1}) + \sqrt{\zeta_j - \zeta_{j-1}} Z_j$,
for $j=1,\dots,t$, where $B(\zeta_0) = \zeta_0 = 0$.  Then,
\eq
  S(\zeta_j) = S(0) e^{(r-\sigma^2/2)\zeta_j + \sigma B(\zeta_j)}
             = S(\zeta_{j-1}) e^{(r-\sigma^2/2)(\zeta_j-\zeta_{j-1})
                          + \sigma \sqrt{\zeta_j - \zeta_{j-1}} Z_j}
                                                      \label{eq:Szetaj}
\endeq
for $j = 1,\dots,t$ and the payoff can be computed via (\ref{eq:payasian}).
This can be replicated $n$ times, independently, and the
option value is estimated by the average discounted payoff.
The Java program of Listing~\ref{lst:Asian} implement this procedure.

Note that generating the sample path and computing the payoff is done
in two different methods.  This way, other methods could eventually be
added to compute payoffs that are defined differently (e.g., based on
the geometric average, or with barriers, etc.) over the same generated
sample path.

\lstinputlisting[label=lst:Asian,
caption={Pricing an Asian option on a GMB process},
lineskip=-1pt,
emph={generatePath,getPayoff,simulateRuns,main}
]{Asian.java}

% \bigskip

The method \texttt{simulateRuns} performs $n$ independent simulation
runs using the given random number stream and put the $n$ observations
of the net payoff in the statistical collector \texttt{statValue}.
In the \texttt{main} program, we first specify the 12 observation
times $\zeta_j = j/12$ for $j=1,\dots,12$, and put them in the array
\texttt{zeta} (of size 13) together with $\zeta_0=0$.
We then construct an \texttt{Asian} object with parameters
$r=0.05$, $\sigma=0.5$, $K = 100$, $S(0)=100$, $t=12$, and the
observation times contained in array \texttt{zeta}.
We then create the statistical collector \texttt{statValue},
perform $10^5$ simulation runs, and print the results.
The discount factor $e^{-rT}$ and the
constants $\sigma \sqrt{\zeta_j - \zeta_{j-1}}$ and
$(r-\sigma^2/2)(\zeta_j - \zeta_{j-1})$ are precomputed in the
constructor \texttt{Asian}, to speed up the simulation.


The program in Listing~\ref{lst:AsianQMC} extends the class \texttt{Asian}
to \texttt{AsianQMC}, whose method \texttt{simulate\-QMC}
estimates the option value via randomized quasi-Monte Carlo.
It uses $m$ independently randomized copies of digital net \texttt{p}
and puts the results in statistical collector \texttt{statAverage}.
The randomization is a left matrix scramble followed by a digital
random shift, applied before each batch of $n$ simulation runs.

% \bigskip

\lstinputlisting[label=lst:AsianQMC,
caption={Pricing an Asian option on a GMB process with quasi-Monte Carlo},
lineskip=-1pt,
lineskip=-1pt,
emph={simulateQMC,main}
]{AsianQMC.java}

% \bigskip

The random number stream passed to the method
\texttt{simulateRuns} is an iterator that enumerates the points and
coordinates of the randomized point set \texttt{p}.
These point set iterators, available for each type of point set
in package \texttt{hups}, implement the \texttt{RandomStream} interface
and permit one to easily replace the uniform random numbers by (randomized)
highly-uniform point sets or sequences, without changing the code of
the model itself.
The method \texttt{resetStartStream}, invoked immediately after each
randomization, resets the iterator to the first coordinate of the
first point of the point set \texttt{p}.
The number $n$ of simulation runs is equal to the number of points.
The points correspond to substreams in the \texttt{RandomStream} interface.
The method \texttt{resetNextSubstream}, invoked after each simulation
run in \texttt{simulateRuns}, resets the iterator to the first
coordinate of the next point.
Each generation of a uniform random number (directly or indirectly)
with this stream during the simulation moves the iterator to the next
coordinate of the current point.

The point set used in this example is a \emph{Sobol' net} with
$n = 2^{16}$ points in $t$ dimensions.
The \texttt{main} program passes this point set to \texttt{simulateQMC}
and asks for $m=20$ independent randomizations.
It then computes the empirical variance and CPU time
\emph{per simulation run} for both MC and randomized QMC.
It prints the ratio of variances, which can be
interpreted as the estimated \emph{variance reduction factor} obtained
when using QMC instead of MC in this example, and the ratio of
efficiencies, which can be interpreted
as the estimated \emph{efficiency improvement factor}.
(The efficiency of an estimator is defined as
1/(variance $\times$ CPU time per run.)
The results are in Listing~\ref{res:AsianQMC}:
QMC reduces the variance by a factor of around 250 and improves the
efficiency by a factor of over 500.
Randomized QMC not only reduces the variance, it also runs faster than MC.
The main reason for this is the call to \texttt{resetNextSubstream}
in \texttt{simulateRuns}, which is rather costly for a random
number stream of class \texttt{MRG32k3a} (with the current implementation)
and takes negligible time for an iterator over a digital net in base 2.
In fact, in the the case of MC, the call to \texttt{resetNextSubstream}
is not really needed.  Removing it for that case reduces the CPU
time by more than 40\%.

\lstinputlisting[label=res:AsianQMC,
caption={Results of the program \texttt{AsianQMC}},%
float=t,lineskip=-1pt]{AsianQMC.res}

\section {Discrete-Event Simulation}
\label {sec:event}

Examples of discrete-event simulation programs, based on the event view
supported by the package \texttt{simevents}, are given in this section.

%%%%%%%%%%%%%%%%%%%%%%%%%%%%%%%%%%%%%%%%%%%%%%%%%%%%%%%%%%%%%%
\subsection {The single-server queue with an event view}
\label {sec:queue-event}

We return to the single-server queue considered in
Section~\ref{sec:queue-lindley}.
This time, instead of simulating a fixed number of customers,
we simulate the system for a fixed time horizon of 1000.

%%%%%%%%%%%%%%%%%
\lstinputlisting[label=lst:QueueEv,%
caption={Event-oriented simulation of an $M/M/1$ queue},%
lineskip=-1pt,emph={simulateOneRun,actions,main}]{QueueEv.java}

% \bigskip

Listing~\ref{lst:QueueEv} gives an event-oriented simulation program,
where a subclass of the class \texttt{Event} is defined for each type
of event that can occur in the simulation:
arrival of a customer (\texttt{Arrival}),
departure of a customer (\texttt{Departure}),
and end of the simulation (\texttt{EndOfSim}).
Each event {\em instance\/} is inserted into the {\em event list\/}
upon its creation, with a scheduled time of occurrence, and is
{\em executed\/} when the simulation clock reaches this time.
Executing an event means invoking its \texttt{actions} method.
Each event subclass must implement this method.
The simulation clock and the event list (i.e., the list of events
scheduled to occur in the future) are maintained behind the
scenes by the class \texttt{Sim} of package \texttt{simevents}.

When \texttt{QueueEv} is instantiated  by the \texttt{main} method,
the program creates
two streams of random numbers,
two random variate generators, two
lists, and two statistical probes (or collectors).
The random number streams
% can be viewed as virtual random number generators that generate random
% numbers in the interval $[0,1)$ according to the uniform probability
are attached to random variate generators
\texttt{genArr} and \texttt{genServ} which are used to generate the times
between successive arrivals and the service times, respectively.
We can use such an attached generator because the means (parameters)
do not change during simulation.
The lists \texttt{waitList} and \texttt{servList} contain the customers
waiting in the queue and the customer in service (if any), respectively.
Maintaining a list for the customer in service may seem exaggerated,
because this list never contains more than one object, but the current
design has the advantage of working with very little change if the
queuing model has more than one server, and in other more general
situations.
Note that we could have used the class \texttt{LinkedListStat} from package
\texttt{simevents} instead of \texttt{java.util.LinkedList}.
However, with our current implementation,
the automatic statistical collection in that \texttt{LinkedListStat}
class would not count the customers whose waiting time is zero, because
they are never placed in the list.
\begin{comment}
Here we use the class \texttt{List} from package \texttt{simevents}.
This class is equivalent to the standard class \texttt{java.util.LinkedList},
except that its implementation is more efficient than the current one
in JDK and it can also collect statistics automatically.
However, the automatic statistical collection on \texttt{waitList}
would not count the customers whose waiting time is zero, because
they are never placed in this list, so we do not use this facility.
\end{comment}

The statistical probe \texttt{custWaits} collects statistics on the
customer's waiting times.  It is of the class \texttt{Tally}, which
is appropriate when the statistical data of interest is  a sequence
of observations $X_1, X_2, \dots$ of which we might want to compute
the sample mean, variance, and so on.
A new observation is given to this probe by the \texttt{add} method
each time a customer starts its service.
Every \texttt{add} to a \texttt{Tally} probe brings a new observation $X_i$,
which corresponds here to a customer's waiting time in the queue.
The other statistical probe, \texttt{totWait}, is of the class
\texttt{Accumulate}, which means that it computes the integral
(and, eventually, the time-average) of a continuous-time
stochastic process with piecewise-constant trajectory.
Here, the stochastic process of interest is the length of the queue
as a function of time.  One must call \texttt{totWait.update} whenever
there is a change in the queue size, to update the (hidden)
{\em accumulator\/} that keeps the current value of the integral
of the queue length.  This integral is equal, after each update,
to the total waiting time in the queue, for all the customers,
since the beginning of the simulation.

Each customer is an object with two fields: \texttt{arrivTime}
memorizes this customer's arrival time to the system, and
\texttt{servTime} memorizes its service time.
This object is created, and its fields are initialized,
when the customer arrives.

The method \texttt{simulateOneRun} simulates this system for a fixed
time horizon.  It first invokes \texttt{Sim.init},
which initializes the clock and the event list.
The method \texttt{Sim.start} actually starts the simulation
by advancing the clock to the time of
the first event in the event list, removing this event
from the list, and executing it.  This is repeated until either
\texttt{Sim.stop} is called or the event list becomes empty.
\texttt{Sim.time} returns the current time on the simulation clock.
Here, two events are scheduled before starting the simulation:
the end of the simulation at time horizon, and the
arrival of the first customer at a random time that has the exponential
distribution with \emph{rate} $\lambda$ (i.e., \emph{mean} $1/\lambda$),
generated by \texttt{genArr} using inversion and its attached random stream.
The method \texttt{genArr.nextDouble} returns this exponential random variate.

The method \texttt{actions} of the class \texttt{Arrival} describes what happens
when an arrival occurs.
Arrivals are scheduled by a domino effect:
the first action of each arrival event schedules the next event in
a random number of time units, generated from the exponential distribution
with rate $\lambda$.
Then, the newly arrived customer is created,
its arrival time is set to the current simulation time,
and its service time is generated from the exponential distribution
with mean $1/\mu$, using the random variate generator \texttt{genServ}.
If the server is busy, this customer is inserted at the end of the
queue (the list \texttt{waitList}) and the statistical probe
\texttt{totWait}, that keeps track of the size of the queue, is updated.
Otherwise, the customer is inserted in the server's list \texttt{servList},
its departure is scheduled to happen in a number of time units
equal to its service time, and a new observation of 0.0 is given to the
statistical probe \texttt{custWaits} that collects the waiting times.

When a \texttt{Departure} event occurs, the customer in service is
removed from the list (and disappears).
If the queue is not empty, the first customer is removed from
the queue (\texttt{waitList}) and inserted in the server's list,
and its departure is scheduled.
The waiting time of that customer (the current time minus its
arrival time) is given as a new observation to the probe
\texttt{custWaits}, and the probe \texttt{totWait} is also updated
with the new (reduced) size of the queue.

The event \texttt{EndOfSim} stops the simulation.
Then the \texttt{main} routine regains control and prints statistical
reports for the two probes.
The results are shown in Listing~\ref{res:QueueEv}.
When calling \texttt{report} on an \texttt{Accumulate} object, an implicit
update is done using the current simulation time and the last
value given to \texttt{update}.  In this example, this ensures
that the \texttt{totWait} accumulator will integrate the total wait
until the time horizon, because the simulation clock is still at that
time when the report is printed.
Without such an automatic update, the accumulator would integrate
only up to the last update time before the time horizon.


%%%%%%%%%%%%%%%%%
\lstinputlisting[label=res:QueueEv,%
caption={Results of the program \texttt{QueueEv}},%
lineskip=-1pt]{QueueEv.res}



%%%%%%%%%%%%%%%%%%%%%%%%%%%%%%%%%%%%%%%%%%%%%%%%%%%%%%%%%%%%%%%%%%%
\subsection {Continuous simulation: A prey-predator system}
\label {sec:preypred}

We consider a classical prey-predator system, where the preys
are food for the predators (see, e.g., \cite{sLAW00a}, page 87).
Let $x(t)$ and $z(t)$ be the numbers of preys and predators
at time $t$, respectively.
These numbers are integers, but as an approximation,
we shall assume that they are real-valued variables evolving
according to the differential equations
\begin{eqnarray*}
  x'(t) &= &\ r x(t) - c x(t) z(t)\\
  z'(t) &= & -s z(t) + d x(t) z(t)
\end{eqnarray*}
with initial values $x(0)=x_0>0$ et $z(0)=z_0>0$.
This is a \emph{Lotka-Volterra} system of differential
equations, which has a known analytical solution.
Here, in the program of Listing~\ref{lst:PreyPred},
we simply simulate its evolution, to illustrate the continuous
simulation facilities of SSJ.

\lstinputlisting[label=lst:PreyPred,%
caption={Simulation of the prey-predator system},
emph={derivative,actions,main}
]{PreyPred.java}

% \bigskip
Note that, instead of using the default simulator, this program
explicitly  creates a discrete-event \class{Simulator} object to manage the
execution of the simulation, unlike the other examples in this section.

The program prints the triples $(t, x(t), z(t))$ at values of
$t$ that are multiples of \texttt{h}, one triple per line.
This is done by an event of class \texttt{PrintPoint}, which is
rescheduled at every \texttt{h} units of time.
This output can be redirected to a file for later use,
for example to plot a graph of the trajectory.
The continuous variables \texttt{x} and \texttt{z} are instances of the
classes \texttt{Preys} and \texttt{Preds}, whose method \texttt{derivative}
give their derivative $x'(t)$ and $z'(t)$, respectively.
The differential equations are integrated by a Runge-Kutta method
of order 4.

%%%%%%%%%%%%%%%%%%%%%%%%%%%%%%%%%%%%%%%%%%%%%%%%%%%%%%%%%%%%%%%%%%%%
\subsection {A simplified bank}
\label {sec:bank}

This is Example~1.4.1 of \cite{sBRA87a}, page~14.
% Bratley, Fox, and Schrage (1987).
A bank has a random number of tellers every morning.
On any given day, the bank has $t$ tellers with probability $q_t$,
where $q_3 = 0.80$, $q_2 = 0.15$, and $q_1 = 0.05$.
All the tellers are assumed to be identical from the modeling viewpoint.


%%%%%%%%%%%%%
\setbox0=\vbox{\hsize=6.0in
%%  {Arrival rate of customers to the bank.}
\beginpicture
\setcoordinatesystem units <1.8cm,2cm>
%%  72.27pt = 1in
\setplotarea x from 0 to 6.5, y from 0 to 1
\axis left
  label {\lines {arrival \ \cr rate }}
  ticks length <2pt> withvalues 0.5 1 / at 0.5 1 / /
\axis bottom
  label {\hbox to 5.4in {\hfill time}}
  ticks length <2pt> withvalues 9:00 9:45 11:00 14:00 15:00 /
  at 0.0 0.75 2.0 5.0 6.0 / /
\shaderectangleson
\putrectangle corners at 0.75 0.0 and 2.0 0.5
\putrectangle corners at 2.0 0.0 and 5.0 1.0
\putrectangle corners at 5.0 0.0 and 6.0 0.5
\endpicture
}

\begin{figure}[htb]
\box0
\caption {Arrival rate of customers to the bank.}
\label {fig:blambda}
\end{figure}

\lstinputlisting[label=lst:BankEv,
caption={Event-oriented simulation of the bank model},
emph={simulOneDay,simulateDays,actions,balk,main}]%
{BankEv.java}

% \bigskip

The bank opens at 10:00 and closes at 15:00 (i.e., {\sc 3 p.m.}).
The customers arrive randomly according to a Poisson process
with piecewise constant rate $\lambda(t)$, $t\ge 0$.
The arrival rate $\lambda(t)$ (see Fig.{}~\ref{fig:blambda})
is 0.5 customer per minute from
9:45 until 11:00 and from 14:00 until 15:00, and
one customer per minute from 11:00 until 14:00.
The customers who arrive between 9:45 and 10:00 join a FIFO queue
and wait for the bank to open.
At 15:00, the door is closed, but all the customers already in will be served.
Service starts at 10:00.

Customers form a FIFO queue for the tellers, with balking.
An arriving customer will balk (walk out) with probability $p_k$ if there
are $k$ customers ahead of him in the queue (not counting the people
receiving service), where
 $$ p_k = \cases { 0       & if $k\le 5$;\cr
                   (n-5)/5 & if $5 < k < 10$;\cr
                   1       & if $k\ge 10$.\cr }$$
The customer service times are independent Erlang random
variables: Each service time is the sum of
two independent exponential random variables with mean one.

We want to estimate the expected number of customers served in a
day, and the expected average wait for the customers
served on a day.
% We could also be interested in the effect of changing the number of tellers,
% changing their speed, and so on.

Listing~\ref{lst:BankEv} gives and event-oriented simulation
program for this bank model.
There are events at the fixed times 9:45, 10:00, etc.
At 9:45, the counters are initialized and the arrival process
is started.  The time until the first arrival,
or the time between one arrival and the next one, is (tentatively)
an exponential with a mean of 2 minutes.
However, as soon as an arrival turns out to be past 11:00,
its time must be readjusted to take into account the increase of the
arrival rate at 11:00.
The event 11:00 takes care of this readjustment,
and the event at 14:00 makes a similar readjustment
when the arrival rate decreases.
We give the specific name \texttt{nextArriv} to the next planned
arrival event in order to be able to reschedule
that particular event to a different time.
Note that a {\em single\/} arrival event is
created at the beginning and this same event is scheduled over and
over again.  This can be done because there is never more than one
arrival event in the event list.
(We could have done that as well for the $M/M/1$ queue in
Listing \ref{lst:QueueEv}.)

At the bank opening at 10:00, an event generates the number
of tellers and starts the service for the corresponding customers.
The event at 15:00 cancels the next arrival.

Upon arrival, a customer checks if a teller is free.
If so, one teller becomes busy and the customer generates its
service time and schedules his departure, otherwise the
customer either balks or joins the queue.
The balking decision is computed by the method \texttt{balk},
using the random number stream \texttt{streamBalk}.
The arrival event also generates the next scheduled arrival.
Upon departure, the customer frees the teller, and the first
customer in the queue, if any, can start its service.
The generator \texttt{genServ} is an \texttt{ErlangConvolutionGen} generator,
so that the Erlang variates are generated by adding two exponentials instead
of using inversion.

The method \texttt{simulateDays} simulates the bank
for \texttt{numDays} days and prints a statistical report.
If $X_i$ is the number of customers served on day $i$ and
$Q_i$ the total waiting time on day $i$, the program estimates
$E[X_i]$ and $E[Q_i]$ by their sample averages $\bar X_n$ and
$\bar Q_n$ with $n = $\texttt{numDays}.
For each simulation run (each day), \texttt{simulOneDay} initializes
the clock, event list, and statistical probe for the waiting times,
schedules the deterministic events, and runs the simulation.
After 15:00, no more arrival occurs and the event list becomes
empty when the last customer departs.
At that point, the program returns to right after the \texttt{Sim.start()}
statement and updates the statistical counters for the number of
customers served during the day and their total waiting time.

The results are given in Listing~\ref{res:Bank}.

\lstinputlisting[label=res:Bank,
caption={Results of the \texttt{BankEv} program}]%
{Bank.res}


%%%%%%%%%%%%%%%%%%%%%%%%%%%%%%%%%%%%%%%%%%%%%%%%%%%%%%%%%%%%%%%%%
\subsection {A call center}
\label {sec:call-center}

We consider here a simplified model of a telephone contact center
(or \emph{call center}) where agents answer incoming calls.
% in FIFO order.
Each day, the center operates for $m$ hours.
The number of agents answering calls and the arrival rate
of calls vary during the day; we shall assume that
they are constant within each hour of operation but depend on the hour.
% We number the hours of operations staring from zero.
Let $n_j$ be the number of agents in the center during hour $j$,
for $j=0,\dots,m-1$.
For example, if the center operates from 8 {\sc am} to 9 {\sc pm},
then $m=13$ and hour $j$ starts at ($j+8$) o'clock.
All agents are assumed to be identical.
When the number of occupied agents at the end of hour $j$ is larger
than $n_{j+1}$, ongoing calls are all completed but new calls are
answered only when there are less than $n_{j+1}$ agents busy.
After the center closes, ongoing calls are completed and calls already
in the queue are answered, but no additional incoming call is taken.

The calls arrive according to a Poisson process with piecewise constant rate,
equal to $R_j = B \lambda_j$ during hour $j$, where the $\lambda_j$
are constants and $B$ is a random variable having the gamma distribution
with parameters $(\alpha_0,\alpha_0)$.
% $E[B] = \alpha/\lambda$ and $\Var[B] = \alpha /\lambda^2$.
Thus, $B$ has mean 1 and variance $1/\alpha_0$, and it
represents the \emph{busyness} of the day; it is more busy than usual
when $B > 1$ and less busy when $B < 1$.
The Poisson process assumption means that conditional on $B$,
the number of incoming calls during any subinterval $(t_1, t_2]$
of hour $j$ is a Poisson random variable with
mean $(t_2 - t_1) B \lambda_j$ and that the arrival counts in
any disjoint time intervals are independent random variables.
This arrival process model is motivated and studied in
\cite{ccWHI99c} and \cite{ccAVR04a}.

Incoming calls form a FIFO queue for the agents.
% with impatient customers.
A call is \emph{lost} (abandons the queue) when its waiting time
exceed its \emph{patience time}.
The patience times of calls are assumed to be i.i.d.{} random variables
with the following distribution: with probability $p$ the patience
time is 0 (so the person hangs up unless there is an agent
available immediately), and with probability $1-p$ it is exponential
with mean $1/\nu$.
The service times
% (times to handle the calls)
are i.i.d.{} gamma random variables with parameters $(\alpha,\beta)$.
%  Or perhaps Erlang?

We want to estimate the following quantities
\emph{in the long run} (i.e., over an infinite number of days):
(a) $w$, the average waiting time per call,
(b) $g(s)$, the fraction of calls whose waiting time is less than
    $s$ seconds for a given threshold $s$, and
(c) $\ell$, the fraction of calls lost due to abandonment.

Suppose we simulate the model for $n$ days.  For each day $i$, let
$A_i$ be the number of arrivals,
$W_i$ the total waiting time of all calls,
$G_i(s)$ the number of calls who waited less than $s$ seconds,
and $L_i$ the number of abandonments.
For this model, the expected number of incoming calls in a day is
$a = E[A_i] = \sum_{j=0}^{m-1} \lambda_j$.
% We have that $w = E[W_i]/a$, $g(s) = E[G_i(s)]/a$, and $\ell = E[L_i]/a$.
Then, $W_i/a$, $G_i(s)/a$, and $L_i/a$, $i=1,\dots,n$,
are i.i.d.{} unbiased estimators of $w$, $g(s)$, and $\ell$, respectively,
and can be used to compute confidence intervals for these quantities
in a standard way if $n$ is large.


\lstinputlisting[label=lst:CallCenter,
caption={Simulation of a simplified call center},
emph={simulateOneDay,actions,generPatience,checkQueue,endWait,readData,main}
]%
{CallCenter.java}

% \bigskip

Listing~\ref{lst:CallCenter} gives an event-oriented simulation
program for this call center model.
When the \texttt{CallCenter} class is instantiated by the \texttt{main} method,
the random streams, list, and statistical probes are created,
and the model parameters are read from a file by the method \texttt{readData}.
The line \texttt{Locale.setDefault(Locale.US)} is added because
real numbers in the data file are read in the anglo-saxon form 8.3
instead of the form 8,3 used by most countries in the world.
The \texttt{main} program then simulates $n = 1000$ operating days and
prints the value of $a$, as well as 90\%{} confidence intervals on
$a$, $w$, $g(s)$, and $\ell$, based on their estimators
$\bar A_n$, $\bar W_n/a$, $\bar G_n(s)/a$, and $\bar L_n/a$,
assuming that these estimators have approximately the Student distribution.
This is justified by the fact that $W_i$, and $G_i(s)$, and $L_i$
are themselves ``averages'' over several observations, so we may expect
their distribution to be not far from a normal.

To generate the service times, we use a gamma random variate
generator called \texttt{genServ}, created in the constructor
after the parameters $(\alpha,\beta)$ of the service time distribution
have been read from the data file.
For the other random variables in the model, we simply create random
streams of i.i.d.{} uniforms (in the preamble) and apply inversion
explicitly to generate the random variates.
The latter approach is more convenient, e.g., for patience times
because their distribution is not standard and for the inter-arrival
times because their mean changes every period.
For the gamma service time distribution, on the other hand,
the parameters always remain the same and inversion is rather slow,
so we decided to create a generator that uses a faster special method.

The method \texttt{simulateOneDay} simulates one day of operation.
It initializes the simulation clock, event list, and counters,
schedules the center's opening
% (start of the first period)
and the first arrival, and starts the simulation.
When the day is over, it updates the statistical collectors.
Note that there are two versions of this method; one that generates the random
variate $B$ and the other that takes its value as an input parameter.
This is convenient in case one wishes to simulate the center with
a fixed value of $B$.

An event \texttt{NextPeriod(j)} marks the beginning of each period $j$.
The first of these events (for $j=0$) is scheduled by \texttt{simulateOneDay};
then the following ones schedule each other successively,
until the end of the day.
This type of event updates the number of agents in the center and
the arrival rate for the next period.
If the number of agents has just increased and the queue is not empty,
some calls in the queue can now be answered.
The method \texttt{checkQueue} verifies this and starts service for the
appropriate number of calls.
The time until the next planned arrival is readjusted to take into account
the change of arrival rate, as follows.
The inter-arrival times are i.i.d.{} exponential with
mean $1/R_{j-1}$ when the arrival rate is fixed at $R_{j-1}$.
But when the arrival rate changes from $R_{j-1}$ to $R_j$,
the residual time until the next arrival should be modified from an
exponential with mean $1/R_{j-1}$ (already generated)
to an exponential with mean $1/R_j$.
Multiplying the residual time by $\lambda_{j-1}/\lambda_j$ is an easy
way to achieve this.
We give the specific name \texttt{nextArrival} to the next arrival
event in order to be able to reschedule it to a different time.
Note that there is a \emph{single} arrival event which is scheduled
over and over again during the entire simulation.
This is more efficient than creating a new arrival event for each
call, and can be done here because there is never more than one arrival
event at a time in the event list.
At the end of the day, simply canceling the next arrival makes sure
that no more calls will arrive.

Each arrival event first schedules the next one.
Then it increments the arrivals counter and creates the new call that just
arrived.  The call's constructor generates its service time and decides
where the incoming call should go.
If an agent is available, the call is answered immediately
(its waiting time is zero), and an event is scheduled for the completion
of the call.   Otherwise, the call must join the queue;
its patience time is generated by \texttt{generPatience} and memorized,
together with its arrival time, for future reference.

Upon completion of a call, the number of busy agents is decremented
and one must verify if a waiting call can now be answered.
The method \texttt{checkQueue} verifies that and if the answer is yes,
it removes the first call from the queue and activates its \texttt{endWait}
method.
This method first compares the call's waiting time with its patience time,
to see if this call is still waiting or has been lost (by abandonment).
If the call was lost, we consider its waiting time
as being equal to its patience time (i.e., the time that the caller
has really waited), for the statistics.
If the call is still there, the number of busy agents is incremented
and an event is scheduled for the call completion.

The results of this program, with the data in file
\texttt{CallCenter.dat}, are shown in Listing~\ref{res:CallCenter}.

\lstinputlisting[label=res:CallCenter,
caption={Simulation of a simplified call center},
float=tp]%
{CallCenter.res}

% \bigskip

This model is certainly an oversimplification of actual call centers.
It can be embellished and made more realistic by considering
different types of agents, different types of calls,
agents taking breaks for lunch, coffee, or going to the restroom,
agents making outbound calls to reach customers when the inbound
traffic is low (e.g., for marketing purpose or for returning calls),
and so on.   One could also model the revenue generated by calls and
the operating costs for running the center,
and use the simulation model to compare alternative operating strategies
in terms of the expected net revenue, for example.

\section {Process-Oriented Programs}
\label {sec:process}

The process-oriented programs discussed in this section are
based on the \texttt{simprocs} package and were initially designed
to run using \emph{Java green threads}, which are actually \emph{simulated}
threads and have been available only in JDK versions 1.3.1 or earlier,
unfortunately.
The green threads in these early versions of JDK were obtained
by running the program with the ``\texttt{java -classic}'' option.

With \emph{native threads}, these programs run more slowly
and may crash if too many threads are initiated.
This is a serious limitation of the package \texttt{simprocs},
due to the fact that Java threads are \emph{not} designed for simulation.

A second implementation of \texttt{simprocs}, available in DSOL
(see the class \texttt{DSOLProcess\-Simulator})
and provided to us by Peter Jacobs, does not use the Java threads.
It is based on a Java reflection mechanism
that interprets the code of processes at runtime and transforms everything into events.
A program using the process view and implemented with the DSOL interpreter
can be 500 to 1000 times slower than the corresponding event-based program.
On the other hand, it is a good and safe teaching tool for process-oriented programming,
and the number of processes is limited only by the available memory.
To use the DSOL system with a program that imports \texttt{simprocs},
it suffices to run the program with
the ``\texttt{java -Dssj.withDSOL ...}'' option.



%%%%%%%%%%%%%%%%%%%%%%%%%%%%%%%%%%%%%%%%%%%
\subsection {The single queue}

\lstinputlisting[%
label=lst:QueueProc,
caption={Process-oriented simulation of an $M/M/1$ queue},float=tp,
emph={simulateOneRun,actions,main}
]%
{QueueProc.java}

Typical simulation languages offer higher-level constructs than those used
in the program of Listing~\ref{lst:QueueEv}, and so does SSJ.
This is illustrated by our second implementation of the single-server
queue model, in Listing~\ref{lst:QueueProc}, based on a
paradigm called the {\em process-oriented\/} approach.
% initiated in the early sixties by the GPSS language (\cite {sSCH90a}).

In the event-oriented implementation, each customer was a {\em passive\/}
object, storing two real numbers, and performing no action by itself.
In the process-oriented implementation given in Listing~\ref{lst:QueueProc},
each customer (instance of the class \texttt{Customer}) is a
\emph{process} whose activities are described by its \texttt{actions} method.
This is implemented by associating a Java \texttt{Thread} to each
\texttt{SimProcess}.
The server is an object of the class \texttt{Resource}, created when
\texttt{QueueProc} is instantiated by \texttt{main}.
It is a {\em passive\/} object, in the sense that it executes no code.
Active resources, when needed, can be implemented as processes.

When it starts executing its actions, a customer first schedules
the arrival of the next customer, as in the event-oriented case.
(Behind the scenes, this effectively schedules an event, in the
event list, that will start a new customer instance.
The class \texttt{SimProcess} contains all scheduling facilities of {\tt
  Event}, which permits one
to schedule processes just like events.)
The customer then requests the server by invoking \texttt{server.request}.
If the server is free, the customer gets it and can continue its
execution immediately.  Otherwise, the customer is automatically
(behind the scenes) placed in the server's queue, is suspended,
and resumes its execution only when it obtains the server.
When its service can start, the customer invokes \texttt{delay} to
freeze itself for a duration equal to its service time, which is
again generated from the exponential distribution with mean $1/\mu$
using the random variate generator \texttt{genServ}.
After this delay has elapsed, the customer
releases the server and ends its life.
Invoking \texttt{delay(d)} can be interpreted as scheduling an event that
% (behind the scenes)
will resume the execution of the process in \texttt{d} units of time.
Note that several distinct customers can
co-exist in the simulation at any given point in time, and
be at different phases of their \texttt{actions} method.
However, only one process is executing at a time.

The constructor \texttt{QueueProc} initializes the simulation,
invokes \texttt{setStatCollecting (true)} to specify that detailed statistical
collection must be performed automatically for the resource \texttt{server},
schedules an event \texttt{EndOfSim} at the time horizon,
schedules the first customer's arrival, and starts the simulation.
The \texttt{EndOfSim} event stops the simulation.
The \texttt{main} program then regains control and prints a detailed
statistical report on the resource \texttt{server}.

It should be pointed out that in the \texttt{QueueProc} program,
the service time of a customer is generated
only when the customer starts its service, whereas for \texttt{QueueEv},
it was generated at customer's arrival.
For this particular model, it turns out that this makes no difference
in the results, because the customers are served in a FIFO order
and because one random number stream is dedicated to the generation
of service times.
However, this may have an impact in other situations.

The process-oriented program here is more compact and
more elegant than its event-oriented counterpart.
This tends to be often true: Process-oriented programming
frequently gives less cumbersome and better looking programs.
On the other hand, the process-oriented implementations also tend to
execute more slowly, because they involve more overhead.
For example, the process-driven single-server queue simulation
is two to three times slower than its event-driven counterpart.
In fact, process management is done via the event list:
processes are started, suspended, reactivated, and so on,
by hidden events.
If the execution speed of a simulation program is really important,
it may be better to stick to an event-oriented implementation.

%%%%%%%%%%%%%%%
\lstinputlisting[label=res:QueueProc,
caption={Results of the program \texttt{QueueProc}}]%
{QueueProc.res}

% \bigskip

\iffalse %%%%%%%%%%%%
\setbox5=\vbox{\hsize=5.8in
\begin{verbatim}
REPORT ON RESOURCE : Server
  From time : 0.0    to time : 1000.0
                 min          max      average    Std. Dev.  num.obs.
  Capacity        1            1          1
  Utilisation     0            1         0.999
  Queue Size      0           12         4.85
  Wait            0         113.721     49.554     22.336      97
  Service        0.065       41.021     10.378     10.377      96
  Sojourn       12.828      124.884     60.251     21.352      96

\end{verbatim}
}

\begin{figure}[htb]
\centerline {\boxit{\box5}}
\caption {Results of the program \texttt{QueueProc}.}
\label {fig:quresProc}
\end{figure}
\fi  %%%%%%%%%%%%

Listing~\ref{res:QueueProc} shows the output of the program \texttt{QueueProc}.
It contains more information than the output of \texttt{QueueEv}.
It gives statistics on the server utilization, queue size,
waiting times, service times, and sojourn times in the system.
(The sojourn time of a customer is its waiting time plus its service time.)

We see that by time $T =1000$, 1037 customers have completed their waiting
and all of them have completed their service.
The maximal queue size has been 10 and its average length
between time 0 and time 1000 was 0.513.
The waiting times were in the range from 0 to 6.262, with an average of 0.495,
while the service times were from 0.00065 to 3.437, with an average of 0.511
(recall that the theoretical mean service time is $1/\mu = 0.5$).
Clearly, the largest waiting time and largest service time
belong to different customers.
% The average waiting and service times do not sum up to
% the average sojourn time because there is one more observation for the
% waiting times.

The report also gives the empirical standard deviations of the waiting,
service, and sojourn times.  It is important to note that these
standard deviations should {\em not\/} be used to compute confidence
intervals for the expected average waiting times or sojourn times
in the standard way, because the observations here (e.g., the successive
waiting times) are strongly dependent, and also not identically distributed.
Appropriate techniques for computing confidence intervals in this type of
situation are described, e.g., in \cite{sFIS96a,sLAW00a}.


%%%%%%%%%%%%%%%%%%%%%%%%%%%%%%%%%%%%%%%%%%%%%%%%%%%%%%%%%%%%%%%%
\subsection {A job shop model}
\label {sec:jobshop}

This example is adapted from \cite[Section 2.6]{sLAW00a},
and from \cite{sLEC88a}.
A job shop contains $M$ groups of machines, the $m$th group having
$s_m$ identical machines, for $m=1,\dots,M$.
It is modeled as a network of queues:
each group has a single FIFO queue, with $s_m$ identical servers for the
$m$th group.  There are $N$ types of tasks arriving to the shop
at random.  Tasks of type $n$ arrive according to a Poisson process
with rate $\lambda_n$ per hour, for $n=1,\dots,N$.
Each type of task requires a fixed sequence of operations,
where each operation must be performed on a specific type of machine
and has a deterministic duration.
A task of type $n$ requires $p_n$ operations, to be performed on
machines $m_{n,1},m_{n,2},\dots,m_{n,p_n}$, in that order, and whose
respective durations are $d_{n,1},d_{n,2},\dots,d_{n,p_n}$, in hours.
A task can pass more than once on the same machine type, so $p_n$ may
exceed $M$.

We want to simulate the job shop for $T$ hours,
assuming that it is initially empty, and start collecting statistics
only after a warm-up period of $T_0$ hours.
We want to compute: (a) the average sojourn time in the shop for
each type of task and
(b) the average utilization rate, average length of the waiting
queue, and average waiting time, for each type of machine,
over the time interval $[T_0,T]$.
For the average sojourn times and waiting times, the counted
observations are the sojourn times and waits that {\em end\/} during
the time interval $[T_0,T]$.
Note that the only randomness in this model is in the task
arrival process.

The class \texttt{Jobshop} in Listing~\ref{lst:Jobshop} performs this
simulation.  Each group of machine is viewed as a resource, with
capacity $s_m$ for the group $m$.
The different {\em types\/} of task are objects of the class \texttt{TaskType}.
This class is used to store the parameters of the different types:
their arrival rate, the number of operations, the machine type
and duration for each operation, and a statistical collector for
their sojourn times in the shop.
(In the program, the machine types and task types are numbered
from 0 to $M-1$ and from 0 to $N-1$, respectively,
because array indices in Java start at 0.)

The tasks that circulate in the shop are objects of the class \texttt{Task}.
The \texttt{actions} method in class \texttt{Task} describes the behavior
of a task from its arrival until it exits the shop.
Each task, upon arrival, schedules the arrival of the next task of the
same type.  The task then runs through the list of its operations.
For each operation, it requests the appropriate type of machine,
keeps it for the duration of the operation, and releases it.
When the task terminates, it sends its sojourn time as a new observation
to the collector \texttt{statSojourn}.

Before starting the simulation, the method \texttt{simulateOneRun}
schedules an event for the end of the simulation and another one
for the end of the warm-up period.  The latter simply starts the
statistical collection.
It also schedules the arrival of a task of each type.
Each task will in turn schedules the arrival of the next task of
its own type.

With this implementation, the event list always
contain $N$ ``task arrival'' events, one for each type of task.
An alternative implementation would be that each task schedules
another task arrival in a number of hours that is an exponential r.v.\
with rate $\lambda$, where $\lambda = \lambda_0 + \cdots + \lambda_{N-1}$
is the global arrival rate, and then the type of each arriving task is
$n$ with probability $\lambda_n/\lambda$, independently of the others.
Initially, a {\em single\/} arrival would be scheduled by the class
\texttt{Jobshop}.
This approach is stochastically equivalent to the current implementation
(see, e.g., \cite{sBRA87a,pWOL89a}), but the event list contains only
one ``task arrival'' event at a time.
On the other hand, there is the additional work of generating the task
type on each arrival.

At the end of the simulation, the \texttt{main} program prints the
statistical reports.  Note that if one wanted to perform several independent
simulation runs with this program, the statistical collectors would
have to be reinitialized before each run, and additional collectors
would be required to collect the run averages and compute confidence
intervals.


%%%%%%%%%%%%%%%%
\lstinputlisting[label=lst:Jobshop,
caption={A job shop simulation},
emph={printReportOneRun,simulateOneRun,actions,readData,performTask,main}
]{Jobshop.java}


%%%%%%%%%%%%%%%%%%%%%%%%%%%%%%%%%%%%%%%%%%%%%%%%%%%%%%%%%%%%%%%%%%%%%
\subsection {A time-shared computer system}
\label {sec:timeshared}

This example is adapted from \cite{sLAW00a}, Section~2.4.
Consider a simplified time-shared computer system comprised of $T$
identical and independent terminals, all busy, using a common server
(e.g., for database requests, or central processing unit (CPU)
consumption, etc.).
Each terminal user sends a task to the server at some random time and
waits for the response.  After receiving the response, he thinks
for some random time before submitting a new task, and so on.

We assume that the thinking time is an exponential  random variable
with mean $\mu$, whereas the server's time needed for a request is a
Weibull random variable with parameters $\alpha$, $\lambda$ and $\delta$.
The tasks waiting for the server form a single queue with a
{\em round robin\/} service policy with {\em quantum size\/} $q$,
which operates as follows.
When a task obtains the server, if it can be completed in less than $q$
seconds, then it keeps the server until completion.
Otherwise, it gets the server for $q$ seconds and returns to the back
of the queue to be continued later.
In both cases, there is also $h$ additional seconds of {\em overhead\/}
for changing the task that has the server's attention.


The {\em response time\/} of a task is defined as the difference
between the time when the task ends (including the overhead $h$ at the
end) and the arrival time of the task to the server.
We are interested in the {\em mean response time}, in steady-state.
We will simulate the system until $N$ tasks have ended, with all
terminals initially in the ``thinking'' state.
To reduce the initial bias, we will start collecting statistics only
after $N_0$ tasks have ended (so the first $N_0$ response times are
not counted by our estimator, and we take the average response
time for the $N-N_0$ response times that remain).
This entire simulation is repeated $R$ times, independently,
so we can estimate the variance of our estimator.
% and compute a confidence interval for the true mean response time.


\unitlength=1in
\begin{picture}(6.0, 2.9)(0.0,0.3)
\thicklines\bf
\put(1.5,1.0){\circle{0.4}}
\put(1.5,1.5){\circle{0.4}}
\put(1.5,2.5){\circle{0.4}}
\put(1.5,1.0){\makebox(0,0){1}}
\put(1.5,1.5){\makebox(0,0){2}}
\put(1.5,2.5){\makebox(0,0){T}}
\multiput(1.5,1.85)(0.0,0.15){3}{\circle*{0.05}}
\put(3.9,1.7){\framebox(0.8,0.4){CPU}}
\multiput(3.5,1.75)(0.1,0.0){4}{\line(0,1){0.3}}
\put(1.1,1.2){\line(0,1){1.1}}
\put(1.9,1.2){\line(0,1){1.1}}
\put(0.7,2.2){\line(0,1){0.5}}
\put(0.9,2.0){\vector(1,0){0.2}}
\put(1.9,2.0){\vector(1,0){1.5}}
\put(3.0,1.8){\vector(1,0){0.4}}
\put(3.0,1.4){\line(1,0){2.1}}
\put(4.7,1.9){\line(1,0){0.4}}
\put(0.9,2.9){\line(1,0){4.2}}
\put(1.3,1.2){\oval(0.4,0.4)[bl]}
\put(1.3,1.7){\oval(0.4,0.4)[bl]}
\put(1.3,2.3){\oval(0.4,0.4)[tl]}
\put(1.7,2.3){\oval(0.4,0.4)[tr]}
\put(1.7,1.7){\oval(0.4,0.4)[br]}
\put(1.7,1.2){\oval(0.4,0.4)[br]}
\put(0.9,2.2){\oval(0.4,0.4)[bl]}
\put(0.9,2.7){\oval(0.4,0.4)[tl]}
\put(3.0,1.6){\oval(0.4,0.4)[l]}
\put(5.1,1.65){\oval(0.4,0.5)[r]}
\put(5.1,2.4){\oval(0.4,1.0)[r]}
\small\rm
\put(1.5,0.65){\makebox(0,0){Terminals}}
\put(4.1,1.25){\makebox(0,0){End of quantum}}
\put(3.2,2.75){\makebox(0,0){End of task}}
\put(3.65,2.2){\makebox(0,0){Waiting queue}}
\end{picture}


Suppose we want to compare the mean response times for two
different configurations of this system, where a configuration is
characterized by the vector of parameters
$(T, q, h, \mu, \alpha, \lambda, \delta)$.
We will make $R$ independent simulation runs (replications)
for each configuration.
To compare the two configurations, we want to use {\em common random
numbers}, i.e., the same streams of random numbers
across the two configurations.
We couple the simulation runs by pairs:
for run number $i$, let $R_{1i}$ and $R_{2i}$ be the mean response times
for configurations 1 and 2, and let
      $$D_i = R_{1i} - R_{2i}.$$
We use the same random numbers to obtain $R_{1i}$ and $R_{2i}$,
for each $i$.
The $D_i$ are nevertheless independent random variables (under the blunt
assumption that the random streams really produce independent uniform
random variables) and we can use them to compute a confidence interval
for the difference $d$ between the theoretical mean response times of the
two systems.
Using common random numbers across $R_{1i}$ and $R_{2i}$ should reduce
the variance of the $D_i$ and the size of the confidence interval.


\lstinputlisting[label=lst:TimeShared,
caption={Simulation of a time shared system},%
emph={simulateConfigs,actions,simulOneRun,main}
]%
{TimeShared.java}

% \bigskip

The program of Listing~\ref{lst:TimeShared} performs this simulation.
In the \texttt{simulateConfigs} method, we perform \texttt{numRuns}
pairs of simulation runs, one with quantum size \texttt{q1} and one
with quantum size {q2}.
Each random stream is reset to the beginning of its \emph{current} substream
after the first run, and to the beginning of its \emph{next} substream
after the second run, to make sure that for each pair of runs,
the generators start from exactly the same
seeds and generate exactly the same random numbers in the same order.
The variable \texttt{mean1} memorizes the values of $R_{1i}$ after the
first run, and the statistical probe \texttt{statDiff} collects the
differences $D_i$, in order to compute a confidence interval for $d$.

For each simulation run, the statistical probe \texttt{meanInRep}
is used to compute the average response time for the $N - N_0$ tasks
that terminate after the warm-up.
It is initialized before each run and updated with a new observation
at the $i$th task termination, for $i = N_0+1,\dots,N$.
At the beginning of a run, a \texttt{Terminal} process is activated for
each terminal.  When the $N$th task terminates, the corresponding
process invokes \texttt{Sim.stop} to stop the simulation and
to return the control to
the instruction that follows the call to \texttt{simulOneRun}.

% \bigskip

\iffalse  %%%%%%%%%
\setbox3=\vbox {\hsize = 6.0in
\begin{verbatim}
REPORT on Tally stat. collector ==> Differences on mean response times
   min          max        average      standard dev.   nb. obs.
   -0.134      0.369        0.168         0.175            10

        90.0% confidence interval for mean (student): ( 0.067, 0.269 )
\end{verbatim}
}

\begin{figure}[ht]
\centerline{\boxit{\box3}}
\caption {Difference in the mean response times for $q=0.1$ and $q=0.2$
    for the time shared system.}
\label {fig:timeshared-res}
\end{figure}
\fi   %%%%%%

For a concrete example, let $T=20$, $h=.001$, $\mu=5$ sec., $\alpha=1/2$,
$\lambda=1$ and $\delta=0$ for the two configurations.
With these parameters, the mean of the Weibull distribution is 2.
Take $q=0.1$ for configuration 1 and $q=0.2$ for configuration 2.
We also choose $N_0=100$, $N=1100$, and $R=10$ runs.
With these numbers, the program gives the results of
Listing~\ref{res:TimeShared}.
The confidence interval on the difference between the response time
with $q=0.1$ and that with $q=0.2$ contains only positive numbers.
We can therefore conclude that
the mean response time is significantly shorter (statistically)
with $q=0.2$ than with $q = 0.1$ (assuming that we can neglect the
bias due to the choice of the initial state).
To gain better confidence in this conclusion, we could repeat the
simulation with larger values of $N_0$ and $N$.

\lstinputlisting[label=res:TimeShared,
caption={Difference in mean response times
  for $q=0.1$ and $q=0.2$, for time shared system}]%
{TimeShared.res}

Of course, the model could be made more realistic by considering,
for example, different types of terminals, with different parameters,
a number of terminals that changes with time,
different classes of tasks with priorities, etc.
SSJ offers the tools to implement these generalizations easily.
% The program would be more elaborate but its structure would be similar.


%%%%%%%%%%%%%%%%%%%%%%%%%%%%%%%%%%%%%%%%%%%%%%%%%%%%%%%%%%%%%%%%%%%%%
\subsection {Guided visits}
\label {sec:visits}

This example is translated from \cite{sLEC88a}.
A touristic attraction offers guided visits, using three guides.
The site opens at 10:00 and closes at 16:00.
Visitors arrive in small groups (e.g., families) and the arrival
process of
those groups is assumed to be a Poisson process
with rate of 20 groups per hour, from 9:45 until 16:00.
The visitors arriving before 10:00 must wait for the opening.
After 16:00, the visits already under way can be completed,
but no new visit is undertaken, so that all the visitors still
waiting cannot visit the site and are lost.

The size of each arriving group of visitors is a discrete random
variable taking the value $i$ with probability $p_i$ given in the
following table:
\begin{center}
\begin{tabular}{r|rrrr}         \hline
   $i$ \ \  & 1  & 2  & 3  & 4\\  \hline
   $p_i$ \  & \ .2 & \ .6 & \ .1 & \ .1\\ \hline
\end{tabular}
\end{center}

Visits are normally made by groups of 8 to 15 visitors.
Each visit requires one guide and lasts 45 minutes.
People waiting for guides form a single queue.
When a guide becomes free, if there is less than 8 people
in the queue, the guide waits until the queue grows to at
least 8 people, otherwise she starts a new visit right away.
If the queue contains more than 15 people, the first 15 will
go on this visit.
At 16:00, if there is less than 8 people in the queue
and a guide is free, she starts a visit with the remaining
people.
At noon, each free guide takes 30 minutes for lunch.
The guides that are busy at that time will take 30 minutes
for lunch as soon as they complete their on-going visit.

Sometimes, an arriving group of visitors may decide to just
go away (balk) because the queue is too long.
%  These visitors are lost.
We assume that the probability of balking when the queue
size is $n$ is given by
$$
   R(n) = \cases {0          & for $n\le 10;$\cr
                  (n-10)/30  & for $10< n< 40$;\cr
                  1          & for $n\ge 40$.}
$$

The aim is to estimate the average number of visitors lost
per day, in the long run.
The visitors lost are those that balk or are still in the
queue after 16:00.

A simulation program for this model is given in
Listing~\ref{lst:Visits}.
Here, time is measured in hours, starting at midnight.
At time 9:45, for example, the simulation clock is at 9.75.
The (process) class \texttt{Guide} describes the daily
behavior of a guide (each guide is an instance of this
class), whereas \texttt{Arrival} generates the arrivals
according to a Poisson process, the group sizes,
and the balking decisions.
The event \texttt{closing} closes the site at 16:00.

The \texttt{Bin} mechanism \texttt{visitReady} is used to
synchronize the \texttt{Guide} processes.
The number of tokens in this bin is 1 if there are
enough visitors in the queue to start a visit (8 or more)
and is 0 otherwise.
When the queue size reaches 8 due to a new arrival,
the \texttt{Arrival} process puts a token into the bin.
This wakes up a guide if one is free.
A guide must take a token from the bin to start a new
visit.  If there are still 8 people or more in the queue
when she starts the visit, she puts the token back to
the bin immediately, to indicate that another visit is
ready to be undertaken by the next available guide.

%%%%%%%%%%%%%%%%
\lstinputlisting[label=lst:Visits,
caption={Simulation of guided visits},
emph={simulateRuns,actions,oneDay,balk,main},%
lineskip=-1pt]{Visits.java}

% \clearpage

The simulation results are in Listing~\ref{res:Visits}.


%%%%%%%%%%%%%%%%
\lstinputlisting[label=res:Visits,
caption={Simulation results for the guided visits model},
 lineskip=-1pt]{Visits.res}

%\bigskip

\iffalse %%%%%%%%%%%%
\setbox3=\vbox {\hsize = 6.0in
\begin{verbatim}
REPORT on Tally stat. collector ==> Nb. of visitors lost per day
   min        max        average      standard dev.   nb. obs.
    3         48          21.78         10.639          100

        90.0 confidence interval for mean ( 20.014, 23.546 )
\end{verbatim}
}

\begin{figure}[ht]
\centerline{\boxit{\box3}}
\caption {Simulation results for the guided visits model.}
\label {fig:visits-res}
\end{figure}
\fi  %%%%%%%%%%

Could we have used a \texttt{Condition} instead of a \texttt{Bin}
for synchronizing the \texttt{Guide} processes?
The problem would be that if several guides are waiting for a
condition indicating that the queue size has reached 8,
{\em all\/} these guides (not only the first one)
would resume their execution
simultaneously when the condition becomes true.


%%%%%%%%%%%%%%%%%%%%%%%%%%%%%%%%%%%%%%%%%%%%%%%%%%%%%%%%%%%%%%%
\subsection {Return to the simplified bank}
\label {sec:bank-proc}

Listing~\ref{lst:BankProc} gives a process-oriented simulation
program for the bank model of Section~\ref{sec:bank}.
Here, the customers are viewed as processes.
There are still events at the fixed times 9:45, 10:00, etc.,
and these events do the same thing,
but they are implicit in the process-oriented implementation.
The next planned arrival event \texttt{nextArriv} is replaced by the
next planned arriving customer \texttt{nextCust}.

Instead of using the default process simulator as in the previous
examples, here one creates a \class{ThreadProcessSimulator} object that
will manage the processes of this simulation.

The process-oriented version of the program is shorter,
because certain aspects (such as the details of an arrival
or departure event) are taken care of automatically by the
process/resource construct, and the events 9:45, 10:00, etc.,
are replaced by a single process.
At 10 o'clock, the \texttt{setCapacity} statement that fixes the
number of tellers also takes
care of starting service for the appropriate number of customers.
The two programs produce exactly the same results, given in
Listing~\ref{lst:BankEv}.
However, the process-oriented program take approximately 4 to 5 times
longer to run than its event-oriented counterpart.

% \clearpage

\lstinputlisting[label=lst:BankProc,
caption={Process-oriented simulation of the bank model},
emph={simulOneDay,simulateDays,actions,balk,main}]%
{BankProc.java}

%%%%%%%%%%%%%%%%%%%%%%%%%%%%%%%%%%%%%%%%%%%%%%%%%%%%%%%%%%%%%%%%


%\section {A single-server queue}
\label {sec:queue}

Our first example is a {\em single-server queue}, where customers
arrive randomly and are served one by one in their order of arrival,
i.e., {\em first in, first out\/} (FIFO).
We suppose that the times between successive arrivals are exponential
random variables with mean 10 minutes, that the service times are exponential
random variables with mean 9 minutes, and that all these random variables are 
mutually independent.
The customers arriving while the server is busy must join the queue.
The system initially starts empty.  We want to simulate the first 1000
minutes of operation and compute statistics such as the mean waiting time
per customer, the mean queue length, etc.

This simple model is well-known in queuing theory: It is called
an $M/M/1$ queue.  Simple formulas are available for this model to
compute the average waiting time per customer, average queue length,
etc., over an {\em infinite\/} time horizon \cite{pKLE75a}.
When the time horizon is finite, these expectations can also
be computed by numerical methods.  The fact that we use this model 
to give a first tasting of SSJ should not be interpreted to mean that 
simulation is necessarily the best tool for it.

We give three examples of simulation programs for the $M/M/1$ queue:
The first one is event-oriented, the second one is process-oriented,
and the third one uses a simple recurrence.
The first program is longer and more complicated than the other two;
it shows how things work at a lower level.


%%%%%%%%%%%%%
\subsection {Event-oriented program}

\lstinputlisting[label=lst:QueueEv,caption={Event-oriented simulation of an $M/M/1$ queue},lineskip=-1pt]{QueueEv.java}

Listing~\ref{lst:QueueEv} gives an event-oriented simulation program,
where a subclass of the class \texttt{Event} is defined for each type
of event that can occur in the simulation:
arrival of a customer (\texttt{Arrival}),
departure of a customer (\texttt{Departure}),
and end of the simulation (\texttt{EndOfSim}).
Each event {\em instance\/} is inserted into the {\em event list\/}
upon its creation, with a scheduled time of occurrence, and is
{\em executed\/} when the simulation clock reaches this time.
Executing an event means invoking its \texttt{actions} method.
Each event subclass must implement this method.
The simulation clock and the event list (i.e., the list of events 
scheduled to occur in the future) are maintained behind the
scenes by the class \texttt{Sim} of package \texttt{simevents}.

When \texttt{QueueEv} is instantiated  by the \texttt{main} method, 
the program creates
two streams of random numbers,
two random variate generators, two
lists, and two statistical probes (or collectors).
The random number streams can be
viewed as virtual random number generators that generate random
numbers in the interval $[0,1)$ according to the uniform probability
distribution.  These streams are attached to random variate generators
\texttt{genArr} and \texttt{genServ} which are used to generate the times
between successive arrivals and the service times, respectively.
We can use such an attached generator because the means (parameters)
do not change during simulation.
The lists \texttt{waitList} and \texttt{servList} contain the customers
waiting in the queue and the customer in service (if any), respectively.
Maintaining a list for the customer in service may seem exaggerated,
because this list never contains more than one object, but the current
design has the advantage of working with very little change if the
queuing model has more than one server, and in other more general 
situations.
Note that we could use the class \texttt{List} from package 
\texttt{simevents} instead of \texttt{java.util.LinkedList}.
However, with our current implementation,
the automatic statistical collection in that \texttt{List} 
class would not count the customers whose waiting time is zero, because
they are never placed in the list.
\begin{comment}
Here we use the class \texttt{List} from package \texttt{simevents}.
This class is equivalent to the standard class \texttt{java.util.LinkedList}, 
except that its implementation is more efficient than the current one
in JDK and it can also collect statistics automatically.
However, the automatic statistical collection on \texttt{waitList} 
would not count the customers whose waiting time is zero, because
they are never placed in this list, so we do not use this facility.
\end{comment}

The statistical probe \texttt{custWaits} collects statistics on the
customer's waiting times.  It is of the class \texttt{Tally}, which 
is appropriate when the statistical data of interest is  a sequence 
of observations $X_1, X_2, \dots$ of which we might want to compute 
the sample mean, variance, and so on.
A new observation is given to this probe by the \texttt{add} method
each time a customer starts its service.
Every \texttt{add} to a \texttt{Tally} probe brings a new observation $X_i$,
which corresponds here to a customer's waiting time in the queue.
The other statistical probe, \texttt{totWait}, is of the class
\texttt{Accumulate}, which means that it computes the integral
(and, eventually, the time-average) of a continuous-time
stochastic process with piecewise-constant trajectory.
Here, the stochastic process of interest is the length of the queue
as a function of time.  One must call \texttt{totWait.update} whenever 
there is a change in the queue size, to update the (hidden)
{\em accumulator\/} that keeps the current value of the integral
of the queue length.  This integral is equal, after each update,
to the total waiting time in the queue, for all the customers,
since the beginning of the simulation.  

Each customer is an object with two fields: \texttt{arrivTime}
memorizes this customer's arrival time to the system, and
\texttt{servTime} memorizes its service time.
This object is created, and its fields are initialized, 
when the customer arrives.

The method \texttt{Sim.init}, invoked in the constructor \texttt{QueueEv},
initializes the clock and the event list, 
whereas \texttt{Sim.start} actually starts the simulation 
by advancing the clock to the time of
the first event in the event list, removing this event
from the list, and executing it.  This is repeated until either
\texttt{Sim.stop} is called or the event list becomes empty.
\texttt{Sim.time} returns the current time on the simulation clock.
Here, two events are scheduled before starting the simulation:
the end of the simulation at time 1000, and the
arrival of the first customer at a random time that has the exponential
distribution with mean 10 (or \emph{rate} $\lambda=1/10$), 
generated by \texttt{genArr} using inversion and its attached random stream.
The method \texttt{genArr.nextDouble} returns this exponential random variate.

The method \texttt{actions} of the class \texttt{Arrival} describes what happens
when an arrival occurs.  
Arrivals are scheduled by a domino effect: 
the first action of each arrival event schedules the next event in
a random number of time units, generated from the exponential distribution
with mean 10.
Then, the newly arrived customer is created,
its arrival time is set to the current simulation time,
and its service time is generated from the exponential
distribution with mean 9, using the random variate generator \texttt{genServ}.
If the server is busy, this customer is inserted at the end of the
queue (the list \texttt{waitList}) and the statistical probe 
\texttt{totWait}, that keeps track of the size of the queue, is updated.
Otherwise, the customer is inserted in the server's list \texttt{servList},
its departure is scheduled to happen in a number of time units 
equal to its service time, and a new observation of 0.0 is given to the
statistical probe \texttt{custWaits} that collects the waiting times.

When a \texttt{Departure} event occurs, the customer in service is 
removed from the list (and disappears).
If the queue is not empty, the first customer is removed from
the queue (\texttt{waitList}) and inserted in the server's list,
and its departure is scheduled.
The waiting time of that customer (the current time minus its
arrival time) is given as a new observation to the probe
\texttt{custWaits}, and the probe \texttt{totWait} is also updated
with the new (reduced) size of the queue.

The event \texttt{EndOfSim} calculates and
prints statistical reports for
the two probes and stops the simulation.  
The program prints the results shown in Figure~\ref{fig:quresEv}.
When calling \texttt{report} on an \texttt{Accumulate} object, an implicit
update is done using the current simulation time and the last
value given to \texttt{update}.  In this example, this ensures
that the end time of the \texttt{Accumulate} will always be 1000,
otherwise it would be a random number.

\setbox4=\vbox{\hsize=5.8in
\begin{verbatim}
REPORT on Tally stat. collector ==> Waiting times
  min        max       average      standard dev   nb. obs.
   0       113.721      49.554         22.336         97

REPORT on Accumulate stat. collector ==> Size of queue

  From time  To time     Min        Max         Average 
    0          1000       0          12          4.85
\end{verbatim}
}

\begin{figure}[htb]
\centerline {\boxit{\box4}}
\caption {Results of the program \texttt{QueueEv}.}
\label {fig:quresEv}
\end{figure}

%%%%%%%%%%%%%
\clearpage
\subsection {Process-oriented program}

\lstinputlisting[label=lst:QueueProc,caption={Process-oriented simulation of an $M/M/1$ queue}]{QueueProc.java}

\clearpage 
Typical simulation languages offer higher-level constructs than those used
in the program of Listing~\ref{lst:QueueEv}, and so does SSJ.
This is illustrated by our second implementation of the single-server
queue model, in Listing~\ref{lst:QueueProc}, based on a
paradigm called the {\em process-oriented\/} approach.
% initiated in the early sixties by the GPSS language (\cite {sSCH90a}).

In the event-oriented implementation, each customer was a {\em passive\/}
object, storing two real numbers, and performing no action by itself.
In the process-oriented implementation given in Listing~\ref{lst:QueueProc},
each customer (instance of the class \texttt{Customer}) is a
\emph{process} whose activities are described by its \texttt{actions} method.
This is implemented by associating a Java \texttt{Thread} to each 
\texttt{SimProcess}.
The server is an object of the class \texttt{Resource}, created when
\texttt{QueueProc} is instantiated by \texttt{main}.
It is a {\em passive\/} object, in the sense that it executes no code. 
Active resources, when needed, can be implemented as processes.

When it starts executing its actions, a customer first schedules
the arrival of the next customer, as in the event-oriented case.
(Behind the scenes, this effectively schedules an event, in the
event list, that will start a new customer instance.
The class \texttt{SimProcess} contains all scheduling facilities of {\tt
  Event}, which permits one
to schedule processes just like events.)
The customer then requests the server by invoking \texttt{server.request}.
If the server is free, the customer gets it and can continue its
execution immediately.  Otherwise, the customer is automatically 
(behind the scenes) placed in the server's queue, is suspended, 
and resumes its execution only when it obtains the server.
When its service can start, the customer invokes \texttt{delay} to 
freeze itself for a duration equal to its service time, which is
again generated from the exponential distribution with mean 9
using the random variate generator \texttt{genServ}.  
After this delay has elapsed, the customer
releases the server and ends its life.
Invoking \texttt{delay(d)} can be interpreted as scheduling an event that
% (behind the scenes)
will resume the execution of the process in \texttt{d} units of time.
Note that several distinct customers can 
co-exist in the simulation at any given point in time, and
be at different phases of their \texttt{actions} method.
However, only one process is executing at a time.

The constructor \texttt{QueueProc} initializes the simulation,
invokes \texttt{setStatCollecting (true)} to specify that detailed statistical
collection must be performed automatically for the resource \texttt{server}, 
schedules an event \texttt{EndOfSim} at time 1000,
schedules the first customer's arrival, and starts the simulation.
The \texttt{EndOfSim} event prints a detailed statistical report on 
the resource \texttt{server}.

It should be pointed out that in the \texttt{QueueProc} program,
the service time of a customer is generated
only when the customer starts its service, whereas for \texttt{QueueEv},
it was generated at customer's arrival.
For this particular model, it turns out that this makes no difference 
in the results, because the customers are served in a FIFO order
and because one random number stream is dedicated to the generation
of service times. 
However, this may have an impact in other situations.

The process-oriented program here is more compact and
more elegant than its event-oriented counterpart.
This tends to be often true: Process-oriented programming 
frequently gives less cumbersome and better looking programs.
On the other hand, the process-oriented implementations also tend to
execute more slowly, because they involve more overhead.
For example, the process-driven single-server queue simulation
is two to three times slower than its event-driven counterpart.
In fact, process management is done via the event list:
processes are started, suspended, reactivated, and so on, 
by hidden events.
If the execution speed of a simulation program is really important,
it may be better to stick to an event-oriented implementation.

%%%%%%%%%%%%%%%
%\subsection {Simulation results for the single-server queue}

\setbox5=\vbox{\hsize=5.8in
\begin{verbatim}
REPORT ON RESOURCE : Server
  From time : 0.0    to time : 1000.0
                 min          max      average    Std. Dev.  num.obs.
  Capacity        1            1          1
  Utilisation     0            1         0.999
  Queue Size      0           12         4.85
  Wait            0         113.721     49.554     22.336      97
  Service        0.065       41.021     10.378     10.377      96
  Sojourn       12.828      124.884     60.251     21.352      96

\end{verbatim}
}

\begin{figure}[htb]
\centerline {\boxit{\box5}}
\caption {Results of the program \texttt{QueueProc}.}
\label {fig:quresProc}
\end{figure}


Figure~\ref{fig:quresProc} shows the output of the program \texttt{QueueProc}.
It contains more information than the output of \texttt{QueueEv}.
It gives statistics on the server utilization, queue size, 
waiting times, service times, and sojourn times in the system.
(The sojourn time of a customer is its waiting time plus its service time.)

We see that by time $T =1000$, 97 customers have completed their waiting
and 96 have completed their service.
The maximal length of the queue has been 12 and its average length
between time 0 and time 1000 was 4.85.
The waiting times were in the range from 0 to 113.721, with an average of 49.554,
while the service times were from 0.065 to 41.021, with an average of 10.378
(recall that the theoretical mean service time is 9.0).
Clearly, the largest waiting time and largest service time
belong to different customers.
The average waiting and service times do not sum up to
the average sojourn time because there is one more observation for the
waiting times.

The report also gives the empirical standard deviations of the waiting,
service, and sojourn times.  It is important to note that these
standard deviations should {\em not\/} be used to compute confidence 
intervals for the expected average waiting times or sojourn times
in the standard way, because the observations here (e.g., the successive
waiting times) are strongly dependent, and also not identically distributed.
Appropriate techniques for computing confidence intervals in this type of
situation are described, e.g., in \cite{sFIS96a,sLAW00a}.


%%%%%%%%%%%%%
\subsection {A simpler program based on Lindley's recurrence}

The aim of the previous programs was
to illustrate the notions of events and processes by a simple example.
However, for a single-server queue, 
if $W_i$ and $S_i$ are the waiting time and service time of the
$i$th customer, and $A_i$ is the time between the arrivals of
the $i$th and $(i+1)$th customers, we have $W_1=0$ and the $W_i$ 
follow the recurrence
\eq
  W_{i+1} = \max(0,\; W_i + S_i - A_i),               \label {eq:lindley}
\endeq
known as Lindley's equation \cite {pKLE75a}.

The program of Listing~\ref{lst:QueueLindley} exploits (\ref{eq:lindley})
to compute the average waiting time of the first $100$ customers.
This is different than for the previous programs, because we now fix
the total number of customers instead of fixing the time horizon.
For a change, the random variates are generated by static methods
instead of via a \texttt{RandomVariateGen} object.
The instruction ``\texttt{Wi += \dots}'' could also be replaced by the
one that is commented out, and which directly implements inversion 
of the exponential distribution.
This illustrates various ways of doing the same thing.


\bigskip
\lstinputlisting[label=lst:QueueLindley,caption={A simulation based on Lindley's recurrence}]{QueueLindley.java}

%%%%%%%%%%%%%%%%%%%%%%%%%%%%%%%%%%
\subsection{Using the observer design pattern}

Listing~\ref{lst:QueueObs} adds a few ingredients to the program
\texttt{QueueLindley}, in order to illustrate the \emph{observer}
design pattern implemented in package \texttt{stat}.  
This mechanism permits one to separate data generation from data 
processing.  It can be very helpful in large simulation programs or 
libraries, where different objects may need to process the same data 
in different ways.  These objects may have the task of storing observations
or displaying statistics in different formats, for example, and they are
not necessarily fixed in advance.

The \emph{observer} pattern, supported by the \texttt{Observable}
class and \texttt{Observer} interface in Java,
offers the appropriate flexibility for that kind of situation.
An \texttt{Observable} object acts in a sense like a \emph{broadcasting}
\emph{distribution agency} that maintains a list of registered 
\texttt{Observer} objects, and send information to all its registered
observers whenever appropriate.

\texttt{StatProbe} in package \texttt{stat}, as well as its subclasses
\texttt{Tally} and \texttt{Accumulate}, extend the \texttt{Observable} class.  
Whenever they receive a new statistical observation, e.g.,  via
\texttt{Tally.add} or \texttt{Accumulate.update}, they send the new value
to all registered observers.
To register as an observer, an object must implement the interface
\texttt{Observer}. 
This implies that it must provide an implementation of the method
\texttt{update}, whose purpose is to recover the information
that the object has registered for.

In the example, the statistical collector \texttt{waitingTimes} 
is the \texttt{Observable} object that transmits to all its registered 
listeners each new statistical observation that it receives via
its \texttt{add} method.
More specifically, each call to \texttt{waitingTimes.add(x)} generates 
in the background a call to \texttt{o.update(waitingTimes, dx)},
where \texttt{dx} is a \texttt{Double} object that contains the value of \texttt{x},
for all registered observers \texttt{o}.

\begin{comment}
The method \texttt{notifyObs} is used to
turn the tally into such an agency.  In fact, the collector is both a
tally and a distribution agency, but its tally functionality can be
disabled using the \texttt{stopCollectStat} method.  This can be useful when
the registered observers already perform statistical collection.
\end{comment}

The single observer that registers to receive observations from 
\texttt{waitingTimes} in the example is an anonymous object of class 
\texttt{ObservationTrace}.  The task of this observer is to
print the waiting times $W_5$, $W_{10}$, $W_{15}$, \dots
The statistical collector \texttt{waitingTimes} himself also stores
appropriate information to be able to provide a statistical report
when asked.
The \texttt{ObservationTrace} object gets informed of any new observation 
$W_i$ via its \texttt{update} method.
The \texttt{Observer} interfaces specifies that \texttt{update} must have two
formal parameters, of classes \texttt{Observable} and \texttt{Object}, respectively. 
A \texttt{StatProbe} always passes \texttt{Double} wrapper object as the 
second parameter of \texttt{update}, so this second parameter is normally
type-casted to a \texttt{Double} inside the \texttt{update} method,
before the observation can be extracted.
In the case where the observer registers to several \texttt{Observable}
objects, the first parameter of \texttt{update} tells him which one
is sending the information and can behave accordingly.

\bigskip
\lstinputlisting[label=lst:QueueObs,caption={A simulation of Lindley's
  recurrence using observers}]{QueueObs.java}


\begin{comment}
%%%%%%%%%%%%%%%%%%%%%%%%


Classes implementing data generation facilities use
the statistical probes as usual but turn on the observation
notification.  In more complex situations, the data generating class
could itself implement \texttt{Observable} to broadcast some custom
observation types, such as calls in a call center simulation program. 
It is a good practice to offer a public access to the statistical
collectors to allow observer registration from other classes.
Here, the only collector used is the \texttt{waitingTimes}.
In the case of a program that also keeps track of the waiting queue
(as in \texttt{QueueEv}), we could also use
a second collector for the queue size.
The \texttt{QueueObs} is the only data generator of the program.
In more complex examples, we could have several data generators,
each with their own properties.

The data processing portion of the simulation program comprises classes
implementing the \texttt{Observer} interface.  These
observers can put data in collectors, in files for later plotting, etc.
In our example, we register one listener for the waiting times.
One can also register observers with the \texttt{Accumulate} class.
However, note that the \texttt{QueueObs} simulation
logic does not know anything about the observation trace.
The \texttt{ObservationTrace}, is a user-defined
class that prints waiting times of some customers.  The output
is only for demonstration purposes, it is not suited for passing
to a plotting program, but it could easily be modified to do so.
Such a listener could also collect waiting times greater or equal
to a certain threshold, use other statistical collection facilities, etc.
The user can implement data processing tools that
the simulation library developers did not think about.

\end{comment}


%%%%%%%%%%%%%%
\section {Continuous simulation: A prey-predator system}
\label {sec:preypred}

We consider a classical prey-predator system, where the preys
are food for the predators (see, e.g., \cite{sLAW00a}, page 87).
Let $x(t)$ and $z(t)$ be the numbers of preys and predators
at time $t$, respectively.
These numbers are integers, but as an approximation,
we shall assume that they are real-valued variables evolving
according to the differential equations
\begin{eqnarray*}
  x'(t) &= &\ r x(t) - c x(t) z(t)\\
  z'(t) &= & -s z(t) + d x(t) z(t)
\end{eqnarray*}
with initial values $x(0)=x_0>0$ et $z(0)=z_0>0$.
This system is actually a Lotka-Volterra system of differential
equations, and has a known analytical solution.
Here, in the program of Listing~\ref{lst:PreyPred},
we simply simulate its evolution, to illustrate the continuous
simulation facilities of SSJ.

%\bigskip
\lstinputlisting[label=lst:PreyPred,caption={Simulation of the prey-predator system},lineskip=-1pt,%
emph={main}]{PreyPred.java}


This program prints the triples $(t, x(t), z(t))$ at values of
$t$ that are multiples of \texttt{h}, one triple per line.
This is done by an event of class \texttt{PrintPoint}, which is
rescheduled at every \texttt{h} units of time.
This output can be redirected to a file for later use,
for example to plot a graph of the trajectory.
The continuous variables \texttt{x} and \texttt{z} are instances of the
classes \texttt{Preys} and \texttt{Preds}, whose method \texttt{derivative}
give their derivative $x'(t)$ and $z'(t)$, respectively.
The differential equations are integrated by a Runge-Kutta method
of order 4.

%\section {A job shop model}
\label {sec:jobshop}

This example is adapted from \cite[Section 2.6]{sLAW00a},
and from \cite{sLEC88a}.
A job shop contains $M$ groups of machines, the $m$th group having 
$s_m$ identical machines, for $m=1,\dots,M$.  
It is modeled as a network of queues:
each group has a single FIFO queue, with $s_m$ identical servers for the
$m$th group.  There are $N$ types of tasks arriving to the shop
at random.  Tasks of type $n$ arrive according to a Poisson process
with rate $\lambda_n$ per hour, for $n=1,\dots,N$.
Each type of task requires a fixed sequence of operations,
where each operation must be performed on a specific type of machine
and has a deterministic duration.
A task of type $n$ requires $p_n$ operations, to be performed on
machines $m_{n,1},m_{n,2},\dots,m_{n,p_n}$, in that order, and whose
respective durations are $d_{n,1},d_{n,2},\dots,d_{n,p_n}$, in hours.
A task can pass more than once on the same machine type, so $p_n$ may
exceed $M$.

We want to simulate the job shop for $T$ hours,
assuming that it is initially empty, and start collecting statistics
only after a warm-up period of $T_0$ hours.
We want to compute: (a) the average sojourn time in the shop for
each type of task and 
(b) the average utilization rate, average length of the waiting
queue, and average waiting time, for each type of machine,
over the time interval $[T_0,T]$.
For the average sojourn times and waiting times, the counted
observations are the sojourn times and waits that {\em end\/} during
the time interval $[T_0,T]$.
Note that the only randomness in this model is in the task
arrival process.

The program \texttt{Jobshop} in Listing~\ref{lst:Jobshop} performs this
simulation.  Each group of machine is viewed as a resource, with
capacity $s_m$ for the group $m$.
The different {\em types\/} of task are objects of the class \texttt{TaskType}.
This class is used to store the parameters of the different types:
their arrival rate, the number of operations, the machine type 
and duration for each operation, and a statistical collector for
their sojourn times in the shop.
(In the program, the machine types and task types are numbered
from 0 to $M-1$ and from 0 to $N-1$, respectively,
because array indices in Java start at 0.)

The tasks that circulate in the shop are objects of the class \texttt{Task}.
The \texttt{actions} method in class \texttt{Task} describes the behavior
of a task from its arrival until it exits the shop.
Each task, upon arrival, schedules the arrival of the next task of the
same type.  The task then runs through the list of its operations.
For each operation, it requests the appropriate type of machine,
keeps it for the duration of the operation, and releases it.
When the task terminates, it sends its sojourn time as a new observation
to the collector \texttt{statSojourn}.

Before starting the simulation, the class \texttt{Jobshop} first
schedules two events: One for the end of the simulation and one
for the end of the warm-up period.  The latter simply reinitializes
the statistical collectors.

With this implementation, the event list always
contain $N$ ``task arrival'' events, one for each type of task.
An alternative implementation would be that each task schedules
another task arrival in a number of hours that is an exponential r.v.\
with rate $\lambda$, where $\lambda = \lambda_0 + \cdots + \lambda_{N-1}$
is the global arrival rate, and then the type of each arriving task is 
$n$ with probability $\lambda_n/\lambda$, independently of the others.
Initially, a {\em single\/} arrival would be scheduled by the class
\texttt{Jobshop}.
This approach is stochastically equivalent to the current implementation
(see, e.g., \cite{sBRA87a,pWOL89a}), but the event list contains only
one ``task arrival'' event at a time.
On the other hand, there is the additional work of generating the task
type on each arrival.

%%%%%%%%%%%%%%%%%%%%%%%%%%%%%%%%%%%%%%%%%%%%%%%%%%%%%
\lstinputlisting[label=lst:Jobshop,caption={A job shop simulation}]{Jobshop.java}

%\section {A time-shared computer system}
\label {sec:timeshared}

This example is adapted from \cite{sLAW00a}, Section~2.4.
Consider a simplified time-shared computer system comprised of $T$
identical and independent terminals, all busy, using a common server
(e.g., for database requests, or central processing unit (CPU) 
consumption, etc.).
Each terminal user sends a task to the server at some random time and
waits for the response.  After receiving the response, he thinks
for some random time before submitting a new task, and so on.

We assume that the thinking time is an exponential  random variable 
with mean $\mu$, whereas the server's time needed for a request is a
Weibull random variable with parameters $\alpha$, $\lambda$ and $\delta$.
The tasks waiting for the server form a single queue with a 
{\em round robin\/} service policy with {\em quantum size\/} $q$,
which operates as follows.
When a task obtains the server, if it can be completed in less than $q$ 
seconds, then it keeps the server until completion.
Otherwise, it gets the server for $q$ seconds and returns to the back
of the queue to be continued later.
In both cases, there is also $h$ additional seconds of {\em overhead\/}
for changing the task that has the server's attention.


\unitlength=1in
\begin{picture}(6.0, 2.9)(0.0,0.3)
\thicklines\bf
\put(1.5,1.0){\circle{0.4}}
\put(1.5,1.5){\circle{0.4}}
\put(1.5,2.5){\circle{0.4}}
\put(1.5,1.0){\makebox(0,0){1}}
\put(1.5,1.5){\makebox(0,0){2}}
\put(1.5,2.5){\makebox(0,0){T}}
\multiput(1.5,1.85)(0.0,0.15){3}{\circle*{0.05}}
\put(3.9,1.7){\framebox(0.8,0.4){CPU}}
\multiput(3.5,1.75)(0.1,0.0){4}{\line(0,1){0.3}}
\put(1.1,1.2){\line(0,1){1.1}}
\put(1.9,1.2){\line(0,1){1.1}}
\put(0.7,2.2){\line(0,1){0.5}}
\put(0.9,2.0){\vector(1,0){0.2}}
\put(1.9,2.0){\vector(1,0){1.5}}
\put(3.0,1.8){\vector(1,0){0.4}}
\put(3.0,1.4){\line(1,0){2.1}}
\put(4.7,1.9){\line(1,0){0.4}}
\put(0.9,2.9){\line(1,0){4.2}}
\put(1.3,1.2){\oval(0.4,0.4)[bl]}
\put(1.3,1.7){\oval(0.4,0.4)[bl]}
\put(1.3,2.3){\oval(0.4,0.4)[tl]}
\put(1.7,2.3){\oval(0.4,0.4)[tr]}
\put(1.7,1.7){\oval(0.4,0.4)[br]}
\put(1.7,1.2){\oval(0.4,0.4)[br]}
\put(0.9,2.2){\oval(0.4,0.4)[bl]}
\put(0.9,2.7){\oval(0.4,0.4)[tl]}
\put(3.0,1.6){\oval(0.4,0.4)[l]}
\put(5.1,1.65){\oval(0.4,0.5)[r]}
\put(5.1,2.4){\oval(0.4,1.0)[r]}
\small\rm
\put(1.5,0.65){\makebox(0,0){Terminals}}
\put(4.1,1.25){\makebox(0,0){End of quantum}}
\put(3.2,2.75){\makebox(0,0){End of task}}
\put(3.65,2.2){\makebox(0,0){Waiting queue}}
\end{picture}


The {\em response time\/} of a task is defined as the difference
between the time when the task ends (including the overhead $h$ at the
end) and the arrival time of the task to the server.
We are interested in the {\em mean response time}, in steady-state.
We will simulate the system until $N$ tasks have ended, with all
terminals initially in the ``thinking'' state.
To reduce the initial bias, we will start collecting statistics only
after $N_0$ tasks have ended (so the first $N_0$ response times are
not counted by our estimator, and we take the average response
time for the $N-N_0$ response times that remain).
This entire simulation is repeated $R$ times, independently,
so we can estimate the variance of our estimator.
% and compute a confidence interval for the true mean response time.


Suppose we want to compare the mean response times for two 
different configurations of this system, where a configuration is
characterized by the vector of parameters $(T, q, h, \mu, \alpha, \lambda, \delta)$.
We will make $R$ independent simulation runs (replications)
for each configuration.
To compare the two configurations, we want to use {\em common random
numbers}, i.e., the same streams of random numbers 
across the two configurations.
We couple the simulation runs by pairs: 
for run number $i$, let $R_{1i}$ and $R_{2i}$ be the mean response times 
for configurations 1 and 2, and let
      $$D_i = R_{1i} - R_{2i}.$$
We use the same random numbers to obtain $R_{1i}$ and $R_{2i}$,
for each $i$.
The $D_i$ are nevertheless independent random variables (under the blunt
assumption that the random streams really produce independent uniform
random variables) and we can use them to compute a confidence interval
for the difference $d$ between the theoretical mean response times of the
two systems.
Using common random numbers across $R_{1i}$ and $R_{2i}$ should reduce
the variance of the $D_i$ and the size of the confidence interval.

\bigskip
\lstinputlisting[label=lst:TimeShared,caption={Simulation of a time shared system}]{TimeShared.java}

The program of Listing~\ref{lst:TimeShared} performs this simulation.
In \texttt{TimeShared}, the variable \texttt{conf} indicates the current
configuration number.  For each configuration, 
% we read the configuration data and 
we make \texttt{nbRep} simulation runs.
The array \texttt{meanConf1} memorizes the values of $R_{1i}$, and the
statistical probe \texttt{statDiff} collect the differences $D_i$, in
order to compute a confidence interval for $d$.
After all the runs for the first configuration have been completed,
the random number streams are reset to their initial seeds,
so that the two configurations get the same random numbers.
The random streams are also reset to the beginning of their next
substream after each run, to make sure that for the corresponding runs
for the two configurations, the generators start from exactly the same
seeds and generate the same numbers.

For each simulation run, the statistical probe \texttt{meanInRep}
is used to compute the average response time for the $N - N_0$ tasks
that terminate after the warm-up.
It is initialized before each run and updated with a new observation
at the $i$th task termination, for $i = N_0+1,\dots,N$.
At the beginning of a run, a \texttt{Terminal} process is activated for
each terminal.  When the $N$th task terminates, the corresponding
process invokes \texttt{Sim.stop} to stop the simulation and
to return the control to
the instruction that follows the call to \texttt{simulOneRun} in
\texttt{TimeShared}.

\setbox3=\vbox {\hsize = 6.0in
\begin{verbatim}
REPORT on Tally stat. collector ==> Differences on mean response times
   min          max        average      standard dev.   nb. obs.
   -0.134      0.369        0.168         0.175            10
 
        90.0 confidence interval for mean ( 0.067, 0.269 )
\end{verbatim}
}

\begin{figure}[ht]
\centerline{\boxit{\box3}}
\caption {Difference in the mean response times for $q=0.1$ and $q=0.2$ 
    for the time shared system.}
\label {fig:timeshared-res}
\end{figure}

For a concrete example, let $T=20$, $h=.001$, $\mu=5$ sec., $\alpha=1/2$,
$\lambda=1$ and $\delta=0$ for the two configurations.
With these parameters, the mean of the Weibull distribution is 2.
Take $q=0.1$ for configuration 1 and $q=0.2$ for configuration 2.
We also choose $N_0=100$, $N=1100$, and $R=10$ runs.
With these numbers in the data file, the program gives the results of
Figure~\ref{fig:timeshared-res}.
The confidence interval on the difference between the response time
with $q=0.1$ and that with $q=0.2$ contains only positive numbers.
We can therefore conclude that 
the mean response time is significantly shorter (statistically) 
with $q=0.2$ than with $q = 0.1$ (assuming that we can neglect the
bias due to the choice of the initial state).
To gain better confidence in this conclusion, we could repeat the 
simulation with larger values of $N_0$ and $N$.

Of course, the model could be made more realistic by considering,
for example, different types of terminals, with different parameters,
a number of terminals that changes with time,
different classes of tasks with priorities, etc.
SSJ offers the tools to implement these generalizations easily.
The program would be more elaborate but its structure would be similar.

%\section {A simplified bank}
\label {sec:bank}

This is Example~1.4.1 of \cite{sBRA87a}, page~14.
% Bratley, Fox, and Schrage (1987).
A bank has a random number of tellers every morning.
On any given day, the bank has $t$ tellers with probability $q_t$,
where $q_3 = 0.80$, $q_2 = 0.15$, and $q_1 = 0.05$.
All the tellers are assumed to be identical from the modeling viewpoint.


%%%%%%%%%%%%%
\setbox0=\vbox{\hsize=6.0in
%%  {Arrival rate of customers to the bank.}
\beginpicture
\setcoordinatesystem units <1.8cm,2cm>
%%  72.27pt = 1in
\setplotarea x from 0 to 6.5, y from 0 to 1
\axis left
  label {\lines {arrival \ \cr rate }}
  ticks length <2pt> withvalues 0.5 1 / at 0.5 1 / /
\axis bottom
  label {\hbox to 5.4in {\hfill time}}
  ticks length <2pt> withvalues 9:00 9:45 11:00 14:00 15:00 /
  at 0.0 0.75 2.0 5.0 6.0 / /
\shaderectangleson
\putrectangle corners at 0.75 0.0 and 2.0 0.5
\putrectangle corners at 2.0 0.0 and 5.0 1.0
\putrectangle corners at 5.0 0.0 and 6.0 0.5
\endpicture
}

\begin{figure}[htb]
\box0
\caption {Arrival rate of customers to the bank.}
\label {fig:blambda}
\end{figure}

\bigskip
\lstinputlisting[label=lst:BankEv,caption={Event-oriented simulation of the bank model}]{BankEv.java}
%\lstlabel{lst:BankEv}

\clearpage
\lstinputlisting[label=st:BankProc,caption={Process-oriented simulation of the bank model}]{BankProc.java}
%\lstlabel{lst:BankProc}

The bank opens at 10:00 and closes at 15:00 (i.e., {\sc 3 p.m.}).
The customers arrive randomly according to a Poisson process
with piecewise constant rate $\lambda(t)$, $t\ge 0$.
The arrival rate $\lambda(t)$ (see Fig.{}~\ref{fig:blambda})
is 0.5 customer per minute from
9:45 until 11:00 and from 14:00 until 15:00, and
one customer per minute from 11:00 until 14:00.
The customers who arrive before 9:45 join a FIFO queue 
and wait for the bank to open.
At 15:00, the door is closed, but all the customers already in will be served.
Service starts at 10:00.

Customers form a FIFO queue for the tellers, with balking.
An arriving customer will balk (walk out) with probability $p_k$ if there
are $k$ customers ahead of him in the queue (not counting the people
receiving service), where
 $$ p_k = \cases { 0       & if $k\le 5$;\cr
                   (n-5)/5 & if $5 < k < 10$;\cr
                   1       & if $k\ge 10$.\cr }$$
The customer service times are independent Erlang random
variables: Each service time is the sum of
two independent exponential random variables with mean one.

We want to estimate the expected number of customers served in a
day, and the expected average wait for the customers
served on a day.
% We could also be interested in the effect of changing the number of tellers,
% changing their speed, and so on.

Listings~\ref{lst:BankEv} and~\ref{lst:BankProc} give simulation
programs for this bank model.
The first program uses only events and the second one 
views the customers as processes.
Both programs have events at the fixed times 9:45, 10:00, etc.;
these events are implicit in the process-oriented implementation.
At 9:45, the counters are initialized and the arrival process
is started.  The time until the first arrival,
or the time between one arrival and the next one, is (tentatively)
an exponential with a mean of 2 minutes.
However, as soon as an arrival turns out to be past 11:00,
its time must be readjusted to take into account the increase of the 
arrival rate at 11:00.
The event 11:00 takes care of this readjustment,
and the event at 14:00 makes a similar readjustment
when the arrival rate decreases.
We give the specific name \texttt{nextArriv} to the next planned
arrival event in the event-oriented case, and \texttt{nextCust} to the 
next customer scheduled to arrive in the process-oriented case,
in order to be able to reschedule
that particular event (or process) to a different time.
Note that in the event-oriented case, a {\em single\/} arrival event is
created at the beginning and this same event is scheduled over and
over again.  This can be done because there is never more than one
arrival event in the event list.
We cannot do this for the customer processes in the
process-oriented case, however, because several processes can be alive
simultaneously.

At the bank opening at 10:00, an event generates the number 
of tellers and starts the service for the corresponding customers.
The event at 15:00 cancels the next arrival.

Upon arrival, a customer checks if a teller is free.
If so, one teller becomes busy and the customer generates its
service time and schedules his departure, otherwise the
customer either balks or joins the queue.
The balking decision is computed by the method \texttt{balk},
using the random number stream \texttt{streamBalk}.
The arrival event also generates the next scheduled arrival.
Upon departure, the customer frees the teller, and the first
customer in the queue, if any, can start its service.

The constructor (\texttt{BankEv} or \texttt{BankProc}) simulates the bank 
for 100 days and prints a statistical report.
It chooses as \texttt{genServ} the \texttt{ErlangConvolutionGen} generator so that
the Erlang variates are generated by adding two exponentials instead
of using inversion.
If $X_i$ is the number of customers served on day $i$ and 
$Q_i$ the total waiting time on day $i$, the program estimates
$E[X_i]$ and $E[Q_i]$ by their sample averages $\bar X_n$ and
$\bar Q_n$ with $n=100$.
For each simulation run (each day), \texttt{simulOneDay} initializes 
the clock, event list, and statistical probe for the waiting times,
schedules the deterministic events, and runs the simulation.
After 15:00, no more arrival occurs and the event list becomes
empty when the last customer departs.
At that point, the program returns to right after the \texttt{Sim.start()}
statement and updates the statistical counters for the number of
customers served during the day and their total waiting time.

The process-oriented version of the program is shorter,
because certain aspects (such as the details of an arrival
or departure event) are taken care of automatically by the
process/resource construct, and the events 9:45, 10:00, etc.,
are replaced by a single process.
At 10 o'clock, the \texttt{setCapacity} statement that fixes the
number of tellers also takes
care of starting service for the appropriate number of customers.

These two programs give the same results, shown in 
Figure~\ref{fig:bank-res}.
However, the process-oriented program take approximately 4 to 5 times
longer to run than its event-oriented counterpart.

\setbox5=\vbox {\hsize = 6.0in
\begin{verbatim}
REPORT on Tally stat. collector ==> Nb. served per day
   min        max        average      standard dev.   nb. obs.
   152        285        240.59         19.21           100
 
REPORT on Tally stat. collector ==> Average wait per day (hours)
   min        max        average      standard dev.   nb. obs.
  0.816      35.613       4.793         5.186           100
\end{verbatim}
}

\begin{figure}[h]
\centerline{\boxit{\box5}}
\caption {Results of the \texttt{BankEv} and \texttt{BankProc} programs.}
\label {fig:bank-res}
\end{figure}

%\section {Guided visits}
\label {sec:visits}

This example is translated from \cite{sLEC88a}.
A touristic attraction offers guided visits, using three guides.
The site opens at 10:00 and closes at 16:00.
Visitors arrive in small groups (e.g., families) and the arrival 
process of
those groups is assumed to be a Poisson process
with rate of 20 groups per hour, from 9:45 until 16:00.
The visitors arriving before 10:00 must wait for the opening.
After 16:00, the visits already under way can be completed,
but no new visit is undertaken, so that all the visitors still
waiting cannot visit the site and are lost.

The size of each arriving group of visitors is a discrete random
variable taking the value $i$ with probability $p_i$ given in the
following table:
\begin{center}
\begin{tabular}{r|rrrr}         \hline
   $i$ \ \  & 1  & 2  & 3  & 4\\  \hline
   $p_i$ \  & \ .2 & \ .6 & \ .1 & \ .1\\ \hline
\end{tabular}
\end{center}

Visits are normally made by groups of 8 to 15 visitors.
Each visit requires one guide and lasts 45 minutes.
People waiting for guides form a single queue.
When a guide becomes free, if there is less than 8 people
in the queue, the guide waits until the queue grows to at 
least 8 people, otherwise she starts a new visit right away.
If the queue contains more than 15 people, the first 15 will
go on this visit.
At 16:00, if there is less than 8 people in the queue 
and a guide is free, she starts a visit with the remaining 
people.
At noon, each free guide takes 30 minutes for lunch.
The guides that are busy at that time will take 30 minutes
for lunch as soon as they complete their on-going visit.

Sometimes, an arriving group of visitors may decide to just
go away (balk) because the queue is too long.
%  These visitors are lost.
We assume that the probability of balking when the queue 
size is $n$ is given by
$$
   R(n) = \cases {0          & for $n\le 10;$\cr
                  (n-10)/30  & for $10< n< 40$;\cr
                  1          & for $n\ge 40$.}
$$

The aim is to estimate the average number of visitors lost
per day, in the long run.
The visitors lost are those that balk or are still in the
queue after 16:00.

A simulation program for this model is given in 
Listing~\ref{lst:Visits}.
Here, time is measured in hours, starting at midnight.
At time 9:45, for example, the simulation clock is at 9.75.
The (process) class \texttt{Guide} describes the daily 
behavior of a guide (each guide is an instance of this 
class), whereas \texttt{Arrival} generates the arrivals
according to a Poisson process, the group sizes,
and the balking decisions.
The event \texttt{closing} closes the site at 16:00.

The \texttt{Bin} mechanism \texttt{visitReady} is used to
synchronize the \texttt{Guide} processes.
The number of tokens in this bin is 1 if there is 
enough visitors in the queue to start a visit (8 or more)
and is 0 otherwise.
When the queue size reaches 8 due to a new arrival,
the \texttt{Arrival} process puts a token into the bin.
This wakes up a guide if one is free.
A guide must take a token from the bin to start a new
visit.  If there is still 8 people or more in the queue
when she starts the visit, she puts the token back to 
the bin immediately, to indicate that another visit is
ready to be undertaken by the next available guide.

\lstinputlisting[label=lst:Visits,caption={Simulation of guided visits}, lineskip=-1pt]{Visits.java}

\setbox3=\vbox {\hsize = 6.0in
\begin{verbatim}
REPORT on Tally stat. collector ==> Nb. of visitors lost per day
   min        max        average      standard dev.   nb. obs.
    3         48          21.78         10.639          100
 
        90.0% confidence interval for mean (student): ( 20.014, 23.546 )
\end{verbatim}
}

\begin{figure}[ht]
\centerline{\boxit{\box3}}
\caption {Simulation results for the guided visits model.}
\label {fig:visits-res}
\end{figure}

The simulation results are in Figure~\ref{fig:visits-res}.

Could we have used a \texttt{Condition} instead of a \texttt{Bin}
for synchronizing the \texttt{Guide} processes?
The problem would be that if several guides are waiting for a
condition indicating that the queue size has reached 8, 
{\em all\/} these guides (not only the first one)
would resume their execution 
simultaneously when the condition becomes true.


\bibliography{simul,random,ift,stat,prob,fin,callc}
\bibliographystyle{plain}

\end{document}
